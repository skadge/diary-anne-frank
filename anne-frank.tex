\documentclass{book}
\usepackage[dutch]{babel}
\usepackage{palatino}
\usepackage{fancyhdr}

\title{Het Achterhuis}
\lhead{Het Achterhuis}
\author{Anne Frank}
\rhead{Anne Franck}

\date{Amsterdam 1947}

\pagestyle{fancy}

\begin{document}

\maketitle

\newpage


\section*{Zondag, 14 Juni 1942}

Vrijdag 12 Juni was ik al om 6 uur wakker en dat is heel begrijpelijk, daar ik
jarig was. Maar om 6 uur mocht ik toch nog niet opstaan, dus moest ik mijn
nieuwsgierigheid bedwingen tot kwart voor zeven. Toen ging het niet langer, ik
ging naar de eetkamer, waar ik door Moortje (de kat) met kopjes verwelkomd werd.

Om even na zevenen ging ik naar papa en mama en dan naar de huiskamer, om mijn
cadeautjes uit te pakken. Het was in de eerste plaats~\emph{jou}~die ik te zien
kreeg, wat misschien wel een van mijn jnste cadeau's is. Dan een bos rozen, een
plantje, twee takken pinkster-rozen, dat waren die ochtend de kinderen van
Flora, die op mijn tafel stonden, maar er kwam nog veel meer.

Van papa en mama heb ik een heleboel gekregen en ook door onze vele kennissen
ben ik erg verwend. Zo ontving ik o.a. de\emph{Camera Obscura}, een
gezelschapsspel, veel snoep, chocola, een puzzle, een broche,~\emph{Nederlandse
Sagen en Legenden}~door Joseph Cohen,~\emph{Daisy's Bergvacantie}, een enig
boek, en wat geld, zodat ik me~\emph{Mythen van Griekenland en Rome}~kan kopen,
mooi!

Dan kwam Lies mij halen en wij gingen naar school. In de pauze tracteerde ik
leraren en leerlingen op boterkoekjes en toen moesten we weer aan het werk.

Nu moet ik ophouden. Dáág, ik vind je zo mooi!

\section*{Maandag, 15 Juni 1942}

Zondagmiddag had ik mijn verjaardag-partijtje. We hebben een film
gehad,~\emph{De Vuurtorenwachter}~met Rin-tin-tin, die zeer in de smaak viel bij
mijn klasgenootjes. We hadden veel pret en het was heel gezellig. Er waren veel
jongens en meisjes. Moeder wil nog altijd weten met wien ik later zou willen
trouwen. Ze kan nooit vermoeden, dat het Peter Wessel is, omdat ik het haar
destijds zo zonder blikken af blozen uit het hoofd gepraat heb. Met Lies Goosens
en~Sanne Houtman ben ik al jaren bevriend, het waren mijn beste vriendinnen.
Intussen heb ik Jopie de Waal op het Joodse Lyceum leren kennen. Wij zijn veel
samen en zij is nu mijn beste vriendin. Lies is nu meer met een ander meisje
samen en Sanne is op een andere school en heeft daar vriendinnen.

\section*{Zaterdag, 20 Juni 1942}

Ik heb een paar dagen niet geschreven, omdat ik eerst eens goed over het hele
dagboek-idee moest nadenken. Het is voor iemand als ik een heel eigenaardige
gewaarwording om in een dagboek te schrijven. Niet alleen dat ik nog nooit
geschreven heb, maar het komt me zo voor, dat later noch ik, noch iemand anders
in de ontboezemingen van een dertienjarig schoolmeisje belang zal stellen. Maar
ja, eigenlijk komt dat er niet op aan, ik heb zin om te schrijven en nog veel
meer om mijn hart over allerlei dingen eens grondig en helemaal te luchten.

`Papier is geduldiger dan mensen', dit gezegde schoot me te binnen toen ik, op
een van mijn licht-melancholieke dagen, verveeld met mijn hoofd op mijn handen
zat en van lamlendigheid niet wist of ik uit moest gaan dan wel thuis blijven,
en zo uiteindelijk op dezelfde plek bleef zitten piekeren. Ja, inderdaad, papier
is geduldig, en daar ik niet van plan ben dat gecartonneerde schrift, dat de
weidse naam `dagboek' draagt, ooit aan iemand te laten lezen, tenzij ik nog eens
ooit in mijn leven een vriend of vriendin krijg die dan `de' vriend of vriendin
is, kan het waarschijnlijk niemand schelen. Nu ben ik bij het punt beland, waar
het hele dagboekidee om begonnen is: ik heb geen vriendin.

Om nog duidelijker te zijn, moet hierop een verklaring volgen, want niemand kan
begrijpen dat een meisje van dertien helemaal alleen op de wereld staat; dat is
ook niet waar: ik heb lieve ouders en een zuster van 16, ik heb alles bij elkaar
geteld zeker wel 30 kennisjes en wat je dan vriendinnen noemt, - ik heb een
stoet aanbidders, die me naar~de ogen zien en als het niet anders kan, met een
gebroken zakspiegeltje in de klas nog een glimp van me trachten op te vangen. Ik
heb familie, lieve tantes en ooms, een goed thuis, neen, ogenschijnlijk
ontbreekt het me aan niets, behalve aan `de' vriendin. Ik kan met geen van mijn
kennisjes iets anders doen dan pret maken, ik kan er nooit toe komen eens over
iets anders te spreken dan over alledaagse dingen. Want wat intiemer te worden
is niet mogelijk, en daar zit hem de knoop. Misschien ligt dat gebrek aan
vertrouwelijkheid bij mij; in ieder geval, het feit is er en het is jammer
genoeg ook niet weg te werken.

Daarom dit dagboek. Om nu het idee van een lang verbeide vriendin nog te
verhogen in mijn fantasie, wil ik niet zo maar gewoon als ieder ander de feiten
in dit dagboek plaatsen, maar wil dit dagboek de vriendin zelf laten zijn en die
vriendin heet Kitty.

Daar niemand iets van mijn verhalen aan Kitty zou snappen, als ik zo met de deur
in huis kom vallen, moet ik in het kort mijn levensgeschiedenis weergeven,
hoewel niet graag.

Mijn vader trouwde pas op zijn 36ste jaar met mijn moeder, die toen 25 was. Mijn
zuster Margot werd in 1926 geboren in Frankfurt a/M., op 12 Juni 1929 volgde ik
en daar we volbloed -Joden zijn, emigreerden we in 1933 naar Nederland, waar
mijn vader aangesteld werd als directeur van de Travies N.V.. Deze staat in
nauwe relatie tot de firma Kolen \& Co. in hetzelfde gebouw, waarvan vader mede
deelgenoot is.

Ons leven verliep met de nodige opwindingen, daar de overgebleven familie niet
door Hitlers Jodenwetten gespaard bleef. In 1938 na de progroms vluchtten mijn
twee ooms, broers van mijn moeder en belandden veilig in U.S.A. Mijn oude
grootmoeder kwam bij ons, ze was toen 73 jaar. Na Mei 1940 ging het bergaf met
de goede tijden: eerst de oorlog, de capitulatie, intocht der Duitsers, waarna
de ellende voor ons Joden begon. Jodenwet volgde op Jodenwet.

Joden moeten een Jodenster dragen. Joden moeten hun fietsen afgeven. Joden mogen
niet in de tram, Joden mogen niet meer in auto's rijden. Joden mogen alleen van
3-5 uur boodschappen doen en alleen in Joodse winkels, waar `Joods lokaal'
opstaat. Joden mogen vanaf 8 uur 's avonds niet op straat zijn en ook niet in
hun tuin zitten, noch bij kennissen. Joden mogen zich niet in schouwburgen,
bioscopen of andere voor vermaak dienende plaatsen ophouden, Joden mogen in het
openbaar generlei sport beoefenen, ze mogen geen zwembad, tennisbaan, hockeyveld
of andere sportplaats betreden. Joden mogen ook niet bij Christenen aan huis
komen. Joden moeten op Joodse scholen gaan en nog veel meer van dergelijke
beperkingen.

Zo ging ons leventje door en we mochten dit niet en dat niet. Jopie zei altijd
tegen me: `Ik durf niets meer te doen, want ik ben bang dat het niet mag'. Onze
vrijheid werd dus zeer beknot, maar het is nog uit te houden.

Oma stierf in Januari 1942; niemand weet hoeveel~\emph{ik}~aan haar denk en nog
van haar houd.

Al in 1934 kwam ik in de kleuterklas van de Montessorischool en ben verder op
deze school gebleven. In 6B kwam ik bij de directrice, mevr.  K., terecht; wij
namen hartroerend afscheid aan het einde van het schooljaar en huilden alle
twee. In 1941 werd ik, evenals mijn zuster Margot, naar het Joodse Lyceum
overgeplaatst, zij in de 4 de en ik in de 1 ste klas.

Met ons gezin van vier gaat het nog steeds goed en zo ben ik dan op de huidige
datum aangeland.

\section*{Zaterdag, 20 Juni 1942}

Lieve Kitty,\\
 Dan begin ik maar meteen; het is nu zo lekker rustig, vader en
moeder zijn uit en~Margot is met wat jongelui bij een vriendin gaan ping-pongen.

Ping-pongen doe ik de laatste tijd ook erg veel. Daar wij ping-pongsters vooral
in~de zomer erg van ijs houden en~ping-pong warm maakt, loopt zo'n spel meestal
uit op een tochtje naar de dichtstbijzijnde ijswinkels, die voor Joden
geoorloofd zijn, namelijk~\emph{Delphi}~of\emph{Oase}. Naar portemonnaie of geld
zoeken we al lang niet meer; bij de~\emph{Oase}~is het meestal zo druk, dat er
onder al de mensen altijd wel wat goedgeefse heren van onze uitgebreide
kennissenkring of de een of andere aanbidder te vinden is, die ons meer ijs
aanbieden dan we in een week opkunnen.

Ik denk, dat je wel een beetje verbaasd zult zijn over het feit, dat ik, zo jong
als ik ben, over aanbidders spreek. Helaas, dit euvel schijnt bij ons op school
niet te vermijden te zijn. Zodra een jongen mij vraagt of hij met mij mee naar
huis mag fietsen en daarop een gesprek begint, kan ik er negen van de tien keer
van overtuigd zijn, dat de desbetreffende jongeling de lastige gewoonte heeft
dadelijk in vuur en vlam te raken en me niet meer uit het oog laat. Na verloop
van tijd zakt die verliefdheid heus wel weer af, vooral omdat ik me van vurige
blikken niet veel aantrek en lustig doorpeddel. Als het soms te bont wordt en ze
over `vader vragen' gaan bazelen, zwaai ik een beetje met mijn fiets, tas valt,
de jongeman moet fatsoenshalve afstappen en na de tas weer afgeleverd te hebben,
heb ik al lang weer een ander gespreksthema aangeknoopt.

Dit zijn nog maar de onschuldigsten, je hebt er natuurlijk ook onder, die
kushandjes sturen of een arm proberen te bemachtigen, maar dan zijn ze beslist
aan het verkeerde adres. Ik stap af en weiger verder van zijn gezelschap gebruik
te maken, of ben zogenaamd beledigd en zeg hem in kernachtige bewoordingen, dat
hij naar huis kan gaan.

Ziezo, de grondslag voor onze vriendschap is gelegd, tot morgen! Je Anne.

\section*{Zondag, 21 Juni 1942}

Lieve Kitty,\\
Onze hele 1 B-klas bibbert; de aanleiding is de in het
vooruitzicht gestelde~leraarsvergadering. De halve klas is aan het~wedden over
zittenblijven of overgaan. Miep de Jong en ik lachen ons naar om onze twee
achterburen Wim en Jacques, die hun hele vacantiekapitaal tegen elkaar verwed
hebben. `Jij gaat over', `Nietes', `Welles', van 's morgens vroeg tot 's avonds
laat, zelfs Mieps smekende blikken om stilte en mijn boze uitvallen kunnen die
twee niet tot kalmte brengen.

Volgens mij moest een vierde van de hele klas blijven zitten, zulke uilen zitten
er in, maar leraren zijn de nukkigste mensen die er bestaan, misschien zijn ze
nu, bij uitzondering, eens nukkig naar de~\emph{goede}~kant.

Voor mijn vriendinnen en mijzelf ben ik niet zo bang, we zullen er wel doorheen
rollen. Alleen in wiskunde ben ik onzeker. En n, afwachten maar. Tot zolang
spreken we elkander moed in.

Ik kan het met al mijn leraren en leraressen nogal goed vinden, het zijn er
negen in getal, waarvan zeven mannelijke en twee vrouwelijke.  Mijnheer Kepler,
de oude wiskundeleraar, was een tijd lang erg kwaad op me, omdat ik zoveel
kletste: vermaning volgde op vermaning, tot ik strafwerk kreeg. Een opstel maken
met tot onderwerp `Een kletskous'. Een kletskous! Wat kan je daar nu over
schrijven? Maar dat was van later zorg en ik stopte mijn agenda na inschrijving
in mijn tas en probeerde me rustig te houden.

's Avonds, thuis, toen al mijn andere huiswerk af was, viel mijn oog op de
aantekening van het opstel. Met het eindje van mijn vulpen in mijn mond begon ik
over het onderwerp na te denken; zo maar gewoon wat bazelen en zo wijd mogelijk
uit elkaar schrijven kan iedereen, maar een afdoende bewijs voor de
noodzakelijkheid van het praten te vinden, dat was de kunst. Ik dacht en dacht,
dan had ik opeens een idee, ik pende mijn 3 opgegeven kantjes vol en was
voldaan. Als argumenten had ik aangevoerd, dat praten vrouwelijk is, dat ik wel
mijn best zou doen het een beetje te temperen, maar afleren zou ik het wel
nooit, daar mijn moeder net zoveel~praatte als ik, zo niet meer en dat aan
overgeërfde eigenschappen nu eenmaal weinig te doen is.

Mijnheer Kepler moest erg om mijn argumenten lachen, maar toen ik mijn
praat-uurtje in de volgende les toch weer voortzette, volgde ook het tweede
opstel. Ditmaal moest het een `onverbeterlijke kletskous' zijn.  Ook dit werd
ingeleverd en Kepler had twee lessen lang niet te klagen.  In de derde les werd
het hem echter weer te bont. `Anne, als strafwerk voor praten, een opstel over
het onderwerp `Kwek, kwek, kwek, zei juffrouw Snaterbek'. De klas schaterde. Ik
moest wel meelachen, hoewel mijn vindingrijkheid op het gebied van
kwebbelopstellen uitgeput was. Ik moest er iets anders, heel origineels op
vinden. Het toeval kwam me te hulp, mijn vriendin Sanne, goede dichteres, bood
haar hulp aan om het opstel van begin tot eind op rijm in te leveren. Ik
juichte. Kepler wou me met dit onzinnige onderwerp in de maling nemen, ik zou
hem met mijn gedicht driedubbel in de maling nemen.

Het gedicht kwam af en was prachtig. Het handelde over een moeder eend en een
vader zwaan, met drie kleine eendjes, die wegens te veel kwebbelen door den
vader doodgebeten werden. Kepler verstond de grap gelukkig goed, hij las het
gedicht met commentaar in de klas voor en in verschillende andere klassen ook
nog.

Sindsdien mocht ik praten en kreeg nooit meer strafwerk, integendeel, Kepler
maakt nu altijd grapjes.

Je Anne.

\section*{Woensdag, 24 Juni 1942}

Lieve Kitty,\\
Het is smoorheet, iedereen puft en bakt enfin die hitte moet ik
alles belopen. Nu zie ik pas hoe mooi een tram toch is, vooral een open, maar
dat genot is voor ons Joden niet langer weggelegd, voor ons is de benenwagen
goed genoeg. Gisteren moest ik tussen de middag naar de Jan Luykenstraat~naar de
tandarts. Het is van onze school in de Stadstimmertuinen een lange weg; op
school viel ik 's middags dan ook haast in slaap. Gelukkig dat de mensen je
vanzelf al iets te drinken aanbieden, de zuster van den tandarts is werkelijk
een hartelijk mens.

Het enige waarop we nog mogen is de pont. Over de Josef Israëlskade vaart een
kleine boot, waarvan de veerman ons dadelijk meenam, toen we hem vroegen ons
over te zetten. Aan de Hollanders ligt het heus niet, dat wij Joden het zo
ellendig hebben.

Ik wou maar, dat ik niet naar school moest, mijn fiets is in de Paasvacantie
gestolen en die van moeder heeft vader bij Christenmensen in bewaring gegeven.
Maar gelukkig nadert de vacantie met snelle schreden, nog een week en het leed
is geleden.

Gistermorgen is me wat leuks overkomen. Ik kwam langs de rijwielstalling, toen
iemand me riep. Ik keek om en zag een aardigen jongen achter me staan, dien ik
de avond tevoren bij Eva, een kennisje, ontmoet had. Hij kwam een beetje
verlegen dichterbij en stelde zich voor als Harry Goldberg. Ik was een beetje
verbaasd en wist niet goed wat hij wilde, maar dat bleek al spoedig. Harry wou
van mijn gezelschap gebruik maken en me naar school brengen. `Als je toch
dezelfde kant op moet, ga ik wel mee', antwoordde ik en zo gingen we samen.
Harry is al 16 en weet over allerlei dingen leuk te vertellen. Vanochtend
wachtte hij weer op me enfin het vervolg zal het nu wel zo blijven.

Je Anne.

\section*{Dinsdag, 30 Juni 1942}

Lieve Kitty,\\
Tot vandaag toe kon ik heus geen tijd vinden om weer te
schrijven. Donderdag~was ik de hele middag bij kennissen, Vrijdag hadden we
visite en zo ging het door tot vandaag. Harry en ik hebben elkaar in die week
goed leren kennen,~hij heeft me veel van zijn leven verteld; hij is zonder zijn
ouders hier in Nederland gekomen bij zijn grootouders. Zijn ouders zijn in
België.

Harry had een meisje, Fanny, ik ken haar als een voorbeeld van zachtheid en
saaiheid. Sinds hij mij ontmoet heeft, is Harry tot de ontdekking gekomen, dat
hij aan Fanny's zijde inslaapt. Ik ben dus zo'n soort wakkerhoudmiddel, een mens
weet nooit waar hij nog eens voor gebruikt wordt.

Zaterdagavond heeft Jopie bij me geslapen, maar Zondagmiddag was ze bij Lies en
ik heb me doodverveeld. Harry zou 's avonds bij mij komen, maar bij zessen belde
hij op. Ik ging naar de telefoon, toen zei hij: `Hier is Harry Goldberg, mag ik
Anne alstublieft even spreken?'

`Ja, Harry, hier is Anne'.\\
`Dag Anne, hoe gaat het met je?'\\
`Goed, dank je
wel'.\\
`Ik moet je tot mijn spijt zeggen, dat ik vanavond niet bij je kan
komen, maar ik

zou je nu nog wel even willen spreken; is het goed dat ik over tien minuten voor
je deur ben?'

`Ja, dat is goed, nou dáág!!!'\\
`Dáág, ik kom direct'.\\
Hoorn neer.\\
Ik ben
me gauw gaan verkleden en heb mijn haar een beetje opgedoft. Toen ben

ik zenuwachtig uit het raam gaan hangen. Eindelijk kwam~\emph{hij}~er aan.
Wonder boven wonder ben ik niet meteen de trap afgesuisd, maar heb rustig
afgewacht tot hij gebeld heeft. Ik ging naar beneden en hij viel meteen met de
deur in huis.

`Zeg Anne, mijn grootmoeder vindt jou nog te jong om geregeld mee om te gaan en
ze zegt, dat ik naar de Leurs moet gaan, maar je weet misschien, dat ik niet
meer met Fanny omga!'

`Neen, hoezo, hebben jullie ruzie gehad?'

`Neen, integendeel, ik heb tegen Fanny gezegd, dat we toch niet goed met elkaar
konden opschieten en dat we~daarom maar niet meer met elkaar om moesten gaan,
maar dat Fanny bij ons nog heel welkom was, en dat ik hoopte, dat ik dat ook bij
haar was. Ik dacht namelijk dat Fanny met een anderen jongen rondzwierf en heb
haar daar ook naar behandeld. Maar dat was helemaal niet waar, en nu zei mijn
oom dat ik Fanny excuus moest aanbieden, maar dat wilde ik natuurlijk niet en
daarom heb ik het uitgemaakt. Maar dat was maar een van de vele redenen. Mijn
grootmoeder wil namelijk dat ik naar Fanny ga, en niet naar jou, maar daar denk
ik niet over; oude mensen hebben soms erg ouderwetse begrippen en daar kan ik me
niet naar richten. Ik heb mijn grootouders wel nodig, maar in zekere zin hebben
zij mij toch ook nodig.

Nu heb ik voortaan Woensdagavonds altijd vrij, omdat ik voor mijn grootouders
naar houtsnijles ga, maar in werkelijkheid ga ik naar zo'n clubje van de
zionistische beweging. Dat mag ik niet, omdat mijn grootouders erg tegen het
zionisme zijn. Ik ben er ook niet fanatiek vóór, maar ik voel er wel wat voor en
interesseer me er voor. Maar de laatste tijd is het daar ook zo'n rommelzootje,
dat ik van plan ben er uit te gaan en daarom is Woensdagavond de laatste avond
dat ik er naar toe ga. Dan kan ik dus Woensdagavond, Zaterdagmiddag,
Zaterdagavond, Zondagmiddag en misschien nog wel vaker met jou omgaan'.

`Maar als je grootouders het nu niet willen, dan moet jij het toch niet achter
hun rug doen!'

`Liefde laat zich nu eenmaal niet dwingen'.

Toen kwamen we langs de boekhandel op de hoek en daar stond Peter Wessel met nog
twee jongens; het is voor het eerst in lange tijd dat hij me weer groette en ik
had er echt plezier in.

Harry en ik liepen intussen steeds maar blokjes en het eind van het liedje was
een afspraak, dat ik de volgende avond om vijf minuten voor zeven in Harry's
stoep zou staan.

Je Anne.

\section*{Vrijdag, 3 Juli 1942}

Lieve Kitty,\\
Gisteren was Harry bij ons thuis om met mijn ouders kennis te
maken. Ik had taart~en snoep gehaald, thee en koekjes, van alles was er, maar
Harry noch ik hadden zin om zo naast elkaar op een stoel te blijven zitten, we
zijn gaan wandelen en pas om 10 over 8 werd ik thuis afgeleverd. Vader was erg
kwaad, vond het geen manier dat ik te laat thuis was, omdat het voor Joden
gevaarlijk is na 8 uur buiten te zijn en ik moest beloven in het vervolg al om
10 minuten vóór acht binnen te zijn.

Morgen ben ik bij hem gevraagd. Mijn vriendin Jopie plaagt me steeds met Harry.
Ik ben heus niet verliefd, o neen, ik mag toch wel vrienden hebben, niemand ziet
er iets in, dat ik een vriendje heb of, zoals moeder het uitdrukt, een cavalier.

Eva heeft me verteld, dat Harry op een avond bij haar was en dat zij aan hem
vroeg: `Wie vind je aardiger, Fanny of Anne?' Toen zei hij: `Dat gaat je niets
aan'. Maar toen hij de deur uitging, (ze hadden de hele avond niet meer samen
gekletst) zei hij: `Nou, Anne hoor, daag, en aan niemand vertellen'. Floep, ging
hij de deur uit.

Je kunt aan alles merken dat Harry nu verliefd op mij is, en dat vind ik voor de
afwisseling wel leuk. Margot zou zeggen: `Harry is een erg geschikt joch', en
dat vind ik ook en nog wel meer dan dat. Moeder zit hem ook ontzettend te
roemen, een knap joch (van gezicht dan), een beleefde jongen en een aardige
jongen; ik ben blij dat Harry bij alle huisgenoten zo in de smaak valt. Hij mag
ze ook graag. Maar mijn vriendinnen vindt hij erg kinderachtig en daar heeft hij
gelijk in.

Je Anne.

\section*{Zondagochtend, 5 Juli 1942}

Lieve Kitty,\\
De promotie Vrijdag in de Joodse Schouwburg is naar wens
verlopen. Mijn rapport~is helemaal niet zo slecht,~ik heb één onvoldoende, een
vijf voor algebra, twee zessen, verder allemaal zevens en twee achten. Thuis
waren ze wel blij, maar mijn ouders zijn in cijferaangelegenheden heel anders
dan andere ouders. Ze trekken zich nooit iets van goede of slechte rapporten aan
en letten er alleen op of ik gezond ben, niet te brutaal en pret heb; als deze
dingen in orde zijn, komt al het andere wel vanzelf. Ik ben het tegendeel, ik
wil geen slechte leerling zijn, ik ben voorwaardelijk op het Lyceum aangenomen,
omdat ik eigenlijk nog in de zevende klas van de 6de Montessorischool had moeten
blijven. Maar toen alle Joodse kinderen naar Joodse scholen moesten, nam de
directeur mij en Lies, na wat heen en weer gepraat, voorwaardelijk op. En ik wil
zijn vertrouwen niet beschamen. Mijn zuster Margot heeft ook haar rapport,
schitterend zoals gewoonlijk. Als cum laude bij ons zou bestaan, was ze zeker
met lof overgegaan, zo'n knappe bol!

Vader is de laatste tijd veel thuis, in de zaak heeft hij niets meer te zoeken;
naar gevoel moet dat zijn om je nu zo overbodig te voelen.  Mijnheer Koophuis
heeft de Travies overgenomen en mijnheer Kraler de firma Kolen \& Co. Toen we
een paar dagen geleden samen om ons pleintje wandelden, begon vader over
onderduiken te praten. Hij had het er over, dat het erg moeilijk voor ons zal
zijn om helemaal afgescheiden van de wereld te leven. Ik vroeg hem, waarom hij
daar nu al over sprak. `Ja, Anne', zei hij daarop, `je weet dat we al meer dan
een jaar kleren, levensmiddelen en meubelen naar andere mensen brengen. Wij
willen onze goederen niet in de handen van de Duitsers laten vallen, maar nog
minder willen wij zelf gepakt worden. Wij zullen daarom uit onszelf weggaan en
niet wachten, tot we gehaald worden'.

`Maar vader, wanneer dan?' Ik werd angstig door de ernst, waarmee vader dit zo
zei.

`Maak je daar maar niet ongerust over, dat regelen wij wel, geniet van je
onbezorgd leventje, zolang je er van~genieten kunt'. Dat was alles. O laat de
vervulling van deze sombere woorden nog lang verre blijven!

Je Anne.

\section*{Woensdag, 8 Juli 1942}

Lieve Kitty,\\
Vanaf Zondagmorgen tot nu lijkt een afstand van jaren. Er is
zoveel gebeurd, dat het is alsof de hele wereld zich plotseling omgedraaid
heeft. Maar Kitty, je merkt dat ik nog leef en dat is de hoofdzaak, zegt vader.

Ja, inderdaad, ik leef nog, maar vraag niet waar en hoe. Ik denk dat je vandaag
helemaal niets van me begrijpt, daarom zal ik maar beginnen met je te vertellen
wat er Zondagmiddag gebeurd is.

Om 3 uur (Harry was even weggegaan, om later terug te komen) belde er iemand aan
de deur. Ik hoorde het niet, daar ik lui in een ligstoel op de veranda in de zon
lag te lezen. Even later verscheen Margot in opgewonden toestand aan de
keukendeur. `Er is een oproep van de S.S.  voor vader gekomen', fluister de ze,
`moeder is al naar mijnheer van Daan gegaan'. (Van Daan is een goede kennis en
medewerker in papa's zaak.) Ik schrok ontzettend, een oproep, iedereen weet wat
dat betekent; concentratiekampen en eenzame cellen zag ik al in mijn geest
opdoemen en daarnaartoe zouden wij vader laten vertrekken. `Hij gaat natuurlijk
niet', verklaarde Margot mij, toen wij in de kamer op moeder zaten te wachten.

`Moeder is naar Van Daan om te bespreken, of we morgen naar onze schuilplaats
zullen vertrekken. De Van Daans gaan met ons mee onderduiken, we zijn dan met
ons zevenen'. Stilte. We konden niet meer spreken, de gedachte aan vader, die
geen kwaad vermoedend op bezoek was bij oude mensen in de Joodse Invalide, het
wachten op moeder, de warmte, de spanning, dat alles deed ons zwijgen.

Plotseling werd er weer gebeld. `Dat is Harry', zei ik.

`Niet open doen', hield Margot mij tegen, maar dat was overbodig, we hoorden
moeder en mijnheer Van Daan beneden met Harry praten, toen kwamen ze binnen en
sloten de deur achter zich dicht. Bij elke bel moesten Margot of ik nu zachtjes
naar beneden om te zien of het vader was, andere mensen lieten we niet toe.

Margot en ik werden uit de kamer gestuurd. Van Daan wou met moeder alleen
spreken. Toen Margot en ik in onze slaapkamer zaten, vertelde zij, dat de oproep
niet vader maar haar betrof. Ik schrok opnieuw en begon toen te huilen. Margot
is zestien; zulke jonge meisjes willen ze dus alleen weg laten gaan. Maar
gelukkig zou ze niet gaan, moeder had het zelf gezegd en daarop zouden dan ook
vaders woorden wel gedoeld hebben, toen hij het met mij over onderduiken had.

Onderduiken, waar zouden we gaan onderduiken, in de stad, op het land, in een
huis, in een hut, wanneer, hoe, waar...? Dat waren vragen, die ik niet mocht
stellen, en die toch steeds weer terugkwamen. Margot en ik begonnen het nodigste
in een schooltas te pakken. Het eerste wat ik er in stopte, was dit
gecartonneerde schrift, daarna krulpennen, zakdoeken, schoolboeken, kam, oude
brieven; ik dacht aan het onderduiken en stopte daardoor de gekste onzin in de
tas. Maar ik heb er geen spijt van; geef meer om herinneringen dan om jurken.

Om 5 uur kwam vader eindelijk thuis, we belden mijnheer Koophuis op en vroegen
of hij nog 's avonds zou kunnen komen. Van Daan ging weg en haalde Miep. Miep is
vanaf 1933 bij vader in de zaak en een intieme kennis geworden, evenals haar
nieuwbakken echtgenoot Henk. Miep kwam, nam wat schoenen, jurken, jasjes,
ondergoed en kousen in een tas mee en beloofde 's avonds terug te komen. Daarna
was het stil in onze woning; geen van vieren wilden we eten, het was nog warm en
alles was erg vreemd. Onze grote bovenkamer hadden we verhuurd aan een zekeren
mijnheer~Goudsmit, gescheiden man van in de dertig, die deze avond schijnbaar
niets te doen had; daarom bleef hij tot 10 uur bij ons rondhangen en was met
goede woorden niet weg te krijgen. Om elf uur kwamen Miep en Henk van Santen.
Weer verdwenen schoenen, kousen, boeken en ondergoed in Mieps tas en Henks diepe
zakken, om half 12 waren ook zij verdwenen. Ik was doodmoe en hoewel ik wist,
dat het de laatste nacht in mijn bed zou zijn, sliep ik dadelijk en werd pas om
half zes in de ochtend door moeder gewekt. Gelukkig was de hitte iets minder dan
Zondag; een warme regen stroomde de hele dag door. We kleedden ons alle vier zo
dik aan, alsof we in een ijskast zouden moeten overnachten en dat alleen om nog
wat kleren mee te nemen. Geen Jood zou het in onze toestand gewaagd hebben met
een koffer vol kleren uit huis te gaan. Ik had 2 hemdjes, 3 broeken, een jurk,
daarover rok, jasje, zomerjas, twee paar kousen, dichte schoenen, muts, sjaal en
nog meer aan, ik stikte thuis al, maar daar vroeg niemand naar.

Margot stopte haar schooltas vol met schoolboeken, haalde haar fiets uit de
stalling en reed achter Miep aan weg naar voor mij onbekende verten.  Ik wist
namelijk nog steeds niet, waar de geheimzinnige plaats van onze bestemming zou
zijn. Om half acht sloten ook wij de deur achter ons; de enige van wie ik
afscheid te nemen had was Moortje, mijn kleine poesje, dat een goed tehuis bij
de buren zou krijgen, zoals aangegeven stond op een briefje aan mijnheer
Goudsmit geadresseerd.

In de keuken een pond vlees voor de kat, de ontbijtboel op tafel, de bedden
afgehaald, dat alles wekte de indruk alsof we hals over kop vertrokken waren.
Indrukken konden ons niet schelen, weg wilden we, alleen maar weg en veilig
aankomen, anders niets.

Morgen vervolg. Je Anne.

\section*{Donderdag, 9 Juli 1942}

Lieve Kitty,\\
Zo liepen we dan door de stromende regen, vader, moeder en ik,
elk met een school- en boodschappentas, tot bovenaan toe volgepropt met de meest
door elkaar liggende dingen.

De arbeiders, die vroeg naar hun werk gingen, keken ons medelijdend na; op hun
gezicht was duidelijk de spijt te lezen, dat ze ons generlei voertuig konden
aanbieden; de opzichtige gele ster sprak voor zichzelf.

Pas toen we op straat waren, vertelden vader en moeder me bij brokjes en beetjes
het hele onderduikplan. Al maandenlang hadden ze zoveel van onze inboedel en ons
lijfgoed als mogelijk was het huis uitgedaan en nu waren we juist zover, dat we
vrijwillig op 16 Juli wilden gaan onderduiken.  Door de oproep was dat
onderduikplan tien dagen vervroegd, zodat we ons met minder goed geordende
appartementen tevreden zouden moeten stellen.  De schuilplaats zelf zou in
vaders kantoorgebouw zijn. Dat is voor buitenstaanders een beetje moeilijk te
begrijpen, daarom zal ik het nader toelichten. Vader heeft niet veel personeel
gehad: mijnheer Kraler, Koophuis, Miep en Elli Vossen, de 23-jarige
stenotypiste, die allen van onze komst op de hoogte waren. In het magazijn
mijnheer Vossen, vader van Elli en twee knechten, die we niets gezegd hadden.

Het gebouw zit als volgt in elkaar: parterre is een groot magazijn, dat ook als
pakhuis gebruikt wordt. Naast de magazijndeur bevindt zich de gewone huisdeur,
die door een tussendeur toegang tot een trapje geeft (A). Boven aan de trap
bereikt men een half-matglazen deur, waarop eenmaal in zwarte letters `kantoor'
stond. Dat is het grote voorkantoor, zeer groot, zeer licht, zeer vol. Overdag
werken daar Elli, Miep en mijnheer Koophuis. Via een kabinetje met brandkast,
garderobe en grote voorraadkast komt men aan het kleine, tamelijk donkere
achterkamertje.  Daar zaten vroeger mijnheer Kraler en mijnheer Van Daan, nu
alleen~nog de eerste. Men kan ook vanuit de gang Kralers kantoor bereiken, maar
dan alleen door een glazen deur die van binnen wel, maar van buiten niet zonder
meer te openen is.

Vanuit Kralers kantoor, de lange smalle gang door, langs het kolenhok, vier
treden op, het pronkstuk van het hele gebouw: het privé-kantoor.  Deftige
donkere meubels, linoleum en kleden op de vloer, radio, sjieke lamp, alles
prima-prima. Daarnaast een grote, ruime keuken met warmwatergeyser en twee
gaspitten. Daarnaast W.C. Dat is de eerste verdieping.

Vanuit de benedengang loopt een gewone houten trap naar boven (B).  Bovenaan is
een klein portaaltje, dat als overloopje wordt betiteld.  Rechts en links van de
overloop zijn deuren, de linker leidt naar het voorhuis met pakhuisruimten,
voorzolder en voorvliering. Van dit voorgebouw loopt aan de andere kant ook nog
een lange, hypersteile, echte Hollandse been-breektrap naar de tweede straatdeur
(C).

De rechterdeur leidt naar `het Achterhuis'. Geen mens zou vermoeden, dat achter
de simpele, grijsgeschilderde deur zoveel kamers schuilgaan. Voor de deur is een
stoepje en dan ben je binnen.

Recht tegenover de ingangsdeur een steile trap (E), links brengt een klein
gangetje je in een kamer, die de huis- en slaapkamer van de familie Frank zou
worden, daarnaast een kleinere kamer, slaap- en werkkamer van de twee jongedames
Frank. Rechts van de trap een kamer zonder raam met wastafel en een afgesloten
W.C.-hokje, ook weer een toegangsdeur naar Margots en mijn kamer.

Als men de trap opgaat en de deur aan het boveneinde opent, staat men verbaasd,
dat in zo'n oud grachtenhuis zo'n groot, licht en ruim vertrek bestaat. In dit
vertrek is een gasfornuis (dat hebben we te danken aan het feit, dat het tot nu
toe als laboratorium diende) en een gootsteen.  Dit nu is de keuken en tevens de
slaapkamer voor het echt-paar Van Daan, alsmede gemeenschappelijke huiskamer,
eetkamer en werkkamer. Een heel klein doorloopkamertje zal Peter van Daans
appartement worden. Dan,~net als vóór, een zolder en vliering. Ziehier, ik
stelde je ons hele, mooie Achterhuis voor.

Je Anne.

\section*{Vrijdag, 10 Juli 1942}

Lieve Kitty,\\
Het is zeer waarschijnlijk, dat ik je met mijn langdradige
woningbeschrijving danig verveeld heb, maar toch vind ik het noodzakelijk, dat
je weet, waar ik beland ben. Nu het vervolg van mijn verhaal, want dat was nog
niet klaar, weet je. Op de Prinsengracht aangekomen nam Miep ons gauw mee naar
boven, het Achterhuis in. Ze sloot de deur achter ons en we waren alleen. Margot
was er met de fiets veel sneller aangekomen en wachtte al op ons. Onze huiskamer
en alle andere kamers waren zo vol rommel, dat het niet te beschrijven is. Alle
cartonnen dozen, die in de loop van de voorafgegane maanden naar kantoor
gestuurd waren, stonden op de grond en op de bedden. Het kleine kamertje was tot
aan het plafond met beddegoed gevuld. Als we 's avonds op behoorlijk opgemaakte
bedden zouden willen slapen, moesten we direct aan de gang gaan en de boel
opruimen. Moeder en Margot waren niet in staat een lid te verroeren, zij lagen
op de kale bedden, waren moe, ellendig en ik weet niet wat al meer. Maar vader
en ik, de twee opruimers in de familie, wilden dadelijk beginnen.

We pakten de hele dag door dozen uit, kasten in, hamerden en ruimden, totdat we
's avonds doodmoe in de schone bedden vielen. De hele dag hadden we geen warm
eten gehad, maar dat hinderde ons niet; moeder en Margot waren te moe en te
overspannen om te eten, vader en ik hadden te veel werk.

Dinsdagochtend begonnen we daar, waar we Maandag opgehouden waren. Elli en Miep
haalden de boodschappen op onze levensmiddelenbonnen, vader verbeterde de
onvoldoende verduistering, we schrobden de keukenvloer en waren opnieuw van 's
ochtends tot 's avonds in de weer. Tijd om over de grote verandering die er in
mijn leven gekomen was na te denken, had ik haast niet tot Woensdag. Toen vond
ik voor het eerst sinds onze aankomst in het Achterhuis gelegenheid om je de
gebeurtenissen mede te delen en tegelijkertijd om me eens goed te realiseren,
wat er nu eigenlijk met me gebeurd was en wat er nog gebeuren zou.

Je Anne.

\section*{Zaterdag, 11 Juli 1942}

Lieve Kitty,\\
Vader, moeder en Margot kunnen nog steeds niet aan het geluid van
de Westertorenklok wennen, die om het kwartier zegt hoe laat het is. Ik wel, ik
vond het dadelijk mooi en vooral 's nachts is het zo iets vertrouwds. Het zal je
wel interesseren te horen hoe het me in mijn `duik' bevalt, welnu, ik kan je
alleen zeggen, dat ik het zelf nog niet goed weet. Ik geloof, dat ik me in dit
huis nooit thuis zal voelen, maar daarmee wil ik helemaal niet zeggen, dat ik
het hier naar vind, ik voel me veeleer als in een heel eigenaardig pension, waar
ik met vacantie ben. Nogal een gekke opvatting van onderduiken, maar het is nu
eenmaal niet anders. Het Achterhuis is als schuilplaats ideaal. Hoewel het
vochtig is en scheefgetrokken zal men nergens in Amsterdam zo iets geriefelijks
voor onderduikers ingericht hebben, ja misschien zelfs nergens in heel
Nederland.

Ons kamertje was met die strakke muren tot nu toe erg kaal; dank zij vader, die
mijn hele filmsterrenverzameling en mijn prentbriefkaarten van tevoren al
meegenomen had, heb ik, na met een lijmpot en kwast de hele muur bestreken te
hebben, van de kamer één plaatje gemaakt. Daardoor ziet het er veel vrolijker
uit en als de Van Daans komen,~zullen we met het hout dat op zolder staat wel
wat muurkastjes en andere aardige prullen maken.

Margot en moeder zijn weer een klein beetje opgeknapt. Gisteren wou moeder voor
het eerst erwtensoep koken, maar toen ze beneden aan het praten was, vergat ze
de soep, die brandde daardoor zo aan, dat de erwten, koolzwart, niet meer van de
pan los te krijgen waren.

Mijnheer Koophuis heeft het~\emph{Boek voor de Jeugd}~voor me meegebracht.
Gisteravond zijn we met zijn vieren naar het privé-kantoor gegaan en hebben de
Engelse radio aangezet. Ik was zo ontzettend bang dat iemand dat zou kunnen
horen, dat ik vader letterlijk smeekte mee naar boven te gaan. Moeder begreep
mijn angst en ging mee. Ook in andere dingen zijn we erg bang, dat de buren ons
zouden kunnen horen of zien.  Direct de eerste dag hebben we de gordijnen
genaaid. Eigenlijk mag men niet van gordijnen spreken, want het zijn maar losse
lichte lappen, totaal verschillend van vorm, kwaliteit en motief, die vader en
ik erg onvakkundig scheef aan elkaar naaiden. Met punaises zijn deze
pronkstukken voor de ramen bevestigd om er vóór het einde van onze onderduiktijd
nooit meer af te komen.

Rechts naast ons ligt een groot zakenpand, links een meubelmakerij; personeel is
er na werktijd niet in de percelen, maar toch zouden er geluiden kunnen
doordringen. We hebben Margot dan ook verboden 's nachts te hoesten, hoewel ze
een zware verkoudheid te pakken heeft en geven haar grote hoeveelheden codeïne
te slikken.

Ik verheug me erg op de komst van de Van Daans, die op Dinsdag vastgesteld is;
het zal veel gezelliger en ook minder stil zijn. De stilte is het namelijk, die
me 's avonds en 's nachts zo zenuwachtig maakt en ik zou er heel wat voor geven
als iemand van onze beschermers hier zou slapen.

Het benauwt me ook meer dan ik zeggen kan, dat we~\emph{nooit}~naar buiten mogen
en ik ben erg bang, dat we ontdekt worden~en dan de kogel krijgen. Dat is
natuurlijk een minder prettig vooruitzicht.

Overdag moeten we altijd erg zacht lopen en zacht spreken, want in het
magazijn~mogen ze ons niet horen. Net word ik geroepen. Je Anne.

\section*{Vrijdag, 14 Augustus 1942}

Lieve Kitty,\\
Een maand lang heb ik je in de steek gelaten, maar er is ook heus
niet zoveel nieuws om je elke dag iets leuks te vertellen. De Van Daans zijn op
13 Juli aangekomen. Wij dachten, dat het de 14de zou worden, maar daar de
Duitsers tussen de 13de en 16de Juli al meer en meer mensen onrustig maakten en
oproepen naar alle kanten stuurden, vonden ze het veiliger een dag te vroeg dan
een dag te laat te vertrekken. 's Ochtends om half 10 (we zaten nog aan het
ontbijt) arriveerde Peter, Van Daans zoon van nog geen 16 jaren, een tamelijk
saaie en verlegen slungel, van wiens gezelschap niet veel te verwachten is en
bracht zijn poes (Mouschi) mee.  Mevrouw en mijnheer kwamen een half uur later,
mevrouw had tot onze grote pret in haar hoedendoos een grote nachtpot. `Zonder
nachtpot voel ik me nergens thuis', verklaarde ze en de pot was dan ook het
eerste, dat een vaste plaats onder het divanbed kreeg. Mijnheer bracht geen po,
maar zijn inklapbare theetafel onder zijn arm mee.

We aten de eerste dag van ons samenzijn gezellig met elkaar en na drie dagen
wisten we allen niet anders meer dan dat we één grote familie waren geworden.
Zoals vanzelf spreekt hadden de Van Daans nog veel over de week, die zij na ons
in de bewoonde wereld doorbrachten, te vertellen. Onder meer interesseerde het
ons sterk wat er met onze woning en met mijnheer Goudsmit gebeurd was.

Mijnheer Van Daan vertelde:

`Maandagmorgen om 9 uur belde mijnheer Goudsmit ons op en vroeg of ik even langs
kon komen. Ik ging dadelijk~en vond G. in grote opwinding.  Hij liet me een
briefje lezen, dat de familie Frank achtergelaten had en wou naar aanwijzing
daarvan de kat naar de buren brengen wat ik zeer goed vond. Mijnheer G. was
bang, dat er huiszoeking gehouden zou worden, daarom liepen we door alle kamers,
ruimden een beetje op en ruimden de tafel af.

Plotseling ontdekte ik op de schrijftafel van mevrouw een blocnote, waarop een
adres in Maastricht stond aangegeven. Hoewel ik wist dat mevrouw dit expres
achtergelaten had, deed ik zeer verbaasd en verschrikt en verzocht mijnheer G.
dringend dit ongelukspapiertje te verbranden.

Ik hield al die tijd vol, dat ik niets van Uw verdwijnen afwist, maar nadat ik
het papiertje had gezien, kreeg ik een goed idee. ``Mijnheer Goudsmit'', zei ik,
``nu valt me opeens in, waarop dit adres betrekking kan hebben. Ik herinner me
nu opeens, dat er ongeveer een half jaar geleden een hooggeplaatst officier op
kantoor was, het bleek een goede jeugdkennis van mijnheer Frank te zijn, die hem
beloofde hem in geval van nood te zullen helpen en die ook in Maastricht
verblijf hield. Ik denk, dat deze officier woord gehouden heeft en de Franks op
de een of andere manier naar België en vandaar naar Zwitserland zal brengen.
Vertelt u dit ook maar aan kennissen, die eventueel naar de Franks zouden
vragen. Maastricht hoeft u dan natuurlijk niet te vermelden''.

En daarmee ging ik weg. De meeste kennissen weten dit nu al, want van
verschillende kanten heb ik op mijn beurt deze verklaring gehoord'.

Wij vonden het verhaal erg grappig, maar lachten nog meer om de
verbeeldingskracht van de mensen, toen mijnheer Van Daan van andere kennissen
vertelde. Zo had een familie ons alle vier vroeg 's ochtends op de fiets langs
zien komen en een andere mevrouw wist pertinent, dat we midden in de nacht in
een militaire auto waren geladen.

Je Anne.

\section*{Vrijdag, 21 Augustus 1942}

Lieve Kitty,\\
Onze `duik' is nu pas een echte schuilplaats geworden.  Mijnheer
Kraler vond het namelijk beter om voor onze toegangsdeur een kast te plaatsen
(omdat er veel huiszoekingen naar verstopte fietsen gehouden worden), maar dan
natuurlijk een kast die draaibaar is en die dan als een deur opengaat.

Mijnheer Vossen heeft het geval getimmerd. We hebben hem intussen over de zeven
onderduikers ingelicht en hij is één en al hulpvaardigheid. Als we nu naar
beneden willen, moeten we eerst bukken en dan springen, want het stoepje is
verdwenen. Na drie dagen liepen we allemaal met een voorhoofd vol builen, omdat
iedereen zich aan de lage deur stootte. Nu is er doek met houtwol tegenaan
gespijkerd. Zien of het helpt!

Leren doe ik niet veel; tot September hou ik voor mezelf vacantie.  Daarna wil
vader me les geven, want ik ben bitter veel van wat ik op school geleerd heb,
vergeten.

Veel verandering komt er in ons leven hier niet. Mijnheer Van Daan en ik liggen
altijd overhoop, daarentegen houdt hij veel van Margot. Mama doet soms alsof ik
een baby ben en dat kan ik niet uitstaan. Verder gaat het wel wat beter. Peter
vind ik nog steeds niet aardiger, het is een vervelende jongen, hij luiert de
hele dag op zijn bed, timmert een beetje en gaat dan weer dutten. Wat een
stomkop!

Het is buiten mooi warm weer en ondanks alles maken wij er zoveel mogelijk
gebruik van door op het harmonicabed op de zolder te gaan liggen, waar de zon
door een open raam naar binnen schijnt.

Je Anne.

\section*{Woensdag, 2 September 1942}

Lieve Kitty,\\
Mijnheer en mevrouw Van Daan hebben erge ruzie gehad,~ik heb zo
iets nog nooit meegemaakt, daar vader en moeder er niet aan zouden denken zo
tegen elkaar te schreeuwen. De aanleiding was zo nietig, dat het niet de moeite
waard was om er één woord over vuil te maken. Maar ja, ieder zijn meug.

Het is natuurlijk erg onaangenaam voor Peter, die er toch ook maar tussen moet
zitten. Ook wordt hij door niemand au sérieux genomen, daar hij verschrikkelijk
kleinzerig en lui is. Gisteren was hij danig ongerust, omdat hij een blauwe in
plaats van een rode tong gekregen had; dit zeldzame verschijnsel verdween
evenwel net zo gauw als het gekomen was. Vandaag loopt hij met een sjaal om zijn
nek, daar deze stijf is en verder klaagt mijnheer over spit in de rug. Pijnen
tussen hart, nier en long zijn hem ook niet vreemd, hij is een echte hypochonder
(dat noem je toch zo, hè?)!

Tussen moeder en mevrouw Van Daan botert het niet zo erg; aanleidingen tot
onaangenaamheden zijn er genoeg. Om een klein voorbeeld te noemen wil ik je
vertellen dat mevrouw uit de gemeenschappelijke linnenkast op drie na al haar
lakens weggehaald heeft. Zij neemt natuurlijk aan, dat moeders goed voor de hele
familie gebruikt kan worden. Zal haar wel lelijk tegenvallen als ze merkt dat
moeder het goede voorbeeld gevolgd heeft.

Verder heeft mevrouw er erg de pé over in, dat ons servies niet, maar het hare
wèl in gebruik is. Ze probeert steeds te weten te komen, waar wij onze borden
wel naar toe gedaan hebben; ze zijn dichter bij dan ze denkt, want ze staan op
zolder in cartonnen dozen, achter een heleboel reclamemateriaal. Gedurende onze
onderduiktijd zijn de borden onbereikbaar en dat is maar goed ook. Mij gebeuren
altijd ongelukken; gisteren smeet ik een soepbord van mevrouws servies aan
stukken. `O!' riep ze woedend uit, `wil je wel eens voorzichtig zijn, dit is het
enige, dat ik nog heb'. Mijnheer Van Daan is de laatste tijd poeslief tegen mij.
Laat hem maar. Mama heeft vanochtend weer zo ellendig~gepreekt; dat kan ik niet
uitstaan. Onze opvattingen staan precies lijnrecht tegenover elkaar. Papa is een
dot, al is hij wel eens 5 minuten kwaad op me.

Verleden week hadden we een kleine interruptie in ons zo eentonige leven: het
kwam door een boek over vrouwen, - en Peter. Je moet namelijk weten, dat Margot
en Peter haast alle boeken, die mijnheer Koophuis ons leent, mogen lezen, maar
dit bijzondere boek over een vrouwenonderwerp hielden de volwassenen toch maar
liever in eigen handen. Dat prikkelde dadelijk de nieuwsgierigheid van Peter.
Wat zou er wel voor verbodens in dat boek staan? Stiekem pakte hij het van zijn
moeder weg, terwijl zij beneden aan het praten was, en ging met zijn buit naar
de vliering. Dat ging een paar dagen goed. Mevrouw Van Daan wist al lang wat hij
deed, maar verklapte niets, totdat mijnheer er achter kwam. Die werd kwaad,
pakte het boek af en dacht, dat daarmee de zaak afgelopen zou zijn. Hij had
echter buiten de nieuwsgierigheid van zijn zoon gerekend, die door het kordate
optreden van papa geenszins van zijn stuk gebracht was.

Peter zon op mogelijkheden om dit meer dan interessante boek toch uit te lezen.
Mevrouw had intussen bij moeder geïnformeerd wat zij van de kwestie dacht.
Moeder vond dit boek niet goed voor Margot, maar in de meeste andere boeken zag
ze geen kwaad.

`Er is een groot verschil, mevrouw Van Daan', zei moeder, `tussen Margot en
Peter, ten eerste is Margot een meisje, en meisjes zijn altijd rijper dan
jongens, ten tweede heeft Margot al meer serieuze boeken gelezen en zoekt niet
naar dingen die voor haar verboden zijn, ten derde is Margot veel ontwikkelder
en verstandiger, wat haar vier H.B.S.- jaren meebrengen'. Mevrouw stemde daarmee
in, maar vond het toch in principe verkeerd jonge kinderen boeken voor
volwassenen te laten lezen.

Intussen had Peter de geschikte tijd gevonden, waarin~niemand naar het boek of
naar hem omkeek. 's Avonds om half 8, toen de hele familie in het privé-kantoor
naar de radio luisterde, nam hij zijn schat weer mee naar de vliering. Om half 9
had hij weer beneden moeten zijn, maar daar het boek zo spannend was, vergat hij
de tijd en kwam net de zoldertrap af, toen zijn vader de kamer binnenkwam. Wat
volgde is begrijpelijk! Een tik, een klap, een ruk, het boek lag op tafel en
Peter zat op de vliering. Zo stonden de zaken toen de familie ten eten kwam.
Peter bleef boven, niemand bekommerde zich om hem, hij moest zonder eten naar
bed.  Vrolijk keuvelend zetten wij onze maaltijd voort, toen er opeens een
doordringend ge uit tot ons doordrong, allen legden de vorken neer en ieder keek
de ander met bleek en verschrikt gezicht aan. Dan hoorden we Peters stem, die
door de kachelpijp riep: `Ik kom toch niet naar beneden, hoor'. Mijnheer Van
Daan sprong op, zijn servet viel op de grond, met vuurrood hoofd schreeuwde hij:
`Maar nu is het genoeg'. Vader pakte hem bij de arm, daar hij erge dingen
vreesde en samen gingen de twee heren naar de zolder. Na veel tegenstribbelen en
trappen belandde Peter in zijn kamer, de deur ging dicht en wij aten door.
Mevrouw wou een boterham voor zoonlief overlaten, mijnheer was onverbiddelijk.
`Als hij niet dadelijk excuus vraagt, moet hij op de vliering slapen'.

Wij protesteerden en vonden `zonder eten' al straf genoeg. Peter mocht eens
verkouden worden en dan zou er geen dokter aan te pas kunnen komen.

Peter vroeg geen excuus, hij zat alweer op de vliering. Mijnheer Van Daan
bemoeide zich er niet meer mee, maar bemerkte 's morgens dat Peters bed wel
beslapen was. Om 7 uur zat Peter alweer op zolder, maar werd toch door de
vriendschappelijke woorden van vader genoopt naar beneden te komen. Drie dagen
norse gezichten, hardnekkig zwijgen en toen liep alles weer in gewone banen.

Je Anne.

\section*{Maandag, 21 September 1942.}

Lieve Kitty,\\
Vandaag zal ik je even het algemene nieuws van het Achterhuis
vertellen. Mevrouw Van Daan is onuitstaanbaar. Ik krijg van boven voortdurend
standjes voor mijn onophoudelijk geklets. Altijd is er iets anders waarmee ze
ons plaagt; nu doet ze het zo, dat ze de pannen niet wil afwassen en als er nog
een tipseltje inzit, doet ze dat niet in een glazen schotel, zoals het tot nu
toe gedaan werd, maar laat het in de pan bederven. Bij de volgende afwas heeft
Margot dan wel eens zeven vuile pannen en dan zegt madame wel: `Margotje,
Margotje, wat heb je veel te doen!'

Met vader ben ik bezig een stamboom van zijn familie te maken, daarbij vertelt
hij van iedereen wat en dat interesseert me geweldig.

Mijnheer Koophuis brengt om de week een paar meisjesboeken voor me mee, ik ben
enthousiast over de~\emph{Joop ter Heul}-serie. De hele Cissy van Marxveldt
bevalt me in het algemeen bijzonder goed.~\emph{Een Zomerzotheid}~heb ik dan ook
al vier keer gelezen en nog moet ik om de potsierlijke situaties lachen.

Het leren heeft een aanvang genomen, ik doe veel aan Frans en pomp elke dag vijf
onregelmatige werkwoorden. Peter heeft zijn Engelse taak vol zuchten opgenomen.
Enkele schoolboeken zijn pas gearriveerd, voorraad schriften, potloden, vlakjes
en etiketten heb ik in ruime mate van thuis meegenomen. Ik luister soms naar de
Oranje-zender, pas sprak Prins Bernhard. Omstreeks Januari zal er een kindje bij
hen geboren worden, vertelde hij. Ik vind het leuk, hier verbazen ze zich er
over, dat ik zo Oranje-gezind ben.

Een paar dagen geleden hadden ze het er over, dat ik toch nog erg dom ben, met
het gevolg dat ik de volgende dag hard aan het werk ben gegaan.  Ik heb heus
geen zin met 14 of 15 jaar nog in de eerste klas te zitten.

Verder kwam in het gesprek ter sprake, dat ik haast niets behoorlijks lezen mag.
Moeder heeft op het ogenblik~\emph{Heeren, vrouwen en knechten}, dat mag ik niet
hebben (Margot wel), ik moet eerst wat meer ontwikkeld zijn, zoals mijn begaafde
zuster. Dan spraken we over mijn onwetendheid omtrent philosophie, psychologie
en physiologie, waar ik niets van af weet. Misschien ben ik het volgend jaar
wijzer! (Deze moeilijke woorden heb ik gauw in Koenen opgezocht.)

Ik ben tot de schrikwekkende conclusie gekomen, dat ik maar één jurk met lange
mouwen en drie vestjes voor de winter heb. Vader heeft me toestemming gegeven
een trui van witte schapenwol te breien; de wol is wel niet erg mooi, daar zal
warmte tegen op moeten wegen. We hebben nog wat kleren bij andere mensen, die
kunnen jammer genoeg pas na de oorlog gehaald worden, als ze er dan tenminste
nog zijn.

Toen ik pas iets aan je schreef over mevrouw, kwam ze er net aan. Flap!  Boek
dicht.

`Hè Anne, mag ik niet eens kijken?' `Neen, mevrouw'.\\
`Alleen de laatste
bladzijde maar?' `Neen, ook niet, mevrouw'.

Ik schrok me natuurlijk een mik, want op deze bladzijde stond zij juist minder
leuk gedeponeerd.

Je Anne.

\section*{Vrijdag, 25 September 1942}

Lieve Kitty,\\
Gisteravond was ik weer eens `op visite' boven bij de Van Daans
om een beetje te praten. Soms is het er wel gezellig. Dan eten we motten-koekjes
(de trommel stond in een klerenkast die `ingemot' was) en drinken limonade.

Het gesprek ging over Peter. Ik vertelde, dat Peter me zo vaak over mijn wang
strijkt, dat ik dat zo vervelend vond en dat ik niet hield van handtastelijke
jongens.

Zij vroegen echter op oudersmanier of ik niet van Peter~kon houden, daar hij
zeker veel van mij hield. Ik dacht `o jee!' en zei `o neen!' Stel je voor!

Wel zei ik, dat Peter een beetje harkerig deed en dat ik dacht, dat hij verlegen
was, want dat is zo met alle jongens, die nog niet vaak met meisjes omgegaan
hebben.

Ik moet zeggen dat de schuilcommissie Achterhuis (afd. heren) zeer vindingrijk
is. Moet je horen, wat ze nu weer bedacht hebben om mijnheer Van Dijk,
hoofdvertegenwoordiger van Travies, goeden kennis en
clandestiene-goederenopberger, een bericht van ons te laten zien! Ze tikken een
brief aan een drogist in Zeeuws-Vlaanderen, klant van de zaak, en wel zo, dat de
man een ingesloten briefje als antwoord moet insturen enfin de bijgevoegde
enveloppe terug moet zenden. Op die enveloppe schrijft vader het adres. Als deze
enveloppe uit Zeeland terugkomt, wordt het inliggende briefje er uit gehaald en
een met de hand geschreven levensteken van vader er in gestopt. Zo kan Van Dijk
het lezen zonder argwaan te krijgen. Ze hebben juist Zeeland gekozen, omdat dit
dicht bij België ligt en het briefje dus makkelijk over de grens gesmokkeld kan
zijn, waarbij nog komt, dat niemand daar zonder speciale vergunning naar toe
mag.

Je Anne.

\section*{Zondag, 27 September 1942}

Lieve Kitty,\\
Ruzie met moeder gehad, voor de zoveelste keer in de laatste
tijd; het gaat tussen ons jammer genoeg niet erg goed, ook met Margot kan ik
niet goed opschieten. Hoewel er in onze familie nooit zo'n uitbarsting als boven
is, is het er voor mij toch lang niet altijd prettig. De naturen van Margot en
moeder zijn zo vreemd voor mij, ik snap mijn vriendinnen nog beter dan mijn
eigen moeder, jammer is dat!

We praten vaak over na-oorlogse problemen, zoals bijvoorbeeld dat men niet
geringschattend over dienstmeisjes moet spreken. Dit vond ik net zo erg als het
verschil tussen mevrouw en juffrouw bij getrouwde vrouwen.

Mevrouw heeft voor de zoveelste keer de bokkepruik op; ze is erg humeurig en
sluit steeds meer van haar privédingen weg. Jammer dat moeder niet elke Van
Daan-verdwijning met een Frank-verdwijning beantwoordt.

Sommige mensen schijnen het een bijzonder genoegen te vinden, niet alleen hun
eigen, maar ook nog de kinderen van hun kennissen op te voeden. De Van Daans
behoren tot die mensen. Aan Margot is niets op te voeden, die is van nature de
goed-, lief- en knapheid zelve, maar ik draag haar deel aan ondeugendheid
ruimschoots mee. Meer dan één keer vliegen aan tafel de vermanende woorden en
brutale antwoorden heen en weer. Vader en moeder verdedigen me altijd vurig,
zonder hen zou ik de strijd niet steeds weer zo zonder blikken of blozen kunnen
opnemen.  Hoewel ze me steeds weer voorhouden dat ik minder moet praten, me met
niets bemoeien moet en bescheidener moet zijn, faal ik meer dan ik slaag. En was
vader niet altijd weer zo geduldig, had ik de hoop om ook nog eens te voldoen
aan de ouderlijke eisen, die toch heus niet zo hoog zijn, al lang opgegeven.

Als ik van een groente, die ik helemaal niet lust, weinig neem en in plaats
daarvan aardappels eet, kunnen Van Daan en vooral mevrouw over die verwendheid
niet heen.

`Neem nog wat groente, Anne, kom', volgt dan al gauw.\\
`Neen, dank u wel,
mevrouw', antwoord ik, `ik heb genoeg aan aardappels'. `Groente is erg gezond,
dat zegt je moeder zelf, neem nog wat', dringt zij dan aan,~totdat vader er
tussen komt en mijn weigering bevestigt.

Dan vaart mevrouw uit: `Dan had u eens bij ons thuis moeten zijn, daar werden~de
kinderen tenminste opgevoed, dit is geen opvoeding, Anne is verschrikkelijk
verwend,~ik zou dat nooit toestaan, als Anne mijn dochter was ...'

Daarmee begint en eindigt altijd de hele woordenstroom: `als Anne mijn
dochter~was'. Nu, gelukkig ben ik dat niet.

Maar om op ons opvoedingsthema terug te komen. Gisteren viel na mevrouws

laatste woorden een stilte in. Dan antwoordde vader: `Ik vind dat Anne zeer goed
opgevoed is, ze heeft toch tenminste al zoveel geleerd dat ze op uw lange preken
geen antwoord meer geeft. Wat de groenten betreft kan ik u niets anders
antwoorden dan vice versa'.

Mevrouw was verslagen en grondig ook. Dat vice versa duidde namelijk
rechtstreeks op haar eigen minieme portie. Mevrouw voert voor haar verwendheid
als reden aan, dat te veel groente vóór het naar bed gaan voor haar ontlasting
slecht is. Laat ze dan over mij in ieder geval haar mond houden. Het is zo
grappig om te zien hoe gauw mevrouw Van Daan een kleur krijgt. Ik lekker niet en
daar ergert ze zich heimelijk ontzettend aan.

Je Anne.

\section*{Maandag, 28 September 1942}

Lieve Kitty,\\
Mijn brief van gisteren was nog lang niet af, toen ik met
schrijven op moest houden.

Ik kan de lust niet bedwingen om je nog een onenigheid mede te delen, maar
voordat ik daaraan begin dit:

Ik vind het heel gek dat volwassen mensen zo gauw, zo veel en over alle
mogelijke kleinigheden ruzie maken; tot nu toe dacht ik altijd dat kibbelen een
kindergewoonte was, die later uit zou slijten. Er is natuurlijk wel eens
aanleiding voor een `echte' ruzie, maar dat woordengeplaag hier is niets anders
dan kibbelarij. Daar deze kibbelarijen tot de orde van de dag behoren, moest ik
er eigenlijk al aan gewend zijn; dit is echter niet het geval en zal ook wel
niet het geval zijn, zolang ik bij haast elke discussie (dit woord wordt hier~in
plaats van ruzie gebruikt) ter sprake kom. Niets, maar dan ook niets laten ze
goed aan me, mijn optreden, karakter, manieren worden stuk voor stuk van onder
tot boven en van boven tot onder beoordeeld en bekletst.  En iets dat ik
helemaal niet gewend was, namelijk harde woorden en geschreeuw aan mijn adres,
moet ik volgens bevoegde zijde welgemoed slikken. Dat kan ik niet! Ik denk er
niet aan om al die beledigingen op me te laten zitten, ik zal ze wel laten zien,
dat Anne Frank niet van gisteren is, ze zullen nog opkijken en gauw hun grote
bek houden, als ik ze duidelijk maak, dat ze niet aan~\emph{mijn}~maar
aan~\emph{hun}~opvoeding het eerst moeten beginnen. Is me dat een manier van
optreden! Barbaars gewoon! Tot nu toe sta ik elke keer weer perplex over zoveel
ongemanierdheid en vooral ... domheid (mevrouw Van Daan), maar zodra ik er aan
ga wennen - en dat zal wel gauw zijn - zal ik ze hun woorden ongezouten
teruggeven, dan zullen ze wel anders praten!

Ben ik dan werkelijk zo ongemanierd, eigenwijs, koppig, onbescheiden, dom, lui
enz. enz. als ze boven zeggen? Ach, welneen, ik weet heus wel dat ik veel fouten
en gebreken heb, maar ze overdrijven het wel in zeer hoge mate.

Als je eens wist, Kitty, hoe ik soms kook bij die schelden schimppartijen! En
het zal heus niet lang meer duren of mijn opgekropte woede komt tot uitbarsting.

Maar nu genoeg hierover, ik heb je lang genoeg met mijn ruzie's verveeld en toch
kan ik niet nalaten, een hoog-interessante tafel-discussie hier te laten volgen.

Door het een of andere thema kwamen we op Pims (Pim - vleinaam voor Papa)
verregaande bescheidenheid. Deze is een vaststaand feit, waaraan zelfs door de
idiootste mensen niet getornd kan worden. Plotseling zei mevrouw, omdat ze toch
in elk gesprek zichzelf betrekken moet: `Ik ben ook erg bescheiden, veel
bescheidener dan mijn man'.

Heb je ooit van je leven! Deze zin laat wel heel duidelijk haar bescheidenheid
uitkomen! Mijnheer Van Daan, die het~nodig vond dat: `dan mijn man' nader te
verklaren, antwoordde heel kalm: `Ik wil ook niet bescheiden zijn, ik heb in
mijn leven altijd ondervonden dat onbescheiden mensen het veel verder brengen
dan bescheidene'. En dan zich tot mij wendend: `Wees maar niet bescheiden, Anne,
daar kom je heus niet verder mee!'

Met deze zienswijze stemde moeder volkomen in. Maar mevrouw Van Daan moest,
zoals gewoonlijk, aan dit opvoedingsonderwerp ook haar woordje toevoegen. Voor
deze keer wendde ze zich echter met rechtstreeks tot mij, maar tot mijn
ouderpaar met de woorden: `U hebt toch een eigenaardige levensopvatting om
zoiets tegen Anne te zeggen, in mijn jeugd was dat toch anders. En ik ben er
zeker van dat dat~\emph{nu}~ook nog anders is, behalve dan in uw moderne
huisgezin!' Dit laatste doelde op de meermalen verdedigde moderne
opvoedingsmethode van moeder.

Mevrouw was vuurrood van opwinding, moeder daarentegen helemaal niet en iemand
die kleurt windt zich door de warmwording steeds meer op en verliest het gauwer
van de tegenpartij. De niet-kleurende moeder, die het hele geval nu maar zo gauw
mogelijk van de baan wilde hebben, bedacht zich maar kort, voordat ze
antwoordde: `Mevrouw Van Daan, ook ik vind inderdaad dat het in het leven veel
beter is wat minder bescheiden te zijn. Mijn man, Margot en Peter zijn alle drie
buitengewoon bescheiden. Uw man, Anne, u en ik zijn niet onbescheiden, maar wij
laten ons ook niet bij alles opzij duwen'.

Mevrouw: `O, maar mevrouw, ik begrijp u niet, ik ben werkelijk buitengewoon
bescheiden, hoe komt u er bij mij onbescheiden te noemen?'

Moeder: `U bent zeker niet onbescheiden, maar niemand zou u bepaald bescheiden
vinden'.

Mevrouw: `Ik zou wel eens willen weten, waarin ik onbescheiden ben! Als ik hier
niet voor mezelf zou zorgen, een ander doet het zeker niet en dan zou ik dus
moeten verhongeren, maar daarom ben ik heus wel net zo bescheiden als uw man'.

Moeder kon om deze belachelijke zelfverdediging niets anders doen dan lachen,

dat irriteerde mevrouw, die haar fraai relaas nog met een lange serie prachtige
Duits-Nederlandse en Nederlands-Duitse bewoordingen vervolgde, totdat de geboren
redenares zich zo in haar eigen woorden verwarde, dat ze ten slotte zich van
haar stoel verhief en de kamer uit wou gaan. Toen viel haar oog op mij. En nu
had je eens iets moeten zien!  Ongelukkigerwijze had ik net op het moment, dat
mevrouw ons haar rug toonde, meewarig en ironisch met mijn hoofd geschud, niet
expres, maar heel onwillekeurig, zo intens had ik de hele woordenvloed gevolgd.
Mevrouw keerde terug en begon te kijven, hard, Duits, gemeen en onbeschaafd,
precies als een dik, rood viswijf, het was een lust om aan te zien. Als ik had
kunnen tekenen, had ik haar het liefst in deze houding getekend, zo grappig was
dat kleine, malle, domme mens!

Maar één ding weet ik nu en dat is dit: je leert de mensen pas goed kennen, als
je een keer echte ruzie met ze gemaakt hebt. Pas dan kan je hun karakter
beoordelen!

Je Anne.

\section*{Dinsdag, 29 September 1942}

Lieve Kitty,\\
Onderduikers beleven rare dingen! Stel je voor, omdat we geen
badkuip hebben, wassen we ons in een wasteiltje en omdat het kantoor (hiermee
bedoel ik altijd de gehele benedenverdieping) warm water heeft, gaan we alle
zeven om de beurt van dit grote voordeel pro teren.

Maar omdat we alle zeven ook zo erg verschillend zijn en het vraagstuk van de
genanterie bij een paar hoger zit dan bij de rest, heeft elk lid van de familie
zich een aparte badplaats uitgezocht. Peter baadt in de keuken, hoewel de keuken
een glazen deur heeft. Als hij van plan is een bad te nemen,~bezoekt hij ons
allemaal apart en deelt mede dat we gedurende een half uur niet langs de keuken
mogen lopen. Deze maatregel vindt hij dan voldoende. Mijnheer baadt helemaal
boven, bij hem weegt de veiligheid van eigen kamer op tegen de lastigheid van
het hete water al de trappen op te dragen. Mevrouw gaat voorlopig helemaal niet
in het bad; die wacht af, welke plaats de beste is. Vader baadt in het
privé-kantoor, moeder in de keuken achter een kachelscherm; Margot en ik hebben
het voorkantoor als ploeterplaats gekozen. 's Zaterdagmiddags gaan daar de
gordijnen dicht, dan reinigen we ons in het donker, terwijl diegene die niet aan
de beurt is, door een kier van het gordijn uit het raam kijkt, en zich over de
grappige mensen verbaast.

Sinds verleden week bevalt me dit bad niet meer en ben ik op zoek gegaan naar
een meer comfortabele inrichting. Het was Peter die me op een idee bracht en wel
om in de ruime kantoor- W.C. mijn teiltje te zetten. Daar kan ik gaan zitten,
licht aansteken, de deur op slot doen, het water zelf zonder vreemde hulp
weggieten en ben veilig voor onbescheiden blikken. Zondag heb ik mijn mooie
badkamer voor het eerst in gebruik genomen en hoe gek het ook klinkt, ik vind
het beter dan welke andere plaats ook.

Verleden week was de loodgieter beneden om de buizen van onze watertoe- en
afvoer van de kantoor- W.C. naar de gang te verleggen. Deze verandering is met
het oog op een eventueel koude winter en buizen-bevriezing geschied. Het
loodgietersbezoek was voor ons allesbehalve aangenaam. Niet alleen mochten we
overdag geen water laten lopen, maar we mochten ook niet naar de W.C. Nu is het
wel heel onnet om je te vertellen wat we dan wel gedaan hebben om dit euvel te
verhelpen; ik ben niet zo preuts om over zulke dingen niet te spreken.

Vader en ik hebben ons van het begin van onze schuiltijd een geïmproviseerde po
aangeschaft, met dien verstande dat we bij gebrek aan een nachtvaasje een glazen
weckpot~voor dit doel opgeofferd hebben. Deze weckpotten hebben we tijdens het
loodgietersbezoek in de kamer gezet en onze behoeften daarin overdag bewaard.
Dit vond ik niet zo akelig dan dat ik de hele dag stil moest blijven zitten en
ook niet mocht praten.  Je kunt je niet indenken hoe moeilijk dat juffrouw
Kwek-kwek-kwek gevallen is. Op gewone dagen moeten we al fluister en; helemaal
niet spreken en bewegen is nog 10 keer erger. Mijn achterste was, na drie dagen
aanéén-stuk-door plat gedrukt te zijn, helemaal stijf en pijnlijk.
Avond-gymnastiek heeft geholpen.

Je Anne.

\section*{Donderdag, 1 October 1942}

Lieve Kitty,\\
Gisteren ben ik ontzettend geschrokken. Er werd om acht uur
plotseling heel hard gebeld. Ik dacht niet anders of er kwam iemand, wie, dat
snap je wel.  Maar toen ze allen beweerden dat het zeker kwajongens waren of de
post was, werd ik rustiger.

De dagen worden hier erg stil; Lewin, een kleine Joodse apotheker en chemicus
werkt voor mijnheer Kraler in de keuken. Hij kent het hele gebouw goed en daarom
zijn we aldoor bang, dat hij het in zijn hoofd krijgt om in het vroegere
laboratorium een kijkje te gaan nemen. We zijn zo stil als babymuisjes. Wie had
drie maanden geleden kunnen vermoeden, dat kwikzilver-Anne urenlang zo rustig
zou moeten en kunnen zitten!?

De 29ste was mevrouw Van Daan jarig. Hoewel het feest niet groots werd gevierd,
werd ze toch wel vereerd met bloemen, kleine cadeau's en goed eten. Rode anjers
van haar heer gemaal schijnt in de familie traditie te zijn. Om nog even bij
mevrouw stil te blijven staan kan ik je zeggen, dat een bron van voortdurende
ergernis voor mij haar flirtpogingen met vader zijn. Zij strijkt hem over wang
en haar, trekt haar rokje heel hoog op, zegt zogenaamd geestige~dingen en
probeert zo Pims aandacht op zich te vestigen. Gelukkig vindt Pim haar niet mooi
en ook niet leuk, hij gaat niet op die flirtations in. Ik ben nogal jaloers
uitgevallen zoals je weet, dus dat kan ik niet hebben. Moeder doet dat toch ook
niet bij mijnheer, dat heb ik haar ook in haar gezicht gezegd.

Peter kan af en toe wel eens grappig uit de hoek komen. Eén voorliefde, die tot
lachen aanleiding geeft, heeft hij althans met mij gemeen, en wel verkleden. Hij
verscheen in een zeer nauwe jurk van mevrouw en ik in zijn pak, hij had een hoed
op en ik een pet. De volwassenen lagen dubbel en wij hadden niet minder schik.

Elli heeft in~\emph{De Bijenkorf}~nieuwe rokken voor Margot en mij gekocht. Het
is rottuig, net jutezakken en ze kosten resp. ƒ 24. - en ƒ 7.50. Wat een
verschil met vroeger!

Nog iets leuks in petto. Elli heeft voor Margot, Peter en mij bij een of andere
vereniging schriftelijke stenographielessen besteld. Je zult eens zien wat een
perfecte stenomensen we het volgend jaar zijn. Ik vind het in ieder geval
reuzegewichtig zo'n geheimschrift echt te leren.

Je Anne.

\section*{Zaterdag, 3 October 1942}

Lieve Kitty,\\
Gisteren was er weer een botsing. Moeder heeft verschrikkelijk
opgespeeld en al mijn zonden aan pappie verteld. Ze begon erg te huilen, ik
natuurlijk ook en ik had al zo'n vreselijke hoofdpijn. Ik heb pappie eindelijk
verteld, dat ik veel meer van hem houd dan van moeder, daar heeft hij op gezegd
dat dat wel weer over zal gaan, maar dat geloof ik niet. Ik moet me met geweld
dwingen tegenover haar kalm te blijven. Pappie wou dat ik, als moeder zich niet
lekker voelt of hoofdpijn heeft, maar eens uit mijzelf moest aanbieden om iets
voor haar te doen, maar dat doe ik niet.

Ik leer vlijtig Frans en ben~\emph{La belle Nivernaise}~aan het lezen.  Je Anne.

\section*{Vrijdag, 9 October 1942}

Lieve Kitty,\\
Niets dan nare en neerdrukkende berichten heb ik vandaag te
vertellen. Onze vele Joodse kennissen worden bij groepjes weggehaald. De Gestapo
gaat met deze mensen allerminst zachtzinnig om, ze worden gewoon in veewagens
naar Westerbork, het grote Jodenkamp in Drente, gebracht. Westerbork moet
vreselijk zijn; voor honderden mensen 1 wasruimte en er zijn veel te weinig
W.C.'s. De slaapplaatsen zijn alle door elkaar gegooid.  Mannen, vrouwen en
kinderen slapen samen. Men hoort daardoor van verregaande zedeloosheid, vele
vrouwen en meisjes, die er wat langer verblijf houden, zijn in verwachting.

Vluchten is onmogelijk; de meeste mensen uit het kamp zijn gebrandmerkt door hun
kaalgeschoren hoofden en velen ook door hun Joods uiterlijk.

Als het in Holland al zo erg is, hoe zullen ze dan in de verre en barbaarse
streken leven, waar ze heengezonden worden? We nemen aan dat de meesten vermoord
worden. De Engelse radio spreekt van vergassing.  Misschien is dat wel de
vlugste sterfmethode. Ik ben helemaal van streek. Miep vertelt al deze
gruwelverhalen zo aangrijpend en zelf is ze eveneens opgewonden. Pas geleden
bijvoorbeeld zat er een oude, lamme Joodse vrouw voor haar deur; ze moest op de
Gestapo wachten, die een auto was gaan halen om haar te vervoeren. Het arme
oudje was zo bang voor het harde schieten op de Engelse vliegers die overvlogen
en ook voor de schel flitsende schijnwerpers. Toch durfde Miep haar niet naar
binnen halen, dat zou niemand gewaagd hebben. De Duitsers zijn niet zuinig met
hun straffen.

Ook Elli is stil; haar jongen moet naar Duitsland. Ze is bang, dat de vliegers
die over onze huizen vliegen, hun bommenlast van vaak wel een millioen kilo's op
Dirks hoofd laten vallen. Grapjes van `een millioen zal hij wel niet krijgen' en
`één bom is al genoeg' vind ik hier niet erg~op zijn plaats. Dirk is heus niet
de enige die moet gaan, elke dag rijden er volle treinen met jongens weg.
Onderweg, als ze op een klein stationnetje stoppen, stappen ze wel eens stiekem
uit en proberen onder te duiken; dat lukt misschien een klein percentage.

Ik ben nog niet klaar met mijn treurzang. Heb je wel eens van gijzelaars
gehoord? Dat voeren ze nu als nieuwste strafsnufje voor saboteurs in.  Het is
het meest vreselijke dat je je kunt voorstellen. Onschuldige vooraanstaande
burgers worden gevangen gezet, om op hun veroordeling te wachten. Als iemand
gesaboteerd heeft en de dader wordt niet gevonden, zet de Gestapo doodgewoon een
stuk of 5 gijzelaars tegen de muur. Vaak staan er doodsberichten van deze mannen
in de krant. Als een `noodlottig ongeval' wordt deze misdaad daar betiteld.
Fraai volk, de Duitsers. En daar behoorde ik ook eens toe! Maar neen, Hitler
heeft ons al lang statenloos gemaakt. En trouwens, er bestaat geen groter
vijandschap op de wereld dan tussen Duitsers en Joden.

Je Anne.

\section*{Vrijdag, 16 October 1942}

Lieve Kitty,\\
Ik heb het vreselijk druk. Net heb ik een hoofdstuk uit~\emph{La
belle Nivernaise}~vertaald en woordjes opgeschreven. Dan een rotsom en drie
bladzijden Franse spraakkunst. Ik vertik het gewoon om elke dag van die sommen
te maken.  Pappie vindt ze ook erg en ik kan ze haast nog beter dan hij, maar
feitelijk kunnen we ze alle twee niet goed, zodat we vaak Margot er bij moeten
halen. In steno ben ik het verste van ons drieën.

Gisteren heb ik~\emph{De Stormers}~uitgelezen. Het is wel leuk, maar het haalt
het lang niet bij~\emph{Joop ter Heul}.

Overigens vind ik dat Cissy van Marxveldt knal schrijft. Ik zal haar boeken
beslist ook aan mijn kinderen laten lezen.

Moeder, Margot en ik zijn weer de beste maatjes, dat is~toch eigenlijk veel
prettiger. Gisteravond lagen Margot en ik samen in mijn bed, het was onnoemelijk
klein, maar juist grappig. Ze vroeg of ze eens mijn dagboek mocht lezen. Ik zei:
`Sommige stukken wel', en toen vroeg ik naar het hare, dat mocht ik dan ook
lezen. Toen kwamen we zo op de toekomst. Ik vroeg haar wat ze wilde worden, maar
dat wil ze niet zeggen en ze maakt er een groot geheim van. Ik heb zoiets
opgevangen van onderwijs; ik weet niet of het goed is, maar ik vermoed van wel.
Eigenlijk mag ik niet zo nieuwsgierig zijn!

Vanmorgen lag ik op Peters bed, nadat ik hem er eerst had afgejaagd. Hij was
woedend op me, maar dat kon me al bijster weinig schelen. Hij mocht wel eens wat
vriendelijker tegen me zijn, want ik heb hem gisteravond nog een appel cadeau
gegeven.

Ik heb Margot eens gevraagd of ze me erg lelijk vond. Ze zei, dat ik er wel
grappig uitzag en dat ik leuke ogen had. Nogal vaag, vind je ook niet?

Tot de volgende keer. Je Anne.

\section*{Dinsdag, 20 October 1942}

Lieve Kitty,\\
Mijn hand trilt nog, hoewel de schrik die we hadden al weer twee
uur voorbij is.

Je moet weten, dat we vijf Minimaxtoestellen in huis hebben tegen brandgevaar.
We wisten dat er iemand zou komen om de apparaten te vullen, maar niemand had
ons gewaarschuwd toen de timmerman of hoe zo'n man anders heet inderdaad kwam.

Gevolg was, dat we helemaal niet stil waren, totdat ik buiten op het overloopje
(tegenover onze kastdeur) hamerslagen hoorde. Ik dacht dadelijk aan den
timmerman en waarschuwde Elli, die boven aan het eten was, dat ze niet naar
beneden kon. Vader en ik vatten post aan de deur om te horen, wanneer de man zou
vertrekken. Na een kwartier aan het werk geweest te zijn, legde hij daarbuiten
zijn hamer~en andere gereedschappen op onze kast (zo meenden we) en klopte aan
onze deur. We werden wit, zou hij dus toch iets gehoord hebben en nu dit
geheimzinnige gevaarte willen onderzoeken? Het scheen zo, het kloppen, trekken,
duwen en rukken hield aan. Ik viel haast flauw van angst bij de gedachte dat het
dien wildvreemden man mocht gelukken onze mooie schuilplaats te ontmantelen. En
net dacht ik, dat ik de langste tijd geleefd had, toen ik mijnheer Koophuis
hoorde zeggen: `Doe even open, ik ben het'.  Dadelijk openden we. De haak,
waarmee de deurkast vastzit en die door ingewijden ook van buiten verwijderd kan
worden, was gaan klemmen; daardoor had niemand ons voor den timmerman kunnen
waarschuwen. De man was nu naar beneden gegaan en Koophuis wilde Elli halen,
doch kreeg de kast weer niet open.

Ik kan je zeggen, dat ik niet weinig opgelucht was. De man, van wien ik meende
dat hij bij ons binnen wou, had in mijn verbeelding steeds groter vormen
aangenomen, op het laatst leek hij op een reus en was zo'n fascist als er geen
ergere bestaat.

Hè, hè, gelukkig is het voor deze keer weer goed met ons afgelopen.

Intussen hadden we Maandag veel plezier. Miep en Henk hebben bij ons overnacht.
Margot en ik waren voor een nacht bij vader en moeder komen slapen, zodat het
echtpaar Van Santen onze plaats kon innemen. Het eten smaakte heerlijk. Een
kleine interruptie was, dat vaders lamp kortsluiting veroorzaakte en wij
eensklaps in het donker zaten. Wat te doen? Nieuwe stoppen waren wel in huis,
maar die moesten helemaal achter in het donkere magazijn ingezet worden en dat
was niet zo'n prettig werkje 's avonds. Toch waagden de heren het er op en na 10
minuten kon onze kaarsenilluminatie weer opgeborgen worden.

Vanochtend was ik al vroeg op. Henk moest al om half negen weggaan. Na een
gezellig ontbijt trok Miep naar~beneden. Het goot en ze was blij dat ze nu niet
op de fiets naar kantoor hoefde te rijden.

Elli komt volgende week ook eens op nachtbezoek. Je Anne.

\section*{Donderdag, 29 October 1942}

Lieve Kitty,\\
Ik ben danig ongerust, vader is ziek. Hij heeft hoge koorts en
rode uitslag, het lijken wel de mazelen. Stel je voor, we kunnen niet eens den
dokter halen! Moeder laat hem goed transpireren, misschien gaat de koorts
daardoor omlaag.

Vanochtend vertelde Miep dat de woning van Van Daan ontmeubeld is. We hebben het
mevrouw nog niet verteld, ze is toch al zo zenuwachtig de laatste tijd en we
hebben geen zin nog een jeremiade aan te horen over haar mooie servies en haar
mooie stoeltjes, die thuis gebleven zijn. Wij hebben ook haast alles wat mooi
was in de steek moeten laten; wat helpt nu nog dat geklaag?

De laatste tijd mag ik wat meer volwassenen-boeken lezen. Ik ben nu bezig
met~\emph{Eva's jeugd}~van Nico van Suchtelen. Het verschil tussen meisjesromans
en dit vind ik niet zo erg groot. Wel staan er ook dingen in over het verkopen
van hun lichaam door vrouwen aan onbekende mannen in straatjes. Ze vragen er een
schep geld voor. Ik zou me doodschamen tegenover zo'n man. Verder staat er ook
in, dat Eva ongesteld is geworden; hè, daar verlang ik zo naar, het lijkt me zo
gewichtig.

Vader heeft Goethes en Schillers drama's uit de grote kast gehaald, hij wil me
elke avond wat gaan voorlezen. Met~\emph{Don Carlos}~zijn we al begonnen.

Om vaders goede voorbeeld te volgen heeft moeder me haar gebedenboek in handen
gestopt. Voor mijn fatsoen heb ik wat gebeden in het Duits gelezen; ik vind het
wel mooi, maar het zegt me niet veel. Waarom dwingt ze me ook om zo
vroom-godsdienstig te doen?

Morgen gaat de kachel voor het eerst aan, we zullen wel danig in de rook zitten.
De schoorsteen is al in lange tijd niet geveegd, laten we hopen dat het ding
trekt!

Je Anne.

\section*{Zaterdag, 7 November 1942}

Lieve Kitty,\\
Moeder is verschrikkelijk zenuwachtig en dat is voor mij altijd
erg gevaarlijk. Zou het toeval zijn dat vader en moeder nooit Margot een standje
geven en dat alles altijd op mij neerkomt? Gisteravond bijvoorbeeld: Margot las
een boek, waarin prachtige tekeningen stonden; zij stond op, ging naar boven en
legde het boek opzij om het straks verder te lezen. Ik had net niets te doen,
pakte het en bekeek de platen. Margot kwam terug, zag `haar' boek in mijn hand,
trok een rimpel in haar voorhoofd en vroeg het boek terug. Ik wilde nog eventjes
verder kijken, Margot werd steeds kwader, moeder mengde zich er in met de
woorden: `Dat boek leest Margot, geef het dus aan haar'. Vader kwam de kamer
binnen, wist niet eens waar het over ging, zag dat Margot iets misdaan werd en
viel tegen mij uit: `Ik zou jou eens willen zien als Margot in jouw boek zou
bladeren!'

Ik gaf dadelijk toe, legde het boek neer, en ging volgens hen beledigd de kamer
uit. Ik was noch beledigd, noch kwaad, maar wel verdrietig.

Het was niet goed van vader te oordelen zonder de strijdvraag te weten.  Ik had
het boek vanzelf aan Margot gegeven en ik zou het nog veel gauwer gedaan hebben
als vader en moeder er zich niet in hadden gemengd en, alsof het het grootste
onrecht betrof, dadelijk Margot in bescherming hadden genomen.

Dat moeder voor Margot opkomt spreekt vanzelf; zij en Margot komen altijd voor
elkaar op. Ik ben daar zo aan gewend dat ik totaal onverschillig voor moeders
standjes en voor Margots prikkelbuien geworden ben.

Ik houd van hen alleen, omdat ze nu eenmaal moeder en Margot zijn. Bij vader is
dat een ander geval. Als hij Margot voortrekt, Margots daden goedkeurt, Margot
prijst en Margot liefkoost, dan knaagt het binnen in me, want op vader ben ik
dol. Hij is mijn grote voorbeeld, van niemand anders in de hele wereld dan van
vader houd ik.

Hij is het zich niet bewust, dat hij met Margot anders omgaat dan met mij.
Margot is nu eenmaal de knapste, de liefste, de mooiste en de beste. Maar een
beetje recht op ernst heb ik toch ook. Ik was altijd de clown en de deugniet van
de familie, moest altijd voor alle daden dubbel boeten, één keer met standjes en
één keer met de wanhopigheid binnen in mezelf. Nu bevredigt me dat oppervlakkige
geliefkoos niet meer, evenmin de zogenaamd ernstige gesprekken. Ik verlang iets
van vader dat hij niet in staat is me te geven.

Ik ben niet jaloers op Margot, was het nooit, ik begeer haar knap- en mooiheid
niet, ik zou alleen zo graag vaders echte liefde, niet alleen als zijn kind maar
als Anne-op-zichzelf voelen.

Ik klamp me aan vader vast, omdat hij de enige is die mijn laatste restje
familiegevoel ophoudt. Vader begrijpt niet, dat ik over moeder mijn hart eens
moet luchten, hij wil niet praten, vermijdt alles wat op moeders fouten
betrekking heeft. En toch ligt moeder met al haar gebreken me het zwaarst op
mijn hart. Ik weet niet hoe me te houden, kan haar haar slordigheid, sarcasme en
hardheid niet onder de neus duwen, kan echter ook niet altijd de schuld bij
mijzelf vinden.

Ik ben in alles precies andersom als zij en dat botst vanzelfsprekend.  Ik
oordeel niet over moeders karakter, want daar kan ik niet over oordelen, ik
bekijk haar alleen als moeder. Voor mij is moeder niet `de' moeder; ik zelf moet
mijn moeder zijn. Ik heb me afgescheiden van hen, ik schipper alleen en zal
later wel zien waar ik beland. Het komt allemaal vooral, doordat ik in mijn
geest een heel groot voorbeeld zie, hoe een moeder en vrouw moet zijn en niets
daarvan terugvind in haar, die ik de naam van moeder moet geven.

Ik neem me altijd voor niet meer naar moeders verkeerde voorbeeld te kijken, ik
wil alleen haar goede kanten zien en wat ik bij haar niet vind, bij mezelf
zoeken. Maar het lukt niet en dan is nog het ergste, dat noch vader noch moeder
beseffen, dat ze in mijn leven tekort schieten en dat ik hen daarvoor
veroordeel. Zou ooit iemand zijn kinderen wel helemaal tevreden kunnen stellen?

Soms geloof ik, dat God me op de proef wil stellen, nu en ook later; ik moet
alleen goed worden zonder voorbeelden en zonder praten. Dan zal ik later het
sterkst zijn.

Wie anders dan ikzelf zal later deze brieven lezen?\\
Wie anders dan ikzelf zal
me troosten? Want vaak heb ik troost nodig, ik ben zo

dikwijls niet sterk genoeg en schiet meer te kort dan ik voldoe. Ik weet het en
probeer altijd weer, elke dag opnieuw, me te verbeteren.

Ik word onevenwichtig behandeld. De ene dag is Anne zo verstandig en mag alles
weten en de volgende dag hoor ik weer, dat Anne nog maar een klein dom schaap is
dat van niets afweet en denkt wonder wat uit boeken geleerd te hebben. Ik ben
niet meer de baby en het troetelkind, dat bovendien bij al haar daden
uitgelachen mag worden. Ik heb mijn eigen idealen, ideeën en plannen, maar kan
ze nog niet onder woorden brengen.  Ach, er komt zoveel rijzen als ik 's avonds
alleen ben, eveneens overdag als ik de mensen moet verdragen die me de keel
uithangen of die mijn bedoelingen altijd verkeerd opvatten. Ik kom daarom
tenslotte altijd weer tot mijn dagboek terug, dat is mijn begin- en mijn
eindpunt, want Kitty is altijd geduldig, ik zal haar beloven, dat ik ondanks
alles vol zal houden, mijn eigen weg zal banen en mijn tranen slikken. Ik zou
alleen zo graag alvast de resultaten zien, of één keer aangemoedigd worden, door
iemand die me liefheeft.

Veroordeel me niet, maar beschouw me als iemand die zich ook eens te vol kan
voelen.

Je Anne.

\section*{Maandag, 9 November 1942}

Lieve Kitty,\\
Gisteren was Peter jarig, hij is 16 geworden. De cadeau's waren
wel leuk. Hij heeft o.a. een beursspel, een scheerapparaat en een
sigarettenaansteker gekregen. Niet dat hij zoveel rookt, helemaal niet, het is
maar voor de elegance.

De grootste verrassing bracht mijnheer Van Daan, toen hij ons om een uur
berichtte, dat de Engelsen in Tunis, Algiers, Casa Blanca en Oran geland waren.
`Dat is het begin van het einde', zeiden ze allemaal, maar Churchill, de Engelse
premier-minister, die waarschijnlijk in Engeland dezelfde uitroepen gehoord had,
zei: `Deze landing is een heel groot feit, toch mag men niet denken, dat dit het
begin van het einde is. Ik zeg veeleer dat het het einde van het begin beduidt'.
Merk je het verschil? Reden tot optimisme is er toch wel. Stalingrad, de
Russische stad, die ze nu al drie maanden verdedigen is nog steeds niet aan de
Duitsers prijsgegeven.

Om in de geest van het Achterhuis te spreken, moet ik toch ook eens iets van
onze levensmiddelenvoorziening schrijven. Je moet weten, dat het op de
bovenafdeling echte smulpapen zijn. Ons brood levert een aardige bakker, een
kennis van Koophuis. We krijgen natuurlijk niet zoveel als we thuis kregen, maar
het is toereikend. Levensmiddelenkaarten worden eveneens clandestien gekocht. De
prijs er van stijgt voortdurend, van ƒ 27. - is het nu al ƒ 33. - geworden. En
dat alleen voor een blaadje bedrukt papier!

Om wat houdbaars in huis te hebben, buiten onze 150 blikken groenten, hebben we
270 pond peulvruchten gekocht. Dat is niet alles voor ons alleen, ook met
kantoor is rekening gehouden. De peulvruchten hingen in zakken in ons gangetje
(binnen de schuildeur), aan haken. Door het zware gewicht zijn een paar naden
van de zakken opengesprongen. We besloten dus onze wintervoorraad maar liever op
zolder te zetten en vertrouwden Peter het hijswerkje toe.

Vijf van de zes zakken waren al heelhuids boven beland, en Peter was net bezig
no zes op te trekken, toen de benedennaad van de zak brak en een regen, neen,
een hagel van bruine bonen door de lucht en de trap af sloeg. In de zak zat
ongeveer 50 pond, het was dan ook een lawaai als een oordeel; beneden dachten ze
niet anders of ze kregen het oude huis met inhoud op hun kop. (Goddank was er
geen vreemde in huis.) Peter schrok even, maar moest verschrikkelijk lachen toen
hij mij beneden aan de trap zag staan, als een eilandje tussen de bonengolven,
zo was ik omringd door het bruine goedje dat me tot aan de enkels reikte. Gauw
gingen we aan het rapen, maar bonen zijn zo glad en klein, dat ze in alle
mogelijke en onmogelijke hoekjes en gaatjes rollen. Elke keer dat nu iemand de
trap afloopt, bukt hij zich om een handvol boontjes aan mevrouw af te leveren.

Haast zou ik vergeten te vermelden, dat vaders ziekte weer helemaal over is. Je
Anne.

P.S. Net komt het bericht door de radio, dat Algiers gevallen is.  Marokko, Casa
Blanca en Oran zijn al een paar dagen in Engelse handen.  Het wachten is nu op
Tunis.

\section*{Dinsdag, 10 November 1942}

Lieve Kitty,\\
Geweldig nieuws, we willen een achtsten onderduiker opnemen! Ja
heus, we zijn altijd van mening geweest, dat hier nog best plaats en eten voor
een achtste persoon is. We waren alleen maar te bang om Koophuis en Kraler nog
meer te belasten. Toen nu de gruwelberichten van buiten aangaande de Joden
steeds erger werden, heeft vader eens de twee beslissende personen gepolst en
deze vonden het plan uitstekend. `Het gevaar is voor zeven even groot als voor
acht', zeiden ze zeer terecht.

Toen dit in orde was, zijn we in gedachten onze kennissenkring langs gegaan om
een alleenstaand mens te vinden, die goed in onze duikfamilie zou passen. Het
was niet moeilijk zo iemand op te scharrelen. Nadat vader alle familieleden van
Van Daan van de hand gewezen had, viel onze keuze op een tandarts genaamd Albert
Dussel, wiens vrouw gelukkig in het buitenland verblijft. Hij staat bekend als
een rustig mens en, zo naar de oppervlakkige kennismaking te oordelen, leek hij
zowel Van Daan als ons sympathiek. Ook Miep is met hem bekend, zodat door haar
het onderduikplan geregeld kan worden. Als hij komt, moet Dussel in mijn kamer
slapen in plaats van Margot, die het harmonicabed tot legerstede krijgt.

Je Anne.

\section*{Donderdag, 12 November 1942}

Lieve Kitty,\\
Dussel was reusachtig blij toen Miep hem vertelde, dat ze een
schuilplaats voor hem had. Ze drukte hem op zijn hart zo gauw mogelijk te komen.
Liefst Zaterdag al. Dat leek hem enigszins bedenkelijk, hij moest zijn
kartotheek nog in orde brengen, twee patiënten helpen en de kas opmaken.  Met
dit bericht kwam Miep vanochtend bij ons. Wij vonden het niet goed dat hij
langer wachtte. Al die voorbereidingen eisen weer uitleggingen aan tal van
mensen, die we er liever buiten zouden houden. Miep zal vragen of het niet toch
zo geregeld kon worden, dat Dussel Zaterdag arriveert.

Dussel zei van neen, hij komt nu Maandag. Ik vind het wel gek, dat hij niet
dadelijk op elk voorstel ingaat. Als hij op straat meegenomen wordt, kan hij
noch de kartotheek in orde brengen, noch de kas opmaken, noch de mensen helpen.
Waarom dan dat uitstel? Ik voor mij vind het stom dat vader toegegeven heeft.

Anders niets nieuws. Je Anne.

\section*{Dinsdag, 17 November 1942}

Lieve Kitty,\\
Dussel is aangekomen. Het is hem allemaal meegelopen. Om elf uur,
had Miep tegen hem gezegd, moest hij vóór het postkantoor op een bepaalde plaats
zijn, daar zou een heer hem meenemen. Dussel stond op de afgesproken plaats
precies op tijd, mijnheer Koophuis, dien Dussel eveneens kende, ging naar hem
toe, berichtte dat de genoemde heer nog niet komen kon en of hij even op het
kantoor bij Miep wilde komen. Koophuis stapte in de tram, reed terug naar
kantoor en Dussel liep dezelfde weg op. Om tien minuten voor half 12 tikte
Dussel aan de kantoordeur. Miep liet hem zijn jas uitdoen, zodat de ster
onzichtbaar was en bracht hem naar het privé-kantoor, waar Koophuis hem bezig
hield, tot de werkster weg was.  Het voorwendsel gebruikend, dat het
privé-kantoor niet langer vrij was, ging Miep vervolgens met Dussel naar boven,
opende de draaikast en stapte voor de ogen van den stomverbaasden man naar
binnen.

Wij zaten bij de Van Daans rondom de tafel met cognac en koffie den mede-duiker
op te wachten. Eerst leidde Miep hem in onze huiskamer; hij herkende dadelijk de
meubels van ons, maar dacht er nog in de verste verte niet aan, dat wij ons
boven zijn hoofd bevonden. Toen Miep hem dat vertelde, viel hij haast flauw van
verbazing. Maar gelukkig liet Miep hem niet te lang de tijd en nam hem mee naar
boven.

Dussel plofte op een stoel neer en keek ons allemaal een poosje sprakeloos aan,
alsof hij eerst even goed de waarheid van onze gezichten wilde lezen. Daarna
stotterde hij: `Maar ... aber, sind u dan niet in België? Ist der Militär nicht
gekomen, das Auto, die vlucht is sie nicht gelukt?' We legden hem het hele geval
uit, dat we dat verhaaltje van de militair en de auto expres rondgestrooid
hadden om de mensen en de Duitsers, die eventueel naar ons zouden zoeken, op een
dwaalspoor te leiden. Dussel was weer sprakeloos over zoveel vernuftigheid en
kon verder niets anders doen~dan verbaasd rondkijken, toen hij ons
hyper-practische en mooie Achterhuisje nader nasnuffelde.

Wij aten samen, hij sliep een beetje en dronk dan thee met ons, ordende zijn
beetje boel, dat Miep van te voren meegebracht had en voelde zich al tamelijk
thuis. Vooral toen hij de volgende getikte Achterhuis-onderduikregels (Maaksel
Van Daan) in handen kreeg:

\emph{Prospectus en leidraad van het Achterhuis}.\\
Speciale instelling voor het
tijdelijk verblijf van Joden en dergelijke.\\
\emph{Gedurende het gehele jaar
geopend}.\\
Mooie, rustige, bosvrije omgeving in het hartje van Amsterdam. Geen
privé-buren.

Te bereiken met de tramlijnen 13 en 17, verder ook met auto of fiets. In
bepaalde gevallen, waar de Duitsers het gebruik van deze vervoermiddelen niet
toelaten, ook lopende.

\emph{Woonprijs:}~gratis.\\
\emph{Dieetkeuken vetvrij}.\\
\emph{Stromend
water}~in badkamer (helaas geen bad) en aan diverse binnen- en~buitenmuren.

\emph{Ruime opslagplaatsen}~voor goederen, van welke aard ook.\\
\emph{Eigen
radiocentrale}, met directe verbinding van London, New York, Tel Aviv en~vele
andere stations. Dit toestel staat vanaf zes uur 's avonds alleen inwonenden ter
beschikking, waarbij geen verboden zenders bestaan met dien verstande, dat
alleen bij uitzondering naar Duitse stations mag worden geluisterd, b.v. naar
klassieke muziek en dergelijke.

\emph{Rusturen:}~10 uur 's avonds tot half 8 's morgens. 's Zondags tot kwart
over 10. In verband met omstandigheden worden ook rusturen overdag gehouden,
volgens aanwijzingen van de directie. Rusturen moeten stipt gehouden worden, in
verband met de algemene veiligheid!!!

\emph{Vacantie:}~Tot nader order vervallen voor zover het betreft het verblijf
buitenshuis.

\emph{Gebruik van taal:}~Vereist is te allen tijde zacht te spreken, toegestaan
zijn alle cultuurtalen, dus geen Duits.

\emph{Gymnastiekoefeningen:}~dagelijks.

\emph{Lessen:}~In stenographie elke week één schriftelijke les, in Engels,
Frans, Wiskunde en Geschiedenis te allen tijde.

\emph{Speciale afdeling voor kleine huisdieren}~met goede verzorging
(uitgezonderd ongedierte, waarvoor speciale vergunning moet worden overgelegd.).

\emph{Het gebruik van maaltijden:}~ontbijt, alle dagen met uitzondering van Zon-
en Feestdagen, 's morgens om 9 uur. Zon- en Feestdagen ca. half 12.

Diner: gedeeltelijk uitgebreid, 's middags kwart over 1 tot kwart voor 2.
Avondeten: koud en/of warm, in verband met de berichtendienst geen vaste
tijd.~\emph{Verplichtingen tegenover de ravitailleringscolonne:}~Altijd bereid
te zijn om met

kantoorwerk te helpen.\\
\emph{Baden:}~'s Zondags vanaf 9 uur staat de teil voor
alle huisgenoten beschikbaar.

Men kan baden in W.C., keuken, privé-kantoor, voorkantoor naar
keuze.~\emph{Sterke dranken:}~alleen op doktersattest.\\
Einde.\\
Je Anne.

\section*{Donderdag, 19 November 1942}

Lieve Kitty,\\
Zoals we allemaal wel veronderstelden is Dussel een erg aardig
mens. Hij vond het natuurlijk goed om het kamertje met me te delen, ik ben er
eerlijk gezegd niet zo erg mee ingenomen, dat een vreemde mijn dingen in gebruik
gaat nemen, maar je moet wat voor het goede doel over hebben en ik breng dit
kleine offer dan ook graag. `Als we maar iemand kunnen redden, is al het andere
bijzaak', zei vader en daar heeft hij volkomen gelijk in.

Dussel heeft me de eerste dag dat hij hier was, dadelijk~naar allerlei dingen
gevraagd; bijvoorbeeld wanneer de werkster komt, hoe de badkamertijden zijn,
wanneer men naar het toilet mag gaan. Je zult lachen, maar dat alles is in een
schuilplaats niet zo gemakkelijk. Wij mogen overdag niet zo'n drukte maken, dat
ze ons beneden horen en als een extra persoon, zoals bijvoorbeeld de werkster,
er is, moeten wij ons allen ook extra in acht nemen. Ik legde Dussel dit alles
mooi uit, maar één ding verbaasde me daarbij en wel, dat hij zo erg traag van
begrip is, alles vraagt hij dubbel en onthoudt het dan nog niet. Misschien gaat
dat wel over, en is hij enkel door de verrassing zo in de war.

Overigens gaat het best. Dussel heeft ons veel van de buitenwereld verteld, die
wij nu al zo lang missen. Het is droevig wat hij allemaal wist. Talloze vrienden
en kennissen zijn weg met een vreselijke bestemming. Avond aan avond tuffen de
groene of grijze militaire auto's langs, de Duitsers bellen aan elke deur en
vragen of er ook Joden wonen.  Zo ja, dan moet de hele familie dadelijk mee, zo
niet, dan gaan ze weer verder. Niemand kan zich aan zijn lot onttrekken als hij
niet onderduikt. Zij gaan ook dikwijls rond met lijsten en bellen alleen daar,
waar ze weten, dat ze een rijke buit zullen binnenhalen. Losgeld wordt er vaak
betaald, per hoofd zoveel. Het lijkt wel op de slavenjacht, zoals ze die vroeger
hielden. Maar het is heus geen mop, daarvoor is het veel te dramatisch. Ik zie
vaak 's avonds in het donker rijen goede, onschuldige mensen lopen met huilende
kinderen, steeds maar lopen, gecommandeerd door zo'n paar kerels, geslagen en
gepijnigd, tot ze haast neervallen. Niemand wordt ontzien, ouden van dagen,
baby's, zwangere vrouwen, zieken, alles, alles gaat mee in de tocht naar de
dood.

Hoe goed hebben we het hier, hoe goed en rustig.

We hoefden ons van al deze ellende niets aan te trekken, als we ons maar niet zo
bang maakten om allen, die ons zo dierbaar waren en die we niet meer kunnen
helpen.

Slecht voel ik me, dat ik in een warm bed lig, terwijl mijn~liefste vriendinnen
ergens buiten neergegooid of neergevallen zijn. Ik word zelf bang als ik aan
allen denk met wie ik me altijd zo innig verbonden voelde en die nu overgeleverd
zijn aan de handen van de wreedste beulen die er bestaan.

En dat alles, omdat ze Joden zijn! Je Anne.

\section*{Vrijdag, 20 November 1942}

Lieve Kitty,\\
We weten geen van allen goed wat voor een houding we moeten
aannemen. Tot nu toe is er van de berichten over de Joden nooit veel tot ons
doorgedrongen en we hebben het het beste gevonden zoveel mogelijk opgewekt te
blijven. Toen Miep af en toe iets losliet over het verschrikkelijke lot van een
kennis, begonnen moeder en mevrouw Van Daan elke keer te huilen, zodat Miep het
beter vond, helemaal niets meer te vertellen. Maar Dussel werd dadelijk bestormd
met vragen en de verhalen, die hij vertelde waren zo gruwelijk en barbaars dat
het niet `het ene oor in en het andere weer uit gaat'.

Toch zullen we, als de berichten een beetje gezakt zijn, wel weer grappen maken
en plagen. Het helpt ons en die daarbuiten niet als we zo somber blijven, als we
allemaal op het ogenblik zijn. En wat heeft het voor zin van het Achterhuis een
melancholiek Achterhuis te maken?

Bij alles wat ik doe moet ik aan de anderen denken, die weg zijn. En als ik om
iets moet lachen, houd ik verschrikt weer op en denk bij mezelf, dat het een
schande is dat ik zo vrolijk ben. Maar moet ik dan de hele dag huilen? Neen, dat
kan ik niet en ze zal ook wel weer overgaan, deze somberheid.

Bij deze narigheid is nog een andere gekomen, maar die is van heel persoonlijke
aard en zinkt in het niet bij de ellende, waarvan ik je zojuist heb verteld.
Toch kan ik niet nalaten je te vertellen, dat ik me de laatste tijd zo verlaten
ga voelen.

Er is een te grote leegte om mij heen. Vroeger dacht ik daar nooit zo over na en
mijn pretjes en vriendinnen vulden mijn gehele denken. Nu denk ik óf aan
ongelukkige dingen óf over mezelf. En ten langen leste ben ik tot de ontdekking
gekomen dat vader, hoe lief hij ook is, toch niet mijn gehele vroegere wereldje
kan vervangen. Maar waarom je met zulke malle dingen lastig vallen, ik ben
ontzettend ondankbaar, Kitty, ik weet het wel, maar het draait me ook vaak als
ik te veel op mijn kop krijg en dan nog aan al die andere narigheid moet denken!

Je Anne.

\section*{Zaterdag, 28 November 1942}

Lieve Kitty,\\
We hebben veel te veel licht gebruikt en nu ons
electriciteitsrantsoen overschreden.

Gevolg: uiterste zuinigheid enfin het vooruitzicht gestelde afsnijding, 14 dagen
geen licht, prettig hè! Maar wie weet, loopt het nog mee! Vanaf vier uur of half
vijf is het te donker om te lezen. We korten onze tijd met allerhande gekke
dingen. Raadsels opgeven, gym doen in het donker, Engels of Frans spreken,
boeken becritiseren - het verveelt allemaal op den duur. Sinds gisteravond heb
ik iets nieuws uitgevonden en wel met een sterke verrekijker in de verlichte
kamers van de achterburen gluren.  Overdag mogen onze gordijnen nooit een
centimeter opzij, maar als het donker is, kan dat geen kwaad.

Ik wist vroeger nooit, dat buren zulke interessante mensen kunnen zijn, althans
de onze. Een paar heb ik bij de maaltijd aangetroffen, een familie was net aan
het filmen en de tandarts van hierover had een oude, angstige dame in
behandeling.

Mijnheer Dussel, de man van wien altijd gezegd was, dat hij uitstekend met
kinderen kan opschieten en ook van alle kinderen houdt, ontpopt zich als de
meest ouderwetse opvoeder en preker van ellenlange manierenreeksen.

Daar ik het zeldzame geluk heb (!) met den HoogEdel-welopgevoeden heer mijn,
helaas zeer nauwe, kamer te mogen delen en daar ik algemeen voor de slechtst
opgevoede van de drie jeugdigen gehouden word, heb ik nogal wat te stellen om de
oude, vaak herhaalde standjes en vermaningen te ontlopen en me een Oostindische
doofheid aan te meten.

Dit alles zou nog tot daar aan toe zijn, als mijnheer maar niet zo'n grote
klikspaan was en zich niet ook nog moeder als overbrengadres uitgezocht had.

Als ik van hem net de wind van voren heb gekregen, doet moeder het nog eens
dunnetjes over, en krijg ik dus de wind van achteren. En als ik dan heel erg
bof, roept mevrouw me vijf minuten later ter verantwoording en komt de wind van
bovenaf geblazen.

Heus, je moet niet denken dat het gemakkelijk is het onopgevoede middelpunt van
een bedillerige onderduikersfamilie te zijn. 's Avonds in bed, als ik over mijn
vele zonden en toegedichte gebreken nadenk, kom ik zo in de war door de grote
massa van beschuldigingen, dat ik óf ga lachen, óf huilen, al naar mijn
innerlijke stemming.

Dan slaap ik in met het gekke gevoel van anders te willen dan ik ben of anders
te zijn dan ik wil, misschien ook anders te doen dan ik wil of dan ik ben. O
hemeltje, nu maak ik jou ook nog in de war, excuseer me, maar doorstrepen daar
hou ik niet van en papier weggooien is in tijden van grote papierschaarste
verboden. Dus kan ik je alleen maar aanraden, de vorige zin niet nog eens door
te lezen en je er vooral niet in te verdiepen, want je komt er toch niet uit!

Je Anne.

\section*{Maandag, 7 December 1942}

Lieve Kitty,\\
Chanuka en St Nicolaas vielen dit jaar haast samen, het verschil
was maar één~dag. Voor Chanuka hebben we niet veel omhaal gemaakt, wat leuke
dingetjes over en weer en dan de kaarsjes. Daar er gebrek aan kaarsen is, worden
ze~maar tien minuten aangestoken, maar als het lied er niet bij ontbreekt is dat
ook wel goed. Mijnheer Van Daan heeft een luchter uit hout vervaardigd, zodat
ook dit voor elkaar is.

St Nicolaasavond, Zaterdag, was veel leuker. Elli en Miep hadden ons erg
nieuwsgierig gemaakt door de hele tijd met vader te fluister en, zodat we wel
vermoedden, dat er iets in voorbereiding was.

En werkelijk, om acht uur gingen we allen de houten trap af, door de stikdonkere
gang (ik griezelde en wenste me alweer veilig boven) naar het kabinetje. Daar
konden we licht aansteken, omdat dit vertrekje geen ramen heeft. Toen dat
gebeurd was, deed vader de grote kast open. `Oh, wat leuk', riepen we allemaal.
In de hoek stond een grote mand met Sint Nicolaaspapier versierd en bovenop was
een Pietmombakkes bevestigd.

Gauw namen we de mand mee naar boven. Er zat voor elk een leuk cadeautje in met
een toepasselijk vers.

Ik kreeg een mikpoppetje, vader boekensteunen, enz. enz. Het was in ieder geval
allemaal leuk bedacht en daar we alle acht nog nooit in ons leven Sinterklaas
gevierd hebben, was deze première goed op zijn plaats.

Je Anne.

\section*{Donderdag, 10 December 1942}

Lieve Kitty,\\
Mijnheer Van Daan komt uit de worst-, vlees- en specerijenhandel.
In de zaak werd hij om zijn vakkennis geëngageerd. Ook nu toont hij zich van de
worstachtige kant, wat ons lang niet onaangenaam is.

We hadden veel vlees (clandestien natuurlijk) besteld om te wecken, als we eens
moeilijke tijden zouden moeten doormaken! Het was een leuk gezicht hoe eerst de
vleesstukken door de vleesmolen gingen, twee of drie keer, dan alle
toebehoorselen door de vleesmassa heengewerkt werden~en hoe de worst dan, door
middel van een tuitje in de darmen, gevuld werd. De braadworst aten we dadelijk
's middags bij de zuurkool op, maar de Gelderse moest eerst goed drogen en
daarvoor werd ze aan een stok met twee touwtjes aan de zoldering opgehangen.
Iedereen die de kamer binnenkwam en de tentoongestelde worsten in het oog kreeg,
begon te lachen. Het was dan ook een allerkoddigst gezicht.

In de kamer was het een rompslomp van jawelste. Mijnheer Van Daan was, met de
schort van mevrouw voor, in zijn hele dikte (hij leek veel dikker dan hij was)
met het vlees bezig. Bloederige handen, rood hoofd en het bemorste schort gaven
hem het aanzien van een echten slager. Mevrouw deed van alles tegelijk,
Nederlands leren uit een boekje, de soep roeren, kijken naar het vlees, zuchten
en klagen over haar gebroken bovenborstrib. Dat komt er van als oudere dames (!)
zulke alleridiootste gymnastiekoefeningen doen om hun dikke achterste weer kwijt
te raken!

Dussel had een ontstoken oog en bette het bij de kachel met kamillen-thee. Pim
zat op een stoel in het straaltje zon dat door het venster kwam, werd van de ene
naar de andere kant geschoven, daarbij had hij zeker weer pijn op zijn
rheumatiekplek, want hij zat tamelijk krom en met een verstoord gezicht mijnheer
Van Daan op zijn vingers te kijken. Hij leek sprekend op zo'n oud en invalide
diakenhuismannetje.  Peter tolde met zijn poes in de kamer rond, moeder, Margot
en ik waren aan het aardappelen schillen, en ten slotte deden we geen van allen
onze bezigheden goed om naar Van Daan te kijken.

Dussel heeft zijn tandartsenpraktijk geopend. Ik zal je voor de grap even
mededelen hoe de eerste behandeling verlopen is. Moeder was aan het strijken en
mevrouw Van Daan de eerste, die er aan moest geloven. Ze ging op een stoel
midden in de kamer zitten. Dussel begon heel gewichtig zijn doosje uit te
pakken, vroeg om eau de cologne als desinfecteermiddel en vaseline als was.

Hij keek in mevrouws mond, raakte een tand en een kies aan, waarbij mevrouw elke
keer in elkaar kromp, alsof ze van de pijn verging, terwijl ze onsamenhangende
klanken uitstootte. Na een langdurig onderzoek (voor mevrouw tenminste, want het
was niet langer dan twee minuten) begon Dussel een gaatje uit te krabben. Maar
neen hoor, er was geen denken aan, mevrouw sloeg wild met haar armen en benen,
zodat op een gegeven ogenblik Dussel het krabbertje losliet, dat ... in mevrouws
tand bleef steken.

Nu had je eerst echt de poppen aan het dansen! Mevrouw sloeg om zich heen,
huilde (voor zover dat mogelijk is met zo'n instrumentje in je mond), probeerde
het krabbertje uit haar mond te krijgen en duwde het er met dat alles nog verder
in. Mijnheer Dussel stond het toneeltje doodkalm met zijn handen in de zij aan
te zien. De rest van de toeschouwers lachte onbedaarlijk. Dat was gemeen, want
ik weet zeker dat ik nog veel harder gegild zou hebben.

Na veel draaien, schoppen, schreeuwen en roepen had mevrouw dan eindelijk het
krabbertje er uit en mijnheer Dussel vervolgde, alsof er niets gebeurd was, zijn
werk.

Hij deed dit zo vlug, dat mevrouw geen tijd had om nog eens te beginnen, maar
hij had ook zoveel hulp, als nog nooit in zijn leven. Twee assistenten is niet
weinig, mijnheer en ik fungeerden goed. Het geheel leek wel op een plaatje uit
de Middeleeuwen met het opschrift `Kwakzalver aan het werk'.

Intussen had de patiënte echter niet veel geduld, ze moest op `haar' soep letten
en op `haar' eten.

Een ding is zeker, mevrouw laat zich zo gauw niet meer behandelen! Je Anne.

\section*{Zondag, 12 December 1942}

Lieve Kitty,\\
Ik zit heel gezellig in het voorkantoor naar buiten te kijken,
door de spleet van het~zware gordijn. Het is schemerdonker, maar nog net licht
genoeg om je te schrijven.

Het is een heel gek gezicht als ik de mensen zie lopen, het lijkt net of ze
allemaal verschrikkelijke haast hebben en bijna over hun eigen voeten
struikelen.

De fietsers, nou, dat tempo is helemaal niet bij te houden, ik kan niet eens
zien, wat voor een soort individu er op het vehikel zit.

De mensen zien er hier in de buurt niet zo erg aanlokkelijk uit en vooral de
kinderen zijn te vies om met een tang aan te pakken. Echte achterbuurtkinderen
met snotneuzen, hun taaltje kan ik nauwelijks verstaan.

Gistermiddag waren Margot en ik hier aan het baden en toen zei ik: `Als we nu
eens de kinderen, die hier langs lopen, stuk voor stuk met een hengel
ophengelden, in het bad stopten, hun goed wasten en verstelden en dan weer
lieten lopen, dan ...' Daarop viel Margot in: `Zouden ze er morgen weer net zo
vies en kapot uitzien als tevoren'.

Maar wat zit ik te bazelen, er zijn ook andere dingen om te zien, auto's,
schepen en de regen. Ik hoor de tram en het gieren er van en amuseer me.

Onze gedachten hebben net zo weinig afwisseling als wijzelf. Ze gaan steeds als
een caroussel van de Joden naar het eten en van het eten naar de politiek.
Tussen haakjes, van Joden gesproken, gisteren heb ik, alsof het een wereldwonder
was, door het gordijn twee Joden gezien; dat was zo'n naar gevoel, net of ik die
mensen verraden heb en nu hun ongeluk zit te beloeren. Hier vlak tegenover ligt
een woonschip, waarop een schipper met vrouw en kinderen huist, deze man heeft
een klein keffertje. Het hondje kennen we alleen van het horen blaffen en van
zijn staartje, dat we kunnen zien als hij in de goot van de boot loopt.

Bah, nu is het gaan regenen en de meeste mensen zijn schuil gegaan onder hun
paraplu. Ik zie niets dan regenjassen en soms een bemutst achterhoofd. Het is
ook eigenlijk niet nodig méér te zien, zo langzamerhand ken ik de vrouwen~ook zó
al, opgeblazen door de aardappels, een rode of groene jas aan, op afgetrapte
hakken en met een tas aan de arm. Ze hebben een grimmig of een goedig gezicht,
al naar het humeur van haar man.

Je Anne.

\section*{Dinsdag, 22 December 1942}

Lieve Kitty,\\
Het Achterhuis heeft met blijdschap vernomen dat ieder met
Kerstmis een kwart pond boter extra krijgt. In de krant staat wel een half pond,
maar dit is alleen voor de gelukkige stervelingen, die hun levensmiddelenkaarten
van de Staat krijgen en niet voor ondergedoken Joden, die voor de duurte maar
vier in plaats van acht clandestien kopen.

We gaan alle acht wat bakken met de boter. Ik heb vanochtend koekjes gemaakt en
twee taarten. Het is erg druk hierboven en moeder heeft me verboden te gaan
leren of lezen, vóórdat het huishoudelijk werk achter de rug is.

Mevrouw Van Daan ligt met haar gekneusde rib in bed, klaagt de hele dag, laat
zich aldoor nieuwe verbanden aanleggen en is met niets tevreden. Ik zal blij
zijn als ze weer op haar twee benen staat en zelf haar boeltje opknapt, want één
ding moet gezegd worden, ze is buitengewoon vlijtig en netjes en zolang ze
lichamelijk en geestelijk in goede conditie verkeert ook vrolijk.

Alsof ik overdag al niet genoeg `sst sst' te horen krijg, omdat ik altijd te
veel lawaai maak, is mijn heer kamergenoot op het idee gekomen me 's nachts ook
herhaaldelijk `sst' toe te roepen. Ik mag dus volgens hem niet eens even
draaien. Ik vertik het om hier aandacht aan te besteden en roep de volgende keer
`sst' terug.

Hij maakt me vooral 's Zondags woedend, als hij zo vroeg het licht aansteekt en
gymnastiek gaat doen. Het lijkt mij, arme geplaagde, wel uren te duren, want de
stoelen, waarmee~mijn bed verlengd is, schuiven gedurig onder mijn slaperig
hoofd heen en weer. Na een paar keer heftig met de armen zwaaiend de
lenigheidsoefeningen beëindigd te hebben, begint meneer met zijn toilet. De
onderbroek hangt aan de haak, dus eerst daar naar toe en weer terug. Maar hij
vergeet de das die op tafel ligt. Dus weer duwende en stotende langs de stoelen
en zo weer terug.

Maar laat ik je niet ophouden over oude nare heren te zeuren, het wordt er toch
niet beter door en al mijn wraakmiddelen, zoals lamp uitschroeven, deur
afsluiten, kleren verstoppen, moet ik om de vrede te bewaren, achterwege laten.

Ach, ik word zo verstandig! Alles moet hier met verstand gebeuren, leren
luisteren, mond houden, helpen, lief zijn, toegeven en weet ik wat nog meer. Ik
ben bang dat ik mijn verstand, dat al niet bijster groot is, veel te vlug
opgebruik en voor na-oorlogse tijden niets meer overhoud.

Je Anne.

\section*{Woensdag, 13 Januari 1943}

Lieve Kitty,\\
Vanochtend ben ik weer met alles gestoord en kon daardoor niets
behoorlijk afmaken.\\
Buiten is het verschrikkelijk. Dag en nacht worden die
arme mensen weggesleept,

met niets anders bij zich dan een rugzak en wat geld. Deze bezittingen worden
hun onderweg ook nog ontnomen. De gezinnen worden uit elkaar gerukt, gesplitst
in mannen, vrouwen en kinderen.

Kinderen, die van school naar huis komen, vinden hun ouders niet meer.  Vrouwen,
die boodschappen doen, vinden bij haar thuiskomst haar huis verzegeld, haar
familie verdwenen.

De Nederlandse Christenen hebben ook al angst, hun zonen worden naar Duitsland
gestuurd, iedereen is bang.

En elke nacht komen er honderden vliegers over Nederland,~vliegen naar de Duitse
steden en ploegen daar de aarde met hun bommen om, en ieder uur vallen in
Rusland en Afrika honderden, duizenden mensen. Niemand kan zich er buiten
houden, de hele aardbol voert oorlog en al gaat het beter met de geallieerden,
een einde is nog niet te zien.

En wij, we hebben het goed, ja beter dan millioenen anderen. We zitten nog
rustig en veilig en eten zogenaamd ons geld op. We zijn zo egoïstisch, dat we
over `na de oorlog' spreken, ons op nieuwe kleren en schoenen verheugen, terwijl
we eigenlijk iedere cent moesten sparen om na de oorlog de andere mensen te
helpen, te redden wat er nog te redden valt.

De kinderen hier lopen rond in een dun blousje en met klompen aan de voeten,
geen jas, geen muts, geen kousen en niemand die hen helpt. Ze hebben niets in
hun buik, maar kauwen op een peenwortel, lopen van hun koude woning weg naar de
koude straat en komen op school in een nog koudere klas. Ja, het is zelfs zo ver
met Holland gekomen, dat talloze kinderen op straat de voorbijgangers aanhouden
en om een stuk brood vragen.

Ik zou je wel urenlang over de ellende, die de oorlog meebrengt, kunnen
vertellen, maar dat maakt me zelf enkel nog mismoediger. Er blijft ons niets
anders over dan zo rustig als het kan het einde van deze misère af te wachten.
Zowel de Joden als de Christenen wachten, de hele aardbol wacht en velen wachten
op hun dood.

Je Anne.

\section*{Zaterdag, 30 Januari 1943}

Lieve Kitty,\\
Ik damp van woede en ik mag het niet tonen. Ik zou willen
stampvoeten, schreeuwen, moeder eens hard door elkaar schudden, huilen en weet
ik wat nog meer om de nare woorden, de spottende blikken, de beschuldigingen die
als pijlen~van een scherp gespannen boog mij elke dag opnieuw treffen en die zo
moeilijk uit mijn lichaam te trekken zijn.

Ik zou moeder, Margot, Van Daan, Dussel en ook vader toe willen schreeuwen:
`Laat me met rust, laat me eindelijk een nacht slapen zonder dat mijn kussen nat
is van tranen, mijn ogen branden en mijn hoofd bonst. Laat me gaan, weg van
alles, liefst weg van de wereld!'

Maar ik kan het niet, ik kan hun mijn vertwijfeling niet laten zien; ik kan hun
geen blik laten slaan in de wonden die zij mij toebrachten, ik zou hun
medelijden en hun goedmoedige spot niet kunnen verdragen, ook dan nog zou ik
moeten schreeuwen. Iedereen vindt me aanstellerig als ik praat, belachelijk als
ik zwijg, brutaal als ik antwoord geef, geslepen als ik een goed idee heb, lui
als ik moe ben, egoïstisch als ik een hap te veel eet, dom, laf berekenend
enzovoort enzovoort. De hele dag hoor ik niets anders dan dat ik een
onuitstaanbaar wicht ben, en hoewel ik er om lach, doe of ik me er niets van
aantrek, kan het me wel wat schelen, zou ik God wel willen vragen me een andere
natuur te geven, die niet alle mensen tegen me in het harnas jaagt.

Het kan niet, mijn natuur is me gegeven en ik kan niet slecht zijn, ik voel het.
Ik doe meer mi{]}n best allen te bevredigen dan ze in de verste verte vermoeden,
ik probeer boven te lachen, omdat ik hun mijn narigheden niet wil tonen.

Meer dan eens heb ik moeder, na een rij onredelijke verwijten, naar het hoofd
geslingerd: `Het kan me toch niets schelen wat je zegt, trek je handen maar
helemaal van me af, ik ben toch een hopeloos geval'. Ik kreeg dan natuurlijk te
horen dat ik brutaal was, werd twee dagen een beetje genegeerd en dan was opeens
weer alles vergeten en ik werd behandeld zoals ieder ander.

Het is me onmogelijk om de ene dag poeslief te zijn en de volgende dag hun mijn
haat in het gezicht te slingeren. Ik kies liever de gulden middenweg, die niet
verguld is, en~houd mijn mond over wat ik denk en probeer ééns net zo minachtend
tegenover hen te worden, als zij het zijn tegenover mij.

Ach, als ik het maar kon. Je Anne.

\section*{Vrijdag, 5 Februari 1943}

Lieve Kitty,\\
Hoewel ik je al lang niets meer van de ruzie's geschreven heb, is
daar toch geen verandering in gekomen. Mijnheer Dussel nam de gauw vergeten
onenigheden in het begin nog tragisch op, maar nu gaat hij er al aan wennen en
probeert niet meer te bemiddelen.

Margot en Peter zijn helemaal niet wat je `jong' noemt, alle twee zo saai en
stil. Ik steek daar verschrikkelijk tegen af en krijg altijd te horen: `Margot
en Peter doen dat ook niet, kijk maar naar hen'.  Verschrikkelijk vind ik dat.

Ik zal je wel bekennen dat ik helemaal niet zo worden wil als Margot, ze is me
veel te slapjes en onverschillig, laat zich door iedereen omverpraten en geeft
in alles toe. Ik wil wat steviger van geest zijn!  Maar zulke theorieën houd ik
voor me, ze zouden me wel uitlachen als ik met deze verdediging aan kwam zetten.

Aan tafel is de stemming meestal gespannen, gelukkig worden de uitbarstingen nog
wel eens door de soep-eters tegengehouden. De soep-eters zijn allen, die van
kantoor komen en dan een kopje soep krijgen.

Vanmiddag had mijnheer Van Daan het er weer over, dat Margot zo weinig eet.
`Zeker om de slanke lijn te houden', vervolgde hij op spottende toon. Moeder,
die het altijd voor Margot opneemt, zei op luide toon: `Ik kan dat domme geklets
van u niet meer horen'.

Mevrouw werd vuurrood, mijnheer keek voor zich en zweeg. Vaak lachen we om de
één of ander; pas geleden kraamde mevrouw zo'n heerlijke onzin uit. Ze vertelde
van vroeger, hoe goed ze met haar vader op kon schieten~en hoeveel ze geflirt
heeft. `En weet u', vervolgde ze, `als een heer een beetje handtastelijk wordt,
zei mijn vader, dan moet je tegen hem zeggen: ``Mijnheer, ik ben een Dame'', dan
weet hij wel wat je bedoelt'. Wij schaterden als om een goede mop.

Ook Peter, hoe stil ook meestal, geeft nog eens aanleiding tot vrolijkheid. Hij
heeft het geluk, dol op vreemde woorden te zijn, maar de betekenis er van niet
altijd te kennen.

Op een middag toen we wegens visite op kantoor niet naar het toilet konden gaan,
moest hij daar toch hoognodig naar toe, trok echter niet door. Om ons nu te
waarschuwen, bevestigde hij een bordje aan de W.C.-deur met: `S.v.p.gas!' er op.
Hij wou natuurlijk zetten: `Voorzichtig, gas', maar vond `s.v.p.' deftiger
staan. Dat de betekenis er van `alstublieft' is, daar had hij geen notie van.

Je Anne.

\section*{Zaterdag, 27 Februari 1943}

Lieve Kitty,\\
Pim verwacht elke dag de invasie. Churchill heeft longontsteking
gehad, hij gaat langzaam vooruit. Gandhi, de Indische vrijheidsman, houdt zijn
zoveelste hongerstaking.

Mevrouw beweert dat ze fataliste is. Maar wie is het bangst als ze schieten?
Niemand anders dan Petronella.

Henk heeft het herderlijk schrijven van de bisschoppen aan de mensen in de kerk
voor ons meegebracht. Het was reuze-mooi en opwekkend geschreven. `Blijft niet
stilzitten, Nederlanders, ieder vechte met zijn eigen wapenen voor de vrijheid
van land, volk en godsdienst! Helpt, geeft, aarzelt niet!' Dat roepen ze zomaar
van de preekstoel. Zou het helpen? Onzen geloofsgenoten stellig niet.

Stel je voor, wat ons nu weer overkomen is. De huiseigenaar van dit perceel
heeft, zonder Kraler en Koophuis in te lichten, het huis verkocht. Op een morgen
kwam de nieuwe huiseigenaar met een architect het huis bezichtigen.

Gelukkig was mijnheer Koophuis aanwezig, die den heren alles heeft laten zien,
op ons Achterhuis na. Hij had zogenaamd de sleutel van de tussendeur thuis
vergeten. De nieuwe huiseigenaar vroeg niets verder.  Als hij maar niet
terugkomt en toch het Achterhuis wil zien, want dan ziet het er lelijk voor ons
uit.

Vader heeft voor Margot en mij een kartotheekdoos leeggemaakt en er kaartjes in
gedaan. Dit wordt de boekenkartotheek, we schrijven namelijk alle twee op welke
boeken we gelezen hebben, door wie ze geschreven zijn, enzovoort. Voor vreemde
woorden heb ik me een apart boekje aangeschaft.

Moeder en ik schieten de laatste tijd wel beter met elkaar op, maar
vertrouwelijk zijn we~\emph{nooit}. Margot is kattiger dan ooit en vader heeft
iets waarmee hij niet voor de dag wil komen, maar is toch altijd een schat.

Nieuwe boter- en margarineverdeling aan tafel! Ieder krijgt zijn stukje smeersel
op zijn bord. Mijns inziens geschiedt de verdeling daarvan door de Van Daans
zeer oneerlijk. Mijn ouders zijn echter veel te bang voor ruzie om er wat van te
zeggen. Jammer, ik vind dat je zulke mensen altijd met gelijke munt terugbetalen
moet.

Je Anne.

\section*{Woensdag, 10 Maart 1943}

Lieve Kitty,\\
Gisteravond hadden we kortsluiting, bovendien paften ze
onophoudelijk. Ik heb mijn angst voor alles wat schieten of vliegers is nog niet
afgeleerd en lig haast elke nacht bij vader in bed, om daar troost te zoeken.
Dat kan nu wel erg kinderachtig zijn, maar je moest het eens meemaken. Je kunt
je eigen woorden niet meer verstaan, zo bulderen de kanonnen. Mevrouw, de
fataliste, begon haast te huilen en sprak met een heel benepen stemmetje: `O,
het is zo onaangenaam, o ze schieten zo hard', dat wil zeggen: `Ik ben zo bang'.

Bij kaarslicht leek het nog niet eens zo erg, als toen het donker was.  Ik rilde
alsof ik koorts had en smeekte vader om het kaarsje weer aan te maken. Hij was
onverbiddelijk, het licht bleef uit. Plotseling schoten machinegeweren, dat is
nog tienmaal erger dan kanonnen. Moeder sprong uit bed en stak, tot ergernis van
Pim, de kaars aan. Haar resolute antwoord op zijn gemopper was: `Anne is toch
geen oude soldaat'. Daarmee basta.

Heb ik je al van mevrouws andere angsten verteld? Ik geloof van niet. Om in al
de Achterhuisavonturen ingelicht te zijn moet je ook dit weten.  Mevrouw meende
op een nacht op zolder dieven te horen, ze vernam echte harde stappen en was zo
bang, dat ze haar man wekte. Net op dat ogenblik verdwenen de dieven en het
lawaai dat mijnheer nog hoorde was het kloppen van het bange hart der fataliste.
`Och Putti (mijnheers troetelnaam), ze hebben zeker de worsten en al onze
peulvruchten meegenomen. En Peter, o zou Peter nog wel in zijn bed liggen?'

`Peter hebben ze heus niet gestolen, hoor, maak je niet bang en laat me slapen'.

Maar daar kwam niets van, mevrouw sliep van angst niet meer in. Een paar nachten
later werd de hele bovenfamilie door het spokenlawaai gewekt.  Peter ging met
een zaklantaarn naar de zolder en rrrrt, wat liep er weg?  Een massa grote
ratten! Toen we wisten wie de dieven waren, hebben we Mouschi op zolder laten
slapen en de ongenode gasten zijn niet meer teruggekomen, tenminste niet 's
nachts.

Peter kwam een paar avonden geleden op de vliering om wat oude kranten te halen.
Om de trap af te dalen moest hij zich goed aan het luik vasthouden. Hij legde
zonder te kijken zijn hand neer en ... viel bijna van schrik en pijn de trap af.
Zonder dat hij het wist had hij zijn hand op een grote rat gelegd, die hem hard
in zijn arm beet. Het bloed liep door zijn pyama heen, toen hij zo wit als een
doek en~met knikkende knieën bij ons kwam. Geen wonder, een grote rat aaien is
niet zo leuk en nog een hap krijgen op de koop toe is verschrikkelijk.

Je Anne.

\section*{Vrijdag, 12 Maart 1943}

Lieve Kitty,\\
Mag ik je voorstellen: Mama Frank, voorvechtster der kinderen!
Extra boter voor de jeugdigen, moderne jeugdproblemen, in alles komt moeder voor
de jeugd op en krijgt na een dosis ruzie altijd haar zin.

Een glas geweckte tong is bedorven. Galadiner voor Mouschi en Mof .

Mof is nog een onbekende voor je, toch is ze al in de zaak geweest, voordat we
gingen onderduiken. Ze is de magazijn- en kantoorkat en houdt in de opslagplaats
de ratten verre. Ook haar politieke naam is makkelijk uit te leggen. Een
tijdlang had de firma twee katten, één voor het magazijn en één voor de zolder.
Het gebeurde wel eens, dat die twee elkaar ontmoetten, wat altijd grote
gevechten ten gevolge had. De magazijner was altijd degene die aanviel, terwijl
het zolderbeest op het eind toch de overwinning behaalde. Net als in de
politiek. Dus werd de magazijnkat, de Duitse of Mof , en de zolderkat, de
Engelse of Tommi, genoemd. Tommi is later afgeschaft en Mof dient ons allen tot
amusement, als we naar beneden gaan.

We hebben zoveel bruine en witte bonen gegeten, dat ik ze niet meer zien kan.
Als ik er alleen maar aan denk, word ik al misselijk. De avond-broodverstrekking
is helemaal ingetrokken.

Pappie heeft net gezegd dat hij niet goed in zijn humeur is, hij heeft weer
zulke droevige oogjes, de arme ziel!

Ik ben gewoon verslaafd aan het boek:~\emph{De klop op de deur}~van Ina
Boudier-Bakker. De familieroman is buitengewoon goed geschreven, wat er omheen
is over oorlog,~schrijvers of de emancipatie der vrouwen, vind ik niet zo goed
en eerlijk gezegd interesseert het me niet zo bar.

Verschrikkelijke bomaanvallen op Duitsland. Mijnheer Van Daan is slecht
gehumeurd, aanleiding sigarettenschaarste. Discussies over het vraagstuk van wel
of niet de blikgroente op te eten is voor onze partij gunstig verlopen.

Geen schoen kan ik meer aan, behalve hoge skischoenen, die in huis erg
onpractisch zijn. Een paar rieten sandalen voor ƒ 6.50 kon ik maar een week
dragen, toen waren ze al buiten gevecht gesteld. Misschien diept Miep wat
clandestiens op. Ik moet vaders haar nog knippen. Pim beweert, dat hij na de
oorlog nooit meer een anderen kapper neemt, zo goed volbreng ik mijn werk. Als
ik maar niet zo vaak zijn oor meeknipte!

Je Anne.

\section*{Donderdag, 18 Maart 1943}

Lieve Kitty,\\
Turkije in de oorlog. Grote opwinding. Wachten met spanning de
radioberichten af.\\
Je Anne.

\section*{Vrijdag, 19 Maart 1943}

Lieve Kitty,\\
De teleurstelling was de vreugde al in een uur gevolgd en had de
laatstgenoemde~achterhaald. Turkije is nog niet in oorlog, de minister daar
sprak enkel van een spoedige opheffing der neutraliteit. Op de Dam stond een
krantenverkoper die schreeuwde: `Turkije aan de kant van Engeland!' De kranten
werden hem op die manier uit de handen gegrist. Zo heeft het verheugende bericht
ook ons bereikt.

De 500- en 1000-gulden-briefjes worden ongeldig verklaard. Dat is een grote
strop voor alle zwartehandelaren en dergelijke, maar nog meer voor eigenaars van
ander zwart geld en ondergedokenen. Men moet, als men een duizend-gulden-biljet
wil inwisselen, precies verklaren en bewijzen boe men er aan is gekomen.
Belastingen mogen er nog wel mee worden betaald, ook dit laatste loopt volgende
week af.

Dussel heeft een trapboormachine ontvangen, ik zal dan wel gauw een ernstige
nakijkbeurt krijgen.

De Führer aller Germanen heeft gesproken voor gewonde soldaten. Het was treurig
om aan te horen. Vraag en antwoord gingen ongeveer zo:

`Heinrich Scheppel is mijn naam'. `Waar gewond?'\\
`Bij Stalingrad'.\\
`Wat
gewond?'

`Twee afgevroren voeten en een gewrichtsbreuk aan de linkerarm'.

Precies zo gaf de radio dit verschrikkelijke marionettentheater aan ons door. De
gewonden leken wel trots op hun verwondingen, hoe meer hoe beter. Eentje kwam
van de ontroering, dat hij den Führer de hand mocht reiken (als hij die
tenminste nog had), haast niet uit zijn woorden.

Je Anne.

\section*{Donderdag, 25 Maart 1943}

Lieve Kitty,\\
Moeder, vader, Margot en ik zaten gisteren genoeglijk bij elkaar;
ineens kwam~Peter binnen en fluister de vader iets in zijn oor. Ik hoorde wat
van `een ton omvallen in het magazijn' en `iemand aan de deur morrelen'. Margot
had dat ook verstaan, maar probeerde mij nog een beetje te kalmeren, want ik was
natuurlijk krijt-wit en zeer nerveus, toen vader dadelijk met Peter de deur
uitging.

Wij drieën wachtten nu af. Nog geen twee minuten later kwam mevrouw, die beneden
in het privé-kantoor naar de radio had geluisterd, naar boven.  Ze vertelde dat
Pim haar verzocht had de radio af te draaien en zachtjes naar boven te gaan.
Maar hoe gaat het, als men heel zachtjes wil lopen,~dan kraken de treden van een
oude trap juist dubbel hard. Weer vijf minuten daarna kwamen Peter en Pim, wit
tot aan hun neuspuntjes, en vertelden ons hun wederwaardigheden.

Ze hadden zich onder aan de trap neergezet en gewacht, eerst zonder resultaat.
Maar opeens, ja hoor, daar hoorden ze twee harde bonzen, alsof hier in huis twee
deuren dichtgeslagen werden. Pim was in één sprong boven, Peter waarschuwde
eerst nog Dussel, die met veel omhaal en geruis eindelijk ook boven aanlandde.
Nu ging het op kousenvoeten een étage hoger naar het gezin Van Daan. Mijnheer
was erg verkouden en lag al in bed, dus schaarden we ons allen aan zijn sponde
en fluister den onze vermoedens.

Telkens als mijnheer hard hoestte, dachten mevrouw en ik dat we de stuipen
kregen van angst. Dat ging zolang, totdat één van ons op het lumineuze idee kwam
hem codeïne te geven, de hoest bedaarde toen onmiddellijk.

Weer wachtten en wachtten we, maar niets werd meer vernomen en nu
veronderstelden we eigenlijk allemaal dat de dieven, doordat ze stappen in het
anders zo stille huis hadden gehoord, de benen genomen hadden. Nu was het
ongeluk, dat de radio beneden nog op Engeland stond en de stoelen er ook nog
netjes omheen geschaard waren. Als nu de deur geforceerd zou zijn en de
luchtbescherming dit zou bemerken en de politie waarschuwen, dan zouden er heel
onaangename gevolgen uit het geval kunnen voortvloeien. Dus stond mijnheer Van
Daan op, trok zijn jas aan, zette een hoed op en ging achter vader heel
voorzichtig de trap af, gevolgd door Peter, die voor alle zekerheid met een
zware hamer bewapend was. De dames boven (inclusief Margot en ik) wachtten met
spanning tot vijf minuten later de heren boven verschenen en vertelden, dat
alles in huis rustig was.

Er werd afgesproken, dat we geen water zouden laten lopen, noch de W.C.
doortrekken. Maar daar de opwinding bijna al de huisgenoten op de maag geslagen
was, kun je je~voorstellen wat een lucht daar heerste, toen we allen na elkaar
onze boodschap achtergelaten hadden.

Als er zo iets is, komen er altijd een heleboel dingen bij elkaar, zo ook nu. No
1 was, dat de Westertorenklok niet speelde, en dat gaf me juist altijd zo'n
geruststellend gevoel.

No 2: Dat mijnheer Vossen de vorige avond eerder weggegaan was en wij dus niet
zeker wisten, of Elli de sleutel nog had kunnen bemachtigen en misschien
vergeten had de deur af te sluiten.

Het was nog altijd avond en wij verkeerden nog steeds in onzekerheid, hoewel we
toch weer een beetje gerustgesteld waren, daar we vanaf circa acht uur, toen de
dief ons huis onveilig had gemaakt, tot half elf niets meer gehoord hadden. Het
leek ons bij nader inzien dan ook erg onwaarschijnlijk dat een dief zo vroeg in
de avond, als er nog mensen op straat kunnen lopen, een deur opengebroken zou
hebben. Bovendien rees bij één van ons de gedachte, dat het wel mogelijk zou
zijn, dat de magazijnmeester van onze buren nog aan het werk was, want in de
opwinding en met de dunne muren kan men zich gemakkelijk in de geluiden
vergissen en ook de verbeelding speelt meestal in zulke hachelijke ogenblikken
een grote rol.

We gingen dus naar bed, maar de slaap wilde niet bij allen komen. Vader, zowel
als moeder als mijnheer Dussel waren veel wakker en ik kan met een beetje
overdrijving zeggen, dat ik geen oog dichtgedaan heb. Vanochtend zijn de heren
naar beneden gegaan en hebben aan de buitendeur getrokken of ze nog afgesloten
was, maar alles bleek veilig.

De gebeurtenissen, die ons zoveel angst hadden aangejaagd, werden in geuren en
kleuren aan het gehele personeel verteld. Iedereen dreef er de spot mee, maar
achteraf kan men om zulke dingen gemakkelijk lachen.  Alleen Elli heeft ons au
sérieux genomen.

Je Anne.

\section*{Zaterdag, 27 Maart 1943}

Lieve Kitty,\\
De steno-cursus is af, we beginnen nu snelheid te oefenen.\\
Wat
worden we knap! Om je verder nog eens iets van mijn dagdoodvakken te~vertellen
(dat noem ik zo, omdat we niets anders doen dan de dagen maar zo gauw mogelijk
laten verlopen, zodat het eind van de onderduiktijd gauw naderbij komt): ik ben
dol op mythologie en wel het meest op de Griekse en Romeinse Goden. Hier denken
ze, dat het voorbijgaande neigingen zijn, ze hebben nog nooit van een bakvis met
Godenappreciaties gehoord. Welnu, dan ben ik de eerste!

Mijnheer Van Daan is verkouden, of beter: hij heeft een weinig keelgekriebel.
Hij maakt er een geweldige poespas om heen. Gorgelen met kamillenthee,
verhemelte penselen met myrrhe-tinctuur, dampo op de borst, neus, tanden en tong
en dan nog een slechte bui.

Rauter, één of andere hoge mof, heeft een rede gehouden: `Alle Joden moeten vóór
1 Juli de Germaanse landen verlaten hebben. Van 1 April tot 1 Mei zal de
provincie Utrecht gezuiverd worden (alsof het kakkerlakken zijn). Van 1 Mei tot
1 Juni de provincies Noord- en Zuid-Holland'. Als een kudde ziek en verwaarloosd
vee worden de arme mensen naar de onzindelijke slachtplaatsen gevoerd. Maar laat
ik hierover zwijgen, ik krijg alleen nachtmerries van mijn eigen gedachten.

Een prettig nieuwtje is, dat de Duitse afdeling van de arbeidsbeurs door
sabotage in brand gestoken is. Een paar dagen daarna volgde de Burgerlijke
Stand. Mannen in uniformen van de Duitse politie hebben de wachtposten gekneveld
en zodoende belangrijke papieren foetsjie gemaakt.

Je Anne.

\section*{Donderdag, 1 April 1943}

Lieve Kitty,\\
Heus geen moppenstemming (zie datum), integendeel, vandaag kan ik
gerust het~gezegde aanhalen: `Een ongeluk komt nooit alleen'.

Ten eerste heeft onze opvrolijker mijnheer Koophuis gisteren een
erge~maagbloeding gehad en moet op zijn minst drie weken het bed houden.  Ten
tweede: Elli griep. Ten derde gaat mijnheer Vossen volgende week naar het
ziekenhuis. Hij heeft waarschijnlijk een maagzweer en moet daaraan geopereerd
worden. En ten vierde waren belangrijke zakenbesprekingen op komst, en alle
voornaamste punten had vader met mijnheer Koophuis in details besproken,
mijnheer Kraler kan in de gauwigheid niet zo goed ingelicht worden.

De verwachte heren kwamen, vader bibberde al van te voren over de afloop van de
bespreking. `Als ik er maar bij kon zijn, was ik maar beneden', riep hij uit.
`Ga dan met je oor op de grond liggen, de heren komen toch in het privé-kantoor,
dan kun je alles horen'. Vaders gezicht klaarde op en om half elf gisterochtend
namen Margot en Pim (twee oren horen meer dan één) hun stelling op de grond in.
De bespreking werd 's morgens niet beëindigd, maar 's middags was vader niet in
staat de luistercampagne voort te zetten. Hij was geradbraakt door de ongewone
en onpractische houding. Ik nam zijn plaats in om half drie, toen we stemmen in
de gang hoorden. Margot hield me gezelschap, het gesprek was ten dele zo
langdradig en vervelend, dat ik plotseling op de harde, koude linoleum-grond
ingeslapen was. Margot durfde me niet aan te raken uit angst, dat ze ons beneden
zouden horen en roepen ging helemaal niet. Ik sliep een goed half uur, werd toen
verschrikt wakker en was alles van de belangrijke bespreking vergeten. Gelukkig
had Margot beter opgelet.

Je Anne

\section*{Vrijdag, 2 April 1943}

Lieve Kitty,\\
Ach, ik heb weer iets verschrikkelijks in mijn zondenregister
staan.\\
Gisteravond lag ik in bed en wachtte tot vader met me zou komen bidden
en dan~goedennacht zou zeggen, toen moeder de kamer binnenkwam, op mijn bed ging
zitten en heel bescheiden vroeg: `Anne, Pappie komt nog niet, zullen wij niet
eens samen bidden?'

`Neen, mansa', antwoordde ik.

Moeder stond op, bleef naast mijn bed staan, liep dan langzaam naar de deur.
Eensklaps draaide ze zich om en met een verwrongen gezicht zei ze: `Ik wil niet
boos zijn, liefde laat zich niet dwingen'. Een paar tranen gleden over haar
gezicht, toen zij de deur uitging.

Ik bleef stil liggen en vond het dadelijk gemeen van mezelf dat ik haar zo ruw
van me gestoten had, maar ik wist ook, dat ik niet anders antwoorden kon. Ik kan
niet huichelen en tegen mijn wil in met haar bidden. Het ging eenvoudig niet.

Ik had medelijden met moeder, heel erg veel medelijden, want voor de eerste keer
in mijn leven heb ik gemerkt, dat mijn koele houding haar niet onverschillig
laat. Ik heb het verdriet op haar gezicht gezien, toen ze over de liefde, die
zich niet dwingen laat, sprak. Het is hard de waarheid te zeggen en toch is het
de waarheid, dat zij zelf mij van zich gestoten heeft, dat zij zelf mij voor
elke liefde van haar kant afgestompt heeft door haar ontactvolle opmerkingen,
haar ruwe scherts over dingen, die bij mij voor grappen niet toegankelijk zijn.
Zoals ik elke keer in elkaar krimp als zij mij haar harde woorden toevoegt, zo
kromp haar hart ineen toen zij merkte, dat de liefde tussen ons werkelijk
verdwenen was.

Zij heeft de halve nacht gehuild en de hele nacht haast niet geslapen.  Vader
kijkt me niet aan en als hij het even~doet, lees ik in zijn ogen de woorden:
`Hoe kon je zo naar zijn, hoe durf je moeder zo'n verdriet te doen'.

Ze verwachten, dat ik excuus zal aanbieden, maar dit is een geval, waarvoor ik
geen excuses kan aanbieden, omdat ik iets gezegd heb, dat waar is en dat moeder
vroeger of later toch moet weten. Ik lijk en ben onverschillig tegenover moeders
tranen en vaders blik geworden, omdat zij alle twee de eerste keer iets voelen
van wat ik onophoudelijk merk.  Ik kan alleen medelijden hebben met moeder, die
zelf haar houding terug moet vinden. Ik voor mij blijf zwijgen en ben koel en
zal ook verder voor de waarheid niet terugdeinzen, omdat zij, naarmate zij
langer wordt uitgesteld, des te moeilijker te horen valt.

Je Anne.

\section*{Dinsdag, 27 April 1943}

Lieve Kitty,\\
Het hele huis dendert van ruzie. Moeder en ik, de Van Daans en
Papa, moeder en mevrouw, alles is kwaad op elkaar, leuke sfeer, hè? Het
gebruikelijke zondenregister van Anne kwam in zijn hele omvang opnieuw op het
tapijt.

Mijnheer Vossen ligt al in het Binnengasthuis, mijnheer Koophuis is weer op, de
maagbloeding was gauwer gestelpt dan anders. Hij heeft verteld, dat de
Burgerlijke Stand nog eens extra toegetakeld is door de brandweer, die in plaats
van alleen het vuur te blussen de heleboel onder water heeft gezet. Doet me
plezier!

Het Carlton Hotel is kapot, twee Engelse vliegers met een grote lading
brandbommen aan boord zijn precies op het `Of ziersheim' gevallen. De hele hoek
Vijzelstraat-Singel is afgebrand. De luchtaanvallen op Duitse steden worden van
dag tot dag sterker. We hebben geen nacht meer rust, ik heb zwarte kringen onder
mijn ogen door het tekort aan slaap. Ons eten is miserabel. Ontbijt met droog
brood en koffiesurrogaat. Diner: al veertien dagen lang spinazie of sla.
Aardappels van 20 cm lengte smaken~zoet en rot. Wie wil vermageren logere in het
Achterhuis! Boven klagen ze steen en been, wij vinden het niet zo tragisch. Alle
mannen, die in 1940 gevochten hebben of gemobiliseerd waren, zijn opgeroepen om
in krijgsgevangenschap voor den Führer te werken. Doen ze zeker als
voorzorgsmaatregel tegen de invasie!

Je Anne.

\section*{Zaterdag, 1 Mei 1943}

Lieve Kitty,\\
Als ik zo wel eens bedenk hoe we hier leven kom ik meestal tot de
conclusie dat~we het hier, in verhouding tot de andere Joden, die niet
ondergedoken zijn, als in een paradijs hebben. Toch zal ik later, als alles weer
gewoon is, wel verwonderd zijn hoe wij, die het thuis erg netjes hadden, zo
afgezakt zijn.

Afgezakt in dien zin van het woord, waar het de manieren betreft. We hebben
bijvoorbeeld al sinds we hier zijn altijd op onze tafel één zeildoek, dat door
het vele gebruik meestal niet tot de schoonste behoort. Wel probeer ik het vaak
wat op te knappen met een vuile afwasdoek, die meer gat dan doek is. Zelfs
hiermee valt op de tafel, met nog zo hard boenen, niet veel eer te behalen. De
Van Daans slapen nu al de hele winter op een lap flanel, die men hier niet
wassen kan, omdat de bonzeeppoeder veel te schaars en bovendien veel te slecht
is. Vader loopt met een gerafelde broek en ook zijn das vertoont slijtage.
Mama's korset is vandaag afgeknapt van ouderdom en is niet meer te repareren,
terwijl Margot met een twee maten te kleine bustehouder loopt.

Moeder en Margot hebben de hele winter samen met drie hemdjes gedaan en de mijne
zijn zo klein dat ze nog niet eens tot aan mijn buik komen.

Dit zijn nu wel allemaal dingen, waar overheen te stappen is, toch bedenk ik wel
eens met schrik: `Hoe willen wij, die vanaf mijn onderbroek tot op de
scheerkwast van vader met versleten dingen rondlopen, later weer eens tot onze
vooroorlogse stand gaan behoren?'

Vannacht heb ik vier keer al mijn bezittingen op moeten pakken, zo hard paften
ze. Vandaag heb ik een koffertje gepakt, waarin ik de nodigste vluchtbehoeften
heb gestopt. Maar terecht zegt moeder: `Waar wil je naar toe vluchten?'

Heel Nederland heeft straf voor het staken in vele gebieden. Daarom is er staat
van beleg afgekondigd en ieder krijgt een boterbon minder. Wat zijn de
kindertjes stout!

Je Anne.

\section*{Dinsdag, 18 Mei 1943}

Lieve Kitty,\\
Ik ben toeschouwster geweest van een hevig luchtgevecht tussen
Duitse en Engelse~vliegers. Een paar Geallieerden moesten jammer genoeg uit hun
brandende toestellen springen. Onze melkboer, die in Halfweg woont, heeft aan de
kant van de weg vier Canadezen zien zitten, waarvan er een vloeiend Hollands
sprak. Deze vroeg den melkboer om vuur voor een sigaret en vertelde hem, dat de
bemanning van de machine uit zes personen bestaan had. De piloot was verbrand en
hun vijfde man had zich ergens verstopt. De groene politie kwam de vier
kerngezonde mensen ophalen. Hoe is het mogelijk dat je na zo'n geweldige
parachute-reis nog zoveel tegenwoordigheid van geest hebt.

Hoewel het tamelijk warm is, moeten wij om de dag onze kachels aanmaken om
groente-afval en vuil te verbranden. In vuilnisemmers kunnen we niets gooien,
omdat we altijd met de magazijnknecht rekening moeten houden.  Hoe licht
verraadt een kleine onvoorzichtigheid je niet!

Alle studenten, die van het jaar willen afstuderen of verder studeren, moeten op
een lijst van de overheid tekenen, dat ze met de Duitsers sympathiseren en de
nieuwe orde goed gezind zijn. 80 pCt heeft het vertikt zijn geweten en
overtuiging te verloochenen, maar de gevolgen zijn niet uitgebleven. Alle
studenten, die niet getekend hebben, moeten in een werkkamp naar Duitsland. Wat
blijft er van de Nederlandse~jeugd nog over, als allen in Duitsland hard moeten
werken?

Wegens de harde knallen had moeder vannacht het raam gesloten; ik was in
Pims~bed. Eensklaps springt boven ons hoofd mevrouw uit haar bed, als door
Mouschi gebeten, direct gevolgd door een harde klap. Het klonk alsof een
brandbom naast mijn bed gevallen was. Ik gilde: `Licht, licht!' Pim knipte de
lamp aan. Ik verwachtte niet anders dan dat binnen een paar minuten de kamer in
lichtelaaie zou staan. Er gebeurde niets.  We haastten ons allen naar boven om
te zien wat daar aan de gang was.  Mijnheer en mevrouw hadden door het open raam
een rose gloed gezien.  Mijnheer dacht dat het hier in de buurt brandde en
mevrouw dacht, dat ons huis vlam gevat had. Bij de klap die volgde stond mevrouw
al op haar trillende benen. Maar er gebeurde verder niets en wij kropen weer in
onze bedden.

Er was nog geen kwartier verlopen, of het schieten begon opnieuw.  Mevrouw rees
dadelijk overeind en liep de trap af naar mijnheer Dussels kamer, om daar de
rust te vinden die haar bij haar ega niet beschoren was. Dussel ontving haar met
de woorden: `Kom in mijn bed, mijn kind!' Wat ons in een onbedaarlijke lach deed
schieten. Het kanonvuur deerde ons niet meer, onze angst was weggevaagd.

Je Anne.

\section*{Zondag, 13 Juni 1943}

Lieve Kitty,\\
Mijn verjaardagsvers van vader is te mooi dan dat ik je dit
gedichtje kan onthouden.

Daar Pim meest in het Duits dicht, moest Margot aan het vertalen slaan.  Oordeel
naar het hier aangehaalde stukje zelf, of Margot zich niet prima van haar
vrijwillige taak gekweten heeft. Na de gebruikelijke korte samenvatting van de
jaarsgebeurtenissen volgt:

`Als jongste van allen en toch niet meer klein Heb je het niet makkelijk; een
ieder wil zijn

Een beetje je leraar, jou dikwijls tot pijn.

``Wij hebben ervaring!'' - ``Neem het van mij aan!'' ``Wij hebben zoiets al
vaker gedaan\\
En weten beter wat kan en wat mag''.\\
Ja ja, zo gaat het sinds
jaar en dag.\\
De eigen gebreken zijn van geen gewicht,\\
Daarom valt berispen
een ieder zo licht.\\
Van andren de fouten die tellen zwaar,\\
't Is moeilijk
voor ons, je ouderpaar\\
Altijd te oordelen zeer rechtvaardig,\\
De ouden
berispen staat eigenaardig.\\
Is men tezamen met zoveel besjes,\\
Dan moet men
slikken al die lesjes\\
Zoals je neemt een bittre pil.\\
Men doet het dan om
vredes wil.\\
De maanden hier zijn niet verspild,\\
Dat had je zelf ook niet
gewild.\\
Met leren en lezen van veel boeken\\
Kon je ``verveling'' met een
lichtje zoeken.\\
Maar nog veel moeilijker is de vraag:\\
``Wat moet ik aandoen?
Wat ik ook draag\\
Is veel te klein. Ik heb geen broek,\\
Mijn hemd is als een
lendendoek\\
En dan de schoenen, 't is niet te dragen\\
Zoveel kwalen als me
plagen'''.

Een stuk over het eetthema kon Margot niet op rijm vertaald krijgen en daarom
laat ik het helemaal weg. Vind je overigens mijn vers niet mooi?  Verder ben ik
erg verwend en heb veel mooie dingen gekregen, onder andere een dik boek over
mijn lievelingsonderwerp: de mythologie van Hellas en Rome. Ook over gebrek aan
snoep kan ik niet klagen, allen hebben hun laatste voorraad aangesproken. Als
Benjamin van de duikfamilie ben ik werkelijk met veel meer vereerd dan me
toekomt.

Je Anne.

\section*{Dinsdag, 15 Juni 1943}

Lieve Kitty,\\
Er is wel een massa gebeurd, maar ik denk meestal dat al mijn
oninteressant geklets~je erg verveelt en dat je blij bent, als je niet zoveel
brieven ontvangt. Ik zal je de berichten dan ook maar in het kort weergeven.

Mijnheer Vossen is niet aan een maagzweer geopereerd. Toen ze hem op de
operatietafel hadden en zijn maag open was, zagen de doktoren, dat hij kanker
had, die al zover gevorderd was, dat er niets meer te opereren viel. Ze hebben
de maag dus weer gesloten, hem toen drie weken in bed gehouden en goed te eten
gegeven en ten slotte maar naar huis gestuurd.  Ik heb ontzettend veel
medelijden met hem en vind het heel erg dat we niet op straat kunnen gaan,
anders zou ik hem beslist vaak bezoeken om hem af te leiden. Voor ons is het een
ramp dat die goede Vossen ons niet meer van alles wat er in het magazijn gebeurt
en gehoord wordt op de hoogte houdt. Hij was onze beste hulp- en steunkracht
voor de voorzichtigheid, we missen hem erg.

Volgende maand zijn wij aan de beurt met de radio-inlevering. Koophuis heeft
thuis een clandestien Baby-toestel, dat wij ter vervanging van onze grote
Philips krijgen. Het is wel jammer dat die mooie kast ingeleverd moet worden,
maar een huis waar onderduikers zitten moet zich in geen geval de overheid
moedwillig op de hals halen. De kleine radio zetten we boven neer. Bij
clandestiene Joden, clandestien geld en clandestien kopen kan ook nog wel een
clandestiene radio.

Alle mensen proberen een oud toestel te bemachtigen om dat in plaats van hun
`moed-houd-bron' in te leveren. Het is heus waar, als de berichten van buiten
steeds slechter worden, helpt de radio met haar wonder-stem, dat we de moed niet
verliezen door elke keer weer te zeggen: `Kop op, houd goede moed, er komen ook
weer andere tijden!'

Je Anne.

\section*{Zondag, 11 Juli 1943}

Lieve Kitty,\\
Om voor de zoveelste keer op het opvoedingsthema terug te komen,
zal ik je~zeggen, dat ik me erg veel moeite geef om hulpvaardig, vriendelijk en
lief te zijn en alles zo te doen, dat de aanmerkingsregen een motregentje wordt.
Het is verhipt moeilijk om voor mensen, die je niet uit kunt staan zo
voorbeeldig te doen, terwijl je er niets van meent. Maar ik zie werkelijk in dat
ik er verder mee kom, als ik een beetje huichel, in plaats van mijn oude
gewoonte te volgen, namelijk om iedereen mijn mening ronduit te zeggen (hoewel
nooit iemand naar mijn mening vraagt of er waarde aan hecht.).

Vaak val ik erg uit mijn rol en kan mijn woede bij onrechtvaardigheden niet
verkroppen, zodat er weer vier weken lang over het brutaalste meisje ter wereld
gekletst wordt. Vind je niet dat ik soms te beklagen ben? Het is maar goed dat
ik geen mopperpot ben, want dan zou ik verzuurd worden en mijn goede humeur niet
kunnen bewaren.

Verder heb ik besloten om steno een beetje te laten schieten. Het heeft lang
geduurd. Ten eerste om nog meer tijd aan mijn andere vakken te kunnen besteden
en ten tweede om mijn ogen, want dat is een ellendige misère. Ik ben erg
bijziend geworden en moest al lang een bril hebben (oei, wat zal ik er uilachtig
uitzien), maar ja, je weet, onderduikers mogen ... Gisteren sprak het hele huis
over niets anders dan over Anne's ogen, omdat moeder geopperd had om mevrouw
Koophuis met mij naar den oogarts te sturen. Bij deze mededeling wankelde ik
even op mijn benen, want dat is toch geen kleinigheid.

Op straat. Stel je voor: op straat! Het is niet om in te denken. Eerst was ik
doodsbenauwd en later blij. Maar zo eenvoudig ging dat niet, niet alle
instanties, die over zulk een stap te beslissen hebben, waren het er zo gauw
over eens. Alle moeilijkheden en risico's moesten eerst op de~weegschaal gelegd
worden, alhoewel Miep direct met me op stap wilde.

Ik haalde alvast mijn grijze jas uit de kast, maar die was zo klein, dat hij wel
van~mijn jongste zusje leek. Ik ben werkelijk benieuwd wat er van komt, maar ik
denk wel niet dat het plan doorgaat, want intussen zijn de Engelsen op Sicilie
geland en vader is weer op `een spoedig einde' ingesteld.

Elli geeft Margot en mij veel kantoorwerk, dat vinden we alle twee gewichtig en
het helpt haar een heel stuk. Correspondentie opbergen en een verkoopboek
inschrijven kan iedereen, maar wij doen het bijzonder nauwgezet.

Miep is precies een pakezeltje, die sjouwt wat af. Haast elke dag heeft ze
ergens groente opgescharreld en brengt alles in grote inkooptassen op de fiets
mee. Zij is het ook, die iedere Zaterdag vijf bibliotheekboeken meebrengt. Wij
kijken altijd reikhalzend naar de Zaterdag uit, omdat dan de boeken komen. Net
als kleine kinderen, die een cadeautje krijgen.

Gewone mensen weten ook niet, hoeveel boeken voor ons opgeslotenen betekenen.
Lezen, leren en de radio zijn onze afleidingen.

Je Anne.

\section*{Dinsdag, 13 Juli 1943}

Lieve Kitty,\\
Gistermiddag had ik met vaders verlof aan Dussel gevraagd of hij
het alstublieft~goed zou vinden (toch erg beleefd) dat ik twee keer in de week
van het tafeltje in onze kamer ook 's middags van vier uur tot half zes gebruik
zou kunnen maken. Van half drie tot vier uur zit ik daar al elke dag, terwijl
Dussel slaapt en verder is de kamer plus het tafeltje verboden terrein. Binnen,
in onze algemene kamer is het 's middags veel te druk, daar kan men niet werken
en trouwens, vader zit 's middags toch ook wel eens graag te werken aan de
schrijftafel.

De reden was dus redelijk en de vraag alleen zuivere beleefdheid. Wat denk je nu
wel dat de hooggeleerde Dussel antwoordde: `Neen'. Botweg en alleen maar `Neen!'
Ik was verontwaardigd en liet me niet zomaar afschepen, vroeg hem dus de reden
van zijn `neen'. Maar ik kwam van een koude kermis thuis. Ziedaar de lading die
volgde:

`Ik moet ook werken, als ik niet 's middags kan werken schiet er voor mij
helemaal geen tijd meer over. Ik moet mijn pensum afkrijgen, anders ben ik voor
niets er aan begonnen. Jij werkt toch aan niets ernstigs.  Die mythologie, wat
is dat nou voor werk, breien en lezen is ook geen werk. Ik ben en blijf aan dat
tafeltje'. Mijn antwoord was: `Mijnheer Dussel, ik werk wèl ernstig; ik kan 's
middags binnen niet werken, ik verzoek u vriendelijk nog eens over mijn vraag na
te denken!'

Met deze woorden draaide de beledigde Anne zich om, en deed of de hooggeleerde
dokter lucht was. Ik ziedde van woede, vond Dussel verschrikkelijk onbeleefd (en
dat was-ie ook) en mij heel vriendelijk.  's Avonds toen ik Pim nog even te
pakken kon krijgen, vertelde ik hem hoe de zaak was afgelopen en besprak met hem
wat ik nu verder moest doen, want opgeven wilde ik de zaak niet en wilde ze ook
liever alleen opknappen. Pim gaf me zo ongeveer weer hoe ik de zaak moest
aanpakken, maar vermaande me om liever tot de volgende dag te wachten, wegens
mijn opgewondenheid.

Deze laatste raad sloeg ik in de wind en wachtte Dussel na de afwas op.  Pim zat
naast ons in de kamer en dat gaf me een grote rust. Ik begon: `Mijnheer Dussel,
ik geloof dat u het niet de moeite waard vond om met mij de zaak eens beter
onder de ogen te zien, maar ik verzoek u dat toch wel te doen'. Met zijn liefste
glimlach merkte Dussel toen op: `Ik ben altijd en te allen tijde bereid om over
deze, intussen al afgehandelde zaak, te spreken!'

Ik vervolgde toen het gesprek, steeds onderbroken door Dussel. `We hebben in het
begin, toen u hier kwam, afgesproken, dat deze kamer voor ons samen zou zijn,
als de indeling dus juist was, zou u de morgen en ik de hele middag krijgen!
Maar dat vraag ik niet eens, en me dunkt, dan zijn twee middagen in de week toch
wel billijk'.

Hierbij sprong Dussel als door een naald gestoken op. `Over recht heb jij hier
helemaal niet te spreken. En waar moet ik dan blijven? Ik zal aan mijnheer Van
Daan vragen of hij op zolder een hokje voor mij bouwen wil, dan kan ik daar
zitten, ik kan ook nergens eens rustig werken. Met jou heeft een mens ook altijd
ruzie. Als je zuster Margot, die toch wel meer reden tot zo'n vraag heeft, met
hetzelfde verzoek tot mij zou komen, zou ik er niet aan denken te weigeren, maar
jij ...'

En toen volgde weer het verhaal van de mythologie en het breien, en Anne was
opnieuw beledigd. Ze liet het echter niet merken en liet Dussel uitspreken:
`Maar ja, met jou is nu eenmaal niet te praten. Je bent een schandelijke egoïst,
als jij maar je zin kunt doordrijven, dan kunnen alle anderen opzij gaan, zo'n
kind heb ik nog nooit gezien. Maar per slot van rekening zal ik toch wel
genoodzaakt zijn je je zin te geven, want anders hoor ik later dat Anne Frank
voor haar examen gezakt is, omdat mijnheer Dussel haar het tafeltje niet wilde
afstaan'.

Het ging maar door en steeds maar door, op het laatst zo'n vloed, dat ik het
haast niet meer bij kon houden. Het ene ogenblik dacht ik: `Ik geef hem direct
een klap voor zijn snuit, dat hij met zijn onwaarheden tegen de muur aanvliegt',
en het volgende ogenblik zei ik tegen mijzelf: `Houd je kalm, die vent is niet
waard, dat je je zo druk over hem maakt!'

Eindelijk was de heer Dussel dan uitgeraasd en ging met een gezicht, waarop
zowel gramschap als triomf te lezen stonden, met zijn jas vol etenswaren de
kamer uit. Ik rende naar vader en vertelde, voor zover hij ons gesprek niet
gevolgd had, het hele relaas. Pim besloot om nog dezelfde~avond met Dussel te
gaan praten en zo gebeurde het. Ze praatten meer dan een half uur. Het onderwerp
van het gesprek was ongeveer als volgt: ze spraken eerst daar over, of Anne nu
aan het tafeltje moest zitten, ja dan neen. Vader zei dat Dussel en hij al een
keer over datzelfde onderwerp gesproken hadden, maar dat hij toen zogenaamd
Dussel gelijk gegeven had om den oudere niet tegenover de jongere in het
ongelijk te stellen. Maar billijk had vader het toen al niet gevonden.  Dussel
vond, dat ik niet mocht spreken, alsof hij een indringer was en op alles beslag
legde, maar dit sprak vader beslist tegen, want hij had zelf gehoord, dat ik
daar met geen woord van gerept had. Het ging zo heen en weer, vader mijn egoïsme
en mijn `pruts'-werk verdedigend, Dussel steeds na-mopperend.

Eindelijk moest Dussel dan toch toegeven en ik kreeg twee middagen in de week
eens gelegenheid om ongestoord tot vijf uur te werken. Dussel keek heel sip,
sprak twee dagen niet tegen me en moest van vijf tot half zes toch nog aan het
tafeltje zitten ... kinderachtig natuurlijk.

Iemand die al 54 jaar oud is en nog zulke pedante en kleinzielige gewoonten
heeft, is door de natuur zo gemaakt en leert die gewoonten ook nooit meer af.

Je Anne.

\section*{Vrijdag, 16 Juli 1943}

Lieve Kitty,\\
Alweer een inbraak, maar nu een echte!\\
Vanochtend ging Peter,
als gewoonlijk, om zeven uur naar het magazijn en zag al~dadelijk, dat zowel de
magazijnals de straatdeur openstonden. Hij berichtte dit aan Pim, die in het
privé-kantoor de radio op Duitsland en de deur op slot deed. Samen gingen ze
toen naar boven.

Het gewone commando in dergelijke gevallen, geen kraan opendraaien, dus niet
wassen, stil zijn, om acht uur kant en klaar zitten, en niet naar de W.C....,
werd als gewoonlijk~stipt opgevolgd. We waren alle acht blij, dat we 's nachts
zo goed geslapen en niets gehoord hadden. Pas om half twaalf vertelde mijnheer
Koophuis dat de inbrekers met een koevoet de buitendeur ingeduwd en de
magazijndeur geforceerd hadden. Er viel in het magazijn echter niet veel te
stelen en daarom beproefden de dieven hun geluk maar een étage hoger.

Ze hebben twee geldkistjes, inhoudende ƒ 40. -, blanco giro- en bankboekjes en
dan het ergste, de hele suikertoewijzing in bons van totaal 150 kg gestolen.

Mijnheer Koophuis denkt, dat deze inbrekers tot hetzelfde gilde behoren als
degene, die zes weken geleden hier was en aan alle drie deuren geprobeerd heeft
binnen te komen. Toen was het niet gelukt.

Het geval heeft weer wat deining in ons gebouw veroorzaakt, maar zonder dit
schijnt het Achterhuis het niet te kunnen stellen. We waren blij, dat de
schrijfmachines en de kassa veilig in onze klerenkast zaten, want die halen ze
elke avond naar boven.

Je Anne.

\section*{Maandag, 19 Juli 1943}

Lieve Kitty,\\
Zondag is Amsterdam-Noord heel zwaar gebombardeerd. De
verwoesting moet~ontzettend zijn. Hele straten liggen in puin en het zal nog
lang duren, voordat al de mensen opgedolven zijn. Tot nu toe zijn er 200 doden
en ontelbare gewonden; de ziekenhuizen zijn propvol. Je hoort van kinderen die
verloren in de smeulende ruïnes naar hun dode ouders zoeken.

Rillingen krijg ik als ik nog aan het doffe, dreunende gerommel in de verte
denk, dat voor ons een teken van de naderende vernieling was.

Je Anne.

\section*{Vrijdag, 23 Juli 1943.}

Lieve Kitty,\\
Voor de aardigheid zal ik je eens vertellen wat nu de eerste wens
van ons achten~is, als we weer eens naar buiten mogen.  Margot en mijnheer Van
Daan wensen zich het meest een heet bad tot boven aan toe en willen daar wel
meer dan een half uur in blijven. Mevrouw Van Daan wil het liefst dadelijk
taartjes gaan eten, Dussel kent niets dan Lotje, zijn vrouw, moeder haar kopje
koffie, vader gaat eerst een bezoek aan mijnheer Vossen brengen, Peter naar de
stad enfin de bioscoop en ik zou van de zaligheid niet weten wat te beginnen.

Ik verlang het meest naar een eigen woning, vrije beweging en eindelijk weer
hulp bij het werk, dus naar school.

Elli heeft ons fruit aangeboden. Kost een kleinigheid. Druiven ƒ 5. - per kg,
kruisbessen ƒ 0.70 per pond. Een perzik ƒ 0.50, 1 kg meloen ƒ 1.50. En dan
zetten ze iedere avond met koeienletters in de krant: `Prijsopdrijving is
woeker!'

Je Anne.

\section*{Maandag, 26 Juli 1943}

Lieve Kitty,\\
Gisteren was het een erg rumoerige dag en we zijn nog steeds
opgewonden. Je~kunt eigenlijk vragen, welke dag er zonder opwinding voorbijgaat.

's Ochtends bij het ontbijt kregen we de eerste keer voor-alarm, maar dat kan
ons niets bommen, want het betekent, dat er vliegers aan de kust zijn. Na het
ontbijt ben ik een uurtje gaan liggen, want ik had erge hoofdpijn en ging toen
naar beneden. Het

was ongeveer twee uur. Om half drie was Margot met haar kantoorwerk klaar; ze
had haar boeltje nog niet gepakt of de sirenes loeiden, dus trok ik weer met
haar naar boven. Het werd tijd, want we waren nog geen vijf minuten boven, of ze
begonnen hard te schieten, zodat we in de gang gingen staan. En ja hoor, daar
dreunde het huis en vielen de bommen.

Ik klemde mijn vluchttas tegen me aan, meer om wat te hebben om vast te houden
dan om te vluchten, want we kunnen toch niet weg. Wanneer we in het uiterste
geval zouden vluchten, betekent de straat evenveel levensgevaar als een
bombardement. Na een half uur werd het vliegen minder, maar de bedrijvigheid in
huis nam toe. Peter kwam van zijn uitkijkpost op de voorzolder, Dussel was in
het voorkantoor, mevrouw voelde zich in het privé-kantoor veilig, mijnheer Van
Daan had vanaf de vliering toegekeken en wij in het portaaltje verspreidden ons
ook en ik klom de trap op, om de rookzuilen boven het IJ te zien opstijgen
waarvan mijnheer Van Daan verteld had. Weldra rook het overal naar brand en het
leek buiten of er dikke mist hing.

Hoewel zo'n grote brand geen leuk gezicht is, was het voor ons gelukkig weer
achter de rug en we begaven ons aan onze respectievelijke bezigheden. 's Avonds
bij het eten: luchtalarm. We hadden lekker eten, maar de trek verdween bij mij
al bij het geluid alleen. Er gebeurde echter niets en drie kwartier later was
alles veilig. De afwas stond aan de kant: luchtalarm, schieten, vreselijk veel
vliegers. `Oh jeminee, twee keer op een dag, dat is erg veel', dachten we allen,
maar dat hielp niets, weer regende het bommen, dit keer aan de andere kant, naar
de Engelsen zeggen op Schiphol. De vliegers doken, stegen, het suisde in de
lucht en het was erg griezelig. Elk ogenblik dacht ik: `Nu valt hij, daar ga
je'.

Ik kan je wel verzekeren dat, toen ik om negen uur naar bed ging, ik mijn benen
nog niet recht kon houden. Klokslag twaalf werd ik wakker: vliegers. Dussel was
zich aan het uitkleden, ik stoorde me er niet aan, sprong toch bij het eerste
schot klaarwakker uit bed. Twee uur weer bij vader en ze vlogen steeds en steeds
nog. Toen werd er geen schot meer gelost en ik kon naar bed gaan. Om half drie
sliep ik in.

Zeven uur. Met een schok zat ik overeind in bed. Van~Daan was bij vader.
Inbrekers, was mijn eerste gedachte. `Alles', hoorde ik Van Daan zeggen, ik
dacht dat alles gestolen was. Maar neen, nu was het een heerlijk bericht, zo
mooi als we in maanden, misschien in al de oorlogsjaren niet gehoord hadden.
`Mussolini is afgetreden, de Keizerkoning van Italië heeft de regering
overgenomen'. We juichten. Na al dat vreselijke van gisteren, eindelijk weer wat
goeds en ... hoop. Hoop op het einde, hoop op de vrede.

Kraler is even langs gekomen en vertelde dat Fokker zwaar geteisterd is.
Intussen hadden we vannacht weer luchtalarm met overvliegers en nog een
voor-alarm. Ik stik zowat in de alarms, ben niet uitgeslapen en heb geen zin in
werk. Maar nu houdt toch de spanning over Italië ons wakker en de hoop op het
einde, misschien nog dit jaar ...

Je Anne.

\section*{Donderdag, 29 Juli 1943}

Lieve Kitty,\\
Mevrouw Van Daan, Dussel en ik waren met de afwas bezig en wat
haast nooit voorkomt en hun zeker op zou vallen was, dat ik buitengewoon stil
was.\\
Om vragen te voorkomen zocht ik dus gauw een tamelijk neutraal onderwerp
en

dacht dat het boek~\emph{Henri van den Overkant}~aan die eis wel zou voldoen.
Maar ik had me misrekend. Krijg ik mevrouw niet op mijn dak, dan is het mijnheer
Dussel. Het kwam hierop neer: mijnheer Dussel had ons dit boek bijzonder
aanbevolen, als een uitstekend exemplaar. Margot en ik vonden het evenwel
allesbehalve uitstekend. Het jongetje was wel goed getekend, maar de rest ...
laat ik daarover liever zwijgen. Iets dergelijks bracht ik bij de afwas op het
tapijt en dat bezorgde me een enorme lading.

`Hoe kan jij de psyche van een man begrijpen! Van een kind is dat niet zo
moeilijk (!). Je bent veel te jong voor~een dergelijk boek, een man van twintig
jaar zou dit niet eens kunnen bevatten'. (Waarom heeft hij speciaal Margot en
mij dit boek zo aanbevolen?)

Nu vervolgden Dussel en mevrouw Van Daan tezamen:

`Je weet veel te veel van dingen af, die niet voor je geschikt zijn, je bent
totaal verkeerd opgevoed. Later, als je ouder bent, heb je nergens meer plezier
in, dan zeg je: ``Dat heb ik twintig jaar geleden al in boeken gelezen''. Je
moet je maar haasten, wil je nog een man krijgen of verliefd worden, jou valt
zeker alles tegen. Je bent in de theorie al helemaal volleerd, alleen de
practijk, die mis je nog!'

Het is zeker hun opvatting van een goede opvoeding om me altijd zo tegen mijn
ouders op te stoken, want dat is het wat ze vaak doen. En om een meisje van mijn
leeftijd niets over `volwassen' onderwerpen te vertellen, is al een even
uitstekende methode. De resultaten van een dergelijke opvoeding treden vaak maar
al te duidelijk aan het licht.

Ik kon die twee, die me daar belachelijk stonden te maken, in hun gezicht slaan
op dat ogenblik. Ik was buiten mezelf van woede en zou werkelijk de dagen gaan
tellen, dat ik van `die' mensen af ben. Die mevrouw Van Daan is me ook een
exemplaar! Daar moet men zijn voorbeelden vandaan halen, wel de voorbeelden ...
maar dan de slechte.

Ze staat bekend als erg onbescheiden, egoïstisch, sluw, berekenend en met niets
tevreden. IJdelheid en coquetterie komen er ook nog bij. Ze is, daar valt niet
aan te tornen, een uitgesproken naar persoontje. Over madame zou ik boekdelen
kunnen vullen en wie weet, kom ik daar nog eens toe. Een mooi vernisje aan de
buitenkant kan iedereen zich opleggen.  Mevrouw is vriendelijk voor vreemden en
vooral voor mannen en daardoor vergist men zich, als men haar pas kent.

Moeder vindt haar te dom om een woord aan te verliezen, Margot te onbelangrijk,
Pim te lelijk (letterlijk en figuurlijk) en ik ben na een lange reis, want ik
ben nooit van begin af~aan vooringenomen, tot de slotsom gekomen, dat ze alle
drie en nog veel meer is. Ze heeft zoveel slechte hoedanigheden, waarom zou ik
dan met één er van beginnen?

Je Anne.

P.S. Wil de lezer even onder de ogen zien, dat toen dit verhaal geschreven werd,
de schrijfster nog niet van haar woede bekoeld was!

\section*{Woensdag, 4 Augustus 1943}

Lieve Kitty,\\
Je weet nu, nadat we ruim een jaar Achterhuizers zijn, wel wat
van ons leven af, maar volledig kan ik je toch niet inlichten. Het is alles zo
uitgebreid, anders dan in gewone tijden en bij gewone mensen. Om je toch een
beetje een nadere blik in ons leven te gunnen zal ik in het vervolg af en toe
een stuk van een gewone dag beschrijven. Vandaag begin ik met de avond en de
nacht:

\emph{'s Avonds negen uur}~begint in het Achterhuis de drukte van het naar bed
gaan en dat is werkelijk altijd een hele drukte.

Stoelen worden verschoven, bedden omvergehaald, dekens opgevouwen; niets blijft
waar het overdag wezen moet. Ik slaap op de kleine divan, die nog geen 1.50 m
lang is. Hier moeten dus stoelen als verlengstuk dienen. Een plumeau, lakens,
kussens, dekens, alles wordt uit Dussels bed gehaald, waar het overdag verblijf
houdt.

Binnen hoort men een vreselijk gekraak; het bed à la harmonica van Margot. Weer
divandekens en kussens, alles om de houten latjes een beetje comfortabeler te
maken. Boven lijkt het of het onweert, het is alleen maar het bed van mevrouw.
Dit wordt~namelijk bij het raam geschoven om hare Hoogheid, met het rose
bedjasje, iets prikkelends in de kleine neusgaatjes te geven.

\emph{Negen uur:}~Na Peter treed ik de waskamer binnen, waar dan een grondige
wasbehandeling volgt, en het gebeurt~soms (alleen in de hete maanden of weken),
dat er een kleine vlo mee in het waswater drijft. Dan tanden poetsen, haren
krullen, nagels behandelen, waterstofwatjes hanteren (dient om zwarte snorharen
te bleken) en dit alles in een klein half uur.

\emph{Half tien}. Gauw badjas aan, zeep in de ene, po, haarspelden, broek,
kruipennen en watten in de andere hand snel ik de badkamer uit, meestal nog eens
teruggeroepen voor de verschillende haren, die in sierlijke, maar voor de
volgende wasser niet al te aangename bogen de wastafel ontsieren.

\emph{Tien uur:}~Verduistering voor, goede nacht. In huis wel een kwartier lang
een gekraak van bedden en een zuchten van kapotte veren, dan is alles stil, als
tenminste de bovenburen geen bedruzie hebben.

\emph{Half twaalf:}~De badkamerdeur kraakt. Een dunne streep licht valt in de
kamer. Gekraak van schoenen, een grote jas, nog groter dan de man er onder ...
Dussel komt van zijn nachtwerk in het kantoor van Kraler terug. Tien minuten
lang schuifelen over de grond, ritselen van papier (dat zijn de op te bergen
etenswaren) en een bed wordt opgemaakt. Dan verdwijnt de gestalte weer en men
hoort alleen van tijd tot tijd van de W.C. een verdacht geluid komen.

\emph{Drie uur:}~Ik moet opstaan voor een kleine boodschap in het metalen blikje
onder mijn bed, waar voorzichtigheidshalve een gummi-matje onder gelegd is, voor
het eventuele lekken. Als dit moet geschieden, houd ik altijd mijn adem in, want
het klatert in het busje als een beekje van een berg. Dan gaat het blikje weer
op zijn plaats en een gestalte in een witte nachtjapon, die iedere avond weer
aan Margot de uitroep ontlokt: `O, die onzedelijke nachtjapon!' stapt in bed.

Een klein kwartiertje ligt dan een zeker iemand te luisteren naar de nachtelijke
geluiden. Allereerst of er misschien een dief beneden zou kunnen zijn, dan naar
de diverse bedden, boven naast enfin de kamer, waaruit men meestal kan opmaken
hoe de verschillende huisgenoten slapen of half-wakker de nacht doorbrengen.

Dit laatste is zeker niet prettig, vooral als het een lid van de familie met
name Dussel betreft. Eerst hoor ik een geluidje alsof een vis naar lucht hapt,
dit herhaalt zich een stuk of tien keer, dan worden met veel omhaal de lippen
bevochtigd, afgewisseld door kleine smakgeluidjes, gevolgd door een langdurig
heen en weer draaien in bed en verschikken van kussens. Vijf minuten volkomen
rust en dan herhaalt zich deze afwikkeling van gebeurtenissen nog wel drie keer,
waarna de dr zich zeker weer voor een poosje in slaap gesust heeft.

Het kan ook gebeuren, dat er 's nachts variërend tussen een en vier uur
geschoten wordt. Dit besef ik nooit helemaal vóór ik uit gewoonte naast mijn bed
sta. Soms ben ik ook zo aan het dromen, dat ik denk aan Franse onregelmatige
werkwoorden of een bovenruzietje. Als dit dan afgelopen is, merk ik pas dat er
geschoten is en ik stil in de kamer gebleven ben.  Maar meestal gebeurt, zoals
boven gezegd. Vlug wordt een kussen plus zakdoek in de hand gestopt, badjas aan,
pantoffels en dan op een holletje naar vader, net zoals Margot in een
verjaardagsgedicht schreef:

`'s Nachts bij het allereerst geschiet,\\
Daar kraakt een deur en ziet ...\\
Een
zakdoek, een kussen en een kleine meid ...'

Eenmaal bij het grote bed aangeland, is de ergste schrik al heen, behalve als
het schieten erg hard mocht zijn.

\emph{Kwart voor zeven:}~Trrrr ... het wekkertje, dat op elk uur van de dag (als
men er naar vraagt of soms ook zonder dat) zijn stemmetje kan verheffen. Knak
... pang ... Mevrouw heeft het afgezet. Kraak ...  mijnheer is opgestaan. Water
opzetten en dan fluks naar de badkamer.

\emph{Kwart over zeven:}~De deur kraakt weer. Dussel kan naar de badkamer gaan.
Eenmaal alleen wordt ontduisterd ... en de nieuwe dag in het Achterhuis is
begonnen.

Je Anne.

\section*{Donderdag, 5 Augustus 1943}

Lieve Kitty,\\
Vandaag nemen we het schaftuurtje maar eens.\\
\emph{Het is half
een:}~De hele rataplan ademt op. Nu zijn de magazijnknechten naar huis.

Boven hoort men het stompen van de stofzuiger over mevrouw haar mooie en enige
kleed. Margot neemt een paar boeken onder haar arm en gaat naar het Nederlandse
onderwijs `voor kinderen die niet verder komen', want daarop lijkt Dussel wel.
Pim gaat met zijn onafscheidelijke Dickens in een hoekje zitten, om toch maar
ergens rust te vinden. Moeder spoedt zich een étage hoger om de nijv're
huisvrouw te helpen en ik ga naar de badkamer om die een beetje op te knappen,
tegelijk met mijzelf.

\emph{Kwart voor een:}~Het emmertje druppelt vol. Eerst mijnheer Van Santen, dan
Koophuis of Kraler, Elli en soms ook even Miep.

\emph{Een uur:}~Gespannen zit alles te luisteren naar de B.B.C.  Geschaard om
het baby-radiotje zitten ze en dit zijn de enige minuten, waar de leden van het
Achterhuis elkaar niet in de rede vallen, want hier spreekt iemand dien zelfs
mijnheer Van Daan niet tegenspreken kan.

\emph{Kwart over een:}~De grote uitdeling. Ieder van beneden krijgt een kopje
soep en als er eens een toetje mocht zijn, ook iets hiervan.  Tevreden gaat
mijnheer Van Santen op de divan zitten of tegen de schrijftafel leunen. De
krant, het kopje en meestal de poes naast zich.  Als één van deze drie mankeert
zal hij niet nalaten te protesteren.  Koophuis vertelt de laatste nieuwtjes uit
de stad, hij is daarvoor inderdaad een uitstekende bron. Kraler komt
holderdebolder de trap op, een korte en stevige tik op de deur en hij komt
handenwrijvend binnen en, al naar gelang de stemming, goed gemutst en druk of
slecht gehumeurd en stil.

\emph{Kwart voor twee:}~De eters verheffen zich en ieder gaat weer naar zijn
eigen bezigheden. Margot en moeder de afwas, mijnheer en mevrouw op hun divan.
Peter op zolder, vader~op de divan beneden, Dussel op de zijne en Anne aan het
werk. Nu volgt het rustigste uurtje, als allen slapen wordt niemand gestoord.
Dussel droomt van lekker eten, daar ziet zijn gezicht echt naar uit en ik kijk
niet lang, want de tijd snelt en om vier uur staat de pedante dr al met de klok
in de hand, omdat ik een minuut over tijd het tafeltje voor hem afruim.

Je Anne.

\section*{Maandag, 9 Augustus 1943}

Lieve Kitty,\\
Dit keer vervolg van de Achterhuis-dagindeling. Na het
schaftuurtje is de~middagtafel aan de beurt:

\emph{Mijnheer Van Daan}~opent de rij. Hij wordt het eerst bediend, neemt
behoorlijk~van alles, als het hem smaakt. Praat meestal mee, geeft altijd zijn
opinie ten beste en als dat eenmaal het geval is, valt er niet meer aan te
tornen. Want als iemand dàt durft, dan is hij lang niet mis. Och ... hij kan als
een kat tegen je blazen ... ik heb het liever niet, hoor ... als je het éénmaal
meegemaakt hebt, geen tweede keer.

Hij heeft de beste mening, hij weet het meest van alles af. Nu goed, hij heeft
een knappe kop, maar `zelfingenomenheid' heeft bij dat heer een hoge graad
bereikt.

\emph{Madame:}~Eigenlijk kan ik beter zwijgen. Op sommige dagen, vooral als er
een slechte bui op til is, is er in haar gezicht niet te kijken.  Op de keper
beschouwd, is zij de schuldige van al de discussies. Niet het onderwerp! Oh
neen, daarvan houdt ieder zich liever afzijdig, maar men zou haar misschien de
aanstichtster kunnen noemen. Stoken, dat is een leuk werk. Stoken tegen mevrouw
Frank en Anne. Tegen Margot en vader gaat dat niet zo gemakkelijk.

Maar nu aan tafel. Mevrouw komt niet te kort, al denkt ze dat wel eens.  De
kleinste aardappels, het lekkerste hapje, het beste van alles; zoeken is
madame's parool. De anderen krijgen wel hun beurt, als ik maar eerst het beste
heb.

Dan praten. Of er iemand luistert, of het iemand interesseert of niet, dat komt
er schijnbaar niet op aan, zij denkt zeker: `wat mevrouw Van Daan zegt,
interesseert iedereen ...'

Coquet glimlachen, doen alsof men van alles wat afweet, ieder een beetje raad
geven en bemoederen, dat~\emph{moet}~toch een goede indruk wekken.  Maar kijk je
langer, dan is het goede er al lang van af.

Vlijtig één, vrolijk twee, coquet drie en soms een leuk snuitje. - Dit is
Petronella van Daan.

\emph{De 3de tafelgenoot:}~Men hoort er niet veel van. De jongeheer Van Daan is
meest stil en laat niet veel van zich merken. Wat eetlust betreft: een
Danaïdenvat, het wordt nooit vol en bij de allerstevigste maaltijd beweert hij
met een doodkalm gezicht dat hij nog wel het dubbele zou kunnen eten.

\emph{No 4 Margot:}~Eet als een muisje, praat helemaal niet. Het enige dat er
ingaat is groente of fruit. `Verwend' is Van Daans oordeel, `te weinig lucht en
sport' onze opinie.

\emph{Daarnaast mama:}~Stevige eetlust, drukke praatster. Niemand heeft als bij
mevrouw Van Daan het idee: dat is de huisvrouw. Waar het verschil zit? Wel,
mevrouw kookt en moeder wast af en poetst.

\emph{No 6 en 7:}~Over vader en mij zal ik maar niet veel zeggen. De eerste is
de bescheidenste van heel de tafel. Hij kijkt eerst of de anderen ook hebben.
Niets heeft hij nodig, de beste dingen zijn voor de kinderen. Daar zit het
voorbeeld van het goede, er naast de zenuwpil van huize Achter.

\emph{Dr Dussel:}~Neemt, kijkt niet, eet, praat niet. En als men praten moet,
dan in 's hemelsnaam maar over eten; daar komt geen ruzie van, alleen
opsnijderij. Enorme porties gaan er in en `neen' wordt nooit gezegd, niet bij
het goede en ook niet vaak bij het slechte. - De broek zit aan de borst, het
rode jasje, zwarte pantoffels en een hoornen bril.  Zo~kan men hem zien aan het
tafeltje, eeuwig werkende, alleen afgewisseld door het middagdutje, eten en ...
het liefste plekje ... de W.C. Drie, vier, vijf keer per dag staat ongeduldig
iemand voor de W.C.-deur en knijpt, hipt van het ene op het andere been en is
haast niet te houden. Stoort hij er zich aan? Niks hoor, van kwart over zeven
tot half acht, van half een tot één uur, van twee uur tot kwart over twee, van
vier uur tot kwart over vier, van zes uur tot kwart over zes en van half twaalf
tot twaalf uur. Men kan het noteren, dit zijn de vaste `zittingstijden'. Er
wordt niet van afgeweken en hij stoort zich niet aan de smekende stem buiten de
deur, die waarschuwt voor een snel naderend onheil.

\emph{No 9:}~Is geen Achterhuis-familielid, maar wel een huis- en tafelgenote.
Elli heeft een gezonde eetlust. Laat niets staan, is niet kieskeurig. Met alles
kan men haar plezieren en dat juist doet ons plezier. Vrolijk en goed gehumeurd,
gewillig en goedig, dat zijn haar kenmerken.

Je Anne.

\section*{Dinsdag, 10 Augustus 1943}

Lieve Kitty,\\
Nieuw idee: ik praat aan tafel meer met mezelf dan met de
anderen, dat is in twee opzichten gunstig. Ten eerste zijn ze allemaal blij, als
ik niet aan één stuk doorbabbel en ten tweede hoef ik me niet aan een andermans
oordeel te ergeren. Mijn mening vind ik zelf niet stom en anderen wel; dus kan
ik hem beter voor mezelf houden. Net zo doe ik, als ik iets moet eten, wat ik
helemaal niet kan uitstaan. Ik neem het bord voor me, verbeeld me dat het iets
heel lekkers is, kijk er zo weinig mogelijk naar en voordat ik het weet is het
op. 's Ochtends bij het opstaan, ook zo iets dat zeer onaangenaam is, spring ik
uit bed, denk bij mezelf `je gaat er dadelijk weer lekker in', loop naar het
raam, ontduister, snuffel zo lang aan de kier tot ik een beetje lucht voel en
ben wakker.

Het bed wordt zo gauw mogelijk uitgelegd, dan is er geen verleiding meer. Weet
je hoe moeder zoiets noemt? `Een levenskunstenaar!' Vind je dat geen grappig
woord? We zijn sinds een week allemaal een beetje met de tijd in de war, daar
onze lieve en dierbare Westertorenklok blijkbaar weggehaald is voor
oorlogsdoeleinden, en wij~dag noch nacht precies weten hoe laat het is. Ik heb
nog wel wat hoop, dat ze iets zullen uitvinden, dat de buurt een beetje aan de
klok herinnert, bijvoorbeeld een tinnen, koperen of weet ik wat voor een ding.

Of ik boven, beneden of waar ook ben, iedereen kijkt me bewonderend op mijn
voeten, waaraan een paar zeldzaam mooie schoenen (voor deze tijd) prijken. Miep
heeft ze tweedehands op de kop getikt voor ƒ 27.50: wijnrood peau de suède met
leer en een tamelijk hoge blokhak. Ik loop als op stelten en zie er nog veel
groter uit dan ik al ben.

Dussel heeft ons indirect in levensgevaar gebracht. Hij liet warempel door Miep
een verboden boek meebrengen, een scheldexemplaar op Mussolini en Hitler.
Onderweg werd ze toevallig door een S.S.-motor aangereden. Ze verloor haar
zelfbeheersing, schreeuwde `ellendelingen' en reed door.  Laat ik er niet aan
denken wat er gebeurd zou zijn, als ze mee naar het bureau had moeten gaan.

Je Anne.

\section*{Woensdag, 18 Augustus 1943}

Lieve Kitty,\\
Boven dit stukje de titel: De taak van de dag in de gemeenschap:
aardappelschillen! De één haalt de kranten, de tweede de mesjes (houdt
natuurlijk het beste voor~zichzelf), de derde de aardappels, de vierde de pan
met water.

Mijnheer Dussel begint, krabt niet altijd goed, maar krabt zonder ophouden,
kijkt~even links en rechts: doet iedereen het wel op dezelfde manier als hij?
Neen: `Anne, kijk maal,~ik neem het mesje zo in mijn hand, krab van boven toe
onder! Nein, zo niet ... maar zo!'

`Ik vind het anders zó gemakkelijker, mijnheer Dussel', merk ik schuchter op.

`Maar dit is toch de beste manier. Du kannst dies toch van mij aannemen.  Het
kan mij natuurlijk niets schelen, aber Du musst het zelf weten'.

We krabben weer verder. Ik kijk tersluiks eens naar mijn buurman. Die schudt in
gedachten nog eens zijn hoofd (zeker over mij), maar zwijgt.

Ik krab weer door. Kijk dan weer even naar de andere kant nu, waar vader zit;
voor vader is aardappels krabben geen karwei, maar een precies werkje. Als hij
leest heeft-ie een diepe rimpel in zijn achterhoofd, maar als hij aardappels,
bonen of andere groente helpt klaarmaken, dan schijnt het of niets tot hem
doordringt. Dan heeft hij zijn aardappelgezicht en nooit zal-ie een minder goed
gekrabde aardappel a everen; dat bestaat eenvoudig niet als hij zo'n gezicht
trekt.

Ik werk weer door en dan kijk ik weer heel even op; nu weet ik al genoeg.
Mevrouw probeert of ze Dussels aandacht niet kan trekken. Eerst kijkt ze even
naar hem en Dussel doet of hij niets merkt. Dan knipoogt ze; Dussel werkt door.
Dan lacht ze; Dussel kijkt niet op. Dan lacht moeder ook; Dussel trekt er zich
niets van aan. Mevrouw heeft niets bereikt, nu moet ze wel weer wat anders
verzinnen. Even stilte, dan: `Putti, doe toch een schort voor! Morgen moet ik
weer de vlekken uit je pak halen!'

`Ik maak me niet vuil!'\\
Weer even stilte.\\
`Putti, waarom ga je niet
zitten?'\\
`Ik sta goed zo, ik sta liever!' Pauze.\\
`Putti, kijk Du spatst
schon!'\\
`Ja, mammi, ik pas wel op'. Mevrouw zoekt een ander onderwerp.

`Zeg, Putti, waarom bombarderen de Engelsen nu niet?'\\
`Omdat het weer te
slecht is, Kerli'.\\
`Maar gisteren was het weer toch mooi en hebben ze ook niet
gevlogen'.\\
`Laten we daar niet over praten'.\\
`Waarom, daar kan men toch wel
over praten of zijn mening over zeggen?' `Neen'.\\
`Waarom nu niet?'\\
`Wees nu
stil, mammi'chen'.\\
`Mijnheer Frank geeft zijn vrouw toch ook altijd
antwoord?'\\
Mijnheer worstelt, dit is zijn tere plek, daar kan hij niet tegen,
en mevrouw begint

altijd weer:\\
`De invasie komt toch nooit!' Mijnheer wordt wit, als mevrouw dat
merkt wordt

ze rood, maar ze gaat toch weer verder:\\
`De Engelsen presteren niets!' De bom
barst.\\
`En nu houd je je mond, donnerwetter-noch-einmal!'\\
Moeder kan haar
lachen haast niet verbijten, ik kijk strak voor me.\\
Zoiets herhaalt zich haast
elke dag, als ze tenminste niet pas een erge ruzie gehad

hebben, want dan houden zowel mijnheer als mevrouw hun mond.\\
Ik moet naar de
zolder nog wat aardappels halen. Daar is Peter bezig de kat-te~ontvlooien. Hij
kijkt op, de kat merkt het, hup ... weg is ie door het open raam, de goot in.
Peter vloekt, ik lach en verdwijn.

Je Anne.

\section*{Vrijdag, 20 Augustus 1943}

Lieve Kitty,\\
Precies half zes gaan de mannen uit het magazijn naar huis en dan
hebben wij de~vrijheid.

Half 6: Elli komt ons de avond-vrijheid schenken. Er~komt dadelijk schot in het
bedrijf. Ik ga eerst met Elli nog even naar boven, waar zij meestal ons
avondtoetje vooruit krijgt.

Elli zit nog niet, of mevrouw begint al haar wensen op te sommen, het klinkt dan
al gauw: `Ach, Elli, ik heb een wens ...'

Elli knipoogt tegen mij, mevrouw zal geen keer overslaan om wie maar boven komt
haar wensen kenbaar te maken. Dat is zeker één van de redenen, waarom ze geen
van allen graag naar boven gaan.

Kwart voor 6: Elli vertrekt. Ik ga twee verdiepingen lager kijken. Eerst naar de
keuken, dan naar het privé-kantoor, vervolgens naar het kolenhok om voor Mouschi
het muizendeurtje open te maken. Na een lange inspectietocht raak ik dan in
Kralers appartement verzeild. Van Daan kijkt in alle laden en mappen, om de post
van de dag te zoeken. Peter haalt de magazijnsleutel en Mof ; Pim sleept
schrijfmachines naar boven; Margot zoekt een rustig plekje om haar kantoorwerk
te maken; mevrouw zet een ketel water op het gasstel; moeder komt met een pan
aardappels de trap af; ieder weet zijn eigen werk.

Al gauw komt Peter uit het magazijn weerom. De eerste vraag geldt het brood. Dit
wordt door de dames altijd in de keukenkast gelegd, maar daar is het niet. Dus
vergeten? Peter wil in het voorkantoor zoeken. Hij maakt zich voor de deur van
dit kantoor zo klein mogelijk en kruipt op handen en voeten, om niet van
buitenaf gezien te worden, naar de stalen kast, pakt het brood, dat daar
opgeborgen lag en verdwijnt, tenminste, hij wil verdwijnen, maar voor ie goed
beseft wat er gebeurd is, is Mouschi over hem heengesprongen en helemaal onder
de schrijftafel gaan zitten.

Peter zoekt naar alle kanten, ha, daar ziet-ie de kat, weer kruipt hij het
kantoor binnen en trekt het dier aan de staart. Mouschi blaast, Peter zucht. Wat
heeft hij bereikt? Mouschi zit nu helemaal bij het raam en likt zich, heel
tevreden aan Peter ontsnapt te zijn. Nu houdt Peter als laatste lokmiddel~de kat
een stukje brood voor en jawel, Mouschi laat zich verleiden en de deur sluit
zich.

Ik heb het allemaal staan aankijken, door de reet van de deur. We werken door.
Tik, tik, tik. ... Drie keer geklopt, etenstijd!

Je Anne.

\section*{Maandag, 23 Augustus 1943}

Lieve Kitty,\\
Vervolg van de Achterhuis-dagindeling: als 's morgens de klok
half negen slaat! Margot en moeder zijn zenuwachtig: `Sst ...  vader, stil Otto,
sst ... Pim! Het is half~negen; kom nu hier, je kunt niet meer het water laten
lopen; loop zachtjes!' Dit zijn de diverse uitroepen voor vader in de badkamer.
Klokslag half negen moet hij in de kamer zijn. Geen druppel water, geen W.C.,
niet lopen, alles stil. Als er nog niemand van het kantoorpersoneel aanwezig is,
is alles voor het magazijn nog gehoriger.

Boven wordt om tien voor half negen de deur geopend en kort daarop drie tikjes
op de grond: de pap voor Anne. Ik klauter de trap op en het honde-schoteltje
wordt afgehaald.

Beneden in mijn kamer gekomen, gaat alles gauw, gauw: haar doen, klaterbus weg,
bed op zijn plaats. Stil, de klok slaat! Mevrouw boven verwisselt van schoenen
en sloft op badsloffen door de kamer, mijnheer ook op sloffen; alles in rust.

Nu is het ideale familie-tafereel wel op zijn hoogtepunt. Ik wil lezen of leren,
Margot ook, evenals vader en moeder. Vader zit (natuurlijk met Dickens en het
woordenboek) op de rand van het uitgezakte piep-bed, waar niet eens behoorlijke
matrassen in liggen: twee peluwen boven op elkaar doen ook dienst. `Moet ik niet
hebben, gaat ook zonder!'

Eenmaal aan het lezen kijkt hij niet op of om, lacht af en toe, doet vreselijke
moeite moeder een verhaaltje op te dringen. Antwoord: `Ik heb nu geen tijd'.
Even kijkt-ie teleurgesteld, dan leest-ie weer door; even later, als er weer
iets typisch leuks komt, probeert hij het weer: `Dit moet je lezen, moeder!'

Moeder zit op het opklapbed, leest, naait, breit of leert, net wat er aan de
beurt is. Opeens valt haar iets in. Gauw even zeggen: `Anne, weet je wel ...
Margot, schrijf even op ...!'

Na een poosje is de rust weer teruggekeerd. Met een klap slaat Margot haar boek
dicht, vader knijpt zijn wenkbrauwen in een grappig boogje, zijn leesrimpeltje
plooit zich opnieuw en hij is weer verdiept; moeder begint met Margot te
kwebbelen, ik word nieuwsgierig, luister ook! Pim wordt in de zaak betrokken ...
Negen uur! Ontbijt!

Je Anne.

\section*{Vrijdag, 10 September 1943}

Lieve Kitty,\\
Elke keer als ik aan je schrijf is er weer iets bijzonders
gebeurd, maar meestal zijn~het meer onaangename dan aangename dingen. Nu echter
is er iets moois aan de gang. Woensdagavond 8 September j.l. zaten we aan de
radio van zeven uur en het eerste wat we hoorden was: `Here follows the best
news of the whole war. Italy had capitulated!' Italië onvoorwaardelijk
gecapituleerd! Om kwart over acht begon de Oranjezender: `Luisteraars, een uur
geleden, toen ik juist klaar was met de kroniek van de dag, kwam het heerlijke
bericht van de capitulatie van Italië binnen. Ik kan zeggen, dat ik nog nooit
met zoveel genoegen mijn papier in de prullemand gedeponeerd heb als vandaag!'
God save the King, het Amerikaanse volkslied en de Internationale werden
gespeeld. Als steeds was de Oranjezender hartverheffend en toch niet te
optimistisch.

Toch hebben we ook narigheid, het gaat om mijnheer Koophuis. Je weet, we houden
allemaal erg veel van hem en hoewel hij altijd ziek is, veel pijn heeft en niet
veel eten en lopen mag, is hij altijd opgewekt en bewonderenswaardig moedig.
`Als mijnheer Koophuis binnenkomt, gaat de zon op', zei moeder pas geleden en
daarmee heeft ze groot gelijk.

Nu moest hij voor een zeer onaangename darmoperatie naar het ziekenhuis en zal
daar minstens vier weken moeten blijven. Je had eens moeten zien hoe hij
afscheid van ons genomen heeft, alsof hij een boodschap ging doen, zo gewoon.

Je Anne.

\section*{Donderdag, 16 September 1943}

Lieve Kitty,\\
De onderlinge verhouding hier wordt hoe langer hoe slechter. Aan
tafel durft~niemand zijn mond open te doen (behalve om er een hap in te laten
glijden), want wat je zegt wordt òf kwalijk genomen òf verkeerd opgevat. Ik slik
elke dag Valeriaantjes tegen angst en depressie, maar dat verhoedt toch niet dat
mijn stemming de volgende dag nog miserabeler is. Eens goed en hard lachen zou
beter helpen dan tien Valeriaantjes, maar lachen zijn wij haast verleerd. Soms
ben ik bang, dat ik van de ernstigheid een uitgestreken gezicht en neerhangende
mond zal krijgen. Met de anderen is het niet beter, allen kijken met bange
voorgevoelens tegen het grote rotsblok, dat winter heet, op.

Nog een feit, dat ons niet opkikkert, is dat de magazijnman v.M.  achterdochtig
wordt aangaande het Achterhuis. Het zou ons niets kunnen schelen wat v.M. van de
toestand denkt, als de man niet een hoge mate van nieuwsgierigheid bezat en zich
niet met een kluitje in het riet laat sturen en bovendien niet betrouwbaar is.

Op een dag wou Kraler eens extra voorzichtig zijn, deed tien minuten voor één
zijn jas aan en ging naar de drogisterij~om de hoek. Nog geen vijf minuten later
was hij weer terug, sloop als een dief de steile trap op, die direct naar boven
leidt en kwam bij ons. Om kwart over één wilde hij weer weggaan, maar kwam Elli
tegen die hem waarschuwde dat v.M. op kantoor zat. Kraler maakte rechtsomkeert
en zat tot half twee bij ons.  Dan nam hij zijn schoenen in zijn hand en liep op
kousenvoeten naar de deur van de voorzolder, ging dan voetje voor voetje de trap
af en na een heel kwartier op de trap gebalanceerd te hebben om het kraken te
voorkomen, belandde hij vanaf de straatzijde op kantoor.

Elli intussen, even van v.M. bevrijd, kwam mijnheer Kraler bij ons afhalen, maar
deze was al lang weg, die zat nog op kousenvoeten op de trap. Wat zullen de
mensen op straat wel gedacht hebben, toen de directeur zijn schoenen buiten weer
aandeed? Ha, die directeur op sokken!

Je Anne.

\section*{Woensdag, 29 September 1943}

Lieve Kitty,\\
Mevrouw Van Daan is jarig. We hebben haar, behalve een kaas-,
vlees- en broodbon alleen nog een potje jam cadeau gegeven. Van haar man, Dussel
en onze beschermers heeft ze ook uitsluitend bloemen gekregen of etenswaren. Zo
zijn de tijden nu eenmaal!

Elli had van de week een halve zenuwuitbarsting, zo vaak werd ze er op
uitgestuurd, steeds werd er op aangedrongen dat ze iets gauw moest halen, dat ze
nog eens moest lopen of dat ze het verkeerd gedaan had.  Als je dan bedenkt, dat
ze beneden op kantoor haar werk af moet hebben, Koophuis ziek is, Miep thuis met
een verkoudheid, zijzelf haar enkel verstuikt, liefdesverdriet en een
mopperenden vader heeft, dan kun je je wel indenken, dat ze met haar handen in
het haar zit. Wij hebben haar getroost en gezegd, dat als ze een paar keer
energiek zou beweren dat ze geen tijd~heeft, die boodschappenlijstjes vanzelf
wel kleiner zouden worden.

Met Van Daan gaat het weer mis, ik zie het al aankomen! Vader is om de een
of~andere reden erg woedend. O, welke uitbarsting hangt ons nu weer boven het
hoofd? Als ik maar niet zo nauw met al die schermutselingen verbonden was, als
ik maar weg kon gaan. Ze maken ons nog gek!

Je Anne.

\section*{Zondag, 17 October 1943}

Lieve Kitty,\\
Koophuis is weer terug, gelukkig! Hij ziet nog wat bleekjes maar
gaat toch~welgemoed op straat om voor Van Daan kleren te verkopen. Het is een
naar feit dat het geld van de Van Daans radicaal op is. Mevrouw wil van haar
stapel jassen, jurken en schoenen niets missen, mijnheer zijn pak is moeilijk
van de hand te doen, omdat hij te hoge prijzen vraagt. Het slot van het liedje
is nog niet in zicht. Mevrouw zal haar bontjas wel moeten afstaan. Boven hebben
ze dienaangaande een knallende ruzie achter de rug en nu is de
verzoeningsperiode van `och, lieve Putti' en `schattige Kerli' ingetreden.

Ik duizel van de scheldwoorden die in de laatste maand door dit eerbare huis
rondgevlogen zijn. Vader loopt met opeengeperste lippen rond; als iemand hem
roept kijkt hij zo schichtig op, alsof hij bang is opnieuw een precaire zaak te
moeten opknappen. Moeder heeft rode plekjes van opgewondenheid op haar wangen.
Margot klaagt over hoofdpijn, Dussel kan niet slapen, mevrouw klaagt de hele dag
en ikzelf ben de kluts helemaal kwijt. Eerlijk gezegd vergeet ik af en toe met
wie we ruzie hebben en met welke persoon de verzoening al heeft plaats gehad.

Het enige wat afleidt is leren en dat doe ik veel. Je Anne.

\section*{Vrijdag, 29 October 1943}

Lieve Kitty,\\
Hier waren weer klinkende ruzie's tussen mijnheer en mevrouw. Het
kwam zo:~zoals ik al schreef is het geld van de Van Daans op. Op een dag, al een
tijd geleden, sprak Koophuis over een bevrienden bontwerker; daardoor kwam
mijnheer op het idee nu toch mevrouws bontjas te verkopen. Het is een bontjas
uit konijnenvellen gemaakt en zeventien jaar lang gedragen. ƒ 325. - kreeg hij
er voor, dit is een enorm bedrag.  Mevrouw wilde het geld echter houden om na de
oorlog nieuwe kleren te kopen en het had heel wat voeten in de aarde, eer
mijnheer haar duidelijk gemaakt had, dat het geld in de huishouding dringend
nodig was.

Dat gegil, geschreeuw, gestamp en gescheld kun je je onmogelijk voorstellen. Het
was angstwekkend. Mijn familie stond met ingehouden adem onder aan de trap, om
zo nodig de vechtenden uit elkaar te houden.  Al dat gegil, gehuil en die
zenuwachtigheid zijn zo spannend en inspannend, dat ik 's avonds huilend in mijn
bed val en de hemel dank dat ik eens een half uurtje voor me alleen heb.

Mijnheer Koophuis is weer afwezig; zijn maag laat hem niet met rust. Hij weet
zelf nog niet eens of het bloed al gestelpt is. Hij was voor de eerste keer erg
down toen hij ons vertelde, dat hij zich niet goed voelde en naar huis ging.

Met mij gaat het in het algemeen goed, behalve dat ik helemaal geen trek heb.
Steeds weer hoor ik: `Wat zie je er toch slecht uit'. Ik moet zeggen, dat ze
zich erg uitsloven om me een beetje op peil te houden.  Druivensuiker,
levertraan, gisttabletten en kalk moeten er aan te pas komen.

Mijn zenuwen ben ik lang niet altijd de baas, vooral 's Zondags voel ik me er
ellendig aan toe. Dan is de stemming in huis drukkend, slaperig en loodzwaar.
Buiten hoor je geen vogel zingen, een doodse en benauwende stilte hangt over
alles heen en dit zware klemt zich aan mij vast, alsof ik mee moest naar een
diepe onderwereld.

Vader, moeder en Margot laten me dan bij tijd en wijle onverschillig, ik dwaal
van de ene naar de andere kamer, de trap af en weer op en heb een gevoel als een
zangvogel, wiens vleugels hardhandig uitgerukt zijn en die in een volslagen
duisternis tegen de spijlen van zijn nauwe kooi aanvliegt. `Naar buiten, lucht
en lachen', schreeuwt het in me. Ik antwoord niet eens meer, ga op een divan
liggen en slaap om de tijd, de stilte, de verschrikkelijke angst ook, te
verkorten, want te doden zijn ze niet.

Je Anne.

\section*{Woensdag, 3 November 1943}

Lieve Kitty,\\
Om ons wat afleiding en ontwikkeling te bezorgen heeft vader een
prospectus van~de Leidse Onderwijs Instelling aangevraagd.  Margot neusde het
dikke boek wel drie keer door, zonder dat ze iets naar haar gading en haar beurs
vond. Vader was vlugger om iets te vinden, hij wilde naar de instelling laten
schrijven en om een proefles `Elementair Latijn' vragen. Zo gezegd, zo gedaan,
de les kwam, Margot trok enthousiast aan het werk en de cursus werd genomen.
Voor mij is hij veel te moeilijk, hoewel ik erg graag Latijn zou leren.

Om mij eveneens aan iets nieuws te laten beginnen vroeg vader aan Koophuis naar
een kinderbijbel, om eindelijk eens iets van het nieuwe testament te weten te
komen. `Wil je Anne voor Chanuka een bijbel geven?' vroeg Margot wat ontdaan.'
`Ja ... eh, ik denk dat Sint Nicolaas een betere gelegenheid is', antwoordde
vader. `Jezus past nu eenmaal niet op Chanuka'.

Je Anne.

\section*{Maandagavond, 8 November 1943}

Lieve Kitty,\\
Als je mijn brievenstapeltje achter elkaar door zou lezen,~dan
zou het je zeker opvallen in wat voor verschillende stemmingen mijn brieven
geschreven zijn. Ik vind het zelf vervelend dat ik hier in het Achterhuis zo erg
van stemmingen afhankelijk ben, dat ben ik trouwens niet alleen, dat zijn wij
allemaal. Als ik een boek lees, dat indruk op me maakt, moet ik in mezelf
grondig orde scheppen, alvorens me onder de mensen te begeven, anders zou men
van mij denken dat ik een wat rare geest had. Op het ogenblik, zoals je wel zult
merken, heb ik een periode, waarin ik neerslachtig ben. Ik zou je echt niet
kunnen zeggen hoe ik zo kom, maar ik geloof dat het mijn lafheid is, waar ik
telkens weer tegen op bots.

Vanavond, toen Elli er nog was, werd er lang, hard en doordringend gebeld. Op
dat ogenblik werd ik wit, kreeg buikpijn en hartkloppingen en dat allemaal van
de angst.

's Avonds in bed zie ik me in een kerker alleen, zonder vader en moeder.  Soms
zwerf ik aan de weg, of ons Achterhuis staat in brand, of ze komen ons 's nachts
weghalen. Ik zie alles, alsof ik het aan mijn eigen lijf beleef en heb daarbij
het gevoel dat dit alles me dadelijk zal kunnen overkomen!

Miep zegt wel vaak, dat ze ons benijdt, omdat we hier rust hebben. Dat kan best
waar zijn, maar aan onze angst denkt ze zeker niet. Ik kan me helemaal niet
voorstellen dat de wereld voor ons ooit weer eens gewoon wordt. Ik spreek wel
over `na de oorlog', maar dan is dat, alsof ik over een luchtkasteeltje spreek,
iets dat nooit werkelijkheid kan worden. Aan ons vroeger thuis, de vriendinnen,
schoolpretjes, aan dat alles denk ik als aan iets, dat een ander dan ikzelf
beleefd heeft.

Ik zie ons achten samen met het Achterhuis, alsof wij een stukje blauwe hemel
waren, omringd door zware, zwarte regenwolken. Het ronde, afgebakende plekje,
waarop wij staan is nog veilig, maar de wolken rukken steeds dichter op ons toe
en de ring, die ons van het naderende gevaar scheidt wordt steeds nauwer
toegehaald. Nu zijn we al zover~door gevaar en donkerte omgeven, dat wij van
vertwijfeling, waar uitredding te vinden, tegen elkaar aanbotsen. We kijken
allen naar beneden waar de mensen tegen elkaar vechten, we kijken allen naar
boven, waar het rustig en mooi is en onderwijl zijn we afgesneden door die
duistere massa, die ons niet naar boven laat gaan, maar die voor ons staat als
een ondoordringbare muur, die ons verpletteren wil, maar nog niet kan. Ik kan
niets anders doen dan roepen en smeken: `O ring, ring wordt wijder en open je
voor ons!'

Je Anne.

\section*{Donderdag, 11 November 1943}

Lieve Kitty,\\
Ik heb net een goede titel voor dit hoofdstuk:

Ode aan mijn Vulpen. In Memoriam.

Mijn vulpen was voor mij altijd een kostbaar bezit; ik waardeerde haar hogelijk,
vooral wegens de dikke punt die zij had, want ik kan alleen met dikke
vulpenpunten werkelijk netjes schrijven. Mijn vulpen heeft een zeer lang en
interessant vulpen-leven gehad, dat ik in het kort wil meedelen.

Toen ik 9 jaar oud was, kwam mijn vulpen in een pakje (in watten gewikkeld) als
`monster zonder waarde' helemaal uit Aken, de woonplaats van mijn grootmoeder,
de goede geefster. Ik lag in bed met griep, terwijl de Februariwind om het huis
gierde. De glorierijke vulpen had een rood leren étuitje om zich heen en werd
dadelijk aan al mijn vriendinnen getoond. Ik, Anne Frank, de trotse bezitster
van een vulpen.

Toen ik 10 jaar was, mocht ik de pen mee naar school nemen en de juffrouw stond
zowaar toe, dat ik er mee schreef.

Met 11 moest mijn schat echter weer opgeborgen worden, daar de juffrouw van de
zesde klas alleen schoolpennen en inktpotjes als schrijfgerei toestond.

Toen ik 12 werd en naar het Joodse Lyceum ging, kreeg~mijn vulpen ter meerdere
ere een nieuw étui, waar een potlood bij kon en dat bovendien veel echter stond,
daar het met een ritssluiting sloot.

Met 13 ging de vulpen mee naar het Achterhuis, waar zij met mij door talloze
dagboeken en geschriften is gerend.

Toen ik 14 jaar oud was, was dat het laatste jaar dat mijn vulpen met mij
voltooid had, en nu ...

Het was op Vrijdagmiddag na vijf uur, dat ik uit mijn kamertje gekomen aan tafel
wilde gaan zitten om te schrijven, toen ik hardhandig opzij geduwd werd en
plaats moest maken voor Margot en vader die hun `Latijn' oefenden. De vulpen
bleef ongebruikt op tafel liggen, en haar bezitster nam zuchtend genoegen met
een heel klein hoekje tafel en ging boontjes wrijven. `Boontjes wrijven' is hier
beschimmelde bruine bonen weer in hun fatsoen brengen.

Om kwart voor zes ging ik de grond vegen en gooide het vuil tezamen met de
slechte bonen op een krant enfin de kachel. Een geweldige vlam sloeg er uit en
ik vond het prachtig, dat op die manier de kachel, die op apegapen gelegen had,
zich herstelde. De rust was weergekeerd, de Latijners opgehoepeld en ik ging aan
tafel zitten om mijn voorgenomen schrijfwerk op te nemen, maar, waar ik ook
zocht, mijn vulpen was nergens te bekennen. Ik zocht nog eens, Margot zocht,
moeder zocht, vader zocht, Dussel zocht, maar het ding was spoorloos verdwenen.

`Misschien is zij wel in de kachel gevallen, tegelijk met de bonen', opperde
Margot. `Ach, wel neen, kind!' antwoordde ik.

Toen mijn vulpen echter 's avonds nog niet te voorschijn wou komen, namen we
allen aan, dat zij verbrand was, temeer daar celluloid reusachtig brandt.

En werkelijk, de droeve verwachting werd bevestigd, toen vader de andere morgen
bij het kachel-uithalen het clipje, waarmee je een vulpen vaststeekt, te midden
van een~lading as terugvond. Van de gouden pen was niets meer te vinden. `Zeker
vastgebakken in de een of andere steen', meende vader.

Eén troost is mij gebleven, al is hij maar schraal: mijn vulpen is gecremeerd,
net wat ik later zo graag wil!

Je Anne.

\section*{Woensdag, 17 November 1943}

Lieve Kitty,\\
Huisschokkende gebeurtenissen aan de gang.\\
Bij Elli thuis
heerst diphterie, daarom mag ze zes weken lang niet met ons in~aanraking komen.
Zowel met eten als met boodschappen doen is het erg lastig, om van de
ongezelligheid nog maar niet te spreken. Koophuis ligt nog steeds en heeft al
drie weken lang niets anders dan melk en pap gehad. Kraler heeft het razend
druk.

Margots ingestuurde Latijnse lessen worden door een leraar gecorrigeerd
teruggezonden. Margot schrijft onder Elli's naam, de leraar is erg aardig en
geestig bovendien. Hij is zeker blij, dat hij zo'n knappe leerlinge gekregen
heeft.

Dussel is helemaal in de war, wij weten geen van allen waarom. Het is er mee
begonnen, dat hij boven zijn mond dichtkneep en noch met mijnheer noch met
mevrouw een woord sprak. Dit viel iedereen op en toen het een paar dagen
aanhield nam moeder de gelegenheid waar om hem te waarschuwen voor mevrouw, die
hem inderdaad nog veel onaangenaamheden zou kunnen bezorgen, als hij zo
doorging.

Dussel zei, dat mijnheer Van Daan met zwijgen begonnen was en hij was dan ook
niet van plan zijn zwijgen te verbreken.

Nu moet je weten, dat het gisteren 16 November was, de dag, waarop hij een jaar
in het Achterhuis is. Moeder werd ter gelegenheid daarvan met een potje bloemen
vereerd, maar mevrouw Van Daan, die al weken van te voren meermalen op deze
datum een toespeling had gemaakt en haar~mening, dat Dussel moest tracteren,
niet onder stoelen of banken stak, kreeg niets. In plaats van voor het eerst
zijn dank voor het onbaatzuchtige opnemen te fluiten,~sprak hij helemaal niet.
En toen ik hem op de morgen van de 16de vroeg, of ik hem feliciteren of
condoleren moest, antwoordde hij, dat hij alles in ontvangst nam. Moeder, die
voor de mooie rol van vredesstichtster wilde fungeren, kwam geen stap met hem
verder en uiteindelijk bleef de toestand gelijk.

`Der Mann hat einen grossen Geist Und ist so klein von Taten!'

Je Anne.

\section*{Zaterdag, 27 November 1943}

Lieve Kitty,\\
Gisteravond vóór het ins'apen kwam opeens voor mijn ogen: Lies.\\
Ik zag haar voor mij, in lompen gekleed met een vervallen en vermagerd gezicht.

Haar ogen waren heel groot en zij keek mij zo droevig en verwijtend aan, dat ik
in haar ogen kon lezen: `O Anne, waarom heb je me verlaten? Help, o help mij,
red mij uit deze hel!'

En ik kan haar niet helpen, ik kan slechts toekijken hoe andere mensen lijden en
sterven en kan God slechts bidden om haar terug te voeren naar ons.

Juist Lies zag ik, geen ander en ik begreep het. Ik heb haar verkeerd
beoordeeld, was te veel kind om haar moeilijkheden te begrijpen. Zij was gehecht
aan haar nieuwe vriendin en het leek, alsof ik haar die wilde ontnemen. Hoe moet
de arme zich gevoeld hebben, ik weet het, ik ken zelf dat gevoel zo goed!

Soms, in een flits, zag ik iets van haar leven, om egoïstisch direct weer in
eigen genoegens en moeilijkheden op te gaan. Het was lelijk van me zoals ik met
haar gehandeld heb en nu keek ze me met haar bleke gezicht en smekende ogen o zo
hulpeloos aan. Kon ik haar maar helpen!

O God, dat ik hier alles heb wat ik me maar wensen kan en dat zij door het harde
noodlot zo aangepakt is. Zij was minstens zo vroom als ik, zij wilde ook het
goede, waarom werd ik dan uitverkoren om te leven en moet zij wellicht sterven?
Welk verschil was er tussen ons? Waarom zijn wij nu zo ver van elkander?

Eerlijk gezegd, heb ik haar maandenlang, ja een jaar haast vergeten.  Niet
helemaal, maar toch heb ik nooit zo aan haar gedacht, dat ik haar in al haar
ellende voor me zag.

Ach Lies, ik hoop dat ik je, als je het einde van de oorlog nog zou beleven en
bij ons terug zou komen, opnemen kan, om je iets te vergoeden van wat er tegen
je misdaan is.

Maar als ik weer in staat ben om haar te helpen, dan heeft zij mijn hulp niet zo
nodig als nu. Zou zij nog een keer aan mij denken en wat zal zij dan voelen?

Goede God, geef haar steun, opdat ze tenminste niet alleen is. O, kon jij haar
maar zeggen, dat ik met liefde en medelijden aan haar denk; het zou haar
misschien in haar uithoudingsvermogen sterken.

Ik mag niet verder denken, want ik kom er niet uit. Ik zie steeds weer haar
grote ogen, die me niet loslaten. Zou Lies werkelijk in zichzelf geloof hebben,
zou zij het geloof niet alleen van buiten opgedrongen gekregen hebben?

Ik weet dat niet eens, nooit heb ik me de moeite genomen om het haar te vragen.

Lies, Lies, kon ik je maar weghalen van waar je nu bent, kon ik je maar laten
delen in alles wat ik geniet. Het is te laat, ik kan niet meer helpen en niet
meer herstellen, wat ik verkeerd gedaan heb. Maar ik zal haar nooit meer
vergeten en altijd voor haar bidden.

Je Anne.

\section*{Maandag, 6 December 1943}

Lieve Kitty,\\
Toen Sinterklaas naderde, dachten we allemaal onwille-keurig aan
de leuk opgemaakte mand van verleden jaar en vooral mij leek het vervelend om
dit jaar alles over te slaan. Ik dacht lang na, totdat ik iets gevonden had,
iets grappigs.

Pim werd geraadpleegd en een week geleden gingen we aan het werk om voor alle
acht een versje te maken.

Zondagavond om kwart over acht verschenen we boven met de grote wasmand tussen
ons in, die versierd was met guurtjes en strikken van rose en blauw
doorslagpapier. Over de mand lag een groot stuk pakpapier heen, waarop een
briefje was bevestigd. Allen waren enigszins verbaasd over de omvang van de
verrassing.

Ik nam het briefje van het papier en las:

\section*{Proloog}

Sinterklaas is ook dit jaar weder gekomen, Zelfs het Achterhuis heeft het
vernomen Helaas, kunnen we het niet zo leuk begaan Als het verleden jaar was
gedaan.

Toen waren we hoopvol en dachten beslist Dat zegevieren zou de optimist\\
En
meenden dat dit jaar vrij\\
Sint Nicolaas te vieren zij.

Toch willen wij deez' dag gedenken, Maar daar nu niets meer is te schenken, Zo
moeten wij iets anders doen:\\
Een ieder kijke in zijn schoen!

(terwijl vader en ik het pakpapier oplichtten).\\
Een daverend gelach volgde,
toen iedere eigenaar zijn schoen uit de mand haalde.

In elke schoen zat een klein pakje in papier gewikkeld met het adres van den
schoeneigenaar.

Je Anne.

\section*{Woensdag, 22 December 1943}

Lieve Kitty,\\
Een zware griep heeft me verhinderd je eerder dan vandaag te
schrijven. Het is~ellendig om hier ziek te zijn. Als ik moest~hoesten kroop ik
één, twee, drie onder de dekens en probeerde mijn keel zo zacht mogelijk tot
stilte te brengen, wat meestal ten gevolge had, dat de kriebel helemaal niet
meer wegging en melk met honing, suiker of pastilles er aan te pas moesten
komen. Als ik aan de kuren denk die ze me door hebben laten maken, dwarrelt het
me. Zweten, omslagen, natte borstlappen, droge borstlappen, heet drinkken,
gorgelen, penselen, stil liggen, warmte-kussen, kruiken, citroenkwast en daarbij
om de twee uur de thermometer.

Kan men op zo'n manier eigenlijk beter worden? Het ergste vond ik wel, als
mijnheer Dussel doktertje ging spelen en met zijn pomadehoofd op mijn blote
borst ging liggen, om de geluiden daarbinnen af te luisteren.  Niet alleen dat
zijn haar me verschrikkelijk kriebelde, ik geneerde me ondanks het feit, dat hij
eens dertig jaar geleden gestudeerd en de artsentitel heeft. Wat heeft die vent
aan mijn hart te gaan liggen? Hij is mijn geliefde toch niet! Trouwens, wat
daarbinnen gezond of niet gezond is, hoort hij toch niet, zijn oren moeten eerst
uitgespoten worden, daar hij angstig veel op een hardhorende gaat lijken.

Maar nu genoeg over de ziekte. Ik ben weer kiplekker, een centimeter gegroeid,
twee pond aangekomen, bleek en leerlustig.

Veel nieuws is er niet te berichten. Bij uitzondering is de verstandhouding hier
goed, niemand heeft ruzie, we hebben zo'n huisvrede zeker een half jaar niet
gehad. Elli is nog steeds van ons gescheiden.

We krijgen voor Kerstmis extra olie, snoep en stroop; het `cadeau' is een
broche, gefabriceerd uit een plak van twee en een halve cent en dit netjes
glimmend. En n, niet uit te leggen prachtig. Mijnheer Dussel heeft aan moeder en
mevrouw Van Daan een mooie taart cadeau gegeven, die Miep op zijn verzoek
gebakken heeft. Bij al het werk moet Miep ook dat nog doen! Ik heb ook wat voor
Miep en Elli. Wel twee maanden lang heb ik namelijk de suiker van mijn~pap
opgespaard en zal daarvan, door bemiddeling van mijnheer Koophuis, borstplaatjes
laten maken.

Het weer is druilerig, de kachel stinkt, het eten drukt zwaar op aller magen wat
aan alle kanten donderende geluiden veroorzaakt; oorlogs-stilstand, rotstemming.

Je Anne.

\section*{Vrijdag, 24 December 1943}

Lieve Kitty,\\
Ik heb al meer geschreven, dat we hier allemaal met stemmingen te
doen hebben~en ik geloof, dat deze kwaal bij mij vooral in de laatste tijd sterk
toeneemt. `Himmelhoch jauchzend und zum Tode betrübt' is hier zeker wel van
toepassing. `Himmelhoch jauchzend' ben ik, als ik er aan denk hoe goed we het
hier hebben en mij vergelijk met andere Joodse kinderen en `zum Tode betrübt'
overvalt me, wanneer bijvoorbeeld, zoals vandaag, mevrouw Koophuis hier geweest
is en vertelt van haar dochter Corry's hockeyclub, kanovaarten,
toneelopvoeringen en vrienden. Ik geloof niet, dat ik jaloers op Corry ben, maar
wel krijg ik dan zo'n sterk verlangen om ook~eens flink pret te maken en te
lachen tot ik er buikpijn van krijg. Vooral nu in de winter met al die vrije
Kerst- en Nieuwjaarsdagen, dan zitten we hier zo als uitgestotenen en toch mag
ik deze woorden eigenlijk niet opschrijven, omdat ik dan ondankbaar lijk en het
is ook wel overdreven. Maar wat je ook van me mag denken, ik kan niet alles voor
mezelf bewaren en haal dan nog maar mijn beginwoorden aan: `Papier is geduldig'.

Als iemand net van buiten komt, met de wind in zijn kleren en de kou op zijn
gezicht, dan zou ik wel mijn hoofd onder de dekens willen stoppen om niet te
denken: `Wanneer is het ons weer gegund lucht te ruiken?' En omdat ik mijn hoofd
niet in dekens mag verbergen, het integendeel rechtop en flink moet houden,
komen de gedachten toch, niet één keer maar och, ontelbare malen. Geloof me,
als~je anderhalf jaar opgesloten zit, dan kan het je op sommige dagen eens
teveel worden. Alle rechtvaardigheid of dankbaarheid ten spijt; gevoelens laten
zich niet verdringen. Fietsen, dansen, fluiten, de wereld inkijken, me jong
voelen, weten dat ik vrij ben, daar snak ik naar en toch mag ik het niet laten
merken, want denk eens aan, als we alle acht ons gingen beklagen of ontevreden
gezichten zetten, waar moet dat naar toe? Ik vraag mezelf wel eens af: `Zou
iemand me hierin begrijpen, heenzien over Jood of niet-Jood, alleen maar in me
zien de bakvis, die zo'n behoefte heeft aan uitgelaten pret?' Ik weet het niet,
en zou er ook met niemand over kunnen spreken, want ik weet dat ik dan ga
huilen. Huilen kan zo'n verlichting brengen.

Ondanks theorieën en moeiten mis ik elke dag de echte moeder die me begrijpt.
Daarom ook denk ik bij alles wat ik doe en wat ik schrijf, dat ik later voor
mijn kinderen wil zijn de `mams' die ik me voorstel. De mams die niet alles zo
ernstig opvat, wat er zo gezegd wordt en wel ernstig, wat van mij komt. Ik merk,
ik kan het niet beschrijven, maar het woord mams zegt al alles. Weet je wat ik
er op gevonden heb om toch zoiets als mams tegen mijn moeder te zeggen? Ik noem
haar vaak `mansa' en daarvan komt dan `mans'. Het is als het ware de onvolkomen
mams, die ik zo graag met nog een pootje aan de `n' zou vereren, maar die dit
niet beseft. Het is gelukkig zo, want zij zou er zich ongelukkig over maken.

Nu genoeg daarover, mijn `zum Tode betrübt' is bij het schrijven een beetje
overgegaan.

Je Anne.

\section*{Zaterdag, 25 December 1943}

Lieve Kitty,\\
In deze dagen, nu Kerstmis nog maar één dag af is, moet ik aldoor
aan Pim denken~en wat hij mij verleden jaar omtrent zijn jeugdliefde verteld
heeft. Verleden jaar begreep ik de~betekenis van zijn woorden niet zo goed, als
ik ze nu begrijp. Sprak hij nog maar een keer, misschien zou ik hem dan kunnen
aantonen dat ik hem begreep.

Ik geloof dat Pim daarover gesproken heeft, omdat hij die `zoveel hartsgeheimen
van anderen weet', zich ook één keer moest fluiten; want Pim zegt anders nooit
wat over zichzelf en ik geloof ook niet, dat Margot vermoedt wat Pim allemaal
heeft moeten doormaken. De arme Pim, hij kan mij niet wijsmaken, dat hij alles
vergeten is. Nooit zal hij dit vergeten. Hij is toegevend geworden. Ik hoop, dat
ik een beetje op hem ga lijken, zonder dat ik dat ook allemaal moet doormaken.

Je Anne.

\section*{Maandag, 27 December 1943}

Lieve Kitty,\\
Vrijdagavond heb ik voor het eerst van mijn leven iets voor
Kerstmis gekregen.

De meisjes, Koophuis en Kraler hadden weer een heerlijke verrassing voorbereid.
Miep heeft een Kerstkoek gebakken, waar `Vrede 1944' op stond. Een pond
boterkoekjes van vooroorlogse kwaliteit verzorgde Elli.  Voor Peter, Margot en
mij was er een flesje yoghurt en voor de volwassenen elk een flesje bier. Alles
was weer zo leuk ingepakt en plaatjes waren op de verschillende pakjes geplakt.
De Kerstdagen zijn voor ons verder gauw voorbijgegaan.

Je Anne.

\section*{Woensdag, 29 December 1943}

Lieve Kitty,\\
Gisteravond was ik weer erg verdrietig. Oma en Lies kwamen voor
mijn geest.

Oma, o die lieve Oma, hoe weinig hebben wij begrepen wat zij geleden heeft, hoe
lief was ze. En daarbij bewaarde ze steeds zorgvuldig het vreselijke geheim, dat
ze met zich meedroeg.

Hoe trouw en goed was Oma altijd, geen van ons zou zij ooit in de steek gelaten
hebben. Wat het ook was, hoe stout ik ook geweest was, Oma verontschuldigde me
altijd.

Oma - heb jij van me gehouden of heb jij mij ook niet begrepen? Ik weet het
niet. Tegen Oma heeft ook nooit iemand wat van zichzelf gezegd. Hoe eenzaam moet
Oma geweest zijn, hoe eenzaam ondanks ons. Een mens kan eenzaam zijn ondanks de
liefde van velen, want voor niemand is hij toch `de liefste'. En Lies, leeft zij
nog? Wat doet zij? O God, bescherm haar en breng haar naar ons terug. Lies, aan
jou zie ik steeds hoe mijn lot ook had kunnen zijn, steeds zie ik mij in jouw
plaats. Waarom ben ik dan vaak nog verdrietig om wat hier gebeurt? Moet ik niet
altijd blij, tevreden en gelukkig zijn, behalve als ik aan haar en haar
lotgenoten denk?

Ik ben zelfzuchtig en laf. Waarom droom en denk ik altijd de ergste dingen en
zou ik het van angst wel willen uitgillen? Omdat ik toch nog, ondanks alles. God
niet genoeg vertrouw. Hij heeft mij zoveel gegeven, wat ik zeker niet verdiend
had en toch doe ik elke dag nog zoveel verkeerds.

Als men aan zijn naasten denkt moet men huilen, men kan eigenlijk wel de hele
dag huilen. Er blijft niets anders over dan te bidden, dat God een wonder laat
gebeuren en nog enigen van hen spaart. En ik hoop, dat ik dat voldoende doe!

Je Anne.

\section*{Zondag, 2 Januari 1944}

Lieve Kitty,\\
Toen ik vanmorgen niets te doen had, bladerde ik eens in mijn
dagboek en kwam~meermalen aan brieven, die het onderwerp `moeder' in zo driftige
woorden behandelden, dat ik, er van geschrokken, me afvroeg: `Anne, ben jij het
die over haat gesproken heeft? O Anne, hoe kon je dat!'

Ik bleef met de open bladzijde in de hand zitten en dacht er over na hoe het
kwam, dat ik zo boordevol woede en~werkelijk met zoiets als haat vervuld was,
dat ik alles aan jou moest toevertrouwen. Ik heb geprobeerd de Anne van een jaar
geleden te begrijpen en te verontschuldigen, want mijn geweten is niet zuiver,
zolang ik je met deze beschuldigingen laat zitten, zonder je nu achteraf te
verklaren hoe ik zo kwam.

Ik lijd en leed aan stemmingen die me (figuurlijk) met mijn hoofd onder water
hielden en de dingen, zoals ze waren, alleen subjectief lieten zien, zonder dat
ik probeerde rustig over de woorden van de tegenpartij na te denken en dan te
handelen in de geest van diegene, die ik in mijn opbruisende temperament
beledigd of verdriet gedaan heb.

Ik heb me in mezelf verstopt, alleen mezelf bekeken en al mijn vreugde, spot en
verdriet ongestoord in mijn dagboek opgepend. Dit dagboek heeft voor mij veel
waarde, omdat het vaak een memoiren boek is geworden, maar op vele bladzijden
zou ik wel `voorbij' kunnen zetten.

Ik was woedend op moeder, ben het soms nog. Zij begreep mij niet, dat is waar,
maar ik begreep haar ook niet. Daar zij wel van mij hield was zij teder, maar
daar zij ook in vele onaangename situaties door mij is gekomen, en dan daardoor
en door vele andere droevige omstandigheden zenuwachtig en geprikkeld was, is
het wel te begrijpen, dat zij mij afsnauwde.

Ik nam dit veel te ernstig op, was beledigd, brutaal en vervelend tegen haar,
wat haar op haar beurt weer verdrietig stemde. Het was dus eigenlijk een heen en
weer van onaangenaamheden en verdrietigheden.  Prettig was het voor ons allebei
zeker niet, maar het gaat voorbij.

Dat ik dit niet wilde inzien en veel medelijden met mezelf had is eveneens te
begrijpen. De zinnen die zo heftig zijn, zijn enkel uitingen van boosheid, die
ik in het gewone leven met een paar maal stampvoeten in een kamer achter slot,
of schelden achter moeders rug botgevierd zou hebben.

De periode, dat ik moeder in tranen veroordeel, is voorbij.

Ik ben wijzer geworden en moeders zenuwen zijn wat gekalmeerd. Ik houd meestal
mijn mond als ik me erger en zij doet dat eveneens en daardoor gaat het
ogenschijnlijk veel beter. Want van moeder zó echt houden met de aanhankelijke
liefde van een kind, dat kan ik niet, daarvoormis ik het gevoel.

Ik sus mijn geweten nu maar met de gedachte, dat scheldwoorden beter op papier
kunnen staan dan dat moeder ze moet meedragen in haar hart.

Je Anne.

\section*{Woensdag, 5 Januari 1944}

Lieve Kitty,\\
Vandaag moet ik je twee dingen bekennen, die heel wat tijd in
beslag zullen nemen,~maar die ik aan iemand~\emph{moet}~vertellen en dat kan ik
dan toch maar het best aan jou doen, omdat ik stellig weet, dat jij altijd en
onder alle omstandigheden zult zwijgen.

Het eerste is over moeder. Je weet dat ik veel geklaagd heb over moeder en dan
toch steeds weer moeite deed aardig tegen haar te zijn.  Plotseling is het me nu
duidelijk geworden wat er aan haar mankeert.  Moeder heeft ons zelf verteld, dat
ze ons meer als haar vriendinnen dan als haar dochters beschouwt. Dat is nu wel
heel mooi, maar toch kan een vriendin de plaats van een moeder niet vervangen.
Ik heb er behoefte aan mijn moeder als een voorbeeld te nemen en eerbied voor
haar te hebben.  Ik heb het gevoel, dat Margot in al deze dingen zo anders denkt
en wat ik je nu verteld heb nooit zou begrijpen. En vader ontwijkt alle
gesprekken, die over moeder zullen handelen.

Een moeder stel ik me voor als een vrouw, die in de eerste plaats veel tact aan
de dag legt, vooral voor haar kinderen die in onze leeftijd zijn en die niet
doet zoals mansa, die, als ik om iets huil, niet om pijn maar om andere dingen,
me uitlacht.

Eén ding, het mag misschien onbenullig lijken, heb ik haar~nooit vergeven. Het
was op een dag dat ik naar den tandarts moest. Moeder en Margot zouden meegaan
en vonden het goed, dat ik mijn fiets meenam. Toen we bij den tandarts klaar
waren en weer voor de deur stonden, vertelden Margot en moeder dat ze naar de
stad gingen, om iets te bekijken of te kopen, ik weet het niet meer zo heel
precies. Ik wilde meegaan, maar dat mocht niet, omdat ik mijn fiets bij me had.
Van woede sprongen me de tranen in de ogen en Margot en moeder begonnen me uit
te lachen. Toen werd ik zo woedend, dat ik op straat mijn tong tegen hen
uitstak, terwijl toevallig een klein vrouwtje langs kwam, dat heel verschrikt
keek. Ik reed op de fiets naar huis en heb zeker nog lang gehuild.

Het is eigenaardig, dat de wond, die moeder me toen toegebracht heeft, nog gaat
branden als ik er aan denk, hoe kwaad ik toen was.

Het tweede is iets dat me erg veel moeite kost om het je te vertellen, want het
gaat over mezelf.

Gisteren las ik een artikel van dra Sis Heyster, die over blozen schreef. Zij
spreekt in dit artikel net alsof het tegen mij alleen was.  Hoewel ik niet zo
gauw bloos, zijn die andere dingen die daarin staan, toch wel op mij van
toepassing. Zij schrijft zo ongeveer, dat een meisje in de puberteitsjaren stil
wordt in zichzelf en gaat nadenken over de wonderen, die er met haar lichaam
verricht worden.

Ook ik heb dat en daarom krijg ik de laatste tijd het gevoel, dat ik me ga
generen voor Margot, moeder en vader. Margot daarentegen is veel verlegener dan
ik en geneert zich helemaal niet.

Ik vind het zo wonderlijk, dat wat er met me gebeurt en niet alleen dat, wat aan
de uiterlijke kant van mijn lichaam te zien is, maar dat wat zich daarbinnen
voltrekt. Juist omdat ik over mezelf en over al deze dingen nooit met iemand
spreek, spreek ik met mezelf er over.

Telkens weer als ik ongesteld ben, en dat is nog maar~drie keer gebeurd, heb ik
het gevoel dat ik, ondanks alle pijn, narigheid en viezigheid, een zoet geheim
met me meedraag en daarom, al heb ik er niets dan last van, verheug ik me in
zekere zin altijd weer op de tijd, dat ik weer dat geheim in me zal voelen.

Verder schrijft Sis Heyster nog, dat jonge meisjes in die jaren niet geheel
zeker van zichzelf zijn en gaan ontdekken, dat zij zelf een mens zijn met
ideeën, gedachten en gewoonten. Ik ben, daar ik hier al kort na mijn 13de jaar
gekomen ben, vroeger dan andere meisjes begonnen met over mezelf na te denken en
te weten, dat ik een `mens op zichzelf' ben. Soms krijg ik 's avonds in bed een
verschrikkelijke behoefte om mijn borsten te bevoelen en te horen, hoe rustig en
zeker mijn hart slaat.

Onbewust heb ik dergelijke gevoelens al gehad, vóór ik hier kwam, want ik weet,
dat ik, toen ik eens bij een vriendin sliep, een sterke behoefte had haar te
zoenen en dat ik dat ook gedaan heb. Ik raak elke keer in extase, als ik een
naakt vrouwen guur zie, zoals bijvoorbeeld een Venus. Ik vind het soms zo
wonderlijk en mooi, dat ik me moet inhouden om mijn tranen niet te laten rollen.

Had ik maar een vriendin! Je Anne.

\section*{Donderdag, 6 Januari 1944}

Lieve Kitty,\\
Mijn verlangen om eens met iemand te praten werd zo groot, dat ik
het op de een of andere manier in mijn hoofd kreeg Peter daarvoor uit te
kiezen.\\
Als ik boven wel eens in Peters kamertje kwam, bij licht, vond ik het
daar altijd

erg gezellig, maar omdat Peter zo bescheiden is en nooit iemand die te lastig
wordt de deur uitzet, durfde ik nooit langer te blijven, omdat ik bang was dat
hij me erg vervelend zou vinden. Ik zocht naar een gelegenheid om onopvallend in
het kamertje te blijven en hem aan de klets te houden en die gelegenheid deed
zich gisteren~voor. Peter heeft namelijk plotseling een manie voor
kruiswoordpuzzles gekregen en doet niets anders meer dan puzzlen. Ik hielp hem
daarbij en al gauw zaten we tegenover elkaar aan zijn tafeltje, hij op de stoel,
ik op de divan.

Het werd mij wonderlijk te moede, telkens als ik in zijn donkerblauwe ogen keek
en hij daar zat met die geheimzinnige glimlach om zijn mond.  Ik kon uit alles
zo zijn innerlijk lezen, ik zag op zijn gezicht nog die hulpeloosheid en die
onzekerheid hoe zich te houden en tegelijkertijd een zweem van het besef van
zijn mannelijkheid. Ik zag zo die verlegen houding en werd zo zacht van binnen,
ik kon het niet laten telkens en telkens weer die donkere ogen te ontmoeten en
smeekte haast met heel mijn hart: o vertel me toch wat er in je omgaat, o kijk
toch over die noodlottige babbelzucht heen.

Maar de avond ging voorbij en er gebeurde niets, behalve dat ik hem dat over dat
blozen vertelde, natuurlijk niet dat wat ik opgeschreven heb, maar alleen dat
hij wel zekerder zou worden met de jaren.

's Avonds in bed vond ik de hele situatie lang niet opwekkend en het idee, dat
ik om Peters gunsten moet smeken, gewoon afstotend. Men doet heel wat om zijn
verlangens te bevredigen, dat zie je wel aan mij, want ik nam me voor vaker bij
Peter te gaan zitten en hem op de een of andere manier aan de praat te krijgen.

Je moet in geen geval denken, dat ik verliefd op Peter ben, geen sprake van. Als
de Van Daans in plaats van een zoon hier een dochter hadden gehad, zou ik ook
geprobeerd hebben met haar vriendschap te sluiten.

Vanochtend werd ik om ongeveer vijf minuten voor zevenen wakker en wist meteen
heel stellig wat ik gedroomd had. Ik zat op een stoel en tegenover me zat Peter
... Wessel, we bladerden een boek met tekeningen van Mary Bos door. Zo duidelijk
was mijn droom dat ik me de tekeningen gedeeltelijk nog herinner. Maar dat was
niet alles, de droom ging verder. Opeens ontmoetten Peters ogen de mijne en~lang
keek ik in die mooie fluweelbruine ogen. Toen zei Peter heel zacht: `Als ik dat
geweten had, was ik al lang bij je gekomen!' Brusk draaide ik me om, want de
ontroering werd me te machtig. En daarna voelde ik een zachte, o zo koele en
weldadige wang tegen de mijne en was alles zo goed, zo goed ...

Op dit punt aangekomen werd ik wakker, terwijl ik nog zijn wang tegen de mijne
aanvoelde en zijn bruine ogen diep in mijn hart voelde kijken, zo diep dat hij
daarin gelezen had hoezeer ik van hem gehouden had en hoeveel ik nog van hem
hield. De tranen sprongen me weer in de ogen en ik was erg bedroefd, omdat ik
hem weer kwijt was, maar tegelijkertijd toch ook blij, omdat ik weer met
zekerheid wist, dat Peter nog steeds mijn uitverkorene is.

Het is eigenaardig dat ik hier vaak zulke duidelijke droombeelden krijg.  Eerst
zag ik Omi1.~op een nacht zo duidelijk, dat ik haar huid van dik, zacht
rimpel-fluweel kon onderscheiden. Toen verscheen Oma als beschermengel, daarna
Lies, die mij nog het symbool van de ellende van al mijn vriendinnen en alle
Joden toelijkt. Als ik voor haar bid, bid ik voor alle Joden en arme mensen
samen. En nu Peter, mijn lieve Peter, nog nooit is hij mij in mijn geest zo
duidelijk voor ogen gekomen, ik hoef geen foto van hem te hebben, ik zie hem
voor mijn ogen o zo goed!

Je Anne.

\section*{Vrijdag, 7 Januari 1944}

Lieve Kitty,\\
Stomkop die ik ben! Ik heb er helemaal niet aan gedacht, dat ik
je de geschiedenis~van mij en al mijn aanbidders nooit heb verteld.

Toen ik nog heel klein was, zelfs nog op de kleuterschool, was mijn
sympathie~gevallen op Karel Samson. Hij had geen vader meer en woonde met zijn
moeder bij een tante in. Een~neefje van Karel, Robby, was een knappe, slanke,
donkere jongen, die steeds meer bewondering opwekte dan de kleine, humoristische
dikzak, die Karel was. Ik keek niet naar knapheid, maar hield jaren lang erg
veel van Karel.

Een tijd lang waren we veel samen, maar overigens bleef mijn liefde
onbeantwoord. Toen kwam Peter op mijn weg en ik kreeg een echte
kinderverliefdheid te pakken. Hij mocht mij eveneens graag en een zomer door
waren we onafscheidelijk. Ik zie ons in gedachten nog hand in hand door de
straten lopen, hij in een wit katoenen pak, ik in een korte zomerjurk. Aan het
einde van de grote vacantie kwam hij in de eerste klas van de middelbare school
en ik in de zesde klas van de lagere school. Hij haalde me van school af en
omgekeerd haalde ik hem. Peter was een beeld van een jongen, groot, knap, slank
met een ernstig, rustig en intelligent gezicht. Hij had donker haar en prachtig
bruine ogen, roodbruine wangen en een spitse neus. Vooral op zijn lach was ik
dol, dan zag hij er zo kwajongensachtig en ondeugend uit. Ik ging in de vacantie
naar buiten; toen ik terugkwam was Peter inmiddels verhuisd en woonde samen met
een veel ouderen jongen. Deze maakte hem er schijnbaar opmerkzaam op, dat ik een
kinderachtige puk was en Peter liet me los. Ik hield zoveel van hem, dat ik de
waarheid niet wilde inzien en hem vasthield, tot de dag kwam, waarop het tot me
doordrong, dat ik, als ik hem nog langer achterna liep, voor jongensgek
uitgemaakt zou worden. De jaren gingen voorbij. Peter ging met meisjes van zijn
eigen leeftijd om en dacht er niet meer aan mij te groeten, maar ik kon hem niet
vergeten.  Ik ging naar het Joodse Lyceum, vele jongens van onze klas werden
verliefd op me, ik vond het leuk, was vereerd, maar verder raakte het mij niet.
Nog weer later was Harry dol op me, maar zoals al gezegd, ik werd nooit meer
verliefd.

Er bestaat een gezegde: `De tijd heelt alle wonden'; zo ging het ook met mij. Ik
verbeeldde mij, dat ik Peter vergeten was en hem helemaal niet meer aardig vond.
De~herinnering aan hem leefde echter in mijn onderbewust zijn zo sterk voort,
dat ik mezelf wel eens bekende, dat ik jaloers was op die andere meisjes en hem
daarom niet meer aardig vond.  Vanochtend heb ik begrepen, dat niets veranderd
is, integendeel, terwijl ik ouder en rijper werd, groeide mijn liefde in mij
mee. Ik kan nu goed begrijpen, dat Peter me toen kinderachtig vond en toch trof
het me steeds weer pijnlijk, dat hij me zo vergeten had. Zijn gelaat vertoonde
zich zo duidelijk aan me en ik weet nu, dat niemand anders zo in me kon blijven
vastzitten.

Na de droom ben ik geheel in de war. Toen vader me vanochtend een zoen gaf,
wilde ik wel schreeuwen: `O, was je Peter maar!' Bij alles denk ik aan hem en de
hele dag herhaal ik niets anders bij mezelf dan: `O Petel, lieve, lieve Petel
...!'

Wie kan me nu helpen? Ik moet gewoon verder leven en God bidden, dat hij, als ik
hier uitkom, Peter op mijn weg zal brengen en dat die, terwijl hij in mijn ogen
mijn gevoelens leest, zal zeggen: `O Anne, als ik dat geweten had, was ik al
lang bij je gekomen!'

Ik heb in de spiegel mijn gezicht gezien en dat ziet er zo anders uit.  Mijn
ogen zien zo helder en zo diep, mijn wangen zijn, wat in weken niet gebeurd is,
roze gekleurd, mijn mond is veel weker, ik zie er uit of ik gelukkig ben en toch
is er zo iets droevigs in mijn uitdrukking, mijn glimlach glijdt meteen van mijn
lippen af. Ik ben niet gelukkig, want ik zou kunnen weten dat Peters gedachten
niet bij mij zijn en toch, toch voel ik steeds weer zijn mooie ogen op me
gericht, en zijn koele zachte wang tegen de mijne ...

O Petel, Petel, hoe kom ik ooit weer van je beeld los? Is ieder ander in je
plaats niet een armzalig surrogaat? Ik houd van je, o met zoveel liefde dat die
niet langer in mijn hart kon groeien, maar te voorschijn springen moest en zich
plotseling, in zo'n geweldige omvang, aan mij openbaarde.

Een week geleden, een dag geleden zou ik, als men me gevraagd had: `Wie van je
kennissen zou je het meest geschikt vinden om mee te trouwen?' geantwoord
hebben: `Ik weet het niet', en nu zou ik schreeuwen: `Petel, want van hem houd
ik met geheel mijn hart, met geheel mijn ziel, in volledige overgave!' Behalve
dat éne, hij mag me niet verder aanraken, dan in mijn gezicht.

Vader zei eens tegen me, toen we over sexualiteit spraken, dat ik de begeerte
toch nog niet kon begrijpen; ik wist altijd dat ik het wel begreep en nu begrijp
ik het helemaal. Niets is me nu zo dierbaar als hij, mijn Petel!

Je Anne.

\section*{Woensdag, 12 Januari 1944}

Lieve Kitty,\\
Sinds 14 dagen is Elli weer bij ons. Miep en Henk konden twee
dagen niet op hun plaats zijn, ze hadden alle twee hun maag bedorven. Ik heb op
het ogenblik dans- en balletbevliegingen en oefen elke avond vlijtig danspassen.
Uit een lichtblauwe onderjurk met kant van Mansa heb ik een hypermoderne
dansjurk vervaardigd. Van boven is er een bandje doorgetrokken, dat boven de
borst sluit; een rose geribbeld lint voltooit het geheel. Tevergeefs probeerde
ik van mijn gymschoenen echte ballet adden te maken. Mijn stijve ledematen zijn
hard op weg om, zoals vroeger, soepel te worden. Een knaloefening vind ik op de
grond zitten, met elke hand een hiel vasthouden en dan beide benen in de hoogte
tillen. Ik moet wel een kussen als onderlegsel gebruiken, anders wordt mijn arme
stuitje te zeer mishandeld.

Hier lezen ze een boek~\emph{Ochtend zonder wolken}~getiteld. Moeder vond het
buitengewoon goed; er zijn veel jeugdproblemen in beschreven.  Een beetje
ironisch dacht ik bij mezelf: `Bemoei je maar eerst meer met je eigen jeugd'.

Ik geloof dat moeder denkt, dat Margot en ik de beste verhouding tot onze ouders
hebben, die er maar bestaat, en dat niemand zich meer met het leven van zijn
kinderen bemoeit dan zij. Ze kijkt in deze stellig alleen naar Margot,~want ik
geloof dat zij zulke problemen en gedachten als ik heb nooit heeft. Moeder wil
ik helemaal niet op de gedachten brengen, dat het bij één van haar spruiten er
heel anders uitziet dan zij zich dat voorstelt, want zij zou geheel verbijsterd
staan en toch niet weten hoe de zaak dan anders aan te pakken; het verdriet dat
daardoor voor haar zou volgen wil ik haar besparen, vooral omdat ik weet, dat
alles voor mij toch hetzelfde zou blijven.

Moeder voelt wel, dat Margot veel meer van haar houdt dan ik, maar zij denkt dat
dit bij periodes zo is! Margot is zo lief geworden, zij lijkt mij heel anders
dan vroeger, ze is lang zo kattig niet meer en wordt een echte vriendin. Ze
beschouwt me in het geheel niet meer als de kleine uk waar je geen rekening mee
te houden hebt.

Het is een gek verschijnsel dat ik mij soms zie als door de ogen van een ander.
Ik bekijk me de zaken van een zekere `Anne' dan op mijn dooie gemak en zit in
mijn eigen levensboek te bladeren, alsof het van een vreemde was. Vroeger,
thuis, toen ik nog niet zoveel nadacht, had ik bij tijd en wijle het gevoel, dat
ik niet bij Mansa, Pim en Margot hoorde en altijd een buitenbeentje zou blijven.
Soms speelde ik dan wel een tijdje de rol van een weeskind, totdat ik mezelf
bestrafte en me verweet, dat het niets dan mijn eigen schuld was, dat ik de
lijdenspersoon speelde, terwijl ik het altijd zo goed had. Dan volgde een
periode waarin ik me dwong vriendelijk te zijn. Elke morgen als iemand de trap
afkwam hoopte ik, dat het moeder zou zijn, die me goeden morgen zou zeggen en ik
begroette haar lief, omdat ik me er ook werkelijk in verheugde, dat ze me zo
lief aankeek. Dan, door de een of andere opmerking, was ze onvriendelijk tegen
me en dan ging ik geheel ontmoedigd naar school. Op weg naar huis
verontschuldigde ik haar, dacht bij mezelf dat ze zorgen had, kwam opgewekt
thuis, praatte honderd uit, totdat zich hetzelfde herhaalde en ik met een
peinzend gezicht met mijn schooltas de deur uitging. Soms nam ik me voor~boos te
blijven, maar uit school thuisgekomen had ik zoveel nieuws, dat ik mijn
voornemen al lang vergeten was en moeder onder alle omstandigheden een open oor
voor al mijn belevenissen moest hebben. Dan kreeg ik weer de tijd, waarin ik 's
morgens niet meer naar stappen op de trap luisterde, me eenzaam voelde en 's
avonds mijn kussentje met tranen overgoot.

Hier is alles veel erger geworden, en n, je weet het.\\
Eén hulp in deze dingen
heeft God me nu gestuurd. `Peter'...\\
Ik voel even aan mijn hangertje, druk er
een zoen op, en denk: `Wat kan me de

hele rommel ook schelen! Peter hoort bij mij en niemand weet er van'. Op die
manier kan ik elke afsnauwing overwinnen.

Wie zou weten, hoeveel er in een bakvisziel omgaat? Je Anne.

\section*{Zaterdag, 15 Januari 1944}

Lieve Kitty,\\
Het heeft geen doel, dat ik je steeds weer tot in de kleinste
bijzonderheden ruzies~en disputen beschrijf. Ik vind het voldoende als ik je
vertel, dat we zeer veel dingen, zoals vet, boter en vlees gedeeld hebben en
onze eigen aardappels bakken. Sinds enige tijd eten we er wat roggebrood extra
bij, omdat we om vier uur al reikhalzend naar het middageten uitzagen en onze
rammelende magen haast niet in bedwang konden houden.

Moeders verjaardag nadert met rasse schreden. Zij heeft van Kraler extra suiker
gekregen, een aanleiding voor jaloezie van de kant van de Van Daans, omdat bij
mevrouws verjaardag de tractatie overgeslagen was. Maar waar dient het voor
elkaar nog verder met harde woorden, huilbuien en nijdige gesprekken te
vervelen! Wees er van overtuigd, Kitty, dat ze ons nòg meer vervelen. Moeder
heeft de voorlopig onuitvoerbare wens geuit, dat ze de Van Daans 14 dagen niet
behoeft te zien.

Ik vraag me herhaaldelijk af, of men met alle mensen, waar men zo lang mee
samenwoont, op den duur ruzie zou krijgen. Of hebben wij misschien erg
gewanboft? Is het merendeel van de mensheid dan zo egoïstisch en schraperig? Ik
vind het best, dat ik hier een klem beetje mensenkennis gekregen heb, maar het
lijkt me nu voldoende. De oorlog stoort zich toch niet aan onze ruzies,
vrijheids- en lucht-neigingen en daarom moeten we proberen het beste van ons
verblijf te maken. Ik zit te preken, maar ik geloof ook dat ik, als ik hier nog
lang blijf, een uitgedroogde bonenstaak word. En ik zou zo graag nog een echte
bakvis zijn!

Je Anne.

\section*{Zaterdag, 22 Januari 1944}

Lieve Kitty,\\
Kan jij me misschien vertellen hoe het komt, dat alle mensen hun
innerlijk zo~angstvallig verbergen? Hoe het komt, dat ik in gezelschap altijd
heel anders doe dan ik moest doen?

Waarom vertrouwt de een de ander zo weinig? Ik weet het, er zal zeker een reden
zijn, maar soms vind ik het erg naar dat je nergens, zelfs bij de mensen die je
het naast staan, vertrouwelijkheid vindt.

Het lijkt me, alsof ik sinds de nacht van mijn droom ouder geworden ben, veel
meer `een persoon op zichzelf'. Je zult wel heel erg gek opkijken als ik je zeg,
dat zelfs de Van Daans een andere plaats bij mij ingenomen hebben. Ik bekijk
ineens al die discussies enzovoort enzovoort niet meer vanuit ons vooringenomen
standpunt.

Hoe kom ik zo veranderd? Ja, zie je, ik dacht er opeens aan dat, als moeder
anders was, een echte mams, onze verhouding dan heel en heel anders geweest zou
zijn. Het is natuurlijk wel waar, dat mevrouw Van Daan allesbehalve een mooi
mens is, maar toch denk ik dat, als moeder niet óók~moeilijk te hanteren werd
bij elk puntig gesprek, de helft van alle ruzies vermeden had kunnen worden.

Mevrouw Van Daan heeft namelijk één zonzijde en dat is, dat je met haar praten
kunt. Ondanks alle egoïsme, schraperigheid en achterbaksheid kun je haar
makkelijk tot toegeven bewegen, als je haar maar niet prikkelt en weerbarstig
maakt. Tot de volgende aanleiding werkt dit middel niet, maar als je een
geduldig iemand bent, kun je dan opnieuw proberen hoe ver je komt.

Al onze opvoedingskwesties, de verwennerij, het eten, alles en alles had een
heel andere loop genomen, als men open en vriendschappelijk was gebleven en niet
steeds alleen de slechte kanten voor ogen had gehad.

Ik weet precies wat je zult zeggen, Kitty. `Maar Anne, komen deze woorden heus
van jou? Van jou, die zoveel harde woorden van boven hebt moeten horen, van jou,
die al het onrecht weet, dat er gebeurd is?' En toch, het komt van mij.

Ik wil opnieuw alles doorgronden en daarbij niet volgens het spreekwoord: `Zoals
de ouden zongen, piepen de jongen' te werk gaan. Ik wil zelf de Van Daans
onderzoeken en zien wat waar en wat overdreven is.  Als ik dan zelf de
teleurstelling ondervonden heb, kan ik weer eenzelfde lijn trekken met vader en
moeder, zo niet, dan zal ik eerst proberen hen van hun verkeerde voorstelling af
te brengen en als dat niet mocht lukken, dan zal ik toch mijn eigen mening en
oordeel hoog houden. Ik zal elke gelegenheid aangrijpen om openlijk met mevrouw
over vele twistpunten te praten en niet bang zijn om, ondanks de naam van
wijsneus, mijn neutrale mening te zeggen.

Ik moet nu niet tegen mijn eigen familie ingaan, maar roddelen van mijn kant
behoort, van vandaag af, tot het verleden.

Tot nu toe dacht ik rotsvast, dat alle schuld van de ruzies bij de Van Daans
ligt, maar een deel lag zeker ook bij ons. Wij hadden wel gelijk met wat de
onderwerpen betreft,~maar van verstandige mensen (waartoe wij ons rekenen!)
verwacht men toch wat meer inzicht, hoe andere mensen te behandelen. Ik hoop dat
ik een tikkeltje van dat inzicht gekresen heb en de gelegenheid zal vinden het
goed aan te wenden.

Je Anne.

\section*{Maandag, 24 Januari 1944}

Lieve Kitty,\\
Mij is iets overkomen, of eigenlijk kan ik van overkomen niet
spreken, dat ik zelf~gek vind.

Vroeger thuis en op school werd er over geslachtsvraagstukken òf geheimzinnig~òf
weerzinwekkend gesproken. Woorden die daar betrekking op hadden werden ge
fluister d en vaak werd iemand, als hij niets wist, uitgelachen. Ik vond dat
raar en dacht: `Waarom spreekt men over deze dingen zo geheimzinnig en
vervelend?' Maar omdat daar toch niets aan te veranderen scheen, hield ik zoveel
mogelijk mijn mond of vroeg soms ook vriendinnen om inlichtingen.

Toen ik van veel op de hoogte was en ook met mijn ouders gesproken had, zei
moeder op een dag: `Anne, ik geef je een goede raad, spreek over dit onderwerp
nooit met jongens en geef geen antwoord als zij er over beginnen'. Ik weet mijn
antwoord heel precies, ik zei: `Neen, natuurlijk niet, stel je voor!'

En daar is het bij gebleven.

In het begin van de onderduiktijd vertelde vader dikwijls wat van dingen, die ik
liever van moeder gehoord had en de rest kwam ik uit boeken of gesprekken wel te
weten. Peter van Daan was nooit zo vervelend op dit gebied als de jongens op
school, in het begin misschien een enkele keer, maar nooit om mij aan het praten
te krijgen.

Mevrouw had ons eens verteld dat zij met Peter over deze dingen nooit gesproken
had en naar zij wist, ook haar~man niet. Schijnbaar wist ze niet eens in
hoeverre Peter ingelicht was.

Gisteren, toen Margot, Peter en ik aan het aardappelschillen waren, kwam
het~gesprek vanzelf op Mof . `We weten nog steeds niet, van welk geslacht Mof
is, hè?' vroeg ik.

`Toch wel', antwoordde hij, `het is een kater'. Ik begon te lachen.  `Mooie
kater die in verwachting is'. Peter en Margot lachten mee om die grappige
vergissing. Peter had namelijk een maand of twee geleden geconstateerd, dat het
niet lang zou duren of Mof had kinderen, haar buik werd opzienbarend dik. De
dikte bleek echter van de vele gestolen boutjes te komen en de kindertjes
groeiden niet hard, laat staan, dat ze geboren werden.

Peter moest zich tegen die beschuldiging toch even verdedigen. `Neen', zei hij,
`Je kunt zelf meegaan om hem te bekijken, ik heb, toen ik eens met hem ravotte,
heel goed gezien, dat hij een kater is'.

Ik was niet in staat mijn nieuwsgierigheid te bedwingen en ging mee naar het
magazijn. Mof had echter geen ontvanguur en was nergens te bekennen.  Wij
wachtten een poosje, kregen het koud en stegen al de trappen weer op. Later op
de middag hoorde ik hem voor de tweede keer naar beneden gaan. Ik raapte al mijn
moed samen om alleen het stille huis door te lopen, en belandde in het magazijn.
Op de paktafel stond Mof te spelen met Peter, die hem juist op de weegschaal
zette om zijn gewicht te controleren.

`Hallo, wil je hem zien?' Hij maakte geen lange toebereidselen, pakte het dier
op, draaide hem op zijn rug, hield heel handig kop en poten vast en de les
begon. `Dit is het mannelijk geslachtsdeel, dit zijn een paar losse haartjes en
dat is zijn achterste'. De kat maakte nogmaals een halve draai en stond weer op
zijn witte sokjes.

Iedere andere jongen, die me `het mannelijk geslachtsdeel' zou hebben gewezen,
zou ik niet meer aangekeken hebben.

Maar Peter sprak doodgewoon verder over het anders zo penibele onderwerp, had
helemaal geen nare bijbedoelingen en stelde me ten slotte in zoverre gerust, dat
ik ook gewoon werd. We speelden met Mof , amuseerden ons, kletsten samen en
gingen daarna slenterend het uitgestrekte pakhuis door naar de deur.

`Ik vind wat ik wil weten altijd toevallig in het een of andere boek, jij niet?'
vroeg ik.

`Waarom, ik vraag het boven wel. Mijn vader weet dat beter dan ik en heeft meer
ervaring in die dingen'.

We stonden al op de trap en ik hield verder mijn mond.

`'t Kan verkeren', zei Bredero. Ja werkelijk, zo gewoon zou ik er nooit met een
meisje over gesproken hebben. Ik weet ook zeker dat moeder dit nooit bedoelde,
toen ze me voor de jongens waarschuwde. Ondanks alles was ik de hele dag een
beetje uit mijn gewone doen, toen ik over ons gesprek nadacht, leek het me toch
eigenaardig. Maar op één punt ben ik tenminste wijzer geworden: er zijn ook
jonge mensen en nog wel van de andere sexe, die ongedwongen en zonder grappen
kunnen spreken.

Zou Peter werkelijk veel aan zijn ouders vragen, zou hij dan heus zo zijn als
hij zich gisteren voordeed?

Ach, wat weet ik er vanaf!!! Je Anne.

\section*{Donderdag, 27 Januari 1944}

Lieve Kitty,\\
In de laatste tijd heb ik een sterke liefde voor stambomen en
genealogische tabellen~van koninklijke huizen opgevat en ben tot de conclusie
gekomen, dat men, als men eenmaal met zoeken begint, steeds verder in de oudheid
moet graven en tot steeds interessanter ontdekkingen komt.

Hoewel ik buitengewoon ijverig ben, wat mijn leervakken betreft, al tamelijk
goed de home-service van de Engelse radio kan volgen, besteed ik nog veel
Zondagen aan het~uitzoeken en sorteren van mijn grote lmsterren-verzameling, die
een zeer respectabele omvang aangenomen heeft.

Mijnheer Kraler maakt me elke Maandag blij, als hij
de~\emph{Cinema}~\&~\emph{Theater}~meebrengt. Hoewel deze verwennerij vaak
betiteld wordt als geldverspilling, door de onmondaine huisgenoten, staan ze
elke keer weer verbaasd over de nauwkeurigheid, waarmee ik na een jaar nog
precies de medespelers in een bepaalde film kan opnoemen.  Elli, die dikwijls
haar vrije dagen met haar vriend in de bios doorbrengt, deelt me de titel van de
voorgenomen films zaterdags mee en ik ratel haar zowel de hoofdrolvertolkers als
de kritiek in enen af. Het is nog niet lang geleden, dat Mans zei, dat ik later
niet naar de bioscoop hoefde, omdat ik inhoud, sterren en kritiek zo goed in
mijn hoofd had.

Als ik op een dag met een nieuw kapsel aan kom zeilen, kijken ze me allemaal met
afkeurende gezichten aan en ik kan er op rekenen dat er eentje vraagt welke
filmster deze coiffure op haar hoofd heeft prijken.  Als ik antwoord, dat het
eigen vinding is, geloven ze me nog maar half.

Wat het kapsel aangaat, dat zit niet langer dan een half uur goed, dan ben ik zo
beu van de afwijzende oordelen, dat ik me naar de badkamer spoed en gauw mijn
gewone huis-, tuin- en keukenkapsel herstel.

Je Anne.

\section*{Vrijdag, 28 Januari 1944}

Lieve Kitty,\\
Vanochtend heb ik me afgevraagd, of jij je niet voorkomt als een
koe, die al de oude nieuwtjes steeds opnieuw weer moet herkauwen en die van het
eenzijdige voedsel ten slotte hard gaapt enfin stilte wenst, dat Anne eens wat
nieuws opduikelt. Helaas, ik weet, het oude is vervelend voor jou, maar denk je
eens in, hoe zeurderig

ik van de oude koeien word, die steeds opnieuw uit de sloot gehaald worden. Als
een tafelgesprek niet over politiek of een heerlijke maaltijd gaat, wel dan
komen moeder of mevrouw maar weer met al lang vertelde jeugdverhalen op de
proppen, of bazelt Dussel over de uitgebreide klerenkast van zijn vrouw, mooie
renpaarden, lekke roeiboten, jongens die met vier jaar zwemmen, spierpijn en
angstige patiënten. Het komt alles hier op neer, dat, als een van de acht zijn
mond open doet, de andere zeven zijn begonnen verhaal kunnen afmaken. De pointe
van elke mop weten we al van te voren en de verteller lacht in zijn eentje om de
geestigheid. De diverse melkboeren, kruideniers en slagers van de ex-huisvrouwen
zien wij in onze verbeelding al met een baard, zoveel zijn ze aan tafel
opgehemeld of afgemaakt; het is onmogelijk dat iets nog jong en fris is, als het
in het Achterhuis ter sprake komt.

Dat alles zou nog te dragen zijn, als de volwassenen er niet zo'n handje van
hadden, verhalen die Koophuis, Henk of Miep ten beste geven, tien keer over te
vertellen en het elke volgende keer opsieren met hun eigen bedenksels, zodat ik
me vaak onder tafel in mijn arm moet knijpen om den enthousiasten verteller niet
de juiste weg te wijzen. Kleine kinderen, zoals Anne er een is, mogen
volwassenen onder geen omstandigheid verbeteren, welke blunders ze ook mogen
slaan of welke onwaarheden en bedenksels ze uit hun duim zuigen.

Een onderwerp dat Koophuis en Henk nogal eens eer aan doen, is het schuilen of
onderduiken. Zij weten heel goed dat alles, wat op andere onderduikers of
verstopte mensen betrekking heeft, ons brandend interesseert en dat wij oprecht
meeleven met opgepakte onderduikers in hun leed, zowel als met bevrijde
gevangenen in hun vreugde.

Onderduiken en schuilen zijn net zo'n gewoon begrip geworden als vroeger de
pantoffels van papa, die voor de kachel moesten staan. Instellingen als `Vrij
Nederland', die persoonsbewijzen vervalsen, geld aan onderduikers verstrekken,
plaatsen om onder te duiken vrijmaken, ondergedoken Christenjongens werk
verlenen, zijn er heel veel~en het is verbazingwekkend hoe veel, hoe nobel en
hoe ons baatzuchtig werk er door de mensen wordt verricht, die met inzet van hun
eigen leven anderen helpen en anderen redden. Het beste voorbeeld daarvan zijn
toch wel onze helpers die ons er tot nu toe doorheengetrokken hebben en ons
hopelijk helemaal op het droge zullen afleveren, anders zullen zij zelf het lot
moeten delen van allen, die gezocht worden. Nooit hebben wij één woord gehoord,
dat op de last duidt, die wij toch zeer zeker zijn, nooit klaagt één van hen,
dat wij hun te veel moeite veroorzaken.

Elke dag komen ze allen boven, spreken met de heren over zaak en politiek, met
de dames over eten en de lasten van de oorlogstijd, met de kinderen over boeken
en kranten. Zij zetten zoveel mogelijk een vrolijk gezicht, brengen bloemen en
cadeau's voor verjaar- en feestdagen mee, staan altijd en overal voor ons klaar.
Dat is wat wij nooit mogen vergeten, dat, hoewel anderen heldenmoed in de oorlog
of tegenover de Duitsers tonen, onze helpers heldenmoed in hun opgewektheid en
liefde aan de dag leggen.

De gekste praatjes doen de ronde en toch zijn de feiten meestal werkelijk
gebeurd. Koophuis deelde ons bijvoorbeeld van de week mee, dat er in Gelderland
twee elftallen tegen elkaar gevoetbald hebben, het ene bestond uitsluitend uit
onderduikers en het tweede was samengesteld uit elf leden van de marechaussée.
In Hilversum worden nieuwe stamkaarten uitgereikt. Opdat de vele onderduikers
ook hun deel van de rantsoenering krijgen, hebben ambtenaren van de uitreiking
alle onderduikers uit de omtrek op een bepaald uur besteld, dan kunnen zij hun
bescheiden aan een apart tafeltje afhalen. Je moet toch maar voorzichtig zijn,
dat dergelijke staaltjes de moffen niet ter ore komen.

Je Anne.

\section*{Donderdag, 3 Februari 1944}

Lieve Kitty,\\
De invasie-stemming stijgt in het land met de dag en als je hier
zou zijn, dan zou~je zeker aan de ene kant net als ik onder de indruk komen van
al die voorbereidingen, maar aan de andere kant zou je ons uitlachen, omdat we
zo'n drukte maken, wie weet voor niets.

Alle kranten staan vol over de invasie en maken de mensen gek, omdat ze
schrijven: `Als de Engelsen eventueel in Nederland zouden landen, zullen de
Duitse autoriteiten alles in het werk stellen om het land te verdedigen, zo
nodig het ook onder water laten lopen'. Daarbij zijn kaartjes gepubliceerd
waarop de delen van Nederland, die onder water gezet kunnen worden, gearceerd
zijn. Daar grote delen van Amsterdam tot het gearceerde behoren, was de eerste
vraag wat te doen als het water een meter hoog in de straten komt te staan.

Op deze moeilijke vraag kwamen van alle kanten de meest verschillende
antwoorden:

`Omdat fietsen of lopen uitgesloten is, zullen we dan door het stilstaande water
moeten waden'.

`Welneen, men moet proberen te zwemmen. We doen allemaal een badmuts en badpak
aan en zwemmen zoveel mogelijk onder water, dan ziet niemand dat we Joden zijn'.

`Ach, wat een praats, ik zie de dames al zwemmen, als de ratjes in hun benen
bijten!' (Dat was natuurlijk een man, zien wie het hardst gilt!)

`We zullen het huis niet meer uit kunnen, het magazijn is zo wankel, dat zakt
bij zo'n overstroming beslist in elkaar'.

`Hoor eens jongens, nu alle gekheid op een stokje, we zullen een bootje zien te
krijgen'.

`Waarvoor is dat nodig? Ik weet iets veel beters, we nemen elk een
melksuikerkist van de voorzolder en roeien met een pollepel'.

`Ik ga stelten lopen, dat kon ik in mijn jeugd primissima'.\\
`Henk van Santen
heeft dat niet nodig, die neemt zijn vrouw zeker op zijn rug,~dan heeft Miep
stelten'.

Nu weet je het wel zo ongeveer, niet Kit? Dit geklets is allemaal erg grappig;
de~waarheid zal het anders leren.

De tweede invasie-vraag kon niet uitblijven. Wat te doen, als de
Duitsers~Amsterdam evacueren?

`Meegaan, ons zo goed mogelijk vermommen'.\\
`In geen geval meegaan. Het enige
is hier blijven! De Duitsers zijn in staat de hele~bevolking steeds verder mee
terug te drijven, totdat ze in Duitsland doodgaan'.

`Ja natuurlijk, we blijven hier, hier is het het veiligst. We zullen proberen
Koophuis~over te halen om met zijn familie hier te komen wonen.  We zullen zien
een zak houtwol te krijgen, dan kunnen we op de grond slapen. Laat Miep en
Koophuis alvast dekens hierheen brengen'.

`We zullen nog bij onze 60 pond wat graan bestellen. Laat Henk peulvruchten zien
te krijgen, we hebben nu ongeveer 60 pond bonen en 10 pond erwten in huis.
Vergeet de 50 blikken groente niet'. `Moeder, tel de andere etenswaren eens
even!' `10 Blikjes vis, 40 blikjes melk, 10 kg melkpoeder, 3 essen olie, 4
weckpotten boter, 4 dito's vlees, 2 mand essen aardbeien, 2 essen
frambozen-bessen, 20 essen tomaten, 10 pond havermout, 8 pond rijst, dat is
alles'.

`Onze voorraad valt nogal mee, maar als je bedenkt, dat we er nog visite bij
zullen moeten voeden en er elke week van gebruikt wordt, dan lijkt het enormer
dan het is. Kolen en brandhout zijn voldoende in huis, kaarsen ook. Laten we
allemaal borstzakjes naaien, om zo nodig al ons geld mee te nemen.

We zullen lijsten opmaken van wat bij vluchten het eerst meemoet en nu maar vast
rugzakken pakken. Als het zover is zetten we twee uitkijkposten op wacht, één op
de voor- en één op de achtervliering.  Zeg, wat beginnen we met zoveel
etenswaren, als we geen water, gas en electra hebben?'

`Dan moeten we op de kachel koken. Water filtreren en koken. We zullen grote
mand essen schoonmaken en daar water in bewaren'.

Deze praatjes hoor ik de hele dag, invasie voor, invasie na, disputen over
hongerlijden, sterven, bommen, brandspuiten, slaapzakken, Jodenbewijzen,
gifgassen enzovoort enzovoort. Allemaal niet opwekkend.  Een goed voorbeeld van
de ondubbelzinnige waarschuwingen van de heren uit het Achterhuis is het
volgende gesprek met Henk:

Achterhuis: `We zijn bang, dat de Duitsers, als ze terugtrekken, de hele
bevolking met zich mee zullen nemen'.

Henk: `Dat is niet mogelijk, daarvoor hebben ze geen treinen ter beschikking'.

A.: `Treinen? Dacht u, dat ze de burgers ook nog in een wagentje zouden zetten?
Geen kwestie van, de benenwagen kunnen ze gebruiken'. (Per pedes apostolorum,
zegt Dussel altijd.)

H.: `Ik geloof er niets van, u ziet alles door een veel te zwarte bril.  Wat
zouden ze er voor belang bij hebben alle burgers mee te drijven?'

A.: `Weet u niet, dat Goebbels gezegd heeft: `Als wij zullen moeten
terugtrekken, slaan we achter ons de deur van alle bezette gebieden dicht?'

H.: `Ze hebben al zoveel gezegd'.

A.: `Denkt u, dat de Duitsers voor zo'n daad te edel of te menslievend zijn? Die
denken: ``Als wij ten onder moeten gaan, dan zullen alle mensen, die binnen het
bereik van onze macht liggen, ook ten onder gaan'''.

H.: `U kunt me veel vertellen, ik geloof er niets van!'

A.: `Het is altijd weer hetzelfde liedje; niemand wil het gevaar inzien dat hem
bedreigt, voordat het zijn eigen lijf geraakt heeft'.

H.: `U weet toch ook niets positiefs; u veronderstelt het toch ook maar'. A.:
`We hebben het allemaal toch zelf meegemaakt, eerst~in Duitsland en toen hier.
En wat gebeurt er in Rusland?'

H.: `De Joden moet u even buiten beschouwing laten, ik geloof dat niemand
weet~wat er in Rusland aan de hand is. De Engelsen en Russen zullen, net als de
Duitsers, voor propagandadoeleinden overdrijven'.

A.: `Geen kwestie van, de Engelse radio heeft altijd de waarheid gezegd.  En
stel, dat de berichten overdreven zijn, dan zijn de feiten nog erg genoeg, want
u kunt niet ontkennen, dat het een feit is, dat er in Polen en Rusland vele
millioenen vreedzame mensen zonder veel omwegen vermoord of vergast worden'.

Verder zal ik je onze gesprekken sparen, ik ben heel rustig en trek me van alle
drukte niets aan. Ik ben zover gekomen, dat het me niet veel meer kan schelen of
ik doodga of blijf leven. De wereld zak ook zonder mij verder draaien en ik kan
me tegen de gebeurtenissen toch niet verzetten.

Ik laat het er op aankomen en doe niets anders dan leren en op een goed einde
hopen.

Je Anne.

\section*{Zaterdag, 12 Februari 1944}

Lieve Kitty,\\
De zon schijnt, de hemel is diep-blauw, er waait een heerlijke
wind en ik verlang~zo -, verlang zo - naar alles ..... Naar praten, naar
vrijheid, naar vrienden, naar alleen-zijn. Ik verlang zo ... naar huilen! Ik heb
een gevoel in me of ik spring en ik weet dat het met huilen beter zou worden; ik
kan het niet. Ik ben onrustig, loop van de ene naar de andere kamer, adem door
de kier van een dicht raam, voel mijn hart kloppen, alsof het zegt: `Voldoe toch
eindelijk aan mijn verlangen'.

Ik geloof, dat ik het voorjaar in me voel, ik voel het lente-ontwaken, ik voel
het in mijn hele lichaam enfin mijn ziel. Ik moet me in bedwang houden om gewoon
te doen, ik ben totaal in de war, weet niet wat te lezen, wat te schrijven, wat
te doen, weet alleen, dat ik verlang ...!

Je Anne.

\section*{Zondag, 13 Februari 1944}

Lieve Kitty,\\
Sinds Zaterdag is er voor mij veel veranderd. Dat komt zo.  Ik
verlangde - en ik~verlang nog - maar ... voor een klein, heel klein deeltje ben
ik al geholpen. Zondagochtend merkte ik al - ik zal eerlijk zijn - tot mijn
grote vreugde dat Peter~me aldoor zo aankeek. Zo heel anders dan gewoonlijk, ik
weet niet, ik kan het niet uitleggen hoe.

Vroeger had ik gedacht, dat Peter verliefd op Margot was, nu had ik opeens het
gevoel, dat dit niet zo is. De hele dag keek ik hem expres niet veel aan, want
als ik dat deed, keek hij ook steeds en dan -, ja dan, dan kreeg ik een mooi
gevoel in me, dat ik toch niet te vaak mocht krijgen.

Ik heb een sterke behoefte om alleen te zijn. Vader merkt, dat ik niet gewoon
ben, maar ik kan hem ook niet alles vertellen. `Laat me met rust, laat me
alleen!' dat zou ik steeds maar willen uitroepen. Wie weet, word ik nog eens
meer alleen gelaten dan me lief is!

Je Anne.

\section*{Maandag, 14 Februari 1944}

Lieve Kitty,\\
Zondagavond zaten ze allen aan de radio, behalve Pim en ik, naar
de `Unsterbliche~Musik Deutscher Meister' te luisteren. Dussel draaide
aanhoudend aan het toestel. Peter ergerde zich en de anderen ook. Na een half
uur van ingehouden zenuwachtigheid verzocht Peter enigszins geprikkeld dat
gedraai te staken. Dussel antwoordde op zijn hooghartigste toontje: `Ich mach'
das schon'. Peter werd kwaad, werd brutaal, mijnheer Van Daan viel hem bij en
Dussel moest toegeven. Dat was alles.

De aanleiding was op zichzelf niet zo buitengewoon belangrijk, maar het schijnt,
dat Peter zich de zaak erg aangetrokken heeft. Hij kwam in ieder geval
vanochtend, toen ik~in de boekenkist op zolder rommelde, naar me toe en begon de
kwestie te vertellen. Ik wist van niets af, Peter merkte dat hij een aandachtige
toehoorster gevonden had, en kwam op dreef.

`Ja, en zie je', zo zei hij, `ik zeg niet gauw iets, want ik weet al vooruit,
dat ik niet uit mijn woorden zal komen. Ik ga stotteren, krijg een kleur en
draai de woorden, die ik zeggen wilde om, tot ik mijn beweringen moet afbreken,
omdat ik zelf de woorden niet meer vind.  Gisteren ging het me net zo, ik wilde
heel iets anders zeggen, maar toen ik eenmaal begonnen was, raakte ik de kluts
kwijt en dat is vreselijk.  Ik had vroeger een slechte gewoonte, die ik nu nog
het liefst zou willen toepassen. Als ik kwaad op iemand was, dan bewerkte ik hem
liever met mijn vuisten dan dat ik met hem redetwistte. Ik weet wel, dat ik met
deze methode niet verder kom en daarom bewonder ik jou, jij komt tenminste recht
uit je woorden, zegt de mensen wat je te zeggen hebt en bent in het minst niet
verlegen'.

`Dan vergis je je erg', antwoordde ik, `ik zeg in de meeste gevallen de dingen
heel anders dan ik me oorspronkelijk voorgenomen had en dan praat ik veel te
veel en veel te lang, dat is net zo'n erge kwaal'.

In stilte moest ik om deze laatste zin lachen, wilde hem echter gerust verder
over zichzelf laten spreken, liet hem van mijn vrolijkheid niets merken, ging op
een kussen op de grond zitten, sloeg mijn armen om mijn opgetrokken benen en
keek hem opmerkzaam aan.

Ik ben reuze blij, dat er nog iemand in huis is, die precies zulke
woedeaanvallen kan krijgen als ik. Het deed Peter zichtbaar goed, dat hij in de
ergste bewoordingen Dussel mocht becritiseren, zonder dat hij bang voor klikken
moest zijn. En ik, ik vond het ook mooi, omdat ik een sterk gevoel van
gemeenschap merkte, wat ik vroeger alleen maar met mijn vriendinnen had.

Je Anne.

\section*{Woensdag, 16 Februari 1944}

Lieve Kitty,\\
Margot is jarig.\\
Om half een kwam Peter de cadeautjes bekijken
en bleef veel langer praten dan~volstrekt nodig was, wat hij anders nooit gedaan
zou hebben. 's Middags ging ik de koffie halen en haalde daarna de aardappels,
omdat ik Margot één keer in het jaar verwennen wilde. Ik kwam door Peters kamer,
hij haalde direct al zijn papieren van de trap af en ik vroeg of ik het luik
naar de vliering moest sluiten. `Ja', antwoordde hij, `doe dat maar. Als je
terugkomt klop je, dan doe ik het wel voor je open'.

Ik bedankte hem, ging naar boven en zocht wel tien minuten lang de kleinste
aardappels uit de grote ton. Toen kreeg ik pijn in mijn rug en werd koud. Ik
klopte natuurlijk niet, deed zelf het luik open, maar hij kwam me toch zeer
gedienstig tegemoet en pakte de pan aan. `Ik heb lang gezocht, ik kon geen
kleinere vinden', zei ik.

`Heb je in de grote ton gekeken?'\\
`Ja, ik heb alles met mijn handen
omgewoeld'.\\
Ik stond intussen onder aan de trap en hij keek onderzoekend in de
pan, die hij~nog in zijn handen hield. `O, maar ze zijn best', zei hij en voegde
er aan toe, toen ik de pan van hem overnam: `Mijn compliment, hoor!' Daarbij
keek hij me aan met zo'n warme, zachte blik, dat ik ook warm en zacht van binnen
werd. Ik kon zo echt merken, dat hij me plezier wilde doen en omdat hij geen
grote lofrede kon houden, legde hij zijn woorden in zijn blik. Ik begreep hem o
zo goed en was hem reuze dankbaar. Nu nog word ik blij, als ik me de woorden en
die blik voor ogen haal.

Toen ik beneden kwam, zei moeder, dat ik nog meer aardappels moest halen, nu
voor het avondeten. Ik bood heel gewillig aan nog eens naar boven te gaan.

Toen ik bij Peter kwam, verontschuldigde ik me, dat ik hem nog eens moest
storen. Hij stond op, ging tussen de~trap en de muur staan, pakte mijn arm vast
toen ik al op de trap stond en wilde me met alle geweld tegenhouden. `Ik ga
wel', zei hij. Ik antwoordde, dat dat heus niet nodig was en dat ik nu geen
kleine behoefde te halen. Toen was hij overtuigd en liet mijn arm los. Op de
terugweg kwam hij het luik openen en pakte weer de pan van me aan. Bij de deur
vroeg ik: `Wat ben je aan het doen?' `Frans', was het antwoord. Ik vroeg of ik
de lessen eens mocht inkijken, waste mijn handen en ging tegenover hem op de
divan zitten.

Nadat ik hem van Frans wat uitgelegd had, gingen we al gauw aan het praten. Hij
vertelde me dat hij later naar Nederlands-Indië wou en daar in de plantages
leven. Hij sprak over zijn leven thuis, over het zwart-handelen en dat hij zo'n
nietsnut was. Ik zei, dat hij wel een heel sterk minderwaardigheidsgevoel had.
Hij sprak ook over de Joden.  Hij zou het veel gemakkelijker gevonden hebben,
als hij Christen was en als hij na de oorlog Christen zou kunnen zijn. Ik vroeg,
of hij zich wou laten dopen, maar dat was ook niet het geval. Na de oorlog zou
toch niemand weten of hij Christen of Jood was, zei hij.

Daarbij ging me even een steek door het hart, ik vind het zo jammer, dat hij nog
altijd een rest oneerlijkheid in zich heeft. Verder spraken we heel gezellig
over vader en over mensenkennis en over alle mogelijke dingen, ik weet zelf niet
meer wat.

Om half vijf ging ik pas weg.

's Avonds zei hij nog iets, dat ik mooi vond. We hadden het over een lmster die
hij eens van mij gekregen heeft en die nu zeker al anderhalf jaar in zijn kamer
hangt. Hij vond die zo leuk en ik bood hem aan hem eens wat andere filmsterren
te geven.

`Neen', antwoordde hij toen, `ik laat het liever zo; deze hier, daar kijk ik
elke dag tegen aan en dat zijn mijn vrienden geworden'.

Waarom hij Mouschi altijd zo tegen zich aandrukt, begrijp~ik nu ook beter. Hij
heeft natuurlijk ook behoefte aan tederheid.

Nog iets heb ik vergeten, waar hij over sprak. Hij zei: `Bangheid ken ik niet;
alleen~wanneer mijzelf iets mankeert. Maar dat leer ik ook al af'.

Dat minderwaardigheidsgevoel bij Peter is heel erg. Zo denkt hij
bijvoorbeeld~altijd, dat hij zo stom is en wij zo knap zijn. Als ik hem met
Frans help, bedankt hij duizend maal. Ik zal beslist eens zeggen: `Schei uit met
die praatjes, jij kent Engels en aardrijkskunde weer veel beter'.

Je Anne.

\section*{Vrijdag, 18 Februari 1944}

Lieve Kitty,\\
Als ik, wanneer ook, naar boven ga, heeft dat altijd ten doel dat
ik `hem' zal zien.

Mijn leven hier is dus eigenlijk veel beter geworden, omdat het nu weer een doel
heeft en ik me op iets kan verheugen.

Het voorwerp van mijn vriendschap is tenminste altijd in huis en ik hoef,
behalve voor Margot, niet bang voor rivalen te zijn. Denk niet, dat ik verliefd
ben, dat is niet waar, maar ik heb aldoor het gevoel, dat er tussen Peter en mij
nog eens iets moois zal groeien, iets dat vriendschap is en vertrouwen geeft.
Als het maar even kan ga ik naar hem toe en het is niet meer zoals vroeger, dat
hij niet goed weet wat met me te beginnen. Integendeel, hij praat nog als ik al
haast de deur uit ben.

Moeder ziet het niet graag dat ik naar boven ga, ze zegt altijd dat ik Peter
lastig val en dat ik hem met rust moet laten. Zou ze nu heus niet begrijpen, dat
ik ook nog intuïtie heb?

Altijd als ik naar binnen ga in het kleine kamertje, kijkt ze me zo gek aan. Als
ik van boven naar beneden kom, vraagt ze, waar ik geweest ben.  Ik kan dat niet
hebben en vind het erg.

Je Anne.

\section*{Zaterdag, 19 Februari 1944}

Lieve Kitty,\\
Het is weer Zaterdag en dat zegt op zichzelf eigenlijk al
genoeg.\\
De ochtend was rustig. Ik heb boven wat geholpen, maar `hem' heb ik
niet meer~dan vluchtig gesproken. Toen om half drie allen hun kamers hadden
opgezocht óf om te lezen óf om te slapen, toog ik met deken en al naar het
privé-kantoor beneden om aan de schrijftafel te gaan zitten lezen of schrijven.
Het duurde niet erg lang, toen werd het me te machtig, mijn hoofd viel voorover
op mijn arm en ik snikte het uit. De tranen stroomden en ik voelde me diep
ongelukkig. O, was `hij' maar gekomen om me te troosten. Het was al vier uur
toen ik weer naar boven ging. Ik haalde aardappels met weer nieuwe hoop in mijn
hart op een ontmoeting, maar toen ik nog in de badkamer mijn haar aan het
opdoffen was, ging hij naar Mof in het magazijn.

Opeens voelde ik weer de tranen opkomen en snelde naar de W.C., nog vlug
onderweg de handspiegel meepakkend. Daar zat ik dan nu, geheel aangekleed,
terwijl mijn tranen donkere vlekken op het rood van mijn schort maakten en ik
diep bedroefd was.

Ik dacht zo ongeveer: `O, zo bereik ik Peter nooit. Wie weet vindt hij me
helemaal niet aardig en heeft hij geen behoefte aan vertrouwen.  Misschien denkt
hij wel nooit meer dan oppervlakkig aan me. Ik moet weer alleen verder, zonder
vertrouwen en zonder Peter. Wellicht gauw weer zonder hoop, troost en
verwachting. O, kon ik mijn hoofd nu maar tegen zijn schouder aanvlijen en me
niet zo hopeloos alleen en verlaten voelen! Wie weet, geeft hij wel helemaal
niet om me en kijkt hij de anderen ook zo zacht aan. Heb ik het me misschien wel
verbeeld, dat dit voor mij was? O Peter, kon je me maar horen of zien. Doch de
waarheid, die misschien zo teleurstellend is, zou ik niet kunnen verdragen'.

Maar even later voelde ik me toch weer hoopvol en vol verwachting, terwijl de
tranen in me nog vloeiden.

Je Anne.

\section*{Woensdag, 23 Februari 1944}

Lieve Kitty,\\
Sinds gisteren is het buiten heerlijk weer en ik ben helemaal
opgekikkerd. Ik ga~haast elke ochtend naar de zolder waar Peter werkt om de
bedompte kamerlucht uit mijn longen te laten waaien. Vanuit mijn
lievelingsplekje op de grond kijk ik naar de blauwe hemel, naar de kale
kastanjeboom aan wiens takken kleine druppeltjes schitteren, naar de meeuwen en
de andere vogels, die in hun scheervlucht wel van zilver lijken.

Hij stond met zijn hoofd tegen de dikke balk aangeleund, ik zat, we ademden de
lucht in, keken naar buiten en voelden, dat dit iets was om niet met woorden te
onderbreken. We keken heel lang naar buiten en toen hij moest gaan om hout te
hakken op de vliering, wist ik, dat hij een jne kerel is. Hij klom de trap op,
ik volgde en gedurende het kwartier dat hij hout hakte spraken we weer geen
woord. Ik keek van mijn standplaats naar hem, hoe hij zichtbaar zijn best deed
goed te hakken, om zijn kracht aan mij te tonen. Maar ik keek ook uit het open
raam, over een groot stuk van Amsterdam heen, over alle daken tot aan de
horizon, die zo licht blauw was, dat de scheidingslijn niet duidelijk te zien
was. `Zolang dit nog bestaat', dacht ik `en ik het mag beleven, deze
zonneschijn, die hemel waaraan geen wolk is, zolang kan ik niet treurig zijn'.

Voor ieder die bang, eenzaam of ongelukkig is, is stellig het beste middel naar
buiten te gaan, ergens waar hij helemaal alleen is, alleen met de hemel, de
natuur en God. Want dan pas, dan alleen voelt men, dat alles is, zoals het zijn
moet en dat God de mensen in de eenvoudige, maar mooie natuur gelukkig wil zien.
Zolang dit bestaat en dat zal wel~altijd zo zijn, weet ik, dat er in welke
omstandigheden ook, een troost voor elk verdriet is. En ik geloof stellig, dat
bij elke ellende de natuur veel ergs kan wegnemen.

O, wie weet duurt het niet lang meer, dat ik dit overstelpende geluksgevoel kan
delen met iemand, die het net zo ondergaat als ik.

Je Anne.

Gedachte:\\
Wij missen hier veel, zeer veel en zeer lang en ik mis het ook net
zoals jij. Ik heb~het niet over uiterlijke dingen, daarvan zijn we hier
voorzien, neen, ik bedoel de innerlijke dingen. Ik verlang net zoals jij naar
vrijheid en lucht, maar nu geloof ik, dat we voor deze ontberingen ruimschoots
vergoeding hebben gekregen. Dit besefte ik plotseling toen ik vanmorgen voor het
raam zat. Ik bedoel vergoeding van binnen.

Toen ik naar buiten keek en eigenlijk God en de natuur recht en diep aankeek,
toen was ik gelukkig, niet anders dan gelukkig. En Peter, zolang er dat geluk
van binnen is, dat geluk om natuur, gezondheid en nog heel veel meer, zolang men
dat met zich meedraagt, zal men altijd weer gelukkig worden.

Rijkdom, aanzien, alles kan je verliezen, maar dat geluk in eigen hart kan
alleen maar versluierd worden, en zal je steeds opnieuw, zolang je leeft, weer
gelukkig maken. Zolang je onbevreesd tot de Hemel kunt opzien, zolang weet je,
dat je zuiver van binnen bent en dat je toch weer gelukkig zult worden.

\section*{Zondag, 27 Februari 1944}

Liefste Kitty,\\
Van 's ochtends vroeg tot 's avonds laat doe ik eigenlijk niets
anders dan aan Peter~denken. Ik slaap met zijn beeld~voor ogen in, droom van hem
en word weer wakker als hij me nog aankijkt.

Ik heb sterk het gevoel, dat Peter en ik helemaal niet zoveel verschillen als
dat van buiten wel lijkt en ik zal je ook uitleggen waarom. Peter en ik missen
alle twee een moeder. De zijne is te oppervlakkig, flirt graag en bekommert zich
niet veel om zijn~gedachten.  De mijne bemoeit zich wel met me, maar mist het
mooie gevoel, het moederbegrip.

Peter en ik worstelen alle twee met ons binnenste, we zijn alle twee nog onzeker
en eigenlijk te teer en te zacht van innerlijk om hard aangepakt te worden. Doet
men dat toch, dan is mijn reactie daarop de drang van `er uit te willen'. Maar
omdat dat onmogelijk is, verberg ik mijn binnenste, gooi met pannen en water en
ben luidruchtig, zodat iedereen maar wou, dat ik weg was.

Hij daarentegen sluit zich op, spreekt haast niet, is stil, droomt en verbergt
zich daardoor ook angstvallig.

Maar hoe en wanneer zullen wij elkaar eindelijk bereiken?\\
Ik weet niet hoelang
ik met mijn verstand dit verlangen nog de baas kan blijven. Je Anne.

\section*{Maandag, 28 Februari 1944}

Liefste Kitty,\\
Het wordt een nacht- en dagmerrie. Ik zie hem haast elk uur en
kan niet bij hem~komen, ik mag niets tonen, moet vrolijk zijn, terwijl binnen in
me alles wanhopig is.

Peter Wessel en Peter van Daan zijn samengevloeid tot één Peter, die goed en
lief is en waar ik verschrikkelijk naar verlang.

Moeder is lastig, vader lief en daardoor nog lastiger, Margot het lastigste,
want ze maakt aanspraak op een lief gezicht en ik wil rust hebben.

Peter kwam niet bij me op zolder, hij ging naar de vliering en timmerde daar
wat. Met elk gekraak en elke slag brokkelde~er een stukje van mijn moed af en
werd ik nog verdrietiger. enfin de verte speelde een klok: `Rechtop van lijf,
rechtop van ziel!' Ik ben sentimenteel, - ik weet het. Ik ben wanhopig en
onverstandig, - dat weet ik ook.

O help mij! Je Anne.

\section*{Woensdag, 1 Maart 1944}

Lieve Kitty,\\
Mijn eigen aangelegenheden zijn op de achtergrond gedrongen en
wel ... door een~inbraak. Ik word vervelend met mijn inbraken, maar wat kan ik
er aan doen, dat de inbrekers er zo'n plezier in hebben Kolen \& Co. met een
bezoek te vereren? Deze inbraak is veel ingewikkelder dan de vorige van Juli
1943.

Toen mijnheer Van Daan gisteravond als gewoonlijk om half acht naar Kralers
kantoor ging, zag hij de glazen tussendeuren en de kantoordeur open staan. Dit
verwonderde hem, hij liep door en werd steeds verbaasder, toen de kabinetdeuren
eveneens geopend waren en op het voorkantoor een vreselijke rommel heerste.
`Hier was een dief', flitste het door zijn hoofd en om dadelijk zekerheid
omtrent het geval te hebben liep hij de trap af, onderzocht de voordeur, voelde
aan het Lipsslot, alles was dicht. `Och, dan zullen zowel Peter als Elli
vanavond erg slordig geweest zijn', veronderstelde hij nu. Hij bleef een poosje
in Kralers kamer zitten, draaide dan de lamp uit, ging naar boven en maakte zich
noch over de open deuren, noch over het rommelige voorkantoor veel zorgen.

Vanochtend klopte Peter al vroeg aan onze kamerdeur en kwam met het minder
prettige nieuwtje, dat de voordeur wijd openstond. Verder wist hij te vertellen,
dat het projectietoestel en Kralers nieuwe actetas uit de muurkast verdwenen
waren. Peter kreeg opdracht de deur te sluiten, Van Daan vertelde zijn
ondervindingen van de vorige avond en wij waren danig ongerust.

Het hele geval kan niet anders uitgelegd worden dan dat de dief een
namaaksleutel van de deur heeft, want die was in het geheel niet opengebroken.
Hij moet 's avonds al heel vroeg hier binnengeslopen zijn, de deur achter zich
gesloten hebben en door Van Daan gestoord zich verstopt hebben, tot deze
weggegaan was om vervolgens met zijn buit te vluchten, waarbij hij in de haast
de deur open liet staan. Wie kan onze sleutel hebben? Waarom is de dief niet
naar het magazijn gegaan? Zou misschien één van onze eigen magazijnmannen de
dader zijn en zou hij ons nu niet gaan verraden, daar hij Van Daan toch gehoord
en misschien zelfs gezien heeft?

Het is erg griezelig, omdat we niet weten, of de desbetreffende inbreker het
niet in zijn hoofd haalt nog een keer onze deur te openen. Of zou hij misschien
zelf van den man, die hier rondliep, geschrokken zijn?

Je Anne.

\section*{Donderdag, 2 Maart 1944}

Lieve Kitty,\\
Margot en ik waren vandaag boven op zolder; al kan ik met haar
samen niet zo~genieten als ik me dat had voorgesteld, toch weet ik, dat zij bij
de meeste dingen net zo voelt als ik.

Aan de afwas begon Elli met moeder en mevrouw Van Daan over haar mismoedigheid
te praten. Hoe helpen die twee haar?

Weet je wat moeder haar voor raad gaf? Ze moest maar aan al die andere mensen
denken, die ondergaan in deze wereld! Wie helpt de gedachte aan ellende nu, als
hij zelf al ellendig is? Dit zei ik ook, het antwoord was: `Jij kunt over zulke
dingen niet meepraten'.

Wat zijn de volwassenen toch idioot en stom! Alsof Peter, Margot, Elli en ik
niet allen hetzelfde voelen en daartegen helpt alleen moederliefde of liefde van
heel, heel goede vrienden. Maar die moeders hier hebben nog niet zo'n greintje
verstand van ons. Mevrouw Van Daan misschien nog iets meer dan moeder. O, ik zou
die arme Elli zo graag iets gezegd hebben, iets waarvan ik bij ervaring weet,
dat het helpt. Maar vader kwam er tussen en duwde me opzij.

Wat zijn ze allemaal stom! Wij mogen geen oordeel hebben. Ja, ze zijn
verschrikkelijk modern. Geen oordeel hebben! Men kan zeggen, je moet je mond
houden, maar geen oordeel hebben bestaat niet. Niemand kan een ander zijn
oordeel verbieden, al is die ander nog zo jong.

Voor Elli, Margot, Peter en mij helpt alleen een grote en toegewijde liefde, die
we alle vier niet krijgen. En niemand, vooral die idiote wijzen hier kunnen ons
begrijpen, want we zijn veel gevoeliger en veel verder met onze gedachten dan
iemand van hier maar in de verste verte zou vermoeden.

Op het ogenblik zit moeder weer te mopperen; ze is zichtbaar jaloers, omdat ik
tegenwoordig meer met mevrouw Van Daan praat dan met haar.

Vanmiddag heb ik Peter te pakken gekregen, we hebben minstens drie kwartier
samen gepraat. Peter had erg veel moeite om wat van zichzelf te vertellen, het
kwam heel langzaam. Hij vertelde dat zijn ouders zo vaak ruzie hadden over
politiek, sigaretten en over allerlei meer. Hij was erg verlegen.

Ik vertelde hem dan over mijn ouders. Vader verdedigde hij: hij vond hem een
`moordkerel'. Dan hadden we het nog over beneden en boven; hij was heus een
beetje verbaasd, dat wij zijn ouders nog altijd niet graag mogen. `Peter', zei
ik, `je weet dat ik eerlijk ben, waarom zou ik het jou niet vertellen, wij weten
hun fouten toch ook'. Onder andere zei ik nog: `Peter, ik zou je zo graag willen
helpen, kan ik dat niet? Je zit hier zo tussen en ik weet, al zeg je het niet,
dat je je het toch wel aantrekt'.

`O, ik zou altijd van je hulp gebruik willen maken'. `Misschien kan je beter
naar vader toegaan, die vertelt~ook niets verder, die kun je het gerust
vertellen, hoor!'

`Ja, die is een echte kameraad'.\\
`Je houdt veel van hem, hè?' Peter knikte en
ik vervolgde: `Nou, hij ook van jou,~hoor!'

Hij keek vlug en rood op, het was werkelijk ontroerend hoe blij hij met die
paar~woorden was.

`Denk je?' vroeg hij.\\
`Ja', zei ik, `dat kun je wel opmaken uit wat hij zo af
en toe loslaat!' Peter is, net als vader, ook een `moordkerel'!

Je Anne.

\section*{Vrijdag, 3 Maart 1944}

Lieve Kitty,\\
Toen ik vanavond in het kaarsje keek1., werd ik weer blij en
rustig. Oma zit eigenlijk~in dat kaarsje en Oma is het ook, die me behoedt en
beschut en die me weer blij maakt.

Maar ... er is een ander, die mijn hele stemming beheerst en dat is ...  Peter.
Toen ik vandaag de aardappels haalde en nog met de volle pan op de trap stond,
vroeg hij al: `Wat heb je tussen-de-middag gedaan?'

Ik ging op de trap zitten en we begonnen te praten. Om kwart over vijf (een uur
na het halen) kwamen de aardappels, die ik intussen op de grond had gezet, pas
in de kamer aan.

Peter sprak geen woord meer over zijn ouders, we praatten alleen over boeken en
over vroeger. Wat heeft die jongen een warme blik; het scheelt, geloof ik, niet
veel meer of ik word verliefd op hem. Daar had hij het vanavond over. Ik kwam
bij hem binnen, na het aardappelschillen, en zei, dat ik het zo warm had.

`Aan Margot en mij kun je zo de temperatuur zien, als het koud is zijn we wit,
als het warm is rood', zei ik.

`Verliefd?' vroeg hij.

`Waarom zou ik verliefd zijn?' Mijn antwoord was vrij onnozel.\\
`Waarom niet!'
zei hij, en toen moesten we gaan eten.\\
Zou hij met die vraag iets bedoeld
hebben? Ik ben er vandaag eindelijk toe gekomen~om hem te vragen of hij mijn
kletsen niet lastig vond, hij zei alleen: `Ik vind het goed, hoor!'

In hoeverre dit antwoord nu verlegen is, kan ik niet beoordelen.

Kitty, ik ben net als een verliefde, die niets anders te vertellen weet dan van
haar schat. Peter is ook inderdaad een schat. Wanneer zou ik hem dat eens kunnen
vertellen? Natuurlijk alleen als hij mij ook een schat vindt, maar ik ben geen
katje om zonder handschoenen aan te pakken, dat weet ik heus wel. En hij houdt
van zijn rust, dus in hoeverre hij mij aardig vindt, daar heb ik geen idee van.
In ieder geval leren we elkaar een beetje kennen, ik wou maar vast, dat we voor
veel meer dingen zouden durven uitkomen. Wie weet komt die tijd gauwer dan ik
denk! Zo'n paar keer per dag krijg ik een blik van verstandhouding van hem, ik
knipoog terug en we zijn alle twee blij.

Ik lijk wel gek om van zijn blijheid te praten, maar ik heb het onweerstaanbare
gevoel, dat hij net zo denkt als ik.

Je Anne.

\section*{Zaterdag, 4 Maart 1944}

Lieve Kitty,\\
Deze Zaterdag is sinds maanden en maanden eens niet zo vervelend,
treurig en~saai als alle vorige. Niemand anders dan Peter is daarvan de oorzaak.

Vanochtend kwam ik op zolder mijn schort ophangen, toen vader vroeg of ik niet
wou blijven om wat Frans te spreken. Ik vond het goed, we spraken eerst wat
Frans,~ik legde Peter wat uit, toen deden we Engels.  Vader las uit Dickens voor
en ik was de hemel te rijk, want ik zat op vaders stoel dicht naast Peter.

Om 11 uur ging ik naar beneden. Toen ik om half 12 weer boven kwam, stond hij al
op de trap op me te wachten. We praatten tot kwart voor één. Als het maar even
te pas komt, wanneer ik de deur uitga, bijvoorbeeld na het eten, als niemand het
hoort, zegt hij: `Dag Anne, tot straks'.

O, ik ben zo blij! Zou hij nu toch van me gaan houden? In ieder geval is hij een
leuke kerel en wie weet hoe mooi ik met hem praten kan!

Mevrouw vindt het wel goed, als ik met hem praat, maar vandaag vroeg ze plagend:
`Kan ik jullie tweeën daar boven wel vertrouwen?'

`Natuurlijk', zei ik protesterend. `U beledigt me, hoor!'\\
Ik verheug me er van
's ochtends tot 's avonds op Peter te zien. Je Anne.

\section*{Maandag, 6 Maart 1944.}

Lieve Kitty,\\
Aan Peters gezicht kan ik zien, dat hij net zoveel denkt als ik
en gisteravond heb~ik me dan ook geërgerd, toen mevrouw zo spottend zei: `De
denker!' Peter werd er rood en verlegen van en ik zat op springen.

Laat die mensen dan toch hun mond houden!

Je kunt je niet voorstellen, hoe naar het is om werkeloos aan te zien hoe
eenzaam hij is. Ik kan me voorstellen, alsof ik het zelf meegemaakt had, hoe
wanhopig hij soms bij ruzie en liefde moet zijn. Arme Peter, hoezeer heeft hij
liefde nodig!

Hoe hard klonk het in mijn oren, toen hij er van sprak, dat hij geen vrienden
nodig had. O, hoe vergist hij zich! Ik geloof ook, dat hij van die woorden niets
meent!

Hij klampt zich vast aan zijn eenzaamheid, zijn gemaakte onverschilligheid en
zijn volwassenheid om maar niet uit zijn rol te vallen, om nooit, nooit te tonen
hoe hij zich voelt. Arme Peter, hoe lang kan deze rol nog duren? Zal op
die~bovenmenselijke inspanning geen verschrikkelijke uitbarsting volgen?

O Peter, kon en mocht ik je maar helpen! Wij samen zouden ons beider~eenzaamheid
wel verdrijven!

Ik denk veel, maar zeg niet veel. Ik ben blij als ik hem zie en als de zon
daarbij~nog schijnt. Gisteren was ik bij het hoofdwassen erg uitgelaten; ik wist
wel, dat hij in de kamer naast ons zat. Ik kon er niets aan doen, hoe stiller en
ernstiger ik van binnen ben, des te luidruchtiger ben ik van buiten.

Wie zal de eerste zijn, die dit pantser ontdekt en doorbreekt? Het is toch goed
dat de Van Daans geen meisje hebben, nooit zou mijn verovering zo moeilijk, zo
mooi en zo mooi kunnen zijn, als niet juist het andere geslacht zou trekken.

Je Anne.

P.S. Je weet, dat ik je eerlijk alles schrijf, daarom moet ik je ook zeggen, dat
ik eigenlijk van de ene ontmoeting op de andere leef. Steeds hoop ik te
ontdekken, dat hij ook zo op mij wacht en ik ben in mezelf opgetogen, als ik
zijn kleine verlegen pogingen merk. Hij zou volgens mij zo graag net zo uit zijn
woorden komen als ik; hij weet ook niet, dat juist zijn onbeholpenheid me treft.

Je Anne.

\section*{Dinsdag, 7 Maart 1944}

Lieve Kitty,\\
Als ik nu zo over mijn leventje van 1942 nadenk, doet dat alles
zo onwerkelijk~aan. Dit godenleventje beleefde een heel andere Anne dan die, die
hier nu wijs geworden is. Een godenleventje, dat was het. Aan elke hoek
aanbidders, een stuk of twintig vriendinnen en kennisjes, de lieveling van het
merendeel der leraren, verwend door vader en moeder van boven tot onder, veel
snoep, genoeg geld, wat wil je meer?

Je zult me allicht willen vragen hoe ik al die mensen dan zo ingepalmd heb. Wat
Peter zegt: `Aantrekkelijkheid', is toch niet helemaal waar.  Elke leraar vond
mijn gewiekste antwoorden, mijn grappige opmerkingen, mijn lachende gezicht en
mijn critische blik leuk, amusant en grappig.  Meer was ik ook niet; een
verschrikkelijke flirt, coquet en amusant. Een paar voordelen had ik, waardoor
ik tamelijk in de gunst bleef, namelijk vlijt, eerlijkheid en gulheid. Nooit zou
ik wie dan ook geweigerd hebben om af te kijken, snoep verdeelde ik met ruime
hand en ik was niet verwaand.

Zou ik van al die bewondering niet overmoedig geworden zijn? Het is één geluk,
dat ik middenin, op het hoogtepunt van het feest, opeens in de werkelijkheid
kwam te staan, en het heeft ruim een jaar geduurd vóór ik er aan gewend was, dat
er nergens meer bewondering kwam.

Hoe zagen ze me op school? De aanvoerster bij grappen en grapjes, altijd haantje
de voorste, nooit in een slechte bui of huilerig. Was het een wonder dat
iedereen graag met me fietste of me een attentie bewees?

Ik kijk op die Anne neer, alsof ze een leuk, maar erg oppervlakkig meisje was,
dat met mij niets meer te maken heeft. Peter zei zeer terecht over me: `Als ik
jou zag, was je steevast door twee of meer jongens en een troep meisjes omringd.
Altijd lachte je en was je het middelpunt!'

Wat is er van dit meisje nog overgebleven? O zeker, ik heb mijn lachen en mijn
antwoorden nog niet verleerd, ik kan nog net zo goed of nog beter de mensen
becritiseren, ik kan nog flirten, als ik ... wil. Daar zit het hem in, ik wil
nog wel eens voor een avondje, een paar dagen, een week zo leven, schijnbaar
onbezorgd en vrolijk. Maar aan het eind van die week zou ik doodaf zijn en de
eerste de beste, die over iets behoorlijks zou praten, erg dankbaar zijn. Ik wil
geen aanbidders, maar vrienden, geen bewonderaars voor een vleiend lachje, maar
voor optreden en karakter. Ik weet heel goed, dat dan de kring om mij heen veel
kleiner~zou zijn. Maar wat hindert dat, als ik nog maar een paar mensen,
oprechte mensen overhoud?

Ondanks alles was ik in 1942 ook niet onverdeeld gelukkig; ik voelde me vaak
verlaten, maar omdat ik van 's morgens tot 's avonds in de weer was, dacht ik
niet na en maakte pret zoveel ik kon. Bewust of onbewust probeerde ik met
grapjes de leegte te verdrijven. Nu bekijk ik mijn eigen leven en werk. Eén
tijdperk er van is onherroepelijk afgesloten.  De onbezorgde, onbekommerde
schooltijd komt nooit meer terug.

Ik verlang er niet eens meer naar, ik ben er boven uit gegroeid, ik kan niet
alleen maar plezier maken, een deeltje van me bewaart altijd zijn ernst.

Ik zie mijn leventje tot aan Nieuwjaar 1944 als onder een scherpe loupe.  Thuis
het leven met veel zon, dan in 1942 hier, de plotselinge overgang, de ruzies, de
beschuldigingen. Ik kon het niet vatten, ik was overrompeld en wist niets anders
om mijn houding te bewaren dan brutaliteit.

De eerste helft van 1943; mijn huilbuien, de eenzaamheid, het langzame inzien
van al die fouten en gebreken, die zo groot zijn en nog dubbel zo groot leken.
Ik praatte overdag over alles heen, probeerde Pim naar me toe te trekken; het
lukte niet. Ik stond alleen voor de moeilijke taak me zó te maken, dat ik geen
verwijten meer zou horen, want die drukten me neer tot een vreselijke
moedeloosheid.

De tweede helft van het jaar werd het iets beter, ik werd bakvis, werd meer als
volwassene beschouwd. Ik begon te denken, verhalen te schrijven en kwam tot de
slotsom, dat de anderen geen recht meer hadden mij als een slingerbol van links
naar rechts te trekken. Ik wilde mezelf hervormen naar mijn eigen wil. Maar één
ding dat me nog méér raakte was, dat ik inzag, dat ook vader nooit mijn
vertrouwde in alles zou worden.  Ik wilde niemand meer vertrouwen dan mezelf.

Na Nieuwjaar: de tweede grote verandering, mijn droom ... Daarmee ontdekte ik
mijn verlangen naar een jongen; niet naar een meisjesvriendin, maar naar een
jongens-vriend. Ontdekte ook het geluk in mezelf en mijn pantser van
oppervlakkigheid en vrolijkheid. Bij tijd en wijle werd ik stil, ontdekte mijn
grenzenloos verlangen naar alles wat mooi en goed is.

En 's avonds, als ik in bed lig en mijn gebed eindig met de woorden: `Ik dank je
voor al het Goede en Lieve en Mooie', dan jubelt het in mij. Dan denk ik aan
`het Goede' van het onderduiken, van mijn gezondheid en mijn hele zelf, aan `het
Lieve' van Peter, dat wat nog klein en gevoelig is en wat we alle twee nog niet
durven noemen of aanraken, dat wat eens komen zal, de liefde, de toekomst, het
geluk en aan `het Mooie' dat de wereld is; de wereld, de natuur, de schoonheid
en al, al het mooie bij elkaar.

Dan denk ik niet aan al de ellende, maar aan het mooie dat nog overblijft.
Hierin ligt voor een groot deel het verschil tussen moeder en mij. Haar raad
voor zwaarmoedigheid is: `Denk aan al de ellende in de wereld en wees blij, dat
jij die niet beleeft!' Mijn raad is: `Ga naar buiten, naar de velden, de natuur
en de zon, ga naar buiten en probeer het geluk in jezelf te hervinden enfin God.
Denk aan al het mooie dat er in en om jezelf nog overblijft en wees gelukkig!'

Volgens mij kan moeders zin niet opgaan, want wat moet je doen, als je zelf de
ellende beleeft? Dan ben je verloren. Daarentegen vind ik, dat er nog altijd
iets moois overblijft, aan de natuur, de zonneschijn, de vrijheid, aan jezelf,
daar heb je wat aan. Kijk daarnaar, dan vind je jezelf weer en God, dan word je
evenwichtig.\\
En wie gelukkig is, zal ook anderen gelukkig maken, wie moed en
vertrouwen heeft, zal nooit in de ellende ondergaan!

Je Anne.

\section*{Zondag, 12 Maart 1944}

Lieve Kitty,\\
De laatste tijd heb ik geen zitvlees meer; ik loop van boven naar
beneden en weer~terug. Ik vind het mooi met Peter te praten, maar ben aldoor
bang, dat ik hem lastig val. Hij heeft me het een en ander over vroeger, zijn
ouders en zichzelf verteld. Ik vind het veel te weinig en vraag me af, hoe ik er
toe kom méér te verlangen. Hij vond me vroeger onuitstaanbaar; het ging vice
versa, nu ben ik van mening veranderd, moet hij dan ook veranderd zijn?

Ik denk van wel, toch sluit dat nog niet in, dat we dikke vrienden moeten
worden, hoewel ik van mijn kant daardoor het hele onderduiken lichter zou kunnen
dragen. Maar laat ik me niet overstuur maken; ik houd me genoeg met hem bezig en
hoef jou niet samen met mij verdrietig te maken, omdat ik me zo lamlendig voel.

Zaterdagmiddag was ik, na een rijtje droevige berichten van buiten, zo
doorgedraaid, dat ik op mijn divan ging liggen om te slapen. Ik wou niets dan
slapen om niet te moeten nadenken. Ik sliep tot vier uur, moest dan naar binnen.
Het viel me heel erg moeilijk om moeder op al haar vragen te antwoorden en voor
vader een smoesje te bedenken, dat mijn slapen verklaarde. Ik wendde hoofdpijn
voor, wat niet gelogen was, daar ik ook hoofdpijn had ... van binnen!

Gewone mensen, gewone meisjes, bakvissen zoals ik, zullen me wel getikt vinden
in mijn zelfbeklag. Maar ja, dat is het juist, ik zeg jou alles wat op mijn hart
ligt en ben de verdere dag zo brutaal, vrolijk en zelfbewust als maar mogelijk
is, om alle vragen te vermijden en me niet inwendig aan mezelf te ergeren.

Margot is erg lief en zou graag mijn vertrouwde zijn, ik kan haar toch niet
alles zeggen. Zij is lief en goed en mooi, maar ze mist de nonchalance om over
diepere dingen te spreken;~ze neemt me ernstig, veel te ernstig op en denkt lang
over haar gekke zusje na, kijkt me bij alles wat ik zeg onderzoekend aan en
denkt bij alles: `Is het nu komedie of meent ze het werkelijk?' Dat komt, omdat
we aldoor samen zijn en ik mijn vertrouwde niet aldoor om me heen zou kunnen
hebben.

Wanneer kom ik weer uit die warboel van gedachten, wanneer zal er weer rust en
vrede in me zijn?

Je Anne.

\section*{Dinsdag, 14 Maart 1944}

Lieve Kitty,\\
Het is misschien vermakelijk voor jou - voor mij allerminst - om
te horen wat we~vandaag zullen eten. Op het ogenblik zit ik, daar de werkster
beneden is, aan de zeildoektafel bij de Van Daans, heb een zakdoek in mijn mond
en tegen mijn neus geduwd, die doordrenkt is van welriekend vóórduiks parfum. Je
zult er op die manier niet veel van snappen, dus weer:

`Bij het begin beginnen'.

Daar onze bonnenleveranciers opgepakt zijn, hebben wij, behalve onze vijf
levensmiddelenkaarten geen extra bonnen en geen vet. Daar zowel Miep als
Koophuis ziek zijn, kan Elli geen boodschappen doen en daar de hele stemming
mistroostig is, is het eten dat ook. Vanaf morgen hebben we geen brokje vet,
boter of margarine meer. We ontbijten niet meer met gebakken aardappelen
(broodbesparing), maar met pap en daar mevrouw denkt dat we verhongeren, hebben
we clandestien volle melk gekocht. Ons middageten van vandaag is
boerenkoolstamppot uit het vat. Vandaar de voorzorgsmaatregel met de zakdoek.
Ongelooflijk zoals boerenkool stinken kan, als die een jaar oud is. Het ruikt
hier in de kamer naar een mengsel van bedorven pruimen, scherp conserveermiddel
en rotte eieren.  Bah, ik word al misselijk bij het idee alleen, dat ik die
rommel eten moet.

Hierbij komt nog, dat onze aardappels zulke vreemdsoortige ziektes opgelopen
hebben, dat van twee emmers pommes de terre er één in de kachel belandt. We
vermaken ons er mee precies de verschillende ziektes uit te zoeken en zijn tot
de conclusie gekomen, dat kanker, pokken en mazelen elkaar afwisselen. O neen,
het is geen pretje om in het vierde oorlogsjaar ondergedoken te zijn. Was al die
rotzooi maar afgelopen!

Eerlijk gezegd zou al het eten me niet zoveel kunnen schelen, als het overigens
maar wat plezieriger hier was. Daar zit hem de kneep: ons allemaal begint dit
saaie leven kregel te maken.

Hier volgt de mening van vijf volwassen onderduikers over de tegenwoordige
toestand:

Mevrouw Van Daan:

`Het baantje van keukenprinses staat me al lang niet meer aan. Zitten en niets
onderhanden hebben verveelt, dus kook ik maar weer. Toch moet ik klagen, zonder
vet koken is onmogelijk, ik word misselijk van al die vieze geurtjes. Niets dan
ondankbaarheid en geschreeuw is het loon voor mijn moeite; ik ben altijd het
zwarte schaap, van alles krijg~\emph{ik}~de schuld. Verder is mijn opvatting,
dat de oorlog niet veel voortgang maakt, de Duitsers zullen tenslotte de
overwinning nog wegdragen. Ik ben vreselijk bang, dat we verhongeren en kaffer
ieder uit als ik een slecht humeur heb'.

Mijnheer Van Daan:

`Ik moet roken, roken, roken, dan is eten, politiek, Kerli's humeur allemaal
niet zo erg. Kerli is een lieve vrouw'.

Maar als hij niets te roken krijgt, deugt niets, dan hoort men: `Ik word ziek,
we leven te slecht, ik moet vlees hebben. Verschrikkelijk dom mens, die Kerli
van me!' Daarna volgt er beslist een hooglopende ruzie.

Mevrouw Frank:

`Het eten is niet erg belangrijk, maar nu zou ik graag een sneetje roggebrood
hebben, want ik heb vreselijke honger.

Als ik mevrouw Van Daan was, had ik achter dat eeuwigdurende roken van mijnheer
al lang een stokje gestoken. Maar nu moet ik beslist een sigaret hebben, want
mijn zenu wen zijn me de baas.

De Engelsen maken veel fouten, maar de oorlog gaat vooruit, ik moet praten en
blij zijn dat ik niet in Polen zit'.

Mijnheer Frank:

`Alles goed, heb niets nodig. Kalm aan, we hebben de tijd. Geef mij mijn
aardappels, dan houd ik mijn mond. Gauw van mijn rantsoen nog wat voor Elli
apart leggen. De politiek loopt in uitstekende banen, ik ben reuze
optimistisch!'

Mijnheer Dussel:

`Ik moet mijn pensum halen alles op zijn tijd afmaken. Politiek
``oitschtekent'', dat we gepakt worden is ``unmooglijk''.

Ik, ik, ik,..!' Je Anne.

\section*{Woensdag, 15 Maart 1944}

Lieve Kitty,\\
Pf, hè, hè, eventjes van de sombere voortaferelen bevrijd!\\
Ik
hoor vandaag niets anders dan `als dit en dat gebeurt, dan zullen we
daarmee~moeilijkheden krijgen en als die nog ziek wordt, zitten we als alleen op
de wereld, en als dan ...' En n, de rest weet je wel, ik vermoed tenminste dat
je de Achterhuizers intussen voldoende kent om hun gesprekken te kunnen raden.

De aanleiding van de `als, als' is, dat mijnheer Kraler opgeroepen is om te
spitten, Elli meer dan snipverkouden is en waarschijnlijk morgen thuis moet
blijven, Miep van haar griep nog niet genezen is en Koophuis een maagbloeding
met bewusteloosheid heeft gehad. Een ware treurlijst!

De magazijnmensen hebben morgen een dag vrij gekregen; mocht Elli thuisblijven,
dan blijft de deur op slot en wij moeten muisstil zijn, opdat de buren ons niet
horen. Henk komt dan de verlatenen om één uur bezoeken en speelt als~het ware de
rol van Artisoppasser. Hij heeft ons vanmiddag voor de eerste keer sinds lange
tijd weer eens iets van de grote wereld verteld. Je had ons achten eens om hem
heen moeten zien zitten, het leek sprekend op een plaatje: Als grootmoeder aan
het vertellen is. - Hij kletste honderd uit voor zijn dankbaar publiek,
natuurlijk over het eten en dan over Mieps dokter, waar wij hem naar vroegen.
`Dokter, praat me niet van den dokter! Ik belde hem vanochtend op, kreeg een
assistentje aan de telefoon en vroeg een recept tegen de griep. Het antwoord
was, dat ik 's morgens tussen acht en negen een recept kon komen halen. Als je
een zeer zware griep hebt, komt de dokter zelf even aan de telefoon en zegt:
``Steek uw tong eens uit, zeg eens aah. Ik hoor het al, u hebt een rode keel. Ik
schrijf wel een receptje voor u uit, kunt u bij de apotheek bestellen, dag
mijnheer.''' En daarmee basta. Makkelijke praktijk is dat, uitsluitend bediening
door de telefoon.

Maar laat ik den dokters niets verwijten, tenslotte heeft ieder mens maar twee
handen enfin de tegenwoordige tijd is er een overvloed aan patiënten en een
minimaal aantal dokters. Toch moesten wij allen even lachen, toen Henk dat
telefoongesprek weergaf.

Ik kan me echt voorstellen hoe een dokterswachtkamer er tegenwoordig uitziet.
Men kijkt niet meer neer op buspatiënten, maar op mensen die niets ergs mankeert
en denkt daarbij: `Mens, wat heb jij hier te zoeken, achter in de rij, hoor,
erge patiënten hebben voorrang'.

Je Anne.

\section*{Donderdag, 16 Maart 1944}

Lieve Kitty,\\
Het weer is heerlijk, niet te beschrijven mooi; ik ga straks gauw
naar de zolder. Nu weet ik wel, waarom ik zoveel onrustiger ben dan Peter. Hij
heeft een eigen~kamer waar hij in werkt, droomt,~denkt en slaapt. Ik word van de
eene hoek naar de andere geduwd.  Alleen in mijn gedeelde kamer ben ik haast
nooit en toch verlang ik daarnaar zo erg. Dat is dan ook de reden, dat ik zo
vaak naar de zolder vlucht. Daar en bij jou kan ik even, heel even mezelf zijn.
Toch wil ik over mijn verlangen niet zeuren, integendeel, ik wil moedig zijn. De
anderen kunnen gelukkig niets van mijn innerlijke gevoelens merken, behalve dat
ik met de dag koeler tegenover moeder sta, vader minder liefkoos en ook
tegenover Margot niets meer loslaat. Ik ben potdicht.  Vóór alles moet ik mijn
uiterlijke zekerheid bewaren, niemand mag weten, dat er in me nog altijd oorlog
heerst. Oorlog tussen mijn verlangen en verstand. Tot nu toe heeft het laatste
de overwinning behaald, maar zal niet toch het eerste het sterkste blijken? Soms
vrees en vaak verlang ik van wel!

O, het is zo vreselijk moeilijk nooit wat tegenover Peter los te laten, maar ik
weet dat hij het eerst moet beginnen; het is zo moeilijk, al die gesprekken en
daden, die ik in mijn dromen beleefd heb, overdag weer ongedaan te moeten
bevinden! Ja Kitty, Anne is een gek kind, maar ik leef ook in een gekke tijd
enfin nog gekkere omstandigheden.

Het mooiste van alles vind ik nog, dat ik wat ik denk en voel tenminste nog kan
opschrijven, anders zou ik compleet stikken! Wat denkt Peter wel over al deze
dingen? Steeds weer hoop ik dat ik op een dag met hem daarover kan spreken. Er
moet toch iets zijn, dat hij van me geraden heeft, want van de uiterlijke Anne,
die hij tot nu toe kent, kan hij toch niet houden.

Hoe kan hij, die zo van rust en vrede houdt, sympathie voor mijn lawaai en
drukte voelen? Kan hij de eerste en enige op de wereld zijn, die achter mijn
betonnen masker gekeken heeft? Zal hij daarachter dan ook gauw belanden? Is er
niet een oud gezegde, dat liefde vaak op medelijden volgt, of dat die twee hand
in hand gaan? Is dat niet ook~met mij het geval? Want ik heb net zoveel
medelijden met hem, als ik het vaak ook met mezelf heb!

Ik weet heus niet, heus niet hoe ik de eerste woorden moet vinden en hoe zou hij
het dan kunnen, die nog veel moeilijker spreekt? Kon ik hem maar schrijven, dan
wist ik tenminste dat hij weet wat ik wou zeggen, want met woorden is het zo
ontzettend moeilijk!

Je Anne.

\section*{Vrijdag, 17 Maart 1944}

Lieve Kitty,\\
Door het Achterhuis waait de wind van opluchting. Kraler is door
de grote raad~vrijgesproken. Elli heeft haar neus een tikje gegeven en hem
streng verboden haar vandaag te hinderen. Alles weer all-right, behalve dat
Margot en ik onze ouders een beetje moe worden.  Je moet me niet verkeerd
begrijpen, met moeder kan ik momenteel niet goed opschieten, dat weet je wel,
van vader houd ik nog net zo veel en Margot van vader en moeder, maar als je zo
oud bent als wij zijn, wil je toch ook een beetje voor jezelf beslissen, wil je
ook zelf eens van oudershandje af.

Als ik naar boven ga, wordt er gevraagd wat ik doen wil, zout mag ik bij het
eten niet hebben, om kwart over acht vraagt moeder steevast elke avond, of ik me
nog niet uitkleden moet, elk boek dat ik lees moet gekeurd worden. Eerlijk
gezegd is die keuring helemaal niet streng en ik mag haast alles lezen, en toch,
al die op- en aanmerkingen plus al dat vragen de hele dag door vinden we beiden
lastig.

Nog iets is vooral bij mij niet naar hun zin, ik wil niet meer zoentjes hier en
kusjes daar geven, gefantaseerde naampjes vind ik aanstellerig, kortom, ik zou
ze wel een poosje kwijt willen. Margot zei gisteravond nog: `Ik vind het echt
vervelend, dat ze, als je even met je hand onder je hoofd twee keer zucht, al
vragen of je hoofdpijn hebt of je niet goed voelt!'

Het is voor ons allebei een hele klap, dat we nu opeens~zien hoe weinig van dat
vertrouwelijke en harmonieuze van thuis over is. En dat komt toch voor een groot
deel, omdat we zo scheef zitten. Hiermee bedoel ik, dat we als kinderen
behandeld worden, wat de uiterlijke dingen betreft en we veel ouder zijn dan
meisjes van onze leeftijd in het algemeen, wat innerlijke dingen betreft.

Al ben ik pas 14, ik weet toch heel goed wat ik wil, ik weet wie gelijk en
ongelijk heeft, ik heb mijn mening, mijn opvatting en mijn principes en al
klinkt het misschien gek voor een bakvis, ik voel me mens, veel meer dan kind,
ik voel me helemaal onafhankelijk van welke andere ziel ook.

Ik weet, dat ik beter debatteren of discuteren kan dan moeder, ik weet, dat ik
objectiever ben, dat ik niet zo overdrijf, netter en handiger ben en daardoor -
je kunt er om lachen - voel ik me in heel veel dingen haar meerdere. Als ik van
iemand houd, moet ik in de eerste plaats bewondering voor hem hebben,
bewondering en respect. Alles zou goed zijn, als ik Peter maar heb, want voor
hem heb ik bewondering in heel veel dingen. Hij is zo'n mooie en knappe jongen!

Je Anne.

\section*{Zondag, 19 Maart 1944}

Lieve Kitty,\\
Gisteren was een heel belangrijke dag voor mij. Ik had besloten
het nu maar met~Peter uit te praten. Net voor we aan tafel gingen, fluister de
ik hem toe: `Doe je vanavond steno, Peter?' `Neen', was zijn antwoord. `Ik wou
je dan later even spreken!' Dat vond hij goed. Na de afwas ging ik fatsoenshalve
eerst aan het raam bij zijn ouders staan, maar het duurde niet lang of ik ging
naar Peter toe. Hij stond aan de linkerkant van het open raam, ik ging aan de
rechter staan en we praatten. Het was veel makkelijker bij het open raam in het
betrekkelijk donker te praten dan bij zoveel licht, en ik geloof, dat Peter dat
ook vond.

We hebben elkaar zoveel verteld, zo ontzettend veel,~dat ik het niet allemaal
kan herhalen, maar het was mooi; de mooiste avond, die ik ooit in het Achterhuis
heb gehad. In het kort zal ik je de verschillende onderwerpen toch wel zeggen.
Eerst hadden we het over de ruzies, dat ik daar nu heel anders tegenover sta,
dan over de verwijdering van ons tegenover onze ouders.

Ik vertelde Peter van moeder en vader, van Margot en van mijzelf.

Op een gegeven ogenblik vroeg hij: `Jullie zeggen elkaar zeker altijd
goedennacht met een nachtzoen?'

`Eén, een heleboel, jij niet, hé?'\\
`Neen, ik heb haast nooit iemand een zoen
gegeven'.\\
`Ook niet voor je verjaardag?'\\
`Ja, dan wel'.\\
We hadden het over
vertrouwen, dat we alle twee onzen ouders niet geschonken~hebben, dat zijn
ouders wel graag zijn vertrouwen wilden hebben, maar dat hij dat niet wilde. Dat
ik mijn verdriet in bed uithuil en hij op de vliering gaat vloeken. Dat Margot
en ik elkaar ook pas zo kort kennen en dat we elkaar toch niet alles vertellen,
omdat we altijd bij elkaar zijn. Over alles en nog wat, o hij was net zoals ik
wel wist dat hij was!

Toen kwamen we over 1942 te praten, hoe anders we toen waren. We kennen elkaar
nu in die personen alle twee niet meer terug. Hoe we elkaar in het begin niet
konden uitstaan. Hij vond me druk en lastig en ik vond al gauw aan die hele
jongen niets. Ik begreep niet, dat hij niet flirtte, maar nu ben ik blij. Hij
had het er ook nog over, dat hij zich zoveel afzonderde. Ik zei, dat tussen mijn
lawaai en zijn stilte niet zoveel verschil ligt. Dat ik ook van rust houd en
niets voor me alleen heb, behalve mijn dagboek. Dat hij blij is, dat mijn ouders
kinderen hier hebben en dat ik blij ben, dat hij hier is. Dat ik hem nu wel
begrijp in zijn teruggetrokkenheid en zijn verhouding tot zijn ouders en dat ik
hem zo graag zou helpen. `Je helpt me toch altijd', zei hij. `Waarmee
dan?'~vroeg ik heel verbaasd. `Met je vrolijkheid!' Dat was wel het mooiste, wat
hij me gezegd heeft. Het was heerlijk, hij moet van me zijn gaan houden, als
kameraad en dat is voorlopig genoeg. Ik heb er geen woorden voor, zo dankbaar en
blij ben ik en ik moet me wel bij je verontschuldigen, Kitty, dat mijn stijl
beneden peil is vandaag.

Ik heb maar opgeschreven wat me ingevallen is. Nu heb ik het gevoel, dat Peter
en ik een geheim delen. Als hij me aankijkt met die ogen, die lach en dat
knipoogje, gaat er bij mij van binnen net een lichtje aan. Ik hoop dat het zo
blijven mag, dat we nog vele, vele prettige uren samen mogen doorbrengen!

Je dankbare en blijde Anne.

\section*{Maandag, 20 Maart 1944}

Lieve Kitty,\\
Vanmorgen vroeg Peter, of ik 's avonds eens meer kwam en zei, dat
ik hem heus~niet stoorde, dat er bij hem in de kamer, zo goed als er plaats voor
één is, ook plaats voor twee is. Ik zei dat ik niet elke avond kon, dat ze dat
beneden niet goed vonden, maar hij vond, dat ik me daar niet aan moest storen.
Ik zei toen, dat ik 's Zaterdagavonds graag eens zou komen en vroeg hem me
vooral eens te waarschuwen als de maan er was. `Dan gaan we naar beneden',
antwoordde hij, `daar de maan bekijken'.

Intussen is er een schaduw op mijn geluk gevallen. Ik dacht het wel al lang, dat
ook Margot Peter meer dan aardig vond. In hoeverre ze van hem houdt weet ik
niet, maar ik vind het erg naar. Elke keer, dat ik Peter nu ontmoet, moet ik
haar opzettelijk pijn doen en het mooiste is, dat zij haast niets laat merken.

Ik weet wel, dat ik wanhopig van jaloezie zou zijn, maar Margot zegt alleen, dat
ik geen medelijden met haar moet hebben.

`Ik vind het zo naar, dat jij er als derde beentje bijstaat',~voegde ik er nog
aan toe. `Dat ben ik gewend', antwoordde zij enigszins bitter.

Dit durf ik Peter toch nog niet te vertellen, misschien later eens, we moeten
eerst~nog zo ontzettend veel uitpraten.

Moeder heeft me gisteravond een kleine opstopper gegeven, die ik heus
wel~verdiend heb. Ik mag me in mijn onverschilligheid tegenover haar niet te ver
laten gaan. Dus maar weer proberen ondanks alles vriendelijk te zijn en mijn
aanmerkingen achterwege te laten.

Ook Pim is niet meer zo hartelijk. Hij probeert nu me minder kinderachtig te
behandelen en is nu veel te koud. Zien wat daarvan komt!

Voorlopig genoeg, ik kan niets anders dan naar Peter kijken en ben boordevol! Je
Anne.

Bewijs van Margots goedheid; dit ontving ik vandaag de 20ste Maart 1944:

`Anne, toen ik gisteren zei dat ik niet jaloers op je was, was ik maar voor 50
\% eerlijk. Het geval is namelijk zo, dat ik noch op jou, noch op Peter jaloers
ben. Ik vind het alleen voor mezelf een beetje jammer, dat~\emph{ik}~nog niemand
gevonden heb en voorlopig zeker niet zal vinden, met wien ik over mijn gedachten
en gevoelens zou kunnen spreken.  Maar daarom zou ik het jullie alle twee van
harte gunnen, als jullie elkaar iets van je vertrouwen zou kunnen schenken. Je
mist hier al genoeg, wat voor vele anderen zo vanzelfsprekend is.

Aan de andere kant weet ik zeker, dat ik met Peter toch nooit zover gekomen zou
zijn, omdat ik het gevoel heb dat ik met dengene, met wien ik veel zou willen
bespreken, op een tamelijk intieme voet zou moeten staan. Ik zou het gevoel
moeten hebben dat hij mij, ook zonder dat ik veel zeg, door en door begrijpt.
Maar daarom moet het iemand zijn, in wien ik geestelijk mijn meerdere voel en
dat is bij~Peter nooit het geval. Bij jou zou ik mij het bovenstaande wel met
Peter kunnen indenken.

Je hoeft je dus helemaal geen verwijten te maken, dat ik te kort kom en dat jij
iets doet wat mij toekwam; niets is minder waar. Jij en Peter zullen beiden
slechts kunnen winnen in de omgang met elkaar'.

Antwoord van mij:

`Lieve Margot,\\
Je briefje vond ik buitengewoon lief, maar ik ben er toch nog
niet helemaal gerust~op en zal dat ook niet worden.

Van vertrouwen in die mate als jij bedoelt is tussen Peter en mij voorlopig
nog~geen sprake, maar aan een open en donker raam zeg je elkaar meer dan in de
lichte zonneschijn. Ook kun je je gevoelens beter uisterend oververtellen dan
als je ze zo moet uitbazuinen. Ik geloof dat jij voor Peter een soort
zusterlijke genegenheid bent gaan voelen en hem graag wilt helpen, minstens zo
graag als ik. Misschien zal je dat ook nog eens kunnen doen, hoewel dat geen
vertrouwen volgens onze bedoeling is. Want ik vind, dat vertrouwen van twee
kanten moet komen; ik geloof dat dat ook de reden is, dat het tussen vader en
mij nooit zo ver is gekomen.

Laten wij er nu over uitscheiden en praat er ook niet meer over; als je nog iets
wilt, doe het dan alsjeblieft schriftelijk, want zo kan ik veel beter zeggen wat
ik bedoel dan mondeling.

Je weet niet hoe ik je bewonder en ik hoop alleen, dat ik nog eens iets van
vaders en jouw goedheid krijg, want daarin zie ik tussen jullie beiden niet veel
verschil meer.

Je Anne'.

\section*{Woensdag, 22 Maart 1944}

Lieve Kitty,\\
Dit ontving ik gisteravond van Margot:

`Beste Anne,\\
Na je briefje van gisteren heb ik het onaangename gevoel, dat je
gewetenswroeging~voelt als je bij Peter gaat werken of praten; daar is heus geen
reden toe. In mijn binnenste heeft iemand recht op wederzijds vertrouwen en ik
zou Peter nog niet in zijn plaats kunnen dulden.

Het is echter zo, net als jij schreef, dat ik het gevoel heb, dat Peter een
soort broer is, maar ... een jongere broer en dat onze gevoelens voelhorens naar
elkaar uitsteken om elkaar misschien later, misschien ook nooit, in een
genegenheid als van broer en zuster te raken; zover is het echter nog lang niet.

Je hoeft dus heus geen medelijden met mij te hebben. Geniet maar zoveel je kunt
van het gezelschap, dat je nu hebt gevonden'.

Het wordt hier intussen steeds mooier. Ik geloof, Kitty, dat we hier in het
Achterhuis misschien nog een echte, grote liefde krijgen. Ik denk heus niet over
trouwen met hem, hoor, ik weet niet hoe hij, als hij volwassen is, eens zal
worden. Ik weet ook niet of we eens zoveel van elkaar zullen houden, dat we
graag zouden trouwen. Dat Peter ook van mij houdt, daarvan ben ik intussen
zeker; op welke manier hij van me houdt, weet ik niet.

Of hij alleen een goede kameraad wenst, of dat ik hem als meisje aantrek, of wel
als zuster, ik kom er niet goed achter.

Toen hij zei, dat ik hem in de ruzies tussen zijn ouders altijd help, was ik
reuze blij en al een stap op weg om aan zijn vriendschap te geloven. Gisteren nu
vroeg ik hem wat hij zou doen als er een dozijn van die Anne's hier waren en
altijd bij hem zouden komen. Zijn antwoord was: `Als ze allemaal waren zoals
jij, zou dat heus niet zo erg zijn!' Hij is voor mij reuze gastvrij en ik geloof
wel, dat hij me werkelijk graag ziet komen. Frans leert hij intussen erg
ijverig, zelfs tot 's avonds kwart over tienen in bed. O, als ik nog~aan
Zaterdagavond denk, aan onze woorden, onze stemming, dan ben ik voor de eerste
keer over mezelf tevreden; ik bedoel dan, dat ik nu hetzelfde zeggen zou en niet
alles anders, wat anders meestal het geval is.

Hij is zo mooi, zowel als hij lacht als wanneer hij zo stil voor zich uitkijkt,
hij is zo lief en goed. Volgens mij is hij het meest door mij overrompeld, toen
hij merkte, dat ik helemaal niet de oppervlakkige Anne van de wereld ben, maar
net zo'n dromerig exemplaar met net zoveel moeilijkheden als hijzelf.

Je Anne.

Antwoord:

`Lieve Margot,\\
Het beste vind ik, dat we nu maar afwachten wat er van komt.
Het kan niet meer~zo heel lang duren, dat de beslissing tussen Peter en mij
valt, òf weer gewoon òf anders. Hoe dat moet gaan weet ik niet, ik denk in deze
nog maar niet verder dan mijn neus lang is. Maar één ding doe ik zeker, als
Peter en ik vriendschap sluiten, dan vertel ik hem ook, dat jij ook veel van hem
houdt en klaar voor hem zoudt staan, als dat nodig was. Dit laatste zul je zeker
niet willen, maar dat kan me nu niets schelen; wat Peter over jou denkt weet ik
niet, maar dat zal ik hem dan ook wel vragen.

Slecht is het zeker niet, integendeel! Kom gerust op zolder of waar we ook zijn,
je stoort ons werkelijk niet, daar we, geloof ik, alle twee stilzwijgend
afgesproken hebben, dat als we praten, we het 's avonds in het donker doen.

Houd moed! Ik doe het ook, hoewel het niet altijd makkelijk is; jouw tijd komt
misschien gauwer dan je denkt.

Je Anne'.

\section*{Donderdag, 23 Maart 1944}

Lieve Kitty,\\
Hier rolt alles weer een beetje. Onze bonnenmannen zijn uit de
gevangenis~ontslagen, gelukkig!

Miep is sinds gisteren weer hier, Elli is beter, hoewel de hoest aanhoudt,
Koophuis zal nog lang thuis moeten blijven.

Gisteren is hier een vliegtuig neergestort, de inzittenden sprongen bijtijds met
parachutes naar beneden. Het toestel kwam neer op een school waar geen kinderen
in waren. Een kleine brand en een paar doden zijn uit het geval voortgekomen. De
Duitsers hebben verschrikkelijk op de dalende vliegers geschoten. De toekijkende
Amsterdammers barstten zowat van woede over zo'n laffe daad. Wij, dat betekent
de dames, schrokken ons een mik, ik vind schieten belabberd.

Ik ga tegenwoordig 's avonds vaak naar boven en hap in Peters kamer de frisse
avondlucht in. Ik vind het gezellig boven op een stoel naast hem te zitten en
naar buiten te kijken.

Van Daan en Dussel doen erg flauw, als ik in zijn kamer verdwijn. `Anne's zweite
Heimat' heet het dan, of `past het voor jonge heren 's avonds nog laat in het
donker jonge meisjes op visite te hebben?' Peter heeft een verbazende
tegenwoordigheid van geest bij dergelijke zogenaamd geestige opmerkingen.

Mijn moeder is trouwens ook niet weinig nieuwsgierig en zou heel graag naar de
onderwerpen van onze gesprekken vragen, als ze niet in stilte bang was voor een
afwijzend antwoord. Peter zegt, dat de volwassenen niets dan afgunst voelen,
omdat wij jong zijn en ons van hun hatelijkheden niet veel aantrekken. Soms
haalt hij me van beneden af, maar hij krijgt ondanks alle voorzorgsmaatregelen
een vuurrod gezicht en kan haast niet uit zijn woorden komen. Ik ben toch maar
blij, dat ik nooit rood word, het lijkt me beslist een hoogst onaangename
gewaarwording.

Vader zegt altijd dat ik een nuf ben, dat is niet waar, ik ben alleen maar
ijdel! Ik heb nog niet van veel mensen gehoord, dat ze me knap van uiterlijk
vinden. Behalve dan van een jongen op school, die zei dat ik er zo leuk uitzag
als ik lachte. Gisteren nu kreeg ik een oprecht compliment van Peter en~ik zal
voor de aardigheid ons gesprek zo ongeveer weergeven:

Peter zei zo vaak: `Lach eens!' Dat viel me op en ik vroeg: `Waarom moet ik
toch~altijd lachen?'

`Omdat dat leuk is; je krijgt dan kuiltjes in je wangen, hoe komt dat
eigenlijk?' `Daar ben ik mee geboren. Dat zit ook in mijn kin. Dat is het enige
schoonheidsding~dat ik bezit'.

`Welneen, dat is niet waar'.\\
`Jawel, ik weet wel dat ik geen mooi meisje ben,
dat ben ik nooit geweest en zal~het ook wel nooit worden'.

`Dat vind ik helemaal niet, ik vind je knap'.\\
`Dat is niet waar'.\\
`Als ik
dat zeg, dan kun je dat van mij aannemen!'\\
Ik zei toen natuurlijk hetzelfde
van hem.\\
Ik moet van allen heel wat over de plotselinge vriendschap horen. We
trekken ons~van al die ouderkletsjes niet veel aan, hun opmerkingen zijn zo
flauw. Zouden de diverse ouders hun jeugd vergeten zijn? Het schijnt van wel, ze
nemen ons tenminste altijd ernstig op als we een grapje maken en lachen om ons
als we ernstig zijn.

Je Anne.

\section*{Maandag, 27 Maart 1944}

Lieve Kitty,\\
Een heel groot hoofdstuk in onze onderduikgeschiedenis op papier
moet eigenlijk~de politiek innemen, maar daar dit onderwerp mij persoonlijk niet
zo erg bezighoudt, heb ik het veel te veel links laten liggen. Daarom zal ik
vandaag eens een hele brief aan de politiek wijden.

Dat er zeer veel verschillende opvattingen omtrent dit vraagstuk bestaan is
vanzelfsprekend, dat er in benarde oorlogstijden ook veel over gesproken wordt
is nog logischer, maar ... dat er zoveel ruzie om gemaakt wordt is eenvoudig
dom.

Laat ze wedden, lachen, schelden, mopperen, laat ze alles doen, als ze maar in
hun eigen vet smoren, maar laat ze geen ruzie maken, want dat heeft meestal
minder goede gevolgen.

De mensen die van buiten komen, brengen zeer veel onwaar nieuws mee; onze radio
heeft echter tot nu toe nooit gelogen. Henk, Miep, Koophuis, Elli en Kraler
vertonen allen in hun politieke stemmingen ups en downs, Henk nog het minst van
hen.

Hier in het Achterhuis is de stemming wat de politiek aangaat wel altijd
dezelfde. Bij de talloze debatten over invasie, luchtbombardementen, speeches
enzovoort enzovoort hoort men ook talloze uitroepen als `onmogelijk', `Um Gottes
Willen, als ze nu pas beginnen willen, wat moet dat dan wel worden!' `Het gaat
uitstekend, prima, best!' Optimisten en pessimisten, vooral de realisten niet te
vergeten, geven met onvermoeide energie hun mening ten beste en zoals dat met
alles gaat, denkt ieder dat hij alleen gelijk heeft. Een zekere dame ergert zich
aan het weergaloze vertrouwen dat haar heer gemaal in de Engelsen stelt, een
zekere heer valt zijn dame aan wegens haar plagende en geringschattende
opmerkingen aangaande zijn geliefde natie.

Het gaat hen nooit vervelen. Ik heb iets uitgevonden en de uitwerking daarvan is
denderend; het is net of je iemand met een speld prikt en hem daardoor op laat
springen. Precies zo werkt mijn middel: begin over de politiek; één vraag, één
woord, één zin en dadelijk zit de hele familie er midden in!

Alsof nu de Duitse Weermachtberichten en de Engelse B.B.C. nog niet genoeg
waren, is er sinds een tijdje een `Luftlagemeldung' ingeschakeld.  Magnifiek in
één woord, maar aan de andere kant ook vaak teleurstellend.  De Engelsen maken
van hun luchtwapen een continubedrijf - alleen te vergelijken met de Duitse
leugens, die dito dito zijn -. Dus, de radio staat al vroeg 's ochtends aan en
wordt elk uur gehoord, tot 's avonds negen, tien of vaak ook elf uur.

Dit is het schoonste bewijs, dat de volwassenen weliswaar~geduld, maar moeilijk
te bereiken hersenen hebben - behoudens uitzonderingen natuurlijk, ik wil
niemand beledigen -. Wij zouden na één, hoogstens twee uitzendingen voor de hele
dag genoeg hebben! Maar die oude ganzen, en n, ik zei immers al!

Arbeiter-Programm, Oranje, Frank Philips of Hare Majesteit Wilhelmina, alles
krijgt een beurt en een even gewillig oor. En zijn ze niet aan het eten of aan
het slapen, dan zitten ze aan de radio en praten over eten, slapen en politiek.

Oef, het wordt vervelend en een hele toer om er geen saaie bes bij te worden.
Bij de ouelui kan dat laatste zo'n kwaad niet meer!

Om een lichtend voorbeeld te geven is een rede van den bij ons allen zeer
geliefden Winston Churchill ideaal.

Zondagavond, negen uur. De thee onder de muts op tafel, de gasten komen binnen.
Dussel naast de radio links, mijnheer er voor, Peter er naast.  Moeder naast
mijnheer, mevrouw achteraan en Pim aan tafel, Margot en ik ernaast. Ik merk, dat
ik niet erg duidelijk schrijf hoe ze zitten. De heren paffen, Peters ogen vallen
van het inspannende luisteren toe, mama in lang donker négligé en mevrouw aan
het bibberen wegens de vliegers die zich van de rede niets aantrekken en lustig
naar Essen vliegen, vader thee slurpend, Margot en ik zusterlijk vereend door de
slapende Mouschi die beslag op twee verschillende knieën heeft gelegd. Margot
heeft krulpennen in het haar; ik ben gekleed in een veel te klein, te nauw en te
kort nachtkostuum.

Het lijkt intiem gezellig, vredig, het is voor deze keer ook zo; toch wacht ik
met schrik de gevolgen van de rede af. Ze kunnen het slot immers haast niet
afwachten, trappelen van ongeduld om er over te kunnen discuteren. Kst, kst,
kst, zo prikkelen ze elkaar, totdat uit de discussie ruzie en onenigheid
ontstaat.

Je Anne.

\section*{Dinsdag, 28 Maart 1944}

Liefste Kitty,\\
Ik zou over de politiek nog veel meer kunnen schrijven, maar ik
heb vandaag eerst~weer een heleboel andere dingen te berichten.  Ten eerste
heeft moeder me eigenlijk verboden zo vaak naar boven te gaan, daar volgens haar
mevrouw Van Daan jaloers is. Ten tweede heeft Peter Margot mee uitgenodigd om
naar boven te komen: of het beleefdheid is of menens weet ik niet. Ten derde ben
ik aan vader gaan vragen, of die vond, dat ik me aan die jaloersheid veel moest
storen en die vond van niet. Wat nu? Moeder is kwaad, misschien ook jaloers.
Vader gunt Peter en mij best die uurtjes en vindt het mooi dat wij het zo goed
met elkaar kunnen vinden. Margot houdt ook van Peter, maar voelt, dat je niet
met zijn drieën kunt bepraten wat je wel met zijn tweeën kunt.

Moeder denkt dat Peter verliefd op mij is, eerlijk gezegd wou ik maar dat het
waar was, dan zijn we quitte en kunnen elkaar veel makkelijker bereiken. Ze zegt
verder, dat hij mij zoveel aankijkt. Nu is het wel waar, dat we elkaar meer dan
eens in de kamer beknipogen en dat hij naar mijn wangenkuiltjes kijkt, daar kan
ik toch niets aan doen! Is het wel?

Ik zit in een heel moeilijke positie. Moeder is tegen me en ik tegen haar, vader
sluit voor de stille strijd tussen moeder en mij zijn ogen.  Moeder is
verdrietig, daar zij toch veel van mij houdt, ik ben helemaal niet verdrietig,
daar ik voel, dat zij me niet begrijpt. En Peter ...  Peter wil ik niet opgeven,
hij is zo lief, ik bewonder hem zo, het zal zo mooi tussen ons kunnen worden,
waarom steken die `ouwen' dan hun neuzen er weer in? Gelukkig ben ik gewend mijn
innerlijk te verbergen en het lukt me dan ook uitstekend niet te laten zien hoe
dol ik op hem ben.  Zal hij ooit wat zeggen? Zal ik ooit zijn wang voelen, zoals
ik Petels wang in mijn droom gevoeld heb? O Peter en Petel, jullie zijn
hetzelfde!  Zij begrijpen ons niet, zouden nooit snappen dat wij al~tevreden
zijn, als we maar bij elkaar zitten en niet praten. Zij begrijpen niet wat ons
zo naar elkaar toedrijft. O, wanneer zouden al die moeilijkheden overwonnen zijn
en toch is het goed ze te overwinnen, want dan is het einde des te mooier. Als
hij zo met zijn hoofd op zijn armen met zijn ogen dicht ligt, dan is hij nog een
kind, als hij met Mouschi speelt, dan is hij liefdevol, als hij aardappels of
andere zware dingen draagt, dan is hij krachtig, als hij gaat kijken bij het
schieten of in het donker naar de dieven, dan is hij moedig en als hij zo
onbeholpen en onhandig doet, dan is hij juist lief.

Ik vind het veel prettiger als hij mij wat uitlegt dan als ik hem wat moet
leren: ik zou zo graag hebben dat hij in bijna alles een overwicht over mij had.

Wat kunnen mij die vele moeders schelen! O, als hij maar wou spreken. Je Anne.

\section*{Woensdag, 29 Maart 1944}

Lieve Kitty,\\
Gisteravond sprak minister Bolkestein voor de Oranjezender er
over, dat er na de~oorlog een inzameling van dagboeken en brieven van deze
oorlog zou worden gehouden. Natuurlijk stormden ze allemaal direct op mijn
dagboek af. Stel je eens voor hoe interessant het zou zijn, als ik een roman van
het Achterhuis zou uitgeven. Aan de titel alleen zouden de mensen denken, dat
het een detectiveroman was.

Maar nu in ernst. Het moet ongeveer tien jaar na de oorlog al grappig aandoen,
als wij vertellen hoe we als Joden hier geleefd, gegeten en gesproken hebben. Al
vertel ik je veel van ons, toch weet je nog maar een heel klein beetje van ons
leven af.

Hoeveel angst de dames hebben als ze bombarderen, bijvoorbeeld Zondag, toen 350
Engelse machines een half millioen kilo bommen op IJmuiden gegooid hebben,
hoe~dan de huizen trillen als een grassprietje in de wind, hoeveel epidemieën
hier heersen. Van al deze dingen weet jij niets af en ik zou de hele dag aan het
schrijven moeten blijven als ik alles in de finesses zou moeten navertellen. De
mensen staan in de rij voor groenten en alle mogelijke andere dingen, de dokters
kunnen niet bij de zieken komen, omdat om de haverklap hun voertuig wordt
gestolen, inbraken en diefstallen zijn er plenty, zo zelfs dat je je gaat
afvragen wat de Nederlanders bezielt dat ze opeens zo stelerig geworden zijn.
Kleine kinderen van acht en elf jaar slaan de ruiten van woningen in en stelen
wat los en vast zit. Niemand durft voor vijf minuten zijn huis te verlaten, want
als je weg bent is je boel ook weg. Iedere dag staan advertenties in de krant
met beloningen voor het terugbezorgen van gestolen schrijfmachines, perzische
kleden, electrische klokken, stoffen enz. enz. Electrische straatklokken worden
gedemonteerd, de telefoons in de cellen tot op de laatste draad uit elkaar
gehaald. De stemming onder de bevolking kan niet goed zijn: iedereen heeft
honger, met het weekrantsoen kan je nog geen twee dagen uitkomen, behalve met
kof esurrogaat. De invasie laat lang op zich wachten, de mannen moeten naar
Duitsland. De kinderen worden ziek of zijn ondervoed, iedereen heeft slechte
kleren en slechte schoenen.

Een zool kost clandestien ƒ 7.50; daarbij nemen de meeste schoenmakers geen
klanten aan of je moet vier maanden op de schoenen wachten, die dikwijls
intussen verdwenen zijn.

Eén ding is hierbij goed, namelijk dat de sabotage tegen de overheid steeds
erger wordt, naarmate het voedsel slechter wordt en de maatregelen tegen het
volk strenger worden. De distributiedienst, de politie, de ambtenaren, alles
doet òf mee om de medeburgers te helpen òf om hen te verklikken en daardoor in
de gevangenis te laten zetten.  Gelukkig staat maar een klein percentage van de
Nederlandse burgers aan de verkeerde kant.

Je Anne.

\section*{Vrijdag, 31 Maart 1944}

Lieve Kitty,\\
Stel je voor, het is nog tamelijk koud, maar de meeste mensen
zitten al ongeveer~een maand lang zonder kolen, plezierig hè! De stemming in het
algemeen is weer optimistisch voor het Russische front, want dat is geweldig! Ik
schrijf wel niet veel over de politiek, maar waar ze nu staan moet ik je toch
even mededelen, ze staan vlak voor het Generaal-Gouvernement en bij Roemenië aan
de Proeth. Vlak bij Odessa staan ze. Hier wachten ze elke avond op een extra
communiqué van Stalin.

In Moskou schieten ze zoveel saluutschoten af, dat de stad elke dag wel dreunen
moet; of ze het nu leuk vinden om te doen of de oorlog weer in de buurt is, of
dat ze geen andere manier weten om hun vreugde te fluiten, ik weet het niet!

Hongarije is door Duitse troepen bezet, daar zijn nog een millioen Joden, die
zullen er nu ook wel aangaan.

Het geklets over Peter en mij is een beetje bedaard. Wij zijn erge goede
vrienden, veel bij elkaar en praten over alle mogelijke onderwerpen. Het is zo
mooi, dat ik me nooit moet inhouden zoals bij andere jongens het geval zou zijn,
als we op précair gebied komen.

Mijn leven hier is beter geworden, veel beter. God heeft mij niet alleen gelaten
en zal me niet alleen laten.

Je Anne.

\section*{Zaterdag, 1 April 1944}

Lieve Kitty,\\
En toch is alles nog zo moeilijk, je weet zeker wel wat ik
bedoel, hé? Ik verlang~zo ontzettend naar een kus, de kus die zo lang uitblijft.
Zou hij mij aldoor nog als een kameraad beschouwen? Ben ik dan niet méér?

Jij weet en ik weet, dat ik sterk ben, dat ik de meeste lasten wel alleen kan
dragen. Ik ben het nooit gewend geweest~mijn lasten met iemand te delen, aan
mijn moeder heb ik me nooit vastgeklemd, maar nu zou ik zo graag eens mijn hoofd
tegen zijn schouder leggen en alleen maar rustig zijn.

Ik kan niet, kan nooit de droom van Peters wang vergeten, toen alles, alles zo
goed was! Zou hij niet ook daarnaar verlangen? Zou hij alleen maar te verlegen
zijn om zijn liefde te bekennen? Waarom wil hij mij zo vaak bij zich hebben? O,
waarom spreekt hij niet?

Laat me ophouden, laat me rustig zijn, ik zal me wel sterk houden en met wat
geduld zal dat andere ook wel komen, maar ... en dat is het erge, het ziet er zo
erg naar uit of ik hem naloop, altijd moet~\emph{ik}~naar boven,
niet~\emph{hij}~gaat naar mij.

Maar dat ligt aan de kamers en hij begrijpt dat bezwaar wel. O ja, hij zal wel
meer begrijpen.\\
Je Anne.

\section*{Maandag, 3 April 1944}

Lieve Kitty,\\
Geheel tegen mijn gewoonte in zal ik je toch maar eens uitvoerig
over het eten~schrijven, want het is niet alleen in het Achterhuis, maar ook in
heel Nederland, in heel Europa en nog verder een zeer voorname en moeilijke
factor geworden.

We hebben in de 21 maanden, die we nu hier zijn al heel wat `eet-periodes'
meegemaakt, wat dat betekent zal je direct horen. Onder `eet-periodes' versta ik
periodes waarin men niets anders te eten krijgt dan een bepaald gerecht of een
bepaalde groente. Een tijdlang hadden we niets anders te eten dan elke dag
andijvie met zand, zonder zand, stamppot, los enfin de vuurvaste schotel, toen
was het spinazie, daarna volgden koolrabie, schorseneren, komkommers, tomaten,
zuurkool enzovoort enzovoort.

Het is heus niet leuk om elke middag en elke avond bijvoorbeeld zuurkool te
eten, maar je doet veel als je honger hebt. Nu hebben we echter de mooiste
periode, want we krijgen~helemaal geen verse groente. Ons weekmenu van 's
middags bestaat uit: bruine bonen, erwtensoep, aardappels met meelballen,
aardappel-chalet, bij de gratie Gods nog eens raapstelen of rotte wortelen en
dan maar weer bruine bonen. Aardappels eten we voor elke maaltijd, te beginnen
bij het ontbijt wegens gebrek aan brood. Voor soep nemen we bruine of witte
bonen, aardappels, julienne uit pakjes, koninginne uit pakjes, bruine bonen uit
pakjes. In alles zit bruine bonen, niet het minst in het brood.

's Avonds eten we altijd aardappels met kunst-jus en, dat hebben we gelukkig
nog, rode-bieten-sla. Over de meelballen moet ik nog spreken, die maken we van
regeringsbloem met water en gist. Ze zijn zo plakkerig en taai, dat het is of er
stenen in je maag liggen, maar a mooi!

Onze grote attractie is een plakje leverworst elke week en de jam op droog
brood. Maar we leven nog en vaak vinden we onze schamele maaltijd zelfs nog
lekker.

Je Anne.

\section*{Dinsdag, 4 April 1944}

Lieve Kitty,\\
Een hele tijd wist ik helemaal niet meer, waarvoor ik nu werk;
het einde van de~oorlog is zo ontzettend ver, zo onwerkelijk, sprookjesachtig.
Als de oorlog in September nog niet afgelopen is, ga ik niet meer naar school.
Want twee jaar wil ik niet achter komen. De dagen bestonden uit Peter, niets dan
Peter, dromen en gedachten, totdat ik Zaterdag zo ontzettend lamlendig werd,
neen vreselijk. Ik zat maar bij Peter mijn tranen te bedwingen, lachte dan met
Van Daan bij de citroen-punch, was opgewekt en opgewonden, maar nauwelijks
alleen wist ik, dat ik nu uithuilen moest. Zo in mijn nachtjapon liet ik me op
de grond glijden en bad eerst heel intens mijn lang gebed, toen huilde ik met
het hoofd op de armen, knieën opgetrokken, op de kale vloer, helemaal
samengevouwen. Bij een harde snik kwam ik weer~in de kamer terug en bedwong mijn
tranen, daar ze binnen niets mochten horen. Toen begon ik mezelf moed in te
spreken, ik zei niets anders dan: `Ik moet, ik moet, ik moet ...' Helemaal stijf
van de ongewone houding viel ik tegen de bedkant aan en vocht toen verder,
totdat ik kort vóór half elf weer in bed stapte. Het was over!

En nu is het helemaal over. Ik moet werken om niet dom te blijven, om vooruit te
komen, om journaliste te worden, want dat wil ik! Ik weet dat
ik~\emph{kan}~schrijven, een paar verhaaltjes zijn goed, mijn
Achterhuisbeschrijvingen humoristisch, veel uit mijn dagboek spreekt, maar ...
of ik werkelijk talent heb, dat staat nog te bezien.

Eva's droom was mijn beste sprookje en het gekke daarbij is, dat ik heus niet
weet, waar het vandaan komt. Veel uit Cady's leven is ook goed, maar het geheel
is niets. Ik zelf ben mijn scherpste en beste beoordelaar hier. Ik weet zelf wat
goed en niet~goed geschreven is.  Niemand die niet schrijft weet hoe mooi
schrijven is; vroeger betreurde ik het altijd, dat ik in het geheel niet tekenen
kon, maar nu ben ik overgelukkig dat ik tenminste schrijven kan. En als ik geen
talent heb om voor kranten of boeken te schrijven, wel, dan kan ik nog altijd
voor mezelf schrijven.

Ik wil verder komen, ik kan me niet voorstellen, dat ik zou moeten leven zoals
moeder, mevrouw Van Daan en al die vrouwen, die hun werk doen en die later
vergelen zijn. Ik moet iets hebben naast man en kinderen, waar ik me aan wijden
kan!

Ik wil nog voortleven ook na mijn dood! En daarom ben ik God zo dankbaar, dat
hij me bij mijn geboorte al een mogelijkheid heeft meegegeven om me te
ontwikkelen en om te schrijven, dus om uit te drukken alles wat in me is.

Met schrijven word ik alles kwijt, mijn verdriet verdwijnt, mijn moed herleeft.
Maar, en dat is de grote vraag, zal ik ooit nog iets groots kunnen schrijven,
zal ik ooit eens journaliste en schrijfster worden?

Ik hoop het, o ik hoop het zo, want in schrijven kan ik alles vastleggen, mijn
gedachten, mijn idealen en mijn fantasieën.

Aan Cady's leven heb ik lang niet meer gewerkt, in mijn gedachten weet ik
precies hoe het verder zal gaan, maar het vlot niet goed. Misschien komt het
nooit af, komt het in de prullemand of kachel terecht ... Dat is een naar idee,
maar daarna denk ik weer, `met 14 jaar en zo weinig ervaring kan je ook nog geen
philosofie schrijven'.

Dus maar weer verder, met nieuwe moed, het zal wel lukken, want schrijven wil
ik!

Je Anne.

\section*{Donderdag, 6 April 1944}

Lieve Kitty,\\
Je hebt me gevraagd, wat mijn hobby's en interessen zijn en
daarop wil ik je~antwoorden. Ik waarschuw je echter, schrik niet, het zijn er
een heleboel.

In de eerste plaats: schrijven, maar dat telt eigenlijk niet als hobby.\\
No
twee: stambomen. Van de Franse, Duitse, Spaanse, Engelse, Oostenrijkse,

Russische, Noorse en Nederlandse vorstenfamilies ben ik in alle kranten, boeken
en papieren naar stambomen aan het zoeken. Met velen ben ik al erg ver
gevorderd, temeer daar ik al lang uit al de biografieën of geschiedenisboeken
die ik lees, aantekeningen maak. Vele stukken geschiedenis schrijf ik zelfs
over.

Mijn derde hobby is dan ook geschiedenis, waartoe vader voor mij al veel boeken
gekocht heeft. Ik kan de dag haast niet afwachten, dat ik in de Openbare
Bibliotheek alles kan napluizen.

Nummer vier is mythologie van Griekenland en Rome. Ook hierover heb ik
verscheidene boeken.

Verdere liefhebberijen zijn filmsterren en familiefoto's. Dol op lezen en op
boeken. Voel veel voor kunstgeschie-denis; vooral schrijvers, dichters en
schilders. Musici komt misschien later nog. Bepaald antipathie heb ik tegen
algebra, meetkunde en rekenen.

Alle overige schoolvakken doe ik met plezier, maar geschiedenis bovenal!\\
Je
Anne.

\section*{Dinsdag, 11 April 1944}

Lieve Kitty,\\
Mijn hoofd klopt, ik weet heus niet waarmee te beginnen.\\
Vrijdag (Goede Vrijdag) speelden we 's middags beursspel,
Zaterdagmiddag~eveneens. Deze dagen verliepen alle heel gauw en gewoon.
Zondagmiddag kwam Peter op mijn uitnodiging om half vijf bij me, om kwart over
vijf gingen we naar de voorzolder, waar we tot zes uur bleven. Van zes uur tot
kwart over zeven werd er aan de radio een mooi Mozart-concert gegeven, vooral
van de\emph{Kleine Nachtmusik}~heb ik erg genoten. Ik kan in de kamer haast niet
luisteren, omdat bij mooie muziek alles in me in beweging komt.

Zondagavond gingen Peter en ik samen om acht uur naar de voorzolder en om zacht
te zitten namen we enige divankussens, die in onze kamer te vinden waren, mee.
We namen op één kist plaats. Kist zowel als kussens waren erg smal, we zaten
helemaal tegen elkaar aan en leunden alle twee tegen andere kisten. Mouschi
hield ons nog gezelschap, dus we waren niet onbespied.

Plotseling, om kwart voor negen, floot mijnheer Van Daan en vroeg of we een
kussen van mijnheer Dussel hadden. Alle twee sprongen we op en gingen met
kussen, kat en Van Daan naar beneden.

Van dit kussen is nog een hele misère gekomen, want Dussel was kwaad, dat een
van zijn kussens er bij was, dat hij als `Nachtkussen' gebruikte. Hij was bang
dat er vlooien in zaten en bracht voor dat éne kussen alles in rep en roer.

Peter en ik stopten toen als wraak twee harde borstels in zijn bed. We hebben om
dit intermezzo wel gelachen.

Onze pret zou niet lang duren. Om half tien klopte Peter zachtjes aan de deur en
vroeg vader of die boven even met een moeilijke Engelse zin wou helpen. `Dat is
niet pluis', zei ik tegen Margot, `het verzinsel ligt er dik bovenop'. Mijn
veronderstelling klopte: in het magazijn waren ze net aan het inbreken. In een
minimum van tijd waren vader, Van Daan, Dussel en Peter beneden, Margot, moeder,
mevrouw en ik wachtten af.

Vier vrouwen in angst moeten praten, zo ook wij, totdat we beneden een slag
hoorden. Daarna werd alles stil, de klok sloeg kwart voor tien. Van ons aller
gezicht was de kleur verdwenen; we waren nog rustig, hoewel bang. Waar zouden de
heren gebleven zijn? Wat was die slag? Zouden ze misschien met de inbrekers
vechten? Tien uur, stappen op de trap: vader bleek en nerveus, kwam binnen,
gevolgd door mijnheer Van Daan. `Licht uit, zachtjes naar boven, we verwachten
politie in huis!'.

Er was geen tijd voor angst, de lichten gingen uit, ik pakte nog gauw een jasje
en we zaten boven. `Wat is er gebeurd, vertel toch gauw!' Er was niemand om te
vertellen, de heren waren weer beneden. Pas tien over tienen kwamen ze alle vier
boven, twee hielden de wacht aan Peters open raam, de deur naar de overloop was
afgesloten, de draaikast dicht. Om het nachtlampje hingen we een trui, daarna
vertelden ze:

Peter hoorde op de overloop twee harde klappen, liep naar beneden en zag dat aan
de linkerkant van de magazijndeur een grote plank miste. Hij holde naar boven,
waarschuwde het weerbare deel der familie en met zijn vieren trokken ze naar
beneden. De inbrekers waren verder aan het breken, toen ze het magazijn
binnenkwamen. Zonder bedenken schreeuwde Van Daan: `Politie!'

Haastige stappen buiten, de inbrekers waren gevlucht. Om te voorkomen, dat de
politie het gat zou opmerken,~werd de plank er weer voor gezet, maar door een
inke trap van buiten vloog die nog eens op de grond. De heren waren perplex over
zoveel brutaliteit, Van Daan zowel als Peter voelden moordneigingen opkomen; met
de bijl sloeg Van Daan hard op de grond, alles was weer stil. Opnieuw wilden ze
het plankje voor het gat zetten. Storing! - Een echtpaar buiten, bescheen met
het schelle schijnsel van een zaklantaarn het hele magazijn door de opening.
`Verroest', mompelde een van de heren en ... nu veranderde hun rol van politie
tot inbrekers. Alle vier slopen ze naar boven, Peter opende vlug de deuren en
ramen van keuken en privé-kantoor, smeet de telefoon op de grond en ten slotte
waren alle vier achter de schuilmuur beland.

Einde van het eerste deel.

Naar alle waarschijnlijkheid zou het echtpaar met de zaklantaarn de politie
gewaarschuwd hebben; het was Zondagavond, de avond van de eerste Paasdag, tweede
Paasdag niemand op kantoor, dus niemand kon zich verroeren vóór Dinsdagmorgen.
Stel je voor, twee nachten en een dag in deze angst te zitten! We stelden ons
niets voor, zaten maar in het stikdonker, omdat mevrouw ook al uit angst de lamp
helemaal uitgedraaid had, de stemmen fluister den, bij elk gekraak klonk `sst.
sst.'.

Het werd half elf, elf uur, geen geluid, om de beurt kwamen vader en Van Daan
bij ons. Toen kwart over elf, gedruis beneden. Bij ons was de ademhaling van het
hele gezin hoorbaar, overigens verroerden we ons niet. Stappen in huis,
privé-kantoor, keuken, dan ... onze trap. Niemand ademde nu nog hoorbaar, acht
harten bonkten, stappen op onze trap, dan gerammel aan de draaikast. Dit moment
is onbeschrijfelijk. `Nu zijn we verloren!' zei ik, en ik zag ons allen
diezelfde nacht nog door de Gestapo weggevoerd. Gerammel aan de draaikast twee
keer, dan viel iets, de stappen~verwijderden zich, tot zover waren we gered. Een
rilling voer door ons allen, van onbestemde kant hoorde ik klappertanden,
niemand zei nog een woord.

Niets werd meer gehoofd in huis, maar er brandde licht op onze overloop, juist
voor de kast. Zou dat zijn, omdat de kast geheimzinnig was? Had de politie
misschien het licht vergeten? Kwam er nog iemand om het uit te doen? De tongen
kwamen los, niemand was meer in huis, misschien wel een bewaker voor de deur.

Drie dingen deden we nu, veronderstellingen fluiten, bibberen van angst en naar
de W.C. moeten. De emmers waren op de zolder, alleen de blikken prullemand van
Peter kon dienst doen. Van Daan maakte een begin, daarna kwam vader, moeder
schaamde zich te veel. Vader bracht de bak mee naar de kamer, waar Margot,
mevrouw en ik er graag gebruik van maakten, eindelijk besloot ook moeder. Steeds
was er vraag naar wat papier, ik had gelukkig in mijn zak.

De bak stonk, alles fluister de en we waren moe, het was twaalf uur. `Ga toch op
de grond liggen en slaap'. Margot en ik kregen elk een kussen plus deken, Margot
lag een stuk van de voorraadkast af, ik tussen de poten van de tafel. Op de
grond stonk het niet zo erg, maar mevrouw haalde toch zachtjes wat bleekpoeder,
een droogdoek diende als tweede beschermmiddel over de pot.

Gepraat, ge fluister , bangigheid, stank, windjes en aldoor iemand op de pot;
slaap daar maar bij! Om half drie werd ik echter te moe en tot half vier hoorde
ik niets. Ik werd wakker toen mevrouw met haar hoofd op mijn voet lag.

`Geeft u me alstublieft wat om aan te doen!' vroeg ik. Ik kreeg, maar vraag niet
wat, een wollen broek over mijn pyama, een rode trui en een zwarte rok, witte
onderkousen en kapotte kniekousen. Mevrouw nam vervolgens weer plaats op de
stoel en mijnheer kwam op mijn voeten te liggen. Vanaf half vier ging ik denken,
ik trilde nog steeds, zodat Van Daan niet slapen kon. Ik bereidde me er op voor,
dat de politie~terug zou komen, dan moeten we zeggen dat we onderduikers zijn;
òf het zijn goede Nederlanders, dan zijn we gered, òf het zijn N.S.B.-ërs, dan
moeten we ze omkopen!

`Doe toch de radio weg', zuchtte mevrouw. `Ja, in het fornuis', antwoordde
mijnheer, `als ze ons vinden, mogen ze ook de radio vinden!'

`Dan vinden ze ook Anne's dagboek', voegde Vader er aan toe. `Verbrand dat dan',
opperde de bangste van ons allen.

Dit en toen de politie aan de kastdeur rammelde, waren mijn bangste ogenblikken.
`Mijn dagboek niet, mijn dagboek alleen samen met mij!' Maar vader antwoordde
niet meer, gelukkig.

Het heeft geen doel al de gesprekken die ik me nog herinner aan te halen; er
werd zoveel gepraat. Ik troostte mevrouw in haar angst. We hadden het over
vluchten en uithoren door de Gestapo, over opbellen en moedig zijn.

`Nu moeten we ons als soldaten gedragen, mevrouw, als we er aan gaan, nu goed,
dan maar voor Koningin en Vaderland, voor vrijheid, waarheid en recht, zoals
radio Oranje steeds zegt. Het enige, dat ontzettend erg is, is dat we ze
allemaal meeslepen in het ongeluk'.

Mijnheer Van Daan ruilde na een uur weer met zijn vrouw, vader kwam bij mij. De
heren rookten aan één stuk door, af en toe een diepe zucht, dan weer een plasje
en dan begon alles opnieuw.

Vier uur, vijf uur, half zes. Nu ging ik bij Peter in de kamer aan het raam
zitten luisteren, zo dicht bijeen, dat we de trillingen van elkanders lichaam
voelden, we spraken zo nu en dan een woord en luisterden scherp. Naast ons in de
kamer ontduisterden ze. Om zeven uur wilden ze Koophuis opbellen en iemand hier
laten komen. Nu schreven ze op wat ze Koophuis aan de telefoon wilden vertellen.
Het risico, dat de wacht voor de deur of in het magazijn dat opbellen hoorde,
was groot, maar het gevaar, dat de politie weer terugkwam, nog groter.

De punten waren:\\
\emph{Ingebroken:}~politie in huis geweest tot aan de
draaideur, niet verder. Inbrekers blijkbaar gestoord, hebben magazijn geforceerd
en zijn door de tuin~ontvlucht.

Hoofdingang gegrendeld, Kraler~\emph{moet}~door de tweede deur weggegaan zijn.
De~schrijfmachines en rekenmachine zijn veilig in de zwarte kist op
privé-kantoor. Henk trachten te waarschuwen en sleutel bij Elli halen, dan naar
kantoor gaan~kijken, voorwendsel: poes voeden.

Alles verliep naar wens. Koophuis werd opgebeld, de schrijfmachines, die bij ons
boven stonden, in de kist gestopt. Dan zaten we weer aan tafel en wachtten op
Henk of de politie.

Peter was ingeslapen, Van Daan en ik lagen op de grond, toen we beneden harde
stappen hoorden. Zachtjes stond ik op: `Dat is Henk'.

`Neen, neen, dat is de politie', hoorde ik van anderen.

Er werd aan onze deur geklopt, Miep floot. Mevrouw Van Daan werd het toen te
machtig, lijkwit en slap hing ze in haar stoel en als de spanning nog één minuut
aangehouden had, was ze flauw gevallen.

Toen Henk en Miep binnenkwamen, leverde onze kamer een heerlijk beeld op, alleen
de tafel was al de moeite van een foto waard.
Een~\emph{Cinema}~\&~\emph{Theater}, vol met jam en een middel tegen diarrhee,
lag op een dansmeisjesbladzijde open, twee jampotten, een half en een kwart
broodje, spiegel, kam, lucifers, as, sigaretten, tabak, asbak, boeken, een
onderbroek, zaklantaarn, closetpapier, enz. enz.  lagen in bonte mengeling
dooreen.

Henk en Miep werden natuurlijk met gejuich en tranen begroet. Henk timmerde het
gat met hout dicht en ging al gauw weer weg om de politie van de inbraak op de
hoogte te stellen. Miep had onder de magazijndeur ook nog een~briefje gevonden
van den nachtwaker Slagter, die het gat open gevonden en de politie gewaarschuwd
had, die zou hij ook even opzoeken.

Een half uurtje hadden we dus om ons op te knappen. Nog nooit heb ik in een half
uur zoveel zien veranderen. Margot en ik legden beneden de bedden uit, gingen
naar de W.C., poetsten onze tanden, wasten ons en maakten ons haar op. Daarna
ruimde ik de kamer nog een beetje op en ging weer naar boven. De tafel was daar
al afgeruimd, we tapten water, zetten koffie en thee, kookten melk en dekten
voor het koffie-uurtje, vader en Peter leegden de po's en maakten ze schoon met
warm water en bleekpoeder.

Om elf uur zaten we met Henk, die terug was, rond de tafel en het werd
langzamerhand weer gezellig. Henks verhaal was als volgt:

`Bij Slagter vertelde de vrouw, want mijnheer sliep, dat haar man op zijn tocht
rond de grachten het gat bij ons ontdekt had en met een er bij gehaalde agent
het perceel doorgelopen was. Hij zou Dinsdag bij Kraler komen en dan verder
vertellen. Op het politiebureau wisten ze nog niets van de inbraak af, maar
noteerden het meteen om ook Dinsdag te komen kijken. Op de terugweg liep Henk
toevallig even bij onze aardappelleverancier om de hoek aan en vertelde dat er
ingebroken was.  `Dat weet ik', zei deze doodleuk, `ik kwam gisteravond met mijn
vrouw langs uw perceel en zag een gat in de deur. Mijn vrouw wou al doorlopen,
maar ik keek even met de lantaarn; toen liepen de dieven meteen weg.  Voor alle
zekerheid heb ik de politie maar niet opgebeld, daar ik dat bij u niet wou doen,
ik weet wel niets, maar ik vermoed veel'.

Henk bedankte en ging heen. Die man vermoedt zeker dat we hier zitten, want hij
brengt de aardappels altijd tussen de middaguren. Fijne kerel!

Nadat Henk vertrokken was en wij afgewassen hadden, was het één uur.  Alle acht
gingen we slapen. Om kwart~voor drie werd ik wakker en zag, dat mijnheer Dussel
al verdwenen was. Heel toevallig kwam ik met mijn slaperige gezicht Peter in de
badkamer tegen, die net van boven kwam. We spraken beneden af.

Ik knapte me op en ging naar beneden. `Durf je nog naar de voorzolder te gaan?'
vroeg hij. Ik stemde toe, haalde mijn beddekussen en we gingen naar de
voorzolder. Het weer was schitterend en al gauw loeiden dan ook de sirenes; we
bleven waar we waren. Peter sloeg zijn arm om mijn schouder, ik sloeg mijn arm
om zijn schouder en zo bleven we, de armen om elkaar heen, rustig afwachten tot
Margot ons om vier uur voor de kof e kwam halen.

We aten ons brood op, dronken limonade en maakten grapjes, dat ging dus weer,
verder liep alles gewoon. Ik bedankte Peter 's avonds, omdat hij de moedigste
van allen was.

Geen van ons allen heeft ooit in zulk een gevaar verkeerd als wij die nacht. God
heeft ons wel heel erg beschermd; denk eens aan, de politie aan onze schuilkast,
het licht er vóór aan en wij blijven nog onopgemerkt.

Als de invasie met bommen komt, kan ieder voor zichzelf opkomen, maar hier was
de angst ook voor onze onschuldige en goede beschermers. `Wij zijn gered, red
ons verder!' Dat is het enige, wat we kunnen zeggen.

Deze geschiedenis heeft heel wat veranderingen meegebracht. Mijnheer Dussel zit
in het vervolg 's avonds niet meer beneden in Kralers kantoor, maar in de
badkamer. Peter gaat om half negen en half tien het huis controleren, Peters
raam mag 's nachts niet meer open. Er mag op de W.C. na half tien niet meer
doorgetrokken worden. Vanavond komt er een timmerman om de magazijndeuren nog te
versterken.

Debatten voor en na zijn er nu in het Achterhuis. Kraler heeft ons onze
onvoorzichtigheid verweten. Ook Henk zei,~dat we in zo'n geval nooit naar
beneden mogen. Wij zijn er heel sterk aan herinnerd dat wij onderduikers zijn,
dat wij geketende Joden zijn, geketend aan een plek, zonder rechten maar met
duizend plichten. Wij Joden mogen ons gevoel niet laten gelden, moeten moedig
zijn en sterk, moeten alle ongemakken op ons nemen en niet mopperen, moeten doen
wat in onze macht ligt en op God vertrouwen. Eens zal deze verschrikkelijke
oorlog toch wel aflopen, eens zullen we toch weer mensen en niet alleen Joden
zijn.

Wie heeft ons dit opgelegd? Wie heeft ons Joden tot een uitzondering onder alle
volkeren gemaakt? Wie heeft ons tot nu toe zo laten lijden?  Het is God geweest
die ons zo heeft gemaakt, maar het zal ook God zijn, die ons opheft. Als wij al
dit leed dragen en er toch nog steeds Joden overblijven, dan zullen de Joden
eenmaal van gedoemden tot voorbeelden worden. Wie weet mag het ons geloof nog
eens zijn, dat de wereld en daarmee alle volkeren het goede leert en daarvoor,
dáárvoor alleen moeten wij ook lijden. Wij kunnen nooit alleen Nederlanders,
alleen Engelsen, of vertegenwoordigers van welk land ook worden, wij zullen
daarnaast altijd Joden blijven, maar wij willen het ook blijven.

Wees moedig! Laten we ons van onze taak bewust blijven en niet mopperen, er zal
een uitkomst komen, God heeft ons volk nooit in de steek gelaten.  Door alle
eeuwen heen zijn er Joden blijven leven, door alle eeuwen heen moesten Joden
lijden, maar door alle eeuwen heen zijn ze ook sterk geworden; de zwakken
vallen, maar de sterken zullen overblijven en nooit ondergaan!

In die nacht wist ik eigenlijk dat ik sterven moest, ik wachtte op de politie,
ik was bereid, bereid zoals de soldaten op het slagveld. Ik wou me graag
opofferen voor het vaderland, maar nu, nu ik weer gered ben, nu is mijn eerste
wens na de oorlog, maak me Nederlander!

Ik houd van de Nederlanders, ik houd van ons land, ik houd van de taal, en wil
hier werken. En al zou ik aan de Koningin zelf moeten schrijven, ik zal niet
wijken vóór mijn doel bereikt is.

Steeds onafhankelijker word ik van mijn ouders, zo jong als ik ben heb ik meer
levensmoed, meer juist en ongeschonden rechtsgevoel dan moeder.  Ik weet wat ik
wil, ik heb een doel, een mening, ik heb een geloof en een liefde. Laat me
mezelf zijn, dan ben ik tevreden. Ik weet dat ik een vrouw ben, een vrouw met
innerlijke sterkte en veel moed.

Als God me laat leven, zal ik meer bereiken dan moeder ooit deed, ik zal niet
onbetekenend blijven, ik zal in de wereld en voor de mensen werken!

En nu weet ik dat moed en blijheid het eerst nodig zijn. Je Anne.

\section*{Vrijdag, 14 April 1944}

Lieve Kitty,\\
De stemming is hier nog zeer gespannen. Pim staat op kookpunt,
mevrouw ligt~met verkoudheid in bed en toetert, mijnheer zonder saffiaantjes is
bleek, Dussel, die veel van zijn gemak opofferde, heeft aanmerkingen enz. enz.

Trouwens, het is waar, we boffen op het ogenblik niet. De W.C. is lek en de
kraan doorgedraaid, dank zij de vele relaties is het één zowel als het andere
spoedig weer te herstellen.

Soms ben ik sentimenteel, dat weet ik wel, maar ... er is hier soms ook plaats
voor sentimentaliteit. Als Peter en ik ergens tussen veel rommel en stof op een
harde houten kist zitten, de armen om elkaars schouder dicht naast elkaar, hij
met een krul van mij in zijn hand, als buiten de vogels zo trillend fluiten, als
je de bomen groen ziet worden, als~de zon naar buiten lokt, als de lucht zo
blauw is, o dan, dan wil ik zoveel!

Niets dan ontevreden en mopperende gezichten ziet men hier, niets dan zuchten en
onderdrukt klagen, het lijkt wel of we het opeens verschrikkelijk slecht
gekregen hebben. Heus, alles is net zo slecht als je het zelf maakt. Hier in het
Achterhuis is er niemand die het goede voorbeeld geeft, hier moet ieder zien hoe
hij zijn buien de baas wordt.  `Was het maar afgelopen', hoor je iedere dag.

Mijn werk, mijn hoop, mijn liefde, mijn moed, dat alles houdt me rechtop en
maakt me goed.

Ik geloof beslist, Kit, dat ik vandaag een beetje getikt ben en ik weet toch
niet waarom. Alles staat hier door elkaar, geen verband valt er te bespeuren en
ik twijfel er soms ernstig aan, of later iemand mijn gedaas zal interesseren.

`De ontboezemingen van een lelijk, jong eendje', wordt dan de titel van al de
onzin. Mijnheer Bolkestein of Gerbrandy zullen aan mijn dagboeken heus niet
zoveel hebben.

Je Anne.

\section*{Zondagmorgen, iets voor 11 uur. 16 April 1944}

Liefste Kitty,\\
Onthoud de dag van gisteren, want hij is heel belangrijk in
mijn leven. Is het niet~voor ieder meisje belangrijk als ze haar eerste zoen te
pakken heeft? Welnu, voor mij is het net zo belangrijk.  De zoen van Bram op
mijn rechterwang telt niet mee en evenmin die van mr Walker op mijn rechter
hand.

Hoe ik zo plotseling aan die zoen gekomen ben, wel, dat zal ik je vertellen.

Gisteravond om acht uur zat ik samen met Peter op zijn divan, al gauw sloeg hij
een arm om mij heen. `Laten~wij een beetje opschikken', zei ik, `dan kan ik mijn
hoofd niet aan het kastje stoten'. Hij schikte op, tot haast helemaal in het
hoekje, ik legde mijn arm onder de zijne door op zijn rug en hij bedolf me
zowat, doordat zijn arm om mijn schouder hing.

Nu hebben we wel meer zo gezeten, maar nooit zo dicht bij elkaar als
gisteravond. Hij drukte me stijf tegen zich aan, mijn linkerborst lag op de
zijne, mijn hart klopte al sneller, maar we waren nog niet klaar. Hij rustte
niet, voordat mijn hoofd op zijn schouder lag en het zijne daar weer bovenop.
Toen ik na ongeveer vijf minuten weer wat rechtop ging zitten, nam hij al gauw
mijn hoofd in zijn handen en legde het weer tegen zich aan. O, het was zo
heerlijk, ik kon niet veel spreken, het genot was te groot. Hij streelde een
beetje onhandig mijn wang en arm, prutste aan mijn krullen en onze hoofden lagen
de meeste tijd helemaal tegen elkaar aan. Het gevoel dat mij daarbij
doorstroomde kan ik je niet vertellen, Kitty, ik was te gelukkig en hij ook,
geloof ik.

Om half negen stonden we op, Peter deed zijn gymschoenen aan om zachter te lopen
bij de ronde door het huis en ik stond er bij. Hoe het zo plotseling kwam, ik
weet het niet, maar voordat wij naar beneden gingen, gaf hij mij een zoen,
tussen mijn haar, half op mijn linkerwang, half op mijn oor. Ik rende naar
beneden, zonder omkijken en ben erg verlangend naar vandaag.

Je Anne.

\section*{Maandag, 17 April 1944}

Lieve Kitty,\\
Geloof jij, dat vader en moeder het zouden goedkeuren, dat ik op
een divan een~jongen zit te zoenen, een jongen van 171⁄2 en een meisje van haast
15? Ik geloof eigenlijk van niet, maar ik moet me in deze maar op mezelf
verlaten. Het is zo rustig en veilig in zijn armen te liggen en te dromen, het
is zo opwindend zijn wang tegen de mijne te voelen, het~is zo heerlijk te weten,
dat er iemand op me wacht. Maar, er is inderdaad een maar, want zal Peter het
hierbij willen laten? Ik ben zijn belofte heus nog niet vergeten, maar ... hij
is een jongen!

Ik weet zelf wel, dat ik er erg vroeg bij ben; nog geen 15 en al zo zelfstandig,
dat is voor andere mensen wel een beetje onbegrijpelijk. Ik weet haast zeker dat
Margot nooit een jongen een kus zou geven, zonder dat er ook sprake is van
verloven of trouwen, maar zulke plannen hebben Peter noch ik. Ook moeder heeft
vóór vader zeker geen man aangeraakt.  Wat zouden mijn vriendinnen er van
zeggen, als ze wisten dat ik in Peters armen lag met mijn hart op zijn borst,
met mijn hoofd op zijn schouder, met zijn hoofd tegen het mijne!

O Anne, wat schandelijk; maar heus, ik vind het niet schandelijk, wij zitten
hier afgesloten, afgesloten van de wereld, in angst en zorg de laatste tijd in
het bijzonder, waarom zouden wij, die van elkaar houden, dan los van elkander
blijven? Waarom zouden wij wachten, totdat wij de gepaste leeftijd hebben?
Waarom zouden wij veel vragen?

Ik heb op me genomen om op mezelf te passen, hij zou me nooit verdriet of pijn
willen doen, waarom zou ik dan niet doen wat mijn hart me ingeeft en ons beiden
gelukkig maken? Toch geloof ik wel, Kitty, dat je een beetje van mijn twijfel
merkt, ik denk dat het mijn eerlijkheid is, die tegen stiekemheid in opstand
komt. Vind jij, dat het mijn plicht is vader te vertellen wat ik doe? Vind jij,
dat ons geheim een derde ter ore moet komen? Veel van het mooie zou er afgaan,
maar zou mijn binnenste daardoor rustig worden? Ik zal er met `hem' over
spreken.

O ja, ik wil met hem nog over zoveel spreken, want alleen elkaar te liefkozen,
daar zie ik geen nut in. Om onze gedachten aan elkaar te vertellen, daar hoort
veel vertrouwen toe, we zullen beiden zeker sterker worden in het besef van dit
vertrouwen!

Je Anne.

\section*{Dinsdag, 18 April 1944}

Lieve Kitty,\\
Hier is alles goed. Vader zei net, dat hij stellig verwacht, dat
er voor 20 Mei nog~heel grootscheepse operaties plaats zullen vinden, zowel in
Rusland en Italië alsook in het Westen; ik kan me een bevrijding uit onze
toestand hoe langer hoe minder voorstellen.

Gisteren zijn Peter en ik dan eindelijk aan ons gesprek toegekomen, dat minstens
al tien dagen uitgesteld was. Ik heb hem alles van de meisjes uitgelegd en niet
geschroomd de meest intieme dingen te bespreken. De avond eindigde met een
wederzijdse zoen, zo'n beetje naast mijn mond, het is werkelijk een mooi gevoel.

Misschien neem ik mijn mooie-zinnen-boek eens mee naar boven om eindelijk eens
wat dieper op de dingen in te gaan. Ik vind er geen bevrediging in om dag aan
dag in elkaars armen te liggen en zou me van hem zo graag hetzelfde voorstellen.

We hebben na ons kwakkel-wintertje weer een prachtig voorjaar; April is
inderdaad schitterend, niet te warm en niet te koud met zo nu en dan een
regenbuitje. Onze kastanje is al tamelijk groen en hier en daar zie je zelfs al
kleine kaarsjes.

Elli heeft ons Zaterdag met vier bosjes bloemen verwend, drie bosjes narcissen
en een bosje blauwedruifjes, het laatste was voor mij.

Ik moet Algebra doen, Kitty, tot ziens. Je Anne.

\section*{Woensdag, 19 April 1944}

Lieve Schat,\\
Wat is er mooier in de wereld dan voor een open raam in de natuur
te kijken, de~vogeltjes te horen fluiten, de zon op je wangen te voelen en een
lieven jongen in je armen te hebben? Het is zo rustig en veilig zijn arm om mij
heen te voelen, om hem dichtbij te weten en toch te zwijgen; het kan niet slecht
zijn, want deze rust is goed. O, om nooit meer gestoord te worden, zelfs niet
door Mouschi.

Je Anne.

\section*{Donderdag, 27 April 1944}

Lieve Kitty,\\
Vanochtend had mevrouw een slecht humeur; niets dan klagen! Eerst
over de~verkoudheid, dat zij geen dropjes krijgt, dat veel neussnuiten niet om
uit te houden is. Dan, dat de zon niet schijnt, dat de invasie niet komt, dat we
niet uit het raam kunnen kijken, enzovoort enzovoort. Wij moesten vreselijk om
haar lachen en het was nog niet zo erg of ze lachte mee.

Op het ogenblik lees ik:~\emph{Keizer Karel V}, door een hoogleraar van de
Universiteit te Göttingen geschreven; deze man heeft 40 jaar aan dit boek
gewerkt. In vijf dagen las ik 50 bladzijden, meer is niet mogelijk.  Het boek
bevat 598 bladzijden, nu kun je uitrekenen hoe lang ik hierover zal doen en dan
nog het tweede deel! Maar ... zeer interessant!

Waar een schoolmeisje in één dag al niet van hoort, neem mij nu eens.  Eerst
vertaalde ik van het Hollands in het Engels een stuk van Nelsons laatste slag.
Daarna nam ik het vervolg van de Noorse oorlog door (1700-1721), van Peter de
Grote, Karel XII, Augustus de Sterke, Stanislaus Lescinsky, Mazeppa, van Görz,
Brandenburg, Voor-Pommeren, Achter-Pommeren en Denemarken plus de gebruikelijke
jaartallen.

Vervolgens belandde ik in Brazilië, las van de Bahia-tabak, de overvloedige
koffie, de anderhalf millioen inwoners van Rio de Janeiro, van Pernambuco en
Sao-Paulo, de Amazone-rivier niet te vergeten. Van negers, mulatten, mestiezen,
blanken, meer dan 50\% analphabeten en de malaria. Daar er nog wat tijd
overschoot, nam ik nog gauw een stamboom door. Jan de Oude, Willem Lodewijk,
Ernst Casimir I, Hendrik Casimir I tot de kleine Margriet Franciska toe (geboren
1943 in Ottawa).

Twaalf uur: Op zolder vervolgde ik mijn programma met kerkgeschiedenis ... Pf!
tot één uur.

Na twee uur zat het arme kind (hm, hm., alweer te werken,~smalen breedneusapen
waren aan de beurt. Kitty, zeg gauw hoeveel tenen een nijlpaard heeft!!

Dan volgt de Bijbel, Ark van Noach, Sem, Cham en Japheth. Daarna Karel V. Bij
Peter: `De kolonel' in het Engels door Thackeray. Franse woordjes overhoren en
dan de Missisippi met de Missouri vergelijken.

Ik ben nog steeds verkouden en heb zowel Margot als vader en moeder aangestoken.
Als Peter het maar niet krijgt, hij moest een zoen hebben en noemde mij zijn
`Eldorado'. Kan niet eens, gekke jongen! Maar lief is hij toch!

Genoeg voor vandaag, adieu! Je Anne.

\section*{Vrijdag, 28 April 1944}

Lieve Kitty,\\
Mijn droom van Peter Wessel heb ik nooit vergeten (zie begin
Januari). Ik voel~als ik er aan denk, vandaag nog zijn wang tegen de mijne met
dat heerlijke gevoel, dat alles goed maakt.

Met Peter hier had ik dat gevoel ook wel eens, maar nooit in zo sterke mate,
totdat ... we gisteravond samenzaten, als gewoonlijk op de divan, in elkanders
armen. Toen gleed de gewone Anne opeens weg en kwam daarvoor de tweede Anne in
de plaats, de tweede Anne, die niet overmoedig en grappig is, maar die alleen
lief wil hebben en zacht wil zijn.

Ik zat tegen hem aangedrukt en voelde de ontroering in me stijgen, de tranen
sprongen in mijn ogen, de linker viel op zijn overall, de rechter druppelde
langs mijn neus door de lucht, ook op zijn overall. Zou hij het gemerkt hebben?
Geen beweging verried het. Zou hij net zo voelen als ik? Hij sprak ook haast
geen woord. Zou hij weten, dat hij twee Anne's voor zich heeft? Het zijn alles
onbeantwoorde vragen.

Om half negen stond ik op, liep naar het raam, daar nemen we altijd afscheid. Ik
trilde nog, ik was nog Anne no twee, hij kwam naar me toe, ik sloeg mijn armen
om zijn hals en drukte een zoen op zijn linkerwang, net wou ik naar de~rechter,
toen mijn mond de zijne ontmoette en we daar op drukten. Duizelend drukten we
ons tegen elkaar aan, nog eens en nog eens, om nooit meer op te houden, o!

Peter heeft behoefte aan tederheid, hij heeft voor het eerst in zijn leven een
meisje ontdekt, heeft voor het eerst gezien, dat de meest plagerige meisjes ook
een innerlijk en een hart hebben en veranderen, zodra ze met je alleen zijn. Hij
heeft voor het eerst in zijn leven zijn vriendschap en zichzelf gegeven, hij
heeft nog nooit ende nimmer eerder een vriend of vriendin gehad. Nu hebben we
elkaar gevonden: ik kende hem ook niet, had ook nooit een vertrouwde en op dit
is het uitgelopen ...

Weer die vraag die me niet loslaat: `Is het goed? Is het goed, dat ik zo gauw
toegeef, dat ik zo heftig ben, net zo heftig en verlangend als Peter zelf? Mag
ik, een meisje, me zo laten gaan?' Er is maar één antwoord op: `Ik verlang zo
... al zo lang, ik ben zo eenzaam en heb nu een troost gevonden!'

's Morgens zijn we gewoon, 's middags ook nog tamelijk, behalve een enkele keer,
maar 's avonds komt het verlangen van de hele dag, het geluk en de zaligheid van
al de vorige keren weer boven en denken we alleen aan elkaar. Iedere avond, na
de laatste kus, zou ik wel weg willen rennen, hem niet meer in de ogen kijken,
weg, weg, in het donker en alleen!

En wat krijg ik, als ik de 14 treden af ben gegaan? Het volle licht, vr gen hier
en lachen daar, ik moet handelen en mag niets laten merken.  Mijn hart is nog te
week om zo'n schok als gisteravond direct weg te duwen. De zachte Anne komt te
weinig en laat zich daarom ook niet dadelijk weer de deur uitjagen. Peter heeft
me geraakt, dieper dan ik ooit in mijn leven geraakt was, behalve in mijn droom!
Peter heeft me aangepakt, mijn binnenste naar buiten gekeerd, is het dan niet
voor ieder mens vanzelfsprekend, dat hij daarna~rust moet hebben om zijn
binnenste weer te herstellen?

O Peter, wat heb je met me gedaan? Wat wil je van me? Waar moet dit naar toe?

O, nu begrijp ik Elli, nu, nu ik dit meemaak, nu begrijp ik haar twijfel; als ik
ouder was en hij zou met me willen trouwen, wat zou ik dan antwoorden? Anne,
wees eerlijk! Je zou niet met hem kunnen trouwen, maar loslaten is ook zo
moeilijk. Peter heeft nog te weinig karakter, te weinig wilskracht, te weinig
moed en sterkte. Hij is nog een kind, niet ouder dan ik in zijn binnenste; hij
wil alleen rust en geluk vinden.

Ben ik werkelijk pas 14? Ben ik werkelijk nog een dom schoolmeisje? Ben ik
werkelijk nog zo onervaren in alles? Ik heb meer ervaring dan de anderen, ik heb
iets meegemaakt, dat haast niemand op mijn leeftijd kent. Ik ben bang voor
mezelf, ik ben bang, dat ik in mijn verlangen me te gauw weggeef, hoe kan dat
later met andere jongens dan goed gaan? O, het is zo moeilijk, altijd is er weer
het hart en het verstand, alles moet op zijn tijd spreken, maar weet ik wel
zeker, dat ik de tijd goed gekozen heb?

Je Anne.

\section*{Dinsdag, 2 Mei 1944}

Lieve Kitty,\\
Zaterdagavond heb ik Peter gevraagd of hij vindt, dat ik vader
iets van ons moet~vertellen, hij vond na een beetje heen en weer van wel. Ik was
blij, het getuigt van zuiver voelen bij hem. Dadelijk toen ik beneden kwam, ging
ik met vader mee water halen, op de trap zei ik al: `Vader, je begrijpt zeker
wel, dat als Peter en ik samen zijn, we geen meter van elkaar afzitten, vind je
dat erg?' Vader antwoordde niet dadelijk, dan zei hij: `Neen, erg vind ik het
niet, maar Anne, hier in die beperkte ruimte moet je voorzichtig zijn'. Hij zei
nog iets in die geest, toen gingen we naar boven. Zondagochtend riep hij me bij
zich en zei: `Anne, ik heb er nog eens over nagedacht' - ik werd al bang -. Het
is~hier in het Achterhuis eigenlijk niet zo goed, ik dacht, dat jullie kameraden
waren. Is Peter verliefd?'

`Geen kwestie van', antwoordde ik.

`Ja, je weet dat ik jullie best begrijp, maar je moet terughoudend zijn, ga niet
meer zo vaak naar boven, wakker hem niet meer aan dan nodig is.  De man is in
zulke dingen altijd de actieve, de vrouw kan tegenhouden.  Het is buiten als je
vrij bent heel iets anders, je ziet andere jongens en meisjes, je kan eens
weggaan, sport doen en van alles, maar hier, als je hier te veel samen zit en je
wilt weg, kun je niet, je ziet elkaar elk uur, altijd eigenlijk. Wees
voorzichtig, Anne, en vat het niet te ernstig op!'

`Dat doe ik niet, vader, maar Peter is fatsoenlijk, hij is een lieve jongen!'

`Ja, maar hij heeft geen sterk karakter, hij is licht naar de goede, maar ook
licht naar de slechte kant te beïnvloeden, ik hoop voor hem dat hij goed blijft,
want in zijn aard is hij goed!'

We praatten nog wat door en spraken af, dat vader ook met hem zou praten.

Zondagmiddag op de voorzolder vroeg hij: `En heb je met je vader gesproken,
Anne?'

`Ja', antwoordde ik, `ik zal het je wel vertellen. Vader vindt het niet erg,
maar hij zegt dat er hier, waar we zo dicht op elkaar zitten, licht botsingen
kunnen komen'.

`We hebben toch afgesproken, dat we geen ruzie zouden maken; ik ben van plan me
daaraan te houden!'

`Ik ook, Peter, maar vader dacht het niet van ons, hij dacht dat we kameraden
waren, vind jij dat dat nu niet kan?'

`Ik wel en jij?'

`Ik ook. Ik heb ook tegen vader gezegd, dat ik je vertrouw. Ik vertrouw op je,
Peter, net zo volkomen als ik het op vader doe en ik geloof, dat je dat waard
bent, is het niet?'

`Ik hoop het'. (Hij was erg verlegen en roodachtig.) `Ik geloof in je, Peter',
vervolgde ik, `ik geloof, dat je~een goed karakter hebt en dat je vooruit zult
komen in de wereld'.

We spraken daarna over andere dingen, later zei ik nog: `Als we hieruit
komen,~weet ik wel, dat je je om mij niet meer zult bekommeren!'

Hij raakte in vuur: `Dat is niet waar, Anne, o neen, dat~\emph{mag}~je niet van
me denken!' Toen werd ik geroepen.

Vader sprak met hem, hij vertelde het mij vandaag. `Je vader dacht, dat die
kameraadschap wel eens op verliefdheid kon uitlopen', zei hij. Maar ik
antwoordde, dat we elkaar wel in bedwang zullen houden.

Vader wil nu, dat ik 's avonds minder naar boven ga, maar dat wil ik niet. Niet
alleen dat ik graag bij Peter ben, ik heb gezegd, dat ik op hem vertrouw. Ik
vertrouw ook op hem en ik wil hem dat vertrouwen ook bewijzen en dat kan ik
nooit doen, als ik uit wantrouwen beneden blijf.

Neen, ik ga!

\section*{Woensdag, 3 Mei 1944}

Lieve Kitty,\\
Eerst even de nieuwtjes van de week. De politiek heeft vacantie;
er is niets, maar~dan ook niets mee te delen. Zo langzamerhand ben ik ook gaan
geloven dat de invasie komt, ze kunnen de Russen toch niet alles alleen laten
opknappen; trouwens die doen ook niets op het ogenblik.

Heb ik je verteld dat onze Mof weg is? Sinds verleden week Donderdag spoorloos
verdwenen. Ze zal zeker al lang in de kattenhemel zijn, terwijl de een of andere
dierenvriend van haar een lekker boutje maakt.  Misschien krijgt een meisje een
muts van haar vel. Peter is over dit feit erg bedroefd.

Sinds Zaterdag lunchen we om half 12 's middags, 's ochtends moeten wij het dus
met een kopje pap volhouden;~dit dient om een maaltijd te sparen. Groente is nog
steeds zeer moeilijk te krijgen: rotte stoofsla hadden we vanmiddag. Gewone sla,
spinazie en stoofsla, anders is er niet. Daarbij nog rotte aardappels, dus een
heerlijke combinatie!

Zoals je je zeker wel kunt indenken wordt hier vaak in vertwijfeling gezegd:
`Waarvoor, o waarvoor dient nu de oorlog? Waarom kunnen de mensen niet vreedzaam
met elkaar leven? Waarom moet alles verwoest worden?'

Deze vraag is begrijpelijk, maar een afdoend antwoord heeft tot nu toe nog
niemand gevonden. Ja, waarom bouwen ze in Engeland steeds grotere vliegtuigen,
construeren ze steeds zwaardere bommen en daarnaast eenheidshuizen voor de
wederopbouw? Waarom worden er elke dag millioenen voor de oorlog besteed en is
er geen cent voor de geneeskunde, voor de kunstenaars, voor de arme mensen
beschikbaar?

Waarom moeten de mensen hongerlijden, als in andere delen van de wereld het
overvloedige voedsel wegrot? O, waarom zijn de mensen zo gek?

Ik geloof nooit dat de oorlog de schuld is alleen van de grote mannen, van de
regeerders en kapitalisten. O neen, de kleine man doet het net zo goed graag,
anders zouden de volkeren er toch al lang tegen in opstand zijn gekomen! Er is
nu eenmaal in de mensen een drang tot vernieling, een drang tot doodslaan, tot
vermoorden en razen en zolang de gehele mensheid, zonder uitzondering, geen
grote metamorphose heeft ondergaan, zal de oorlog woeden, zal alles wat
opgebouwd, aangekweekt en gegroeid is, weer geschonden en vernietigd worden,
waarna de mensheid opnieuw moet beginnen.

Ik ben vaak neerslachtig geweest, maar nooit wanhopig, ik beschouw dit
onderduiken als een gevaarlijk avontuur, dat romantisch en interessant is. Ik
beschouw elke ontbering als een amusement in mijn dagboek. Ik heb me nu~eenmaal
voorgenomen, dat ik een ander leven zal leiden dan andere meisjes en later een
ander leven dan gewone huisvrouwen. Dit is het goede begin van het interessante
en daarom, daarom alleen moet ik in de meest gevaarlijke ogenblikken lachen om
het humoristische van de situatie.

Ik ben jong en bezit nog veel opgesloten eigenschappen, ik ben jong en sterk en
beleef dit grote avontuur, ik zit er nog midden in en kan niet de hele dag
klagen. Ik heb veel meegekregen, een gelukkige natuur, veel vrolijkheid en
sterkte. Elke dag voel ik hoe mijn innerlijk groeit, hoe de bevrijding nadert,
hoe mooi de natuur is, hoe goed de mensen in mijn omgeving zijn, hoe interessant
dit avontuur is! Waarom zou ik dan wanhopig zijn?

Je Anne.

\section*{Vrijdag, 5 Mei 1944}

Lieve Kitty,\\
Vader is ontevreden over mij; hij dacht, dat ik na ons gesprek
van Zondag vanzelf~niet meer elke avond naar boven zou gaan. Hij wil die
`Knutscherei' niet hebben. Dat woord kon ik niet horen, het was al naar genoeg
om er over te spreken, waarom moet hij mij nu ook nog zo naar maken? Ik zal
vandaag met hem spreken. Margot heeft me goede raad gegeven, zie hier wat ik zo
ongeveer wil zeggen:

`Ik geloof, vader, dat je een verklaring van mij verwacht, ik zal je die geven.
Je bent teleurgesteld in mij, je had meer terughouding van mij verwacht, je wilt
zeker, dat ik net zo ben als een 14-jarige behoort te zijn. Daar vergis je je
in!

Sinds we hier zijn, vanaf Juli 1942 tot een paar weken geleden had ik het heus
niet makkelijk. Als je eens wist, wat ik 's avonds al niet afgehuild heb, hoe
ongelukkig ik was, hoe eenzaam ik me voelde, dan zou je wel kunnen begrijpen,
dat ik naar boven wil!

Ik heb het niet van de ene op de andere dag klaar gespeeld, dat ik zóver gekomen
ben, dat ik helemaal zonder moeder~en zonder steun van wie dan ook kan leven;
het heeft me veel, veel strijd en tranen gekost om zo zelfstandig te worden, als
ik nu ben. Je kunt lachen en me niet geloven, het kan me niets schelen, ik weet,
dat ik een mens alleen ben en ik voel me niet voor een cent verantwoordelijk
tegenover jullie. Ik heb je dit alleen verteld, omdat ik dacht, dat je me anders
te stiekem vond, maar voor mijn daden heb ik alleen verantwoording tegenover
mezelf af te leggen.

Toen ik moeilijkheden had, hebben jullie en ook jij je ogen dicht gedaan en je
oren dicht gestopt, je hebt me niet geholpen, integendeel, niets dan
waarschuwingen heb ik gekregen, dat ik niet zo luidruchtig moest zijn. Ik was
luidruchtig, alleen om niet aldoor verdrietig te zijn, ik was overmoedig om niet
steeds die stem van binnen te horen. Ik heb komedie gespeeld anderhalf jaar
lang, dag in dag uit, ik heb niet geklaagd, ben niet uit mijn rol gevallen,
niets van dat alles, en nu, nu ben ik uitgestreden. Ik heb overwonnen! Ik ben
zelfstandig naar lichaam en geest, ik heb geen moeder meer nodig, ik ben door al
die strijd sterk geworden.

En nu, nu ik er bovenop ben, nu ik weet, dat ik uitgevochten heb, nu wil ik ook
zelf mijn weg verder gaan, de weg die ik goed vind. Je kunt en mag me niet als
14 beschouwen, ik ben door alle narigheid ouder geworden; ik zal geen spijt van
mijn daden hebben, ik zal handelen zo als ik denk dat ik dat kan doen.

Je kunt me niet zachtzinnig van boven weghouden, óf je verbiedt me alles, óf je
vertrouwt me door dik en dun, laat me dan ook met rust!'

Je Anne.

\section*{Zaterdag, 6 Mei 1944.}

Lieve Kitty,\\
Gisteren vóór het eten heb ik een brief in vaders zak gestopt,
waarin ik dat schreef,~wat ik je gisteren uitgelegd heb. Na het lezen was hij,
volgens Margot, de hele avond van streek. (Ik was boven aan de afwas.) Arme Pim,
ik kon wel~weten welke uitslag dit epistel zou opleveren. Hij is zo gevoelig!
Dadelijk heb ik tegen Peter gezegd, dat hij niets meer vragen of zeggen moest.
Pim heeft tegen mij niets meer over het geval gezegd, zou het nog komen?

Hier gaat alles weer zo'n beetje. Wat ze ons van de prijzen en mensen buiten
vertellen is haast niet te geloven; een half pond thee kost ƒ 350, een pond
koffie ƒ 80. -, boter ƒ 35. - per pond, een ei ƒ 1.45; voor Bulgaarse tabak
wordt ƒ 14. - per ons betaald! Iedereen handelt zwart, elke loopjongen biedt wat
aan. Onze bakkers jongen heeft stopzijde bezorgd, ƒ 0.90 een dun strengetje, de
melkboer komt aan clandestiene levensmiddelenkaarten, een begrafenisonderneming
bezorgt kaas.  Ingebroken, vermoord en gestolen wordt er elke dag,
politieagenten en nachtwakers doen net zo hard mee als beroepsdieven, ied reen
wil wat in zijn maag hebben en daar salarisverhoging verboden is, moeten de
mensen wel zwendelen. De kinderpolitie blijft aan de gang met opsporingswerk,
meisjes van 15, 16, 17 en ouder worden elke dag vermist.

Je Anne.

\section*{Zondagmorgen, 7 Mei 1944}

Lieve Kitty,\\
Vader en ik hebben gistermiddag een lang gesprek gehad; ik moest
vreselijk huilen~en hij deed mee. Weet je wat hij tegen me zei, Kitty?

`Ik heb al veel brieven in mijn leven gekregen, maar dit is wel de lelijkste!
Jij,

Anne, die zoveel liefde van je ouders ondervonden hebt, die ouders hebt, die
altijd voor je klaar staan, die je altijd verdedigd hebben wat er ook was, jij
spreekt van geen verantwoording voelen? Jij voelt je verongelijkt en alleen
gelaten, neen Anne, dat was een groot onrecht wat je ons hebt aangedaan!

Misschien heb je het niet zo bedoeld, maar het was zo neergeschreven, neen Anne,
zo'n verwijt hebben~\emph{wij}~niet verdiend!'

O, ik heb vreselijk gefaald, dit is wel de ergste daad die~ik in mijn leven
gedaan heb. Ik wou niets dan opscheppen met mijn huilen en mijn tranen, niets
dan me groot voordoen om hem respect voor me te laten hebben. Zeker, ik heb veel
verdriet gehad, maar om dien goeden Pim zo te beschuldigen, hij die alles voor
me gedaan heeft en nog alles voor me doet, neen, dat was meer dan gemeen.

Het is heel goed, dat ik eens uit mijn onbereikbare hoogte neergehaald ben, dat
mijn trots eens een beetje geknakt is, want ik was weer veel te ingenomen met
mezelf. Wat mejuffrouw Anne doet is nog lang niet altijd goed! Iemand die een
ander, waarvan hij zegt te houden, zo'n verdriet aandoet en dat nog wel
opzettelijk, is laag, heel laag!

En het meest schaam ik me nog over de manier, waarop vader me vergeven heeft;
hij zal de brief in de kachel gooien en is nu zo lief tegen me
alsof~\emph{hij}~iets misdaan heeft. Neen Anne, je moet toch nog ontzettend veel
leren, begin daar maar eerst weer eens mee, in plaats van zo op anderen neer te
kijken en anderen te beschuldigen!

Ik heb veel verdriet gehad, maar heeft niet iedereen op mijn leeftijd dat? Ik
heb veel komedie gespeeld, maar ik was het mij nog niet eens bewust, ik voelde
me eenzaam, maar was haast nooit wanhopig! Ik moet me diep schamen en ik schaam
me diep.

Gedane zaken nemen geen keer, maar wel kan men verder voorkomen. Ik wil weer van
voren af aan beginnen en het kan niet moeilijk zijn, daar ik Peter nu heb. Met
hem als steun~\emph{kan}~ik het!

Ik ben niet meer alleen, hij houdt van me, ik houd van hem, ik heb mijn boeken,
mijn schrijfboek en mijn dagboek, ik ben niet zo erg lelijk, niet zo heel erg
dom, heb een vrolijke natuur en wil een goed karakter krijgen!

Ja Anne, je hebt heel goed gevoeld dat je brief te hard en onwaar was, maar je
was er nog trots op! Laat ik vader als voorbeeld weer opnemen, en
ik~\emph{zal}~me beteren.

Je Anne.

\section*{Maandag, 8 Mei 1944}

Lieve Kitty,\\
Heb ik je eigenlijk al wel eens wat van onze familie verteld?\\
Ik geloof van niet en daarom zal ik er maar dadelijk mee beginnen. Mijn
vader~had zeer rijke ouders. Zijn vader had zich helemaal opgewerkt en zijn
moeder was van voorname en rijke familie. Zo had vader in zijn jeugd een echt
rijke-zoontjesleven, elke week partijtjes, bals, feesten, mooie meisjes, diners,
vele kamers enz. enz.

Al dat geld ging na opa's dood door de wereldoorlog en inflatie verloren.  Vader
is dus prima opgevoed en moest gisteren verschrikkelijk lachen, omdat het de
eerste keer in zijn 55 jarig leven was, dat hij aan tafel de koekepan uitkrabde.

Ook moeder is van rijke ouders en vaak luisteren wij met open mond naar verhalen
van verlovingen met 250 mensen, particuliere bals en diners. Nu kan men ons in
geen geval meer rijk noemen, maar al mijn hoop is op na de oorlog gevestigd.

Ik verzeker je, dat ik helemaal niet op zo'n bekrompen leventje gesteld ben als
moeder en Margot. Ik zou graag een jaar naar Parijs en een jaar naar Londen gaan
om de taal te leren en kunstgeschiedenis te studeren.  Vergelijk dat maar met
Margot, die kraamverzorgster in Palestina wil worden! Ik stel me nog altijd veel
van mooie jurken en interessante mensen voor. Ik wil wat zien en beleven in de
wereld, dat heb ik je al meer verteld. En een beetje geld kan daarbij geen
kwaad.

Miep vertelde vanmorgen van een verloving waar ze naar toe was. Zowel de bruid
als de bruidegom zijn van rijke ouders en het was dus ook bijzonder mooi. Miep
maakte ons even lekker met het eten, dat ze kregen: groentesoep met balletjes
gehakt, kaas, broodjes, hors d'oeuvre met eieren en roastbeef, moscovisch gebak,
wijn en sigaretten, van alles zoveel als je maar wilt (clandestien).

Miep heeft 10 borrels gedronken, is dat de anti-alcoholische vrouw? Als Miep al
zoveel op had, hoeveel zou haar ega er dan wel naar binnen geslagen hebben? Ze
waren op dat feest natuurlijk allemaal wat aangeschoten. Er waren twee agenten
van de knokploeg van de politie, die foto's van het paar genomen hebben. Het
blijkt, dat Miep haar onderduikers geen minuut uit haar gedachten kan zetten,
want ze heeft direct naam en adres van deze mannen genoteerd, voor het geval dat
er eens iets mocht gebeuren en men goede Nederlanders nodig heeft.

Ze heeft ons zo lekker gemaakt, wij die voor ons ontbijt niets anders kregen dan
twee lepels pap en die rammelden van de honger, wij die dag in dag uit niets
anders dan half rauwe spinazie (voor de vitaminen) en rotte aardappels krijgen,
wij die in onze holle maag niets dan sla, stoofsla, spinazie en nog eens
spinazie slaan. Misschien worden we nog eens zo sterk als Popeye, hoewel ik er
nog niet veel van zie!

Als Miep ons naar de verloving had meegenomen, dan was er van de broodjes niets
meer voor de andere gasten overgebleven. Ik kan je zeggen, dat we de woorden uit
Mieps mond trokken, dat we om haar heen stonden, alsof we nog nooit in ons leven
van lekker eten of sjieke mensen hadden gehoord!

En dat zijn nu de kleindochters van een millionnair. Het loopt toch wel gek in
de wereld!

Je Anne.

\section*{Dinsdag, 9 Mei 1944}

Lieve Kitty,\\
Het verhaaltje van Ellen de fee is af. Ik heb het overgeschreven
op mooi postpapier,~met rode inkt versierd enfin elkaar genaaid. Het geheel ziet
er wel leuk uit, maar is het niet wat weinig voor vaders verjaardag? Ik weet het
niet. Margot en moeder hebben elk een verjaardagsgedicht gemaakt.

Mijnheer Kraler kwam vanmiddag met het bericht boven, dat mevrouw B., die
vroeger als demonstratrice in de zaak werkte, volgende week elke middag om twee
uur hier op kantoor wil koffiedrinken. Stel je voor!  Niemand kan dan meer naar
boven komen, de aardappels kunnen niet gebracht worden, Elli krijgt geen eten,
we kunnen niet naar de W.C., we mogen ons niet verroeren, en zo voort en zo
voort.

We kwamen met de meest uiteenlopende voorstellen voor de dag om haar af te
wimpelen. Van Daan vond, dat een goed laxeermiddel in haar koffie misschien
afdoende zou zijn. `Neen', antwoordde mijnheer Koophuis, `alsjeblieft niet, dan
komt ze helemaal niet meer van de doos!' Daverend gelach. `Van de doos?' vroeg
mevrouw, `Wat betekent dat?' Een uitleg volgde. `Kan ik dat altijd gebruiken?'
vroeg ze daarna heel onnozel.  `Stel je voor', gichelde Elli, `dat je in de
Bijenkorf naar de doos vroeg, ze zouden je niet eens begrijpen!'

O Kit, het is zulk mooi weer, kon ik maar naar buiten! Je Anne.

\section*{Woensdag, 10 Mei 1944}

Lieve Kitty,\\
We zaten gistermiddag op zolder Frans te leren, toen ik opeens
achter me gekletter~van water hoorde. Ik vroeg Peter wat dat te betekenen had,
maar deze antwoordde niet eens, rende naar de vliering waar de bron van het
onheil was en duwde Mouschi, die wegens een te natte kattebak er naast had
plaatsgenomen met een hardhandig gebaar op de juiste plek. Een luid spektakel
volgde en de inmiddels uitgeplaste Mouschi rende naar beneden.

Mouschi was om nog een beetje kattebakachtig gemak te ondervinden op een beetje
zaagsel gaan zitten. De plas liep dadelijk van de vliering door het plafond naar
de zolder en ongelukkigerwijs precies in en naast de aardappelton.

Het plafond droop en daar de zoldergrond op zijn beurt ook niet vrij van gaatjes
is, vielen er verscheidene gele druppels door het plafond van de kamer, tussen
een stapel kousen en een paar boeken, die op tafel lagen.  Ik lag dubbel van het
lachen, het was ook een te gek gezicht, de ineengedoken Mouschi onder een stoel,
Peter met water, bleekpoeder en dweil en Van Daan aan het sussen. Het onheil was
al gauw hersteld, maar het is een bekend feit, dat katteplasjes verschrikkelijk
stinken. Dat bewezen de aardappels gisteren maar al te duidelijk en bovendien
ook het houtafval, dat vader in een emmer naar beneden haalde om te verbranden.
Arme Mouschi! Kan jij weten, dat er geen turfmolm te krijgen is?

Je Anne.

P.S. Gisteren en vanavond sprak onze geliefde Koningin, ze neemt vacantie om
gesterkt naar Nederland terug te kunnen keren. Ze sprak van `straks als ik terug
ben, spoedige bevrijding, heldenmoed en zware lasten'.

Een speech van minister Gerbrandy volgde. Een dominee besloot de avond met een
bede tot God om voor de Joden, de mensen in concentratiekampen, in gevangenissen
enfin Duitsland te zorgen.

Je Anne.

\section*{Donderdag, 11 Mei 1944}

Lieve Kitty,\\
Ik heb het vreselijk druk op het ogenblik en hoe gek het ook
klinkt, ik heb te weinig~tijd om door mijn berg werk heen te komen.  Zal ik je
eens in het kort vertellen, wat ik alzo moet doen? Nu dan, tot morgen moet ik
het eerste deel van de levensgeschiedenis van Galileo Galilei uitlezen, daar het
naar de bibliotheek terug moet. Gisteren ben ik er mee begonnen, maar ik krijg
het wel uit.

Volgende week moet ik~\emph{Palestina op de tweesprong}~en~het tweede deel
van~\emph{Galilei}~lezen. Verder heb ik gisteren het eerste deel van de
biographie van Keizer Karel V uitgelezen en moet hoog nodig de vele
aantekeningen en stambomen, die ik daaruit gehaald heb, uitwerken.  Vervolgens
heb ik drie bladzijden vreemde woorden, die allen opgezegd, opgeschreven en
geleerd moeten worden, uit de verschillende boeken gehaald. No vier is, dat al
mijn filmsterren in vreselijke wanorde door elkaar liggen en naar een opruiming
snakken; daar echter zo'n opruiming verscheidene dagen in beslag zou nemen en
professor Anne op het ogenblik, zoals gezegd, in het werk stikt, wordt de chaos
toch nog maar chaos gelaten.

Dan wachten Theseus, Oedipus, Peleus, Orpheus, Jason en Hercules op een
ordeningsbeurt, daar verschillende van hun daden als bonte draden in een jurk in
mijn hoofd dooreenliggen, ook Myron en Phidias moeten hoognodig een behandeling
ondergaan, willen ze niet helemaal uit hun samenhang raken. Evenzo gaat het
bijvoorbeeld met de zeven- en negenjarige oorlog; ik haal op die manier alles
door elkaar. Ja, wat moet je ook met zo'n geheugen beginnen! Stel je eens voor
hoe vergeetachtig ik zal zijn als ik tachtig ben!

O nog wat, de Bijbel, hoe lang zal het nog duren of ik ontmoet het verhaal van
de badende Suzanna? En wat bedoelen ze met de schuld van Sodom en Gomorrha? O er
is nog zo ontzettend veel te vragen en te leren.  En Liselotte von der Pfalz,
die heb ik intussen helemaal in de steek gelaten.

Kitty, zie je wel, dat ik overloop?

Nu over iets anders: Je weet al lang dat mijn liefste wens is eenmaal
journaliste en later een beroemd schrijfster te worden. Of ik deze
grootheidsneigingen (of - waanzin?) ooit tot uitvoering zal kunnen brengen, zal
nog moeten blijken, maar onderwerpen heb ik tot nu toe nog wel. Na de oorlog wil
ik in ieder geval een boek getiteld `Het Achterhuis' uitgeven. Of dat lukt
blijft ook nog de vraag, maar mijn dagboek zal daarvoor kunnen dienen.

Behalve `Het Achterhuis' heb ik nog meer onderwerpen op stapel staan.  Daarover
schrijf ik je nog wel eens uitvoerig, als ze vaste vorm hebben aangenomen.

Je Anne.

\section*{Zaterdag, 13 Mei 1944}

Liefste Kitty,\\
Gisteren was vader jarig; vader en moeder waren 19 jaar
getrouwd, het was geen~werksterdag en de zon scheen, zoals ze in 1944 nog nooit
geschenen had. Onze kastanjeboom staat van onder tot boven in volle bloei, hij
is zwaar beladen met bladeren en veel mooier dan verleden jaar.

Vader heeft van Koophuis een biographie over het leven van Linnaeus gekregen,
van Kraler een boek over de natuur, van Dussel\emph{Amsterdam te water}, van Van
Daan een reuze doos opgemaakt als door de beste decorateur met drie eieren, een
fles bier, een fles yoghurt en een groene das er in. Onze pot stroop stak wel
wat af. Mijn rozen geuren heerlijk in tegenstelling tot Miep en Elli's anjers,
die geurloos maar ook heel mooi zijn. Hij is wel verwend. Er zijn 50 taartjes
gekomen, heerlijk!  Vader tracteerde zelf op kruidkoek, voor de heren bier en
voor de dames yoghurt. Alles viel in de smaak.

Je Anne.

\section*{Dinsdag, 16 Mei 1944}

Liefste Kitty,\\
Ter afwisseling, omdat we het er zo lang niet meer over gehad
hebben, wil ik je~een kleine discussie oververtellen, die mijnheer en mevrouw
gisteravond hadden. Mevrouw: `De Duitsers zullen de Atlantik-Wall wel erg sterk
gemaakt hebben, ze zullen zeker alles doen wat in hun macht ligt om de Engelsen
tegen te houden.

Het is toch enorm hoeveel kracht de Duitsers hebben!' Mijnheer: `O ja,
ontzettend'.

Mevrouw: `Ja-ah'.\\
Mijnheer: `Zeker zullen de Duitsers op het eind de oorlog
nog winnen, zo sterk~zijn ze!'

Mevrouw: `Dat kan best, ik ben van het tegendeel nog niet overtuigd'.~

Mijnheer: `Ik zal maar geen antwoord meer geven'.\\
Mevrouw: `Je antwoordt mij
toch altijd weer, je laat je toch telkens weer~meeslepen'.

Mijnheer: `Welneen, mijn antwoorden zijn heel miniem'.\\
Mevrouw: `Maar je
antwoordt toch en je moet ook altijd gelijk hebben! Je~voorspellingen komen lang
niet altijd uit!'

Mijnheer: `Tot nu toe zijn mijn voorspellingen uitgekomen'.\\
Mevrouw: `Dat is
niet waar. De invasie was er vorig jaar al, de Finnen hadden al~vrede, Italië
was in de winter afgelopen, de Russen hadden Lemberg al, o neen, om jouw
voorspellingen geef ik niet veel'.

Mijnheer (opstaand): `En hou je nu eindelijk je grote bek, ik zal je nog eens
bewijzen, dat ik gelijk heb, één keer zul je er toch genoeg van krijgen, ik kan
dat gekanker niet meer horen, met je neus zal ik je in al je plagerijen duwen!'

Einde eerste bedrijf.

Ik moest eigenlijk ontzettend lachen, moeder ook en Peter zat zich eveneens te
verbijten. O die domme volwassenen, laat ze zelf liever beginnen te leren, vóór
ze zoveel op de kinderen hebben aan te merken!

Je Anne.

\section*{Vrijdag, 19 Mei 1944}

Lieve Kitty,\\
Gisteren was ik al heel ellendig, overgegeven (en dat Anne!),
buikpijn, alle~narigheid die je je maar kunt indenken. Vandaag gaat het weer
veel beter, ik heb erge honger, maar van die bruine bonen die we vandaag
krijgen, zal ik liever afblijven.

Met Peter en mij gaat het best, de arme jongen heeft nog~meer behoefte aan wat
tederheid dan ik. Hij bloost nog elke avond bij zijn nachtkus en bedelt gewoon
om nog een. Zou ik de betere vervanging van Mof zijn? Ik vind het niet erg, hij
is al zo gelukkig, nu hij weet dat er iemand van hem houdt.

Ik sta na mijn moeizame verovering een beetje boven de situatie, maar je mag
niet denken, dat mijn liefde ver flauwd is. Hij is een schat, maar mijn
innerlijk heb ik gauw weer dichtgesloten; als hij nu nog eens het slot wil
verbreken moet het breekijzer wel harder zijn!

Je Anne.

\section*{Zaterdag, 20 Mei 1944}

Lieve Kitty,\\
Gisteravond kwam ik van de zolder naar beneden, en zag dadelijk,
toen ik de~kamer binnenging, dat de mooie vaas met anjers op de grond lag,
moeder op haar knieën aan het dweilen was, terwijl Margot mijn papieren van de
grond viste.

`Wat is hier gebeurd?' vroeg ik met angstige voorgevoelens, en hun antwoord niet
eens afwachtend bekeek ik van een afstand de schade. Mijn hele stambomen-map,
schriften, boeken, alles dreef. Ik huilde haast en was zo opgewonden, dat ik me
van mijn woorden niets meer herinneren kan, maar Margot zei, dat ik iets
uitkraamde van `unübersehbare Schaden, verschrikkelijk, ontzettend, nooit meer
goed te maken' en nog meer.  Vader schaterde, moeder en Margot vielen in, maar
ik kon wel huilen om al dat verloren werk en de goed uitgewerkte aantekeningen.

Bij nader bekijken viel de `unübersehbare Schaden' gelukkig mee, zorgvuldig
zocht ik op de zolder de samengeplakte papiertjes bij elkaar en maakte ze los.
Daarna hing ik ze naast elkaar aan de waslijnen te drogen. Het was een grappig
gezicht en ik moest toen toch weer lachen: Maria de Medici naast Karel de V,
Willem van Oranje en Marie~Antoinette; dat is `Rassenschande', grapte mijnheer
Van Daan. Na Peter de verzorging van mijn papiertjes te hebben toevertrouwd,
ging ik weer naar beneden.

`Welke boeken zijn bedorven?' vroeg ik aan Margot, die mijn boekenschat aan het
controleren was. `Algebra', zei Margot. Ik kwam gauw naderbij, maar jammer
genoeg was ook het algebraboek nog niet bedorven. Ik wou dat dat in de vaas was
gevallen: ik heb nog nooit aan één boek zo'n hekel gehad als aan dat
algebraboek. Voorin staan minstens 20 namen van meisjes, die het al vóór mij in
hun bezit hadden, het is oud, geel, volgekrabbeld en verbeterd. Als ik nog eens
heel erg baldadig ben, scheur ik dat rotding aan stukken!

Je Anne.

\section*{Maandag, 22 Mei 1944}

Lieve Kitty,\\
Vader heeft op 20 Mei vijf esjes yoghurt aan mevrouw Van Daan
verloren met~wedden. De invasie is werkelijk nog niet gekomen; ik mag gerust
zeggen dat heel Amsterdam, heel Nederland, ja heel de Westkust van Europa tot
Spanje toe, dag en nacht over de invasie spreekt en debatteert, er op wedt en
... hoopt.

De spanning stijgt ten top. Lang niet alle mensen die wij tot de `goede'
Nederlanders rekenen hebben hun vertrouwen in de Engelsen behouden, lang niet
allen vinden de Engelse bluf een meesterlijk staaltje, o neen, de mensen willen
nu eindelijk eens daden zien, grote en heldhaftige daden.  Niemand denkt verder
dan zijn neus lang is, niemand denkt er aan, dat de Engelsen voor zichzelf en
hun land strijden, iedereen denkt dat ze verplicht zijn zo gauw en zo goed
mogelijk Holland te redden.

Welke verplichtingen hebben de Engelsen dan aan ons? Waaraan hebben de
Hollanders de edelmoedige hulp verdiend, die zij zo stellig verwachten?  O neen,
de Nederlanders~zullen zich nog erg vergissen, de Engelsen hebben zich ondanks
al hun bluf toch zeker niet meer geblameerd dan alle andere landen en landjes,
die nu bezet zijn. De Engelsen zullen ons heus geen verontschuldigingen
aanbieden, want, al zouden wij hun verwijten, dat zij hebben geslapen gedurende
de jaren dat Duitsland zich bewapende, wij kunnen niet ontkennen, dat al de
andere landen, vooral de landen die aan Duitsland grenzen, ook geslapen hebben.
Met struisvogelpolitiek komen wij er niet, dat heeft Engeland en dat heeft de
hele wereld ingezien en daarvoor moeten alle geallieerden nu stuk voor stuk en
Engeland niet het minst zware offers brengen.

Geen land zal voor niets zijn mannen opofferen en vooral niet voor de belangen
van een ander, ook Engeland zal dat niet doen. De invasie, de bevrijding en de
Vrijheid zullen eenmaal komen, doch Engeland en Amerika zullen het tijdstip
vaststellen en niet al de bezette gebieden samen.

Tot ons grote leedwezen en onze grote ontzetting hebben wij gehoord, dat de
stemming tegenover ons Joden bij vele mensen omgeslagen is. We hebben gehoord,
dat er anti-semitisme is gekomen in kringen, die daaraan vroeger niet dachten.
Ons achten allemaal heeft dit feit diep, heel diep geraakt. De oorzaak van deze
Jodenhaat is begrijpelijk, soms zelfs menselijk, maar niet goed. De Christenen
verwijten de Joden, dat ze bij de Duitsers hun mond voorbij praten, dat ze hun
helpers verraden, dat vele Christenen door toedoen van de Joden het vreselijke
lot en de vreselijke straf van zovelen ondergaan.

Dit is alles waar, maar wij moeten als bij alle dingen ook de keerzijde van de
medaille bekijken. Zouden de Christenen in onze plaats anders handelen? Kan een
mens, onverschillig Jood of Christen, voor Duitse middelen blijven zwijgen?
Iedereen weet, dat dit haast onmogelijk is, waarom eist men dan het onmogelijke
van de Joden?

Er wordt in ondergrondse kringen over gemompeld, dat Duitse Joden, die naar
Nederland geëmigreerd zijn en nu in Polen zitten, niet meer naar Nederland terug
zullen mogen komen; ze hadden in Nederland asylrecht, maar zullen, als Hitler
weg is, weer naar Duitsland terug moeten.

Als men dat hoort, vraagt men zich dan niet vanzelfsprekend af, waarom deze
lange en moeilijke oorlog gevoerd wordt? Wij horen toch altijd, dat wij allen
samen vechten voor vrijheid, waarheid en recht! En begint er nog tijdens dat
gevecht al tweedracht te komen, is toch de Jood weer minder dan de ander? O het
is treurig, heel erg treurig, dat weer voor de zoveelste maal de oude wijsheid
bevestigd is: `Wat één Christen doet, moet hijzelf verantwoorden, wat één Jood
doet, valt op alle Joden terug'.

Eerlijk gezegd kan ik niet begrijpen, dat Nederlanders, mensen van dit goede,
eerlijke en rechtschapen volk zó oordelen over ons, zó oordelen over het meest
verdrukte, het ongelukkigste en het meest meelijwekkende van alle volkeren van
heel de wereld misschien.

Ik hoop maar één ding, namelijk dat deze Jodenhaat van voorbijgaande aard zal
zijn, dat de Nederlanders toch zullen laten zien wie zij zijn, dat zij nu niet
en nooit zullen wankelen in hun rechtsgevoel. Want anti-semitisme is
onrechtvaardig!

En als dit vreselijke inderdaad waarheid zou worden, dan zal het armzalige
restje Joden uit Nederland weggaan. Wij ook, wij zullen weer verder trekken met
ons bundeltje, uit dit mooie land dat ons zo hartelijk onderdak heeft aangeboden
en ons nu de rug toekeert.

Ik houd van Nederland, ik heb eenmaal gehoopt dat het mij, vaderlandsloze, als
vaderland zal mogen dienen, ik hoop het nog!

Je Anne.

\section*{Donderdag, 25 Mei 1944}

Lieve Kitty,\\
Iedere dag wat anders, vanochtend is onze groenteman gepakt, hij
had twee Joden~in huis. Het is een zware slag voor ons, niet alleen dat die arme
Joden weer aan de rand van de afgrond staan, maar voor dien man is het
vreselijk.

De wereld staat hier op zijn kop, de fatsoenlijke mensen worden weggestuurd naar
concentratiekampen, gevangenissen en eenzame cellen en het uitschot regeert over
jong en oud, rijk en arm. De één vliegt er in door de zwarte handel, de tweede
door het helpen van Joden of andere onderduikers, niemand die niet bij de N.S.B.
is, weet wat er morgen gebeurt.

Ook voor ons is deze man een zeer zwaar gemis. De meisjes kunnen en mogen die
porties aardappelen niet aanslepen, het enige wat er op zit is minder te eten.
Hoe we het zullen doen zal ik je nog wel meedelen, prettiger wordt het zeker
niet. Moeder zegt, dat we 's ochtends helemaal geen ontbijt, 's middags pap en
brood, 's avonds gebakken aardappels en eventueel één of twee keer per week
groente of sla krijgen, meer niet.  Dat zal hongeren worden, maar alle
ontberingen zijn niet zo erg als ontdekt worden.

Je Anne.

\section*{Vrijdag, 26 Mei 1944}

Lieve Kitty,\\
Eindelijk, eindelijk ben ik dan zover, dat ik rustig aan mijn
tafeltje voor de spleet~van het raam kan zitten en je alles, alles kan
schrijven.

Ik voel me zo ellendig als in maanden niet het geval is geweest, zelfs niet na
de~inbraak was ik zo van binnen en buiten kapot. Aan de ene kant de groenteman,
de Jodenkwestie, die in het hele huis uitvoerig besproken wordt, de uitblijvende
invasie, het slechte eten, de spanning, de miserabele stemming, de
teleurstelling om Peter en aan de andere kant, Elli's verloving,
Pinksterreceptie, bloemen,~Kralers verjaardag, taarten en verhalen van cabarets,
films en concerten. Dat verschil, dat grote verschil, dat is er altijd, de ene
dag lachen we om het humoristische van onze onderduik-situatie, maar de andere
dag en nog veel meer dagen zijn we bang en staan angst, spanning en wanhoop op
ons gezicht te lezen.

Miep en Kraler ondervinden het meest de last van ons achten, Miep in haar werk,
Kraler, die de kolossale verantwoording soms te machtig wordt en die bijna niet
meer spreken kan van ingehouden zenuwen en opwinding.  Koophuis en Elli zorgen
ook goed voor ons, zeer goed zelfs, maar eenmaal is voor hen ook het Achterhuis
vergeten, al is het dan maar voor een paar uurtjes, een dag, twee dagen
misschien. Ze hebben hun eigen zorgen, Koophuis voor zijn gezondheid, Elli om
haar verloving die er niet erg rooskleurig uitziet en naast die zorgen hebben ze
ook hun uitstapjes, hun visites, hun hele leven van gewone mensen. Voor hen
wijkt de spanning soms, al is het maar voor een korte tijd, voor ons wijkt ze
nooit. Twee jaar lang duurt het nu al en hoe lang nog zullen wij aan deze haast
ondraaglijke, steeds groeiende druk weerstand moeten blijven bieden?

De riolering is verstopt, er mag geen water aflopen of alleen druppelsgewijs, we
mogen niet naar de W.C. of moeten een borstel meenemen, het vuile water bewaren
we in een grote Keulse pot. Voor vandaag kunnen we ons helpen, maar wat te doen
als de loodgieter het niet alleen af kan? De stadsreinigingsdienst komt niet
vóór Dinsdag.

Miep stuurde ons een krentenmik met het opschrift `Vrolijke Pinksteren'.  Het is
haast alsof ze spot, onze stemming en onze angst zijn heus niet `vrolijk'. We
zijn allemaal banger geworden na de aangelegenheid met den groenteman, je hoort
weer van alle kanten `sst. sst.', alles gebeurt zachter. De politie heeft daar
de deur geforceerd, dus daar zijn wij ook niet veilig voor! Als ook wij eens ...
neen, ik mag het niet neerschrijven, maar de vraag laat zich vandaag
niet~wegduwen, integendeel, al de eens doorgemaakte angst staat weer voor me in
al zijn verschrikking.

Ik moest alleen naar beneden naar de W.C. om acht uur vanavond, niemand was
beneden, allen zaten aan de radio, ik wou moedig zijn, maar het was moeilijk. Ik
voel me hierboven nog altijd veiliger dan alleen beneden in dat grote, stille
huis; alleen met de geheimzinnige, stommelachtige geluiden van boven en het
getoeter van de claxons op straat. Ik tril als ik niet gauw voortmaak en even
nadenk over de situatie.

Ik vraag me steeds weer af, of het niet beter voor ons allemaal was geweest als
we niet waren gaan onderduiken, als we nu dood waren en deze ellende niet
meemaakten, vooral, omdat dan onze beschermers geen gevaar meer zouden lopen.
Maar ook voor die gedachte deinzen we allen terug, we houden nog van het leven,
we zijn de stem van de natuur nog niet vergeten, we hopen nog, hopen op alles.
Laat er nu gauw wat gebeuren, desnoods schieten, dat kan ons niet méér
vermorzelen dan deze onrust doet. Laat het einde komen, al is het hard, dan
weten we tenminste of we uiteindelijk zullen overwinnen of ten onder gaan.

Je Anne.

\section*{Woensdag, 31 Mei 1944}

Lieve Kitty,\\
Zo'n mooie, warme, men kan gerust zeggen hete Pinksteren is het
nog zelden~geweest. Hitte is hier in het Achterhuis vreselijk; om je een indruk
van de vele klachten te geven, zal ik je in het kort de warme dagen beschrijven:

Zaterdag: `Heerlijk, wat een weer', zeiden we allen 's ochtends. `Was het maar
wat minder warm', 's middags toen de ramen dicht moesten.

Zondag: `Het is niet om uit te houden, die hitte. De boter smelt, er is geen
koel plekje in huis, het brood wordt droog,~de melk bederft, geen raam kan open,
wij arme uitgestotenen zitten hier te stikken, terwijl de andere mensen
Pinkstervacantie hebben'.

Maandag: `Mijn voeten doen me pijn. Ik heb geen dunne kleren. Ik kan in die
warmte niet afwassen', aldus mevrouw. Het was reuze naar.

Ik kan nog steeds niet tegen warmte en ben blij dat vandaag de wind behoorlijk
blaast en het zonnetje toch schijnt.

Je Anne.

\section*{Maandag, 5 Juni 1944}

Lieve Kitty,\\
Nieuwe Achterhuis-narigheden, ruzie tussen Dussel en Franks over
iets heel~onbelangrijks: de boterverdeling. Capitulatie van Dussel. Dikke
vriendschap tussen mevrouw Van Daan en laatstgenoemde, irtpartijen, zoentjes en
vriendelijke lachjes. Dussel begint vrouwen-verlangens te krijgen.

Inneming van Rome door het vijfde leger, de stad is noch verwoest noch
gebombardeerd.

Weinig groente en aardappelen. Weer slecht. Aanhoudende zware bombardementen op
Pas de Calais en Franse kust.

Je Anne.

\section*{Dinsdag, 6 Juni 1944}

Liefste Kitty,\\
`This is D.-day', zei om 12 uur de Engelse radio en terecht,
`This is~\emph{the}~day', de~invasie is begonnen!

Vanochtend om acht uur berichtten de Engelsen: zwaar bombardement van
Calais,~Boulogne, Le Hâvre en Cherbourg, alsmede Pas de Calais (Zoals
gewoonlijk). Verder een veiligheidsmaatregel voor de bezette gebieden, alle
mensen die in de zône van 35 km van de kust wonen, moeten zich op bombardementen
voorbereiden. Zo mogelijk zullen de Engelsen een uur van te voren pamfletten
uitwerpen.

Volgens Duitse berichten zijn er Engelse parachutetroepen aan de Franse kust
geland. Engelse landingsboten in gevecht met Duitse mariniers, aldus de B.B.C.

Discussie in het Achterhuis om negen uur aan het ontbijt: Is dit een
proeflanding net als twee jaar geleden bij Dieppe?

Engelse uitzending in het Duits, Nederlands, Frans en andere talen om 10 uur:
`The invasion has begun!' Dus: de `echte' invasie. Engelse uitzending in het
Duits 11 uur: Speech van opperbevelhebber Generaal Dwight Eisenhower.

Engelse uitzending in het Engels 12 uur: `This is D.-day'. Generaal Eisenhower
zei tegen het Franse volk: `Stiff ghting will come now, but after this the
victory. The year 1944 is the year of complete victory, good luck!'

Engelse uitzending in het Engels één uur (vertaald):

11000 vliegtuigen staan gereed en vliegen onophoudelijk af en aan om troepen
neer te laten en achter de linies te bombarderen. 4000 landingsvaartuigen plus
kleine boten brengen tussen Cherbourg en Le Hâvre onophoudelijk troepen en
materiaal aan wal. Engelse en Amerikaanse troepen zijn al in harde gevechten
gewikkeld. Speeches van Gerbrandy, den eersten Minister van België, Koning
Haakon van Noorwegen, de Gaulle voor Frankrijk, den Koning van Engeland en niet
te vergeten Churchill.

Het Achterhuis in opschudding! Zou dan nu werkelijk de lang verbeide bevrijding
naderen, de bevrijding waarover zo veel gesproken is, maar die toch tè mooi is,
tè sprookjesachtig om ooit werkelijkheid te kunnen worden? Zou dit jaar, 1944,
ons de overwinning schenken? We weten het ook nu nog niet, maar de hoop doet ons
herleven, maakt ons weer moedig, maakt ons weer sterk. Want moedig moeten wij de
vele angsten, de ontberingen en het lijden doorstaan, nu komt het er op aan om
kalm en standvastig te blijven. Nu meer dan ooit is het zaak de nagels in het
vlees te drukken en niet te schreeuwen. Schreeuwen van ellende kunnen~Frankrijk,
Rusland, Italië en ook Duitsland, maar wij hebben daar het recht nog niet toe!

O Kitty, het mooiste van die invasie is, dat ik het gevoel heb, dat er vrienden
in aantocht zijn. Die vreselijke Duitsers hebben ons zo lang onderdrukt en het
mes op de keel gezet, dat de gedachten aan vrienden en uitredding ons met
vertrouwen bezielt!

Nu geldt het niet meer de Joden, nu geldt het Nederland en heel bezet Europa.
Misschien, zegt Margot, kan ik met September of October toch nog naar school.

Je Anne.

P.S. Ik zal je van de nieuwste berichten op de hoogte houden!

\section*{Vrijdag, 9 Juni 1944}

Lieve Kitty,\\
Met de invasie gaat het prima-prima. De geallieerden hebben
Bayeux, een dorpje~aan de Franse kust ingenomen en strijden nu om Caen. Het is
duidelijk, dat het de bedoeling is, om het schiereiland waarop Cherbourg ligt,
af te snijden. Iedere avond vertellen oorlogscorrespondenten van de
moeilijkheden, moed en geestdrift van het leger; de ongelofelijkste staaltjes
wisten ze te vertellen. Ook gewonden, die nu alweer in Engeland terug zijn,
waren voor de microfoon.  Ondanks het miserabele weer vliegen ze vlijtig. Het is
ons via de B.B.C.  ter ore gekomen, dat Churchill de invasie met de troepen wou
meemaken, alleen op afraden van Eisenhower en andere Generaals is dit plan niet
doorgegaan. Denk eens aan, wat een moed van zo'n ouden man, hij is zeker al 70
jaar.

Hier is de opwinding weer wat bedaard, toch hopen wij dat de oorlog eind van het
jaar eindelijk afgelopen zal zijn. Het zal tijd worden! Mevrouw Van Daans gezeur
is niet om aan te horen, nu ze ons met de invasie niet meer gek kan~maken,
zanikt ze de hele dag over het slechte weer. Je zou zin hebben haar in een emmer
koud water op de vliering te zetten.

Je Anne.

\section*{Dinsdag, 13 Juni 1944}

Lieve Kitty,\\
Mijn verjaardag is weer voorbij, nu ben ik dus 15. Ik heb nogal
veel gekregen. De vijf delen Springer~\emph{Kunstgeschiedenis}, een stel
ondergoed, twee ceintuurs, een~zakdoek, twee esjes yoghurt, een potje jam, een
kruidkoek, een plantkundeboek van vader en moeder. Doublé armband van Margot,
Patria-boek van de Van Daans, biomals en lathyrus van Dussel, snoep en schriften
van Miep en Elli en het hoogtepunt: het boek~\emph{Maria Theresia}~en drie
plakjes volvette kaas van Kraler. Van Peter een mooie bos pioenrozen, de arme
jongen heeft zich zoveel moeite gegeven om wat te vinden, maar niets is gelukt.

Met de invasie gaat het nog steeds uitstekend, ondanks het miserabele weer, de
talloze stormen, gietbuien en hoge zeeën.

Churchill, Smuts, Eisenhower en Arnold waren gisteren in de Franse dorpen, die
door de Engelsen veroverd en bevrijd zijn. Churchill was op een torpedoboot die
de kust beschoot. Die man schijnt, als zovele mannen, geen angst te kennen,
benijdenswaardig!

De stemming in Nederland is vanuit ons Achterfort niet te peilen.  Ongetwijfeld
zijn de mensen blij, dat het nietsdoende (!) Engeland eindelijk ook zelf eens de
handen uit de mouwen steekt. Alle Nederlanders die nu nog op de Engelsen
neerkijken, Engeland en de oude-heren-regering beschimpen, Engelsen laf noemen,
maar toch de Duitsers haten, moeten eens net zo opgeschud worden als dat met~een
kussen gebeurt. Misschien leggen zich die verwarde hersenen dan in een wat
betere plooi.

Je Anne.

\section*{Woensdag, 14 Juni 1944}

Lieve Kitty,\\
Veel wensen, veel gedachten, veel beschuldigingen en veel
verwijten spoken er~in mijn hoofd om. Ik ben heus niet vol verbeelding, zoals
vele mensen menen, ik weet mijn talloze fouten en gebreken beter dan wie ook,
alleen met dit verschil dat ik ook weet, dat ik me ook beteren wil, beteren zal
en me al veel verbeterd heb.

Hoe komt het dan, vraag ik me zo vaak af, dat iedereen me nog zo ontzettend
eigenwijs en onbescheiden vindt? Ben ik zo eigenwijs?  Ben~\emph{ik}~het
werkelijk, of zijn het misschien niet ook de anderen?  Dat klinkt gek, ik merk
het, maar ik streep mijn laatste zin niet door, omdat hij heus niet zo gek is.
Mevrouw Van Daan, een van mijn voornaamste beschuldigers, staat bekend als
onintelligent en laat ik het maar gerust uitspreken `dom'. Domme mensen kunnen
het meestal niet verkroppen, als anderen iets beter doen dan zijzelf.

Mevrouw vindt mij dom, omdat ik niet in zo grote mate als zijzelf in begrip
tekort schiet, ze vindt mij onbescheiden, omdat zij het nog erger is, ze vindt
mijn jurken te kort, omdat de hare nog korter zijn. En daarom vindt zij mij ook
eigenwijs, omdat zij zelf nog wel tweemaal zoveel als ik meepraat over
onderwerpen, waarvan zij totaal geen verstand heeft. Maar één van mijn meest
geliefde spreuken is: `Van ieder verwijt is wel wat waar', en ik geef dan ook
grif toe, dat ik eigenwijs ben.

Nu is het lastige van mijn natuur, dat ik van niemand zoveel standjes en zoveel
kritiek krijg als van mijzelf. Als moeder daar nog haar portie raadgevingen aan
toevoegt, wordt de hoop preken zo onoverkomelijk groot, dat ik van wanhoop om er
nog ooit uit te komen, brutaal word en ga tegenspreken~en dan komt vanzelf het
bekende en al zo oude Anne-woord terug: `Niemand begrijpt me ook!'

Dit woord zit in me en hoe onwaar het ook mag schijnen, ook hier is een tipje
waarheid in. Mijn zelf beschuldigingen nemen vaak zo'n omvang aan, dat ik naar
een troostende stem snak, die ze weer in gezonde vormen terugbrengt en zich ook
iets van mijn binnenkamer aantrekt, maar helaas, ik kan lang zoeken, gevonden is
diegene nog niet.

Ik weet, dat je nu aan Peter denkt, hé Kit? Het is zo, Peter houdt van me, niet
als verliefde maar als vriend, zijn toegenegenheid stijgt met de dag, maar wat
dat geheimzinnige is, dat ons alle twee tegenhoudt, dat begrijp ik zelf niet.
Soms denk ik wel eens, dat dat verschrikkelijke verlangen van mij naar hem
overdreven was, maar dat is toch heus niet zo, want als ik eens twee dagen niet
boven was, verlang ik weer zo hevig naar hem als nooit te voren. Peter is lief
en goed, toch mag ik niet loochenen, dat veel in hem mij teleurstelt. Vooral
zijn godsdienstafkeer, eetgesprekken en nog meer van dergelijke uiteenlopende
dingen bevallen me niet. Toch ben ik er vast van overtuigd, dat wij naar onze
eerlijke afspraak nooit ruzie zullen krijgen, Peet is vredelievend, verdraagzaam
en erg toegevend. Hij laat zich door mij veel meer dingen zeggen dan hij zijn
moeder toestaat, hij probeert met veel hardnekkigheid orde op zijn zaken te
houden. Maar toch, waarom blijft zijn binnenste dan binnen en mag ik daar nooit
aan raken? Hij is een veel meer gesloten natuur dan ik, dat is waar, maar ik
weet - en nu werkelijk uit de praktijk - dat zelfs de meest gesloten naturen op
hun tijd net zo hard of nog harder naar een vertrouwde verlangen.

Peter en ik hebben alle twee onze denkjaren in het Achterhuis doorgebracht, we
praten vaak over toekomst, verleden en heden, maar zoals gezegd, het echte mis
ik en toch weet ik zeker dat het er is.

Je Anne.

\section*{Donderdag, 15 Juni 1944}

Lieve Kitty,\\
Zou het komen, omdat ik zolang mijn neus niet in de buitenlucht
kon steken, dat~ik zo dol ben geworden op alles wat natuur is? Ik weet nog heel
goed, dat een stralend blauwe hemel, piepende vogels, maneschijn en bloeiende
bloemen mijn aandacht vroeger nog lang niet konden vasthouden. Hier is dat
anders geworden.

Ik heb met Pinksteren bijvoorbeeld, toen het zo warm was, 's avonds met geweld
mijn ogen open gehouden, om bij half 12 de maan aan het open raam eens één keer
goed en alleen te kunnen bekijken. Helaas liep deze opoffering op niets uit,
daar de maan te veel licht verspreidde en ik een open raam niet mocht riskeren.
Een andere keer, alweer heel wat maanden geleden, was ik toevallig boven, toen
het raam 's avonds open was. Ik ging niet eerder naar beneden, voordat de
luchttijd afgelopen was. De donkere regenachtige avond, de storm, de jagende
wolken hielden me in hun macht gevangen; na anderhalf jaar had ik voor het eerst
weer de nacht van aangezicht tot aangezicht gezien. Na die avond was mijn
verlangen dit nog eens te zien groter dan mijn angst voor dieven, donker
rattenhuis of overvallen. Ik ging geheel alleen naar beneden en keek naar buiten
uit het raam van het privé-kantoor en de keuken. Vele mensen vinden de natuur
mooi, velen slapen eens onder de blote hemel, velen verlangen in gevangenissen
of ziekenhuizen naar de dag, dat ze weer vrij van de natuur kunnen genieten,
maar weinigen zijn met hun verlangen zo afgesloten en geïsoleerd van datgene wat
voor arm en rijk hetzelfde is.

Het is geen verbeelding dat het zien van de hemel, de wolken, de maan en de
sterren mij rustig en afwachtend maakt. Dit middel is veel beter dan valeriaan
of broom, de natuur maakt mij deemoedig en gereed om alle slagen dapper op te
vangen.

Het heeft helaas zo moeten zijn, dat ik de natuur - en dan nog maar bij
uitzondering - door dikbestoven en vuil begordijnde ramen mag zien. En om
daardoor te kijken is geen plezier meer, want de natuur is het enige dat
werkelijk geen surrogaat kan verdragen.

Je Anne.

\section*{Vrijdag, 16 Juni 1944}

Lieve Kitty,\\
Nieuwe problemen: mevrouw is wanhopig, spreekt van een kogel door
de kop,~gevangenis, ophangen en zelfmoord. Zij is jaloers, dat Peter mij wel en
haar niet zijn vertrouwen schenkt. Ze is beledigd dat Dussel niet voldoende op
haar flirterijen ingaat, bang dat haar man al het bontmantel-geld oprookt, maakt
ruzie, scheldt, huilt, beklaagt zich, lacht en begint dan weer met ruzie.

Wat moet je nu met zo'n snotterend en mal exemplaar beginnen? Au sérieux wordt
ze door niemand genomen, karakter heeft ze niet, klagen doet ze bij ieder en ze
loopt rond als: von hinten Lyceum, von vorne Museum.  Daarbij is nog het ergste
dat Peter brutaal, mijnheer Van Daan prikkelbaar en moeder cynisch wordt. Ja,
dat is me een toestand! Er is maar één regel die je goed voor ogen moet houden:
lach om alles en stoor je niet aan anderen! Het lijkt egoïstisch, het is in
werkelijkheid het enige geneesmiddel voor wie troost bij zichzelf moet vinden.

Kraler heeft weer een oproep om vier weken te gaan spitten.\\
Hij probeert vrij
te komen door een dokters-attest en een brief van de zaak.

Koophuis wil een maagoperatie ondergaan. Gisteren om 11 uur is van alle
particulieren de telefoon afgesneden.

Je Anne.

\section*{Vrijdag, 23 Juni 1944}

Lieve Kitty,\\
Niets bijzonders aan de hand hier. De Engelsen zijn de grote
aanval op Cherbourg~begonnen, volgens Pim en Van Daan zijn we op 10 October
zeker vrij! De Russen nemen deel aan de actie, gisteren is hun offensief bij
Witebsk begonnen, dat is precies op de dag af drie jaar na de Duitse inval.

We hebben haast geen aardappels, in het vervolg willen we de aardappels voor
alle acht aftellen, dan kan ieder zien wat hij doet.

Je Anne.

\section*{Dinsdag, 27 Juni 1944}

Liefste Kitty,\\
De stemming is omgeslagen, het gaat enorm goed.  Cherbourg,
Witebsk en Slobin~zijn vandaag gevallen. Veel buit en gevangenen. Nu kunnen de
Engelsen aan land brengen wat ze willen, want ze hebben een haven, het hele
Cotentin, drie weken na de invasie Engels!  Geweldige prestatie. In de drie
weken na de D.-day is nog geen dag zonder regen en storm geweest, zowel hier als
in Frankrijk, maar die pech verhindert den Engelsen en Amerikanen niet hun
enorme kracht te tonen en hoe te tonen! Wel is de Wuwa (wonderwapen) in volle
actie, maar wat beduidt zo'n sisser anders dan wat schade in Engeland en volle
kranten bij de moffen? Trouwens als ze in dat Moffrika merken, dat het
`Bolsjewistische gevaar' nu werkelijk in aantocht is, zullen ze nog meer de
bibberatie krijgen.

Alle Duitse vrouwen en kinderen die niet voor de Weermacht werken, worden uit de
kuststreken geëvacueerd naar Groningen, Friesland en Gelderland. Mussert heeft
verklaard, dat hij, als de invasie hier komt, een soldatenpakje aantrekt. Wil de
dikkert misschien gaan vechten? Had ie wel eerder in Rusland kunnen doen.
Finland heeft het~vredesaanbod toentertijd geweigerd, ook nu zijn de
onderhandelingen dienaangaande weer afgebroken, wat zullen die een berouw
krijgen, die stomkoppen!

Hoever denk je, dat we op 27 Juli zijn? Je Anne.

\section*{Vrijdag, 30 Juni 1944}

Lieve Kitty,\\
Slecht weer of: bad weather at a stretch to the 30th of June.\\
Is dat niet netjes, o zeker, ik kan al een mondje vol Engels; om dat te
bewijzen~lees ik:~\emph{An ideal husband}~met woordenboek. Oorlog uitstekend!
Bobroisk, Mogilef en Orsja gevallen, veel gevangenen.

Hier alles allright, humeuren gaan vooruit. Onze hyperoptimisten triomferen.
Elli heeft haar kapsel veranderd, Miep heeft een week vrij.  Dat is het laatste
nieuws.

Je Anne.

\section*{Donderdag, 6 Juli 1944}

Lieve Kitty,\\
Mij wordt het bang om het hart als Peter er van spreekt, dat hij
later misschien misdadiger wordt of gaat speculeren; hoewel het natuurlijk als
grap bedoeld is, heb ik toch het gevoel, dat hij zelf bang voor zijn
karakterzwakheid is. Steeds weer hoor ik, zowel van Margot als van Peter: `Ja,
als ik zo sterk en moedig was als jij, als ik zo precies mijn wil doorzette, als
ik zo'n volhardende energie had, ja dan ...!'

Is het werkelijk een goede eigenschap, dat ik me niet laat beïnvloeden?  Is het
wel goed, dat ik haast uitsluitend de weg van mijn eigen geweten volg?

Eerlijk gezegd kan ik me niet goed indenken hoe iemand zeggen kan: `Ik ben
zwak', en dan nog zwak blijft. Als men zoiets toch al weet, waarom dan niet er
tegen in gewerkt, waarom zijn karakter niet trainen? Het antwoord was: `Omdat
het zoveel makkelijker is!' Dat antwoord heeft me een beetje~mismoedig gemaakt.
Makkelijk? Betekent een lui en bedrieglijk leven, dat het ook een makkelijk
leven is? O neen, het kan niet waar zijn, het mag niet zo zijn, dat slapheid en
... geld iemand zo gauw verleiden kunnen.

Ik heb er lang over nagedacht wat ik dan wel voor een antwoord moet geven, hoe
ik Peet er toe moet brengen in zichzelf te geloven en vooral zichzelf te
verbeteren; of mijn gedachtengang juist is, weet ik niet.

Ik heb me zo vaak voorgesteld hoe mooi het zou zijn als iemand me zijn
vertrouwen schenkt, maar nu, nu het zover is, zie ik pas hoe moeilijk het is
helemaal met de gedachte van die ander te denken en dan~\emph{het}~antwoord te
vinden. Vooral omdat `makkelijk' en `geld'-begrip voor mij iets volkomen vreemds
en nieuws is. Peter begint een beetje op mij te steunen en dat mag onder geen
omstandigheid. Op eigen benen in het leven staan is moeilijk voor een type als
Peter, maar nog moeilijker is het om op eigen benen te staan, als bewust levend
mens. Want als je dat bent, is het dubbel zwaar de weg te vinden door de zee van
problemen en toch standvastig te blijven. Ik dobber maar zo'n beetje rond, zoek
al dagenlang, zoek naar een volkomen afdoend middel tegen het vreselijke woord:
`makkelijk'.

Hoe kan ik hem duidelijk maken, dat wat zo makkelijk en mooi lijkt hem naar de
diepte zal trekken, de diepte waar geen vrienden, geen steun en niets moois meer
is, de diepte vanwaar opstaan haast onmogelijk is?

Wij leven allen, maar weten niet waarom en waarvoor, wij leven allen met het
doel gelukkig te worden, we leven allen verschillend en toch gelijk.  Wij drieën
zijn opgevoed in een goede kring, wij kunnen leren, wij hebben de mogelijkheid
iets te bereiken, wij hebben veel redenen om op een mooi geluk te hopen, maar
... wij moeten dat zelf verdienen. En dat is wat nooit met iets makkelijks te
bereiken valt. Geluk verdienen betekent er voor werken en goed~te doen en niet
te speculeren en lui te zijn. Luiheid mag aantrekkelijk~\emph{schijnen},
werken~\emph{geeft}~bevrediging.

Mensen die niet van werken houden kan ik niet begrijpen, maar dat is bij Peter
ook niet het geval; hij heeft alleen maar geen vast doel voor ogen, vindt
zichzelf te dom en te min om iets te presteren. Arme jongen, hij heeft nog nooit
het gevoel van anderen gelukkig maken gekend en dat kan ik hem ook niet leren.
Hij heeft geen godsdienst, spreekt spottend over Jezus Christus, vloekt met de
naam van God; hoewel ik ook niet orthodox ben, doet het me elke keer weer pijn
als ik merk hoe verlaten, hoe minachtend, hoe arm hij is.

Mensen die een godsdienst hebben, mogen blij zijn, want het is niet elk gegeven
aan bovenaardse dingen te geloven. Het is niet eens nodig bang te zijn voor
straffen na de dood; het vagevuur, de hel en de hemel zijn dingen die velen niet
aannemen kunnen, maar toch houdt de een of andere godsdienst, welke het is doet
niets ter zake, den mens op het goede pad.  Het is geen angst voor God, maar de
hooghouding van eigen eer en geweten. Hoe mooi en goed zouden alle mensen zijn,
als ze elke avond voor het inslapen zich de gebeurtenissen van de hele dag voor
ogen riepen en dan precies zouden nagaan wat goed en slecht geweest is in hun
eigen optreden. Onwillekeurig probeer je elke dag weer van voren af aan je te
verbeteren, allicht dat je dan na verloop van tijd heel wat bereikt. Dit
middeltje is voor ieder te gebruiken, het kost niets en is beslist erg nuttig.
Want wie het niet weet, moet het leren en ervaren: `Een rustig geweten maakt
sterk!'

Je Anne.

\section*{Zaterdag, 8 Juli 1944}

Lieve Kitty,\\
De hoofdvertegenwoordiger van de zaak, mijnheer B. was in
Beverwijk en heeft~zo van de veiling aardbeien gekregen. Ze kwamen hier aan,
stof g, vol zand, maar in grote~mate. Niet minder dan 24 kistjes voor kantoor en
ons. Dadelijk 's avonds werden de eerste zes glazen geweckt en acht potjes jam
gemaakt. De volgende ochtend wilde Miep voor kantoor jam koken.

Om half één geen vreemden in huis, buitendeur op slot, kistjes afhalen, Peter,
vader, Van Daan, stommelen op de trap: Anne warm water van de geyser tappen,
Margot emmer halen, alle hens aan dek! Met een heel gek gevoel in mijn maag
stapte ik de overvolle kantoorkeuken binnen. Miep, Elli, Koophuis, Henk, vader,
Peter: onderduikers en ravitaillerings-colonne, alles door elkaar en dat midden
op de dag!

Door de gordijnen kan men niet naar binnen kijken, maar hard gepraat, slaande
deuren, ik kreeg de bibberatie van opwinding. Ja, maar zijn we heus nog
ondergedoken, ging het door mij heen, zo'n gevoel moet je hebben, als je je weer
aan de wereld mag vertonen. De pan was vol, gauw naar boven. In onze keuken aan
tafel stond de rest van de familie en plukte, het moest tenminste plukken
verbeelden; er ging meer in de monden dan in de emmers. Nog een emmer was weldra
nodig, Peter ging weer naar de keuken beneden - er werd twee keer gebeld; de
emmer bleef staan, Peter holde naar boven, de kastdeur op slot! We trappelden
van ongeduld, de kraan moest dicht blijven en al wachtten de half gewassen
aardbeien nog zo op hun bad, de duikregel: `Iemand in huis, alle kranen dicht
voor het lawaai dat de watertoevoer maakt', bleef gehandhaafd.

Om één uur komt Henk met de mededeling, het was de postbode. Peter snelt de trap
weer af. Rang, de bel, rechtsomkeert. Ik ga luisteren of er iemand komt, eerst
aan de kastdeur, dan zachtjes boven aan de trap. Ten slotte hangen Peter en ik
als twee dieven over de leuning en luisteren naar het lawaai dat van beneden
komt. Geen vreemde stem, Peter gaat stiekem de trap af, blijft halverwege staan
en roept: `Elli!' Geen antwoord, nog eens: `Elli!' Het lawaai~in de keuken is
harder dan Peters stem. Hij de trap af, de keuken in. Ik sta gespannen naar
beneden te kijken: `Maak dat je naar boven komt, Peter, de accountant is er, je
moet weg!' Het is de stem van Koophuis. Zuchtend komt Peter boven, de kastdeur
gaat dicht. Half twee komt Kraler eindelijk. `O jeminee, ik zie niets anders
meer dan aardbeien, mijn ontbijt aardbeien, Henk eet aardbeien, Koophuis snoept
aardbeien, Miep kookt aardbeien, ik ruik aardbeien, wil ik dat rode goedje kwijt
en ga naar boven, dan worden hier gewassen ... aardbeien'.

De rest van de aardbeien gaat in de weck. 's Avonds: twee potten open.  Vader
maakt er gauw jam van. De volgende ochtend: twee weckglazen open, 's middags
vier weckglazen open, Van Daan had ze niet heet genoeg gesteriliseerd. Nu kookt
vader elke avond jam.

We eten pap met aardbeien, karnemelk met aardbeien, boterham met aardbeien,
aardbeien als dessert, aardbeien met suiker, aardbeien met zand. Twee dagen
dansten overal aardbeien, aardbeien, aardbeien, toen was de voorraad op of
achter slot en grendel in de potten.

`Hoor eens, Anne', roept Margot, `we hebben doppertjes gekregen van den
groenteboer om de hoek, 19 pond'. `Dat is aardig van hem', antwoord ik.
Inderdaad is het aardig, maar het werk ... poeh!

`Jullie moeten Zaterdagochtend allemaal meedoppen', kondigt moeder aan tafel
aan. En werkelijk, vanochtend na het ontbijt verscheen de grote emaille pan op
tafel, tot aan de rand met doppers gevuld. Het doppen is een vervelend werk,
maar dan moet je eens het `schilletjes uithalen' proberen. Ik denk dat het
merendeel van de mensen niet weet hoe lekker en zacht de peul van de erwt smaakt
als het binnenste velletje er uit gehaald wordt. De net aangehaalde voordelen
halen het echter nog niet bij het geweldige voordeel, dat de portie die gegeten
kan worden wel drie maal zo groot is als wanneer je alleen de erwtjes eet.

Dit velletje trekken is een buitengewoon nauwkeurig en pietepeuterig werkje, dat
misschien wel voor pedante tandartsen of preciese kantoormensen geschikt is;
voor een ongeduldige bakvis als ik is het verschrikkelijk. Om half tien zijn we
begonnen, om half elf sta ik op, om half twaalf ga ik weer zitten. Het gonst in
mijn oren: hoekje knikken, velletje trekken, draadje halen, peultje gooien en zo
voort en zo voort, het draait voor mijn ogen, groen, groen, wormpje, draadje,
rotte peul, groen, groen, groen.

Van lamlendigheid om toch wat te doen, klets ik de hele ochtend alle onzin die
er bestaat, maak ze allen aan het lachen en voel me zelf haast vergaan van
stompzinnigheid. Met elk draadje dat ik trek weet ik weer beter, dat ik nooit,
nooit alleen maar huisvrouw wil zijn!

Om 12 uur gaan we eindelijk ontbijten, maar van half één tot kwart over één
moeten we weer velletjes trekken. Ik ben zowat zeeziek als ik ophoud, de anderen
ook een beetje. Ik ga slapen tot vier uur en ben dan nog in de war door die
ellendige erwten.

Je Anne.

\section*{Zaterdag, 15 Juli 1944}

Lieve Kitty,\\
We hebben van de bibliotheek een boek gehad met de uitdagende
titel:~\emph{Hoe vindt~u het moderne jonge meisje?}~Over dit onderwerp zou ik
het vandaag graag eens hebben.

De schrijfster van het boek becritiseert `de jeugd van tegenwoordig' van top tot
teen, zonder echter alles wat jong is helemaal af te wijzen als: tot niets goeds
in staat. Integendeel, zij is eerder van mening, dat als de jeugd zou willen,
zij een grote, mooiere en betere wereld zou kunnen opbouwen, dat de jeugd de
middelen heeft, maar zich bezig houdt met oppervlakkige dingen zonder het
wezenlijk mooie een blik te gunnen.

Bij enkele passages had ik sterk het gevoel, dat de schrijfster mij bedoelde met
haar afkeuringen en daarom wil ik me~nu eindelijk eens helemaal aan je
openleggen en me verdedigen tegen deze aanval. Ik heb een sterk uitkomende
karaktertrek die iedereen die me langer kent moet opvallen, en wel mijn
zelfkennis. Ik kan mezelf bij al mijn handelingen bekijken,~alsof ik een vreemde
was. Helemaal niet vooringenomen of met een zak verontschuldigingen sta ik dan
tegenover de Anne van elke dag en kijk toe wat die goed en wat ze slecht doet.
Dat `zelfgevoel' laat me nooit los en bij elk woord dat ik uitspreek weet ik
dadelijk als het uitgesproken is: `Dit had anders moeten zijn', of `Dat is goed
zo als het is'. Ik veroordeel mezelf in zo onnoemelijk veel dingen en zie steeds
meer hoe waar dat woord van vader was: `Ieder kind moet zichzelf opvoeden'.
Ouders kunnen alleen raad of goede aanwijzingen meegeven, de uiteindelijke
vorming van iemands karakter ligt in zijn eigen hand.

Daarbij komt nog, dat ik buitengewoon veel levensmoed heb, ik voel me altijd zo
sterk en tot dragen in staat, zo vrij en zo jong! Toen ik dat voor het eerst
opmerkte was ik blij, want ik geloof niet, dat ik gauw zal buigen voor de slagen
die ieder moet opvangen.

Maar over die dingen heb ik het al zo vaak gehad, ik wilde nu even op het
hoofdstuk `vader en moeder begrijpen me niet' komen. Mijn vader en moeder hebben
me altijd erg verwend, waren lief voor me, verdedigden me en hebben gedaan wat
ouders maar doen kunnen. En toch heb ik me lang zo ontzettend eenzaam gevoeld,
buitengesloten, verwaarloosd en niet begrepen. Vader probeerde al het mogelijke
om mijn opstandigheid te temperen, het baatte niet, zelf heb ik me genezen door
mezelf het verkeerde van mijn doen en laten voor te houden.

Hoe komt het nu, dat vader me in mijn strijd nooit tot steun is geweest, dat hij
helemaal missloeg toen hij me de helpende hand wou bieden? Vader heeft de
verkeerde middelen aangepakt, hij heeft altijd tot me gesproken als tot een
kind,~dat moeilijke kindertijden moest doormaken. Dat klinkt gek, want niemand
anders dan vader heeft me steeds veel vertrouwen geschonken en niemand anders
dan vader heeft me het gevoel gegeven, dat ik verstandig ben. Maar één ding
heeft hij verwaarloosd: hij heeft er namelijk niet aan gedacht, dat mijn vechten
om er bovenop te komen voor mij belangrijker was dan al het andere. Ik wou niet
van `leeftijdsverschijnsels', `andere meisjes', `gaat vanzelf over' horen, ik
wou niet als meisje-zoals-alle-anderen, maar als Anne-op-zichzelf behandeld
worden. Dat begreep Pim niet. Trouwens ik kan iemand niet mijn vertrouwen geven,
die mij ook niet heel veel van zichzelf vertelt en omdat ik van Pim zeer weinig
afweet, zal ik de weg naar het intieme tussen ons niet kunnen betreden.

Pim stelt zich altijd op het standpunt van den ouderen vader, die ook wel eens
zulke voorbijgaande neigingen gehad heeft, maar die met mij toch niet meer als
vriend kan meeleven, hoe ijverig hij daar ook naar zoekt.

Door deze dingen ben ik er toe gekomen mijn levensbeschouwingen en mijn goed
overdachte theorieën nooit aan iemand anders mee te delen dan aan mijn dagboek
en een enkele keer aan Margot. Voor vader verborg ik alles wat mezelf beroerde:
ik heb hem nooit in mijn idealen laten delen, heb hem willens en wetens van me
vervreemd.

Ik kon niet anders, ik heb helemaal volgens mijn gevoel gehandeld, maar ik heb
gehandeld, zoals het voor mijn rust goed was. Want mijn rust en zelfvertrouwen,
dat ik zo wankel opgebouwd heb, zou ik weer helemaal verliezen, als ik nu
kritiek op mijn half-affe werk zou moeten doorstaan. En dat heb ik zelfs voor
Pim niet over, hoe hard het ook mag klinken, want niet alleen heb ik Pim in
niets in mijn innerlijk leven laten delen, ik stoot hem vaak door mijn
geprikkeldheid nog verder van me af.

Dit is een punt waarover ik veel denk: hoe komt het, dat~Pim me zo ergert? Dat
ik haast niet met hem leren kan, dat zijn liefkozingen me gemaakt lijken, dat ik
rust wil hebben en het liefst zag, dat hij me een beetje links liet liggen, tot
ik weer zekerder tegenover hem sta? Want nog steeds knaagt het verwijt aan me
van die gemene brief, die ik hem in mijn opgewondenheid durfde schrijven. O, wat
is het moeilijk om werkelijk naar alle kanten sterk en moedig te zijn!

Toch heeft dit me niet mijn ergste teleurstelling bezorgd, neen, nog veel meer
dan over vader peins ik over Peter. Ik weet heel goed, dat ik hem veroverd heb
inplaats van omgekeerd: ik heb me een droombeeld van hem geschapen, zag hem als
een stillen, gevoeligen, lieven jongen, die liefde en vriendschap zo nodig
heeft. Ik moest me eens uitspreken bij een levende, ik wou een vriend hebben die
me weer op weg hielp, ik heb het moeilijke werk volbracht en heb hem langzaam
maar zeker naar me toegedraaid. Toen ik hem ten slotte tot vriendschappelijke
gevoelens voor mij gebracht had, kwamen we vanzelf tot intimiteiten, die me nu
bij nader inzien ongehoord lijken.

We spraken over de meest verborgen dingen, maar over de dingen waarvan mijn hart
vol was en is, hebben we tot nu toe gezwegen. Ik kan nog steeds niet goed uit
Peter wijs, is hij oppervlakkig, of is het verlegenheid die hem zelfs tegenover
mij tegenhoudt? Maar dat achterwege gelaten, ik heb één fout begaan door alle
andere mogelijkheden om tot vriendschap te komen uit te schakelen en door te
trachten met intimiteiten hem nader te komen. Hij hunkert naar liefde en hij
begint me elke dag liever te vinden, dat merk ik heel goed. Hem geven onze
ontmoetingen bevrediging, bij mij werken ze alleen maar de drang uit om het
steeds weer opnieuw met hem te proberen. En toch kom ik er niet toe de
onderwerpen aan te raken, die ik zo graag aan het licht zou laten komen. Ik heb
Peter, meer dan hijzelf weet, met geweld naar mij toegehaald. Nu houdt hij zich
aan me vast en ik zie voorlopig nog geen afdoend middel hem weer van me te
scheiden en~op eigen benen te zetten.  Toen ik namelijk al heel gauw bemerkte,
dat hij geen vriend voor mijn begrip kon zijn, heb ik er naar gestreefd hem dan
tenminste op te heffen uit zijn bekrompenheid en hem groot te maken in zijn
jeugd.

`Want in zijn diepste grond is de jeugd eenzamer dan de ouderdom'. Dit gezegde
is mij uit een of ander boek bijgebleven en heb ik waar bevonden.

Is het dan wel waar, dat de volwassenen het hier moeilijker hebben dan de jeugd?
Neen, dat is zeker niet waar. Oudere mensen hebben een mening over alles en
wankelen niet meer in hun daden door het leven. Wij jongeren, hebben dubbele
moeite onze meningen te handhaven in een tijd waar alle idealisme vermeld en
verpletterd wordt, waar de mensen zich van hun lelijkste kant laten zien, waar
getwijfeld wordt aan waarheid en recht en God.

Iemand die dan nog beweert, dat de ouderen het hier in het Achterhuis veel
moeilijker hebben, realiseert zich dan zeker niet in hoeveel groter omvang de
problemen op ons toe komen stormen, de problemen waarvoor we misschien nog veel
te jong zijn, maar die zich toch zolang aan ons opdringen, totdat we na heel
lange tijd een oplossing gevonden menen te hebben, een oplossing meestal die
niet bestand blijkt tegen de feiten, die haar weer te niet doen. Dat is het
moeilijke in deze tijd: idealen, dromen, mooie verwachtingen komen nog niet bij
ons op of ze worden door de gruwelijke werkelijkheid getroffen en zo totaal
verwoest.

Het is een groot wonder, dat ik niet al mijn verwachtingen heb opgegeven, want
ze lijken absurd en onuitvoerbaar. Toch houd ik ze vast, ondanks alles, omdat ik
nog steeds aan de innerlijke goedheid van den mens geloof. Het is me ten
enenmale onmogelijk alles op te bouwen op de basis van dood, ellende en
verwarring. Ik zie hoe de wereld langzaam steeds meer in een woestijn herschapen
wordt, ik hoor steeds harder de aanrollende donder, die ook ons zal doden, ik
voel het~leed van millioenen mensen mee en toch, als ik naar de hemel kijk, denk
ik, dat alles zich weer ten goede zal wenden, dat ook deze hardheid zal
ophouden, dat er weer rust en vrede in de wereldorde zal komen.

Intussen moet ik mijn denkbeelden hoog en droog houden, in de tijden die komen
zijn ze misschien toch nog uit te voeren.

Je Anne.

\section*{Vrijdag, 21 Juli 1944}

Lieve Kitty,\\
Nu word ik hoopvol, nu eindelijk gaat het goed. Ja heus, het gaat
goed!

Knalberichten! Er is een moordaanslag op Hitler gepleegd en nu eens niet door
Joodse communisten of Engelse kapitalisten, maar door een edel-germaanse Duitse
generaal, die graaf en bovendien nog jong is. De Goddelijke voorzienigheid heeft
den Führer het leven gered en hij is er jammer genoeg met een paar schrammetjes
en wat brandwonden afgekomen.  Een paar officieren en generaals uit zijn naaste
omgeving zijn gedood of gewond. De hoofddader is gefusilleerd.

Het beste bewijs toch wel, dat er veel officieren en generaals zijn die de pé
aan de oorlog hebben en Hitler graag in de diepste gewelven zagen afdalen. Hun
streven is na Hitlers dood een militaire dictatuur te vestigen, door middel
daarvan vrede met de Geallieerden te sluiten, zich opnieuw te bewapenen en na
een twintigtal jaren opnieuw een oorlog te beginnen. Misschien heeft de
voorzienigheid wel expres nog een beetje gedraald om hem uit de weg te ruimen,
want het is voor de Geallieerden veel makkelijker en ook voordeliger als de
smetteloze Germanen elkander doodslaan; des te minder werk blijft er over voor
de Russen en Engelsen en des te gauwer kunnen ze weer met de opbouw van hun
eigen steden beginnen.

Maar zo ver zijn we nog niet en ik wil ook volstrekt niet op~de glorierijke
feiten vooruitlopen. Toch merk je wel dat wat ik nu zeg nuchtere realiteit is,
die met de beide benen op de grond staat; bij uitzondering zit ik nu eens niet
over hogere idealen te bazelen. Hitler is voorts nog zo vriendelijk geweest aan
zijn trouw en aanhankelijk volk mede te delen, dat alle militairen vanaf vandaag
de Gestapo te gehoorzamen hebben en dat elke soldaat die weet dat zijn superieur
betrokken geweest is bij deze laffe en minderwaardige aanslag, dezen zonder vorm
van proces mag neerknallen.

Mooie geschiedenis zal dat worden. Pietje Wijs heeft pijnlijke voeten van het
lange lopen, zijn baas de officier snauwt hem af. Pietje grijpt zijn geweer,
roept: `Jij wou den Führer vermoorden, daar is je loon!' Een knal en de
hoogmoedige chef, die het heeft gewaagd Pietje standjes te geven, is het eeuwige
leven (of is het eeuwige dood?) ingegaan. Op het laatst zal het er dan zo
uitzien, dat de heren officieren hun broek voldoen van angst als ze een soldaat
tegenkomen of ergens leiding moeten geven, omdat de soldaten meer zullen durven
zeggen en doen dan zijzelf.  Snap je het een beetje, of ben ik weer
hakketakkerig geweest? Ik kan er niets aan doen, ben veel te vrolijk om logisch
te zijn, bij het vooruitzicht dat ik met October wel weer eens in de
schoolbanken zou kunnen zitten! O lala, heb ik niet net nog verteld, dat ik niet
voorbarig wil zijn? Vergeef me, ik heb niet voor niets de naam een bundeltje
tegenspraak te zijn!

Je Anne.

\section*{Dinsdag, 1 Augustus 1944}

Lieve Kitty,\\
`Een bundeltje tegenspraak' is de laatste zin van mijn vorige
brief en de eerste van~mijn huidige. `Een bundeltje tegenspraak', kun jij me
precies uitleggen wat dat is? Wat betekent tegenspraak? Zoals zovele woorden
heeft het twee betekenissen, tegenspraak van buiten en tegenspraak van binnen.

Het eerste is het gewone `zich niet neerleggen bij andermans meningen, het zelf
beter weten, het laatste woord hebben', en n, allemaal onaangename eigenschappen
waarom ik bekend sta. Het tweede, daar sta ik niet om bekend, dat is mijn eigen
geheim.

Ik heb je al eens meer verteld, dat mijn ziel als het ware in tweeën gesplitst
is. De ene kant herbergt mijn uitgelaten vrolijkheid, spotternijen om alles,
mijn levenslust en vooral het opvatten van de lichte kant van alles. Daaronder
versta ik geen aanstoot nemen aan irten, een zoen, een omhelzing, een onnette
mop. Deze kant zit meestal op de loer en verdringt de andere, die veel mooier,
reiner en dieper is.  Niet waar, de mooie zijde van Anne kent niemand en daarom
kunnen mij ook zo weinig mensen lijden.

Zeker, ik ben een amusante clown voor één middag, dan heeft iedereen weer voor
een maand genoeg van me. Eigenlijk precies hetzelfde wat een liefdes film voor
diepdenkende mensen is, eenvoudig een afleiding, vermaak voor één keer, iets om
gauw te vergeten, niet slecht maar nog minder goed. Het is heel naar voor me je
dit te moeten vertellen, maar waarom zou ik het niet doen, als ik toch weet dat
het de waarheid is? Mijn lichtere oppervlakkige kant zal de diepere altijd te
vlug af zijn en daarom steeds overwinnen. Je kunt je niet voorstellen hoe vaak
ik niet al geprobeerd heb deze Anne, die maar de helft is van alles wat Anne
heet, weg te duwen, lam te slaan, te verbergen: het gaat niet en ik weet ook
waarom het niet gaat.

Ik ben erg bang dat allen die me kennen zoals ik altijd ben, zullen ontdekken
dat ik een andere kant heb, een mooiere en betere kant. Ik ben bang dat ze met
me zullen spotten, me belachelijk en sentimenteel vinden, me niet ernstig nemen.
Ik ben gewend niet ernstig genomen te worden, maar alleen de `lichte' Anne is
het gewend en kan het verdragen, de `zwaardere' is daarvoor te zwak. Als ik
werkelijk eens een kwartier de goede Anne met geweld voor het
voetlicht~geplaatst heb, dan trekt zij zich als een kruidje roer-me-niet samen,
zodra ze moet spreken, laat Anne no 1 aan het woord en is, vóór ik het weet,
verdwenen.

In gezelschap is de lieve Anne dus nog nooit, nog niet één keer tevoorschijn
gekomen, maar in het alleen-zijn voert zij haast altijd de boventoon. Ik weet
precies hoe ik zou willen zijn, hoe ik ook ben ...  van binnen, maar helaas, ik
ben het enkel voor mezelf. En dat is misschien, neen, heel zeker de reden waarom
ik mezelf een gelukkige binnennatuur noem en andere mensen me een gelukkige
buitennatuur vinden.  Van binnen wijst de reine Anne me de weg, van buiten ben
ik niets dan een van uitgelatenheid losgerukt geitje.

Zoals ik al zei, ik voel alles anders aan dan ik het uitspreek en daardoor heb
ik de naam van jongensnaloopster, flirt, wijsneus en romannetjesleester
gekregen. De vrolijke Anne lacht er om, geeft brutale antwoorden, trekt haar
schouders onverschillig op, doet of het haar niets kan schelen, maar o wee,
precies omgekeerd reageert de stille Anne. Als ik helemaal eerlijk ben, dan wil
ik je wel bekennen, dat het me wel deert, dat ik onnoemelijk veel moeite doe
anders te worden, maar dat ik telkens weer tegen sterkere legers vecht.

Het snikt in me: `Zie je wel, dat is er van je terecht gekomen: slechte
meningen, spottende en verstoorde gezichten, mensen die je antipathiek vinden en
dat alles omdat je niet naar de goede raad van je eigen goede helft luistert.
Ach, ik zou wel willen luisteren, maar het gaat niet; als ik stil en ernstig
ben, denkt iedereen dat het een nieuwe komedie is, en dan moet ik me wel met een
grapje er uit redden, om nog maar niet eens te spreken van mijn eigen familie,
die beslist denkt dat ik ziek ben, me hoofdpijnpillen en zenuwtabletten laat
slikken, me hals en voorhoofd voelt of ik koorts heb, naar mijn ontlasting
vraagt en mijn slechte bui becritiseert. Dat houd ik niet vol: als er zo op me
gelet wordt, word ik eerst snibbig, dan verdrietig en ten slotte draai ik mijn
hart~weer om, draai het slechte naar buiten, het goede naar binnen en zoek
aldoor naar een middel om te worden, zoals ik zo erg graag zou willen zijn en
zoals ik zou kunnen zijn, als ... er geén andere mensen in de wereld zouden
wonen.

Je Anne.

\emph{\textbf{Slotwoord}}
\end{document}
